
% Default to the notebook output style

    


% Inherit from the specified cell style.




    
\documentclass{article}

    
    
    \usepackage{graphicx} % Used to insert images
    \usepackage{adjustbox} % Used to constrain images to a maximum size 
    \usepackage{color} % Allow colors to be defined
    \usepackage{enumerate} % Needed for markdown enumerations to work
    \usepackage{geometry} % Used to adjust the document margins
    \usepackage{amsmath} % Equations
    \usepackage{amssymb} % Equations
    \usepackage{eurosym} % defines \euro
    \usepackage[mathletters]{ucs} % Extended unicode (utf-8) support
    \usepackage[utf8x]{inputenc} % Allow utf-8 characters in the tex document
    \usepackage{fancyvrb} % verbatim replacement that allows latex
    \usepackage{grffile} % extends the file name processing of package graphics 
                         % to support a larger range 
    % The hyperref package gives us a pdf with properly built
    % internal navigation ('pdf bookmarks' for the table of contents,
    % internal cross-reference links, web links for URLs, etc.)
    \usepackage{hyperref}
    \usepackage{longtable} % longtable support required by pandoc >1.10
    \usepackage{booktabs}  % table support for pandoc > 1.12.2
    \usepackage{ulem} % ulem is needed to support strikethroughs (\sout)
    

    
    
    \definecolor{orange}{cmyk}{0,0.4,0.8,0.2}
    \definecolor{darkorange}{rgb}{.71,0.21,0.01}
    \definecolor{darkgreen}{rgb}{.12,.54,.11}
    \definecolor{myteal}{rgb}{.26, .44, .56}
    \definecolor{gray}{gray}{0.45}
    \definecolor{lightgray}{gray}{.95}
    \definecolor{mediumgray}{gray}{.8}
    \definecolor{inputbackground}{rgb}{.95, .95, .85}
    \definecolor{outputbackground}{rgb}{.95, .95, .95}
    \definecolor{traceback}{rgb}{1, .95, .95}
    % ansi colors
    \definecolor{red}{rgb}{.6,0,0}
    \definecolor{green}{rgb}{0,.65,0}
    \definecolor{brown}{rgb}{0.6,0.6,0}
    \definecolor{blue}{rgb}{0,.145,.698}
    \definecolor{purple}{rgb}{.698,.145,.698}
    \definecolor{cyan}{rgb}{0,.698,.698}
    \definecolor{lightgray}{gray}{0.5}
    
    % bright ansi colors
    \definecolor{darkgray}{gray}{0.25}
    \definecolor{lightred}{rgb}{1.0,0.39,0.28}
    \definecolor{lightgreen}{rgb}{0.48,0.99,0.0}
    \definecolor{lightblue}{rgb}{0.53,0.81,0.92}
    \definecolor{lightpurple}{rgb}{0.87,0.63,0.87}
    \definecolor{lightcyan}{rgb}{0.5,1.0,0.83}
    
    % commands and environments needed by pandoc snippets
    % extracted from the output of `pandoc -s`
    \providecommand{\tightlist}{%
      \setlength{\itemsep}{0pt}\setlength{\parskip}{0pt}}
    \DefineVerbatimEnvironment{Highlighting}{Verbatim}{commandchars=\\\{\}}
    % Add ',fontsize=\small' for more characters per line
    \newenvironment{Shaded}{}{}
    \newcommand{\KeywordTok}[1]{\textcolor[rgb]{0.00,0.44,0.13}{\textbf{{#1}}}}
    \newcommand{\DataTypeTok}[1]{\textcolor[rgb]{0.56,0.13,0.00}{{#1}}}
    \newcommand{\DecValTok}[1]{\textcolor[rgb]{0.25,0.63,0.44}{{#1}}}
    \newcommand{\BaseNTok}[1]{\textcolor[rgb]{0.25,0.63,0.44}{{#1}}}
    \newcommand{\FloatTok}[1]{\textcolor[rgb]{0.25,0.63,0.44}{{#1}}}
    \newcommand{\CharTok}[1]{\textcolor[rgb]{0.25,0.44,0.63}{{#1}}}
    \newcommand{\StringTok}[1]{\textcolor[rgb]{0.25,0.44,0.63}{{#1}}}
    \newcommand{\CommentTok}[1]{\textcolor[rgb]{0.38,0.63,0.69}{\textit{{#1}}}}
    \newcommand{\OtherTok}[1]{\textcolor[rgb]{0.00,0.44,0.13}{{#1}}}
    \newcommand{\AlertTok}[1]{\textcolor[rgb]{1.00,0.00,0.00}{\textbf{{#1}}}}
    \newcommand{\FunctionTok}[1]{\textcolor[rgb]{0.02,0.16,0.49}{{#1}}}
    \newcommand{\RegionMarkerTok}[1]{{#1}}
    \newcommand{\ErrorTok}[1]{\textcolor[rgb]{1.00,0.00,0.00}{\textbf{{#1}}}}
    \newcommand{\NormalTok}[1]{{#1}}
    
    % Additional commands for more recent versions of Pandoc
    \newcommand{\ConstantTok}[1]{\textcolor[rgb]{0.53,0.00,0.00}{{#1}}}
    \newcommand{\SpecialCharTok}[1]{\textcolor[rgb]{0.25,0.44,0.63}{{#1}}}
    \newcommand{\VerbatimStringTok}[1]{\textcolor[rgb]{0.25,0.44,0.63}{{#1}}}
    \newcommand{\SpecialStringTok}[1]{\textcolor[rgb]{0.73,0.40,0.53}{{#1}}}
    \newcommand{\ImportTok}[1]{{#1}}
    \newcommand{\DocumentationTok}[1]{\textcolor[rgb]{0.73,0.13,0.13}{\textit{{#1}}}}
    \newcommand{\AnnotationTok}[1]{\textcolor[rgb]{0.38,0.63,0.69}{\textbf{\textit{{#1}}}}}
    \newcommand{\CommentVarTok}[1]{\textcolor[rgb]{0.38,0.63,0.69}{\textbf{\textit{{#1}}}}}
    \newcommand{\VariableTok}[1]{\textcolor[rgb]{0.10,0.09,0.49}{{#1}}}
    \newcommand{\ControlFlowTok}[1]{\textcolor[rgb]{0.00,0.44,0.13}{\textbf{{#1}}}}
    \newcommand{\OperatorTok}[1]{\textcolor[rgb]{0.40,0.40,0.40}{{#1}}}
    \newcommand{\BuiltInTok}[1]{{#1}}
    \newcommand{\ExtensionTok}[1]{{#1}}
    \newcommand{\PreprocessorTok}[1]{\textcolor[rgb]{0.74,0.48,0.00}{{#1}}}
    \newcommand{\AttributeTok}[1]{\textcolor[rgb]{0.49,0.56,0.16}{{#1}}}
    \newcommand{\InformationTok}[1]{\textcolor[rgb]{0.38,0.63,0.69}{\textbf{\textit{{#1}}}}}
    \newcommand{\WarningTok}[1]{\textcolor[rgb]{0.38,0.63,0.69}{\textbf{\textit{{#1}}}}}
    
    
    % Define a nice break command that doesn't care if a line doesn't already
    % exist.
    \def\br{\hspace*{\fill} \\* }
    % Math Jax compatability definitions
    \def\gt{>}
    \def\lt{<}
    % Document parameters
    \title{sujet}
    
    
    

    % Pygments definitions
    
\makeatletter
\def\PY@reset{\let\PY@it=\relax \let\PY@bf=\relax%
    \let\PY@ul=\relax \let\PY@tc=\relax%
    \let\PY@bc=\relax \let\PY@ff=\relax}
\def\PY@tok#1{\csname PY@tok@#1\endcsname}
\def\PY@toks#1+{\ifx\relax#1\empty\else%
    \PY@tok{#1}\expandafter\PY@toks\fi}
\def\PY@do#1{\PY@bc{\PY@tc{\PY@ul{%
    \PY@it{\PY@bf{\PY@ff{#1}}}}}}}
\def\PY#1#2{\PY@reset\PY@toks#1+\relax+\PY@do{#2}}

\expandafter\def\csname PY@tok@nf\endcsname{\def\PY@tc##1{\textcolor[rgb]{0.00,0.00,1.00}{##1}}}
\expandafter\def\csname PY@tok@kn\endcsname{\let\PY@bf=\textbf\def\PY@tc##1{\textcolor[rgb]{0.00,0.50,0.00}{##1}}}
\expandafter\def\csname PY@tok@sh\endcsname{\def\PY@tc##1{\textcolor[rgb]{0.73,0.13,0.13}{##1}}}
\expandafter\def\csname PY@tok@kt\endcsname{\def\PY@tc##1{\textcolor[rgb]{0.69,0.00,0.25}{##1}}}
\expandafter\def\csname PY@tok@o\endcsname{\def\PY@tc##1{\textcolor[rgb]{0.40,0.40,0.40}{##1}}}
\expandafter\def\csname PY@tok@nb\endcsname{\def\PY@tc##1{\textcolor[rgb]{0.00,0.50,0.00}{##1}}}
\expandafter\def\csname PY@tok@gu\endcsname{\let\PY@bf=\textbf\def\PY@tc##1{\textcolor[rgb]{0.50,0.00,0.50}{##1}}}
\expandafter\def\csname PY@tok@vc\endcsname{\def\PY@tc##1{\textcolor[rgb]{0.10,0.09,0.49}{##1}}}
\expandafter\def\csname PY@tok@ss\endcsname{\def\PY@tc##1{\textcolor[rgb]{0.10,0.09,0.49}{##1}}}
\expandafter\def\csname PY@tok@vg\endcsname{\def\PY@tc##1{\textcolor[rgb]{0.10,0.09,0.49}{##1}}}
\expandafter\def\csname PY@tok@gs\endcsname{\let\PY@bf=\textbf}
\expandafter\def\csname PY@tok@sd\endcsname{\let\PY@it=\textit\def\PY@tc##1{\textcolor[rgb]{0.73,0.13,0.13}{##1}}}
\expandafter\def\csname PY@tok@sc\endcsname{\def\PY@tc##1{\textcolor[rgb]{0.73,0.13,0.13}{##1}}}
\expandafter\def\csname PY@tok@ow\endcsname{\let\PY@bf=\textbf\def\PY@tc##1{\textcolor[rgb]{0.67,0.13,1.00}{##1}}}
\expandafter\def\csname PY@tok@gh\endcsname{\let\PY@bf=\textbf\def\PY@tc##1{\textcolor[rgb]{0.00,0.00,0.50}{##1}}}
\expandafter\def\csname PY@tok@kr\endcsname{\let\PY@bf=\textbf\def\PY@tc##1{\textcolor[rgb]{0.00,0.50,0.00}{##1}}}
\expandafter\def\csname PY@tok@kc\endcsname{\let\PY@bf=\textbf\def\PY@tc##1{\textcolor[rgb]{0.00,0.50,0.00}{##1}}}
\expandafter\def\csname PY@tok@nd\endcsname{\def\PY@tc##1{\textcolor[rgb]{0.67,0.13,1.00}{##1}}}
\expandafter\def\csname PY@tok@ge\endcsname{\let\PY@it=\textit}
\expandafter\def\csname PY@tok@mi\endcsname{\def\PY@tc##1{\textcolor[rgb]{0.40,0.40,0.40}{##1}}}
\expandafter\def\csname PY@tok@no\endcsname{\def\PY@tc##1{\textcolor[rgb]{0.53,0.00,0.00}{##1}}}
\expandafter\def\csname PY@tok@ne\endcsname{\let\PY@bf=\textbf\def\PY@tc##1{\textcolor[rgb]{0.82,0.25,0.23}{##1}}}
\expandafter\def\csname PY@tok@se\endcsname{\let\PY@bf=\textbf\def\PY@tc##1{\textcolor[rgb]{0.73,0.40,0.13}{##1}}}
\expandafter\def\csname PY@tok@mf\endcsname{\def\PY@tc##1{\textcolor[rgb]{0.40,0.40,0.40}{##1}}}
\expandafter\def\csname PY@tok@sr\endcsname{\def\PY@tc##1{\textcolor[rgb]{0.73,0.40,0.53}{##1}}}
\expandafter\def\csname PY@tok@s2\endcsname{\def\PY@tc##1{\textcolor[rgb]{0.73,0.13,0.13}{##1}}}
\expandafter\def\csname PY@tok@si\endcsname{\let\PY@bf=\textbf\def\PY@tc##1{\textcolor[rgb]{0.73,0.40,0.53}{##1}}}
\expandafter\def\csname PY@tok@sx\endcsname{\def\PY@tc##1{\textcolor[rgb]{0.00,0.50,0.00}{##1}}}
\expandafter\def\csname PY@tok@kd\endcsname{\let\PY@bf=\textbf\def\PY@tc##1{\textcolor[rgb]{0.00,0.50,0.00}{##1}}}
\expandafter\def\csname PY@tok@vi\endcsname{\def\PY@tc##1{\textcolor[rgb]{0.10,0.09,0.49}{##1}}}
\expandafter\def\csname PY@tok@s1\endcsname{\def\PY@tc##1{\textcolor[rgb]{0.73,0.13,0.13}{##1}}}
\expandafter\def\csname PY@tok@gd\endcsname{\def\PY@tc##1{\textcolor[rgb]{0.63,0.00,0.00}{##1}}}
\expandafter\def\csname PY@tok@sb\endcsname{\def\PY@tc##1{\textcolor[rgb]{0.73,0.13,0.13}{##1}}}
\expandafter\def\csname PY@tok@ni\endcsname{\let\PY@bf=\textbf\def\PY@tc##1{\textcolor[rgb]{0.60,0.60,0.60}{##1}}}
\expandafter\def\csname PY@tok@m\endcsname{\def\PY@tc##1{\textcolor[rgb]{0.40,0.40,0.40}{##1}}}
\expandafter\def\csname PY@tok@c1\endcsname{\let\PY@it=\textit\def\PY@tc##1{\textcolor[rgb]{0.25,0.50,0.50}{##1}}}
\expandafter\def\csname PY@tok@gt\endcsname{\def\PY@tc##1{\textcolor[rgb]{0.00,0.27,0.87}{##1}}}
\expandafter\def\csname PY@tok@il\endcsname{\def\PY@tc##1{\textcolor[rgb]{0.40,0.40,0.40}{##1}}}
\expandafter\def\csname PY@tok@bp\endcsname{\def\PY@tc##1{\textcolor[rgb]{0.00,0.50,0.00}{##1}}}
\expandafter\def\csname PY@tok@mb\endcsname{\def\PY@tc##1{\textcolor[rgb]{0.40,0.40,0.40}{##1}}}
\expandafter\def\csname PY@tok@nv\endcsname{\def\PY@tc##1{\textcolor[rgb]{0.10,0.09,0.49}{##1}}}
\expandafter\def\csname PY@tok@s\endcsname{\def\PY@tc##1{\textcolor[rgb]{0.73,0.13,0.13}{##1}}}
\expandafter\def\csname PY@tok@gi\endcsname{\def\PY@tc##1{\textcolor[rgb]{0.00,0.63,0.00}{##1}}}
\expandafter\def\csname PY@tok@go\endcsname{\def\PY@tc##1{\textcolor[rgb]{0.53,0.53,0.53}{##1}}}
\expandafter\def\csname PY@tok@ch\endcsname{\let\PY@it=\textit\def\PY@tc##1{\textcolor[rgb]{0.25,0.50,0.50}{##1}}}
\expandafter\def\csname PY@tok@gr\endcsname{\def\PY@tc##1{\textcolor[rgb]{1.00,0.00,0.00}{##1}}}
\expandafter\def\csname PY@tok@err\endcsname{\def\PY@bc##1{\setlength{\fboxsep}{0pt}\fcolorbox[rgb]{1.00,0.00,0.00}{1,1,1}{\strut ##1}}}
\expandafter\def\csname PY@tok@cpf\endcsname{\let\PY@it=\textit\def\PY@tc##1{\textcolor[rgb]{0.25,0.50,0.50}{##1}}}
\expandafter\def\csname PY@tok@kp\endcsname{\def\PY@tc##1{\textcolor[rgb]{0.00,0.50,0.00}{##1}}}
\expandafter\def\csname PY@tok@nt\endcsname{\let\PY@bf=\textbf\def\PY@tc##1{\textcolor[rgb]{0.00,0.50,0.00}{##1}}}
\expandafter\def\csname PY@tok@nc\endcsname{\let\PY@bf=\textbf\def\PY@tc##1{\textcolor[rgb]{0.00,0.00,1.00}{##1}}}
\expandafter\def\csname PY@tok@cm\endcsname{\let\PY@it=\textit\def\PY@tc##1{\textcolor[rgb]{0.25,0.50,0.50}{##1}}}
\expandafter\def\csname PY@tok@nl\endcsname{\def\PY@tc##1{\textcolor[rgb]{0.63,0.63,0.00}{##1}}}
\expandafter\def\csname PY@tok@cs\endcsname{\let\PY@it=\textit\def\PY@tc##1{\textcolor[rgb]{0.25,0.50,0.50}{##1}}}
\expandafter\def\csname PY@tok@c\endcsname{\let\PY@it=\textit\def\PY@tc##1{\textcolor[rgb]{0.25,0.50,0.50}{##1}}}
\expandafter\def\csname PY@tok@k\endcsname{\let\PY@bf=\textbf\def\PY@tc##1{\textcolor[rgb]{0.00,0.50,0.00}{##1}}}
\expandafter\def\csname PY@tok@gp\endcsname{\let\PY@bf=\textbf\def\PY@tc##1{\textcolor[rgb]{0.00,0.00,0.50}{##1}}}
\expandafter\def\csname PY@tok@cp\endcsname{\def\PY@tc##1{\textcolor[rgb]{0.74,0.48,0.00}{##1}}}
\expandafter\def\csname PY@tok@w\endcsname{\def\PY@tc##1{\textcolor[rgb]{0.73,0.73,0.73}{##1}}}
\expandafter\def\csname PY@tok@mo\endcsname{\def\PY@tc##1{\textcolor[rgb]{0.40,0.40,0.40}{##1}}}
\expandafter\def\csname PY@tok@mh\endcsname{\def\PY@tc##1{\textcolor[rgb]{0.40,0.40,0.40}{##1}}}
\expandafter\def\csname PY@tok@nn\endcsname{\let\PY@bf=\textbf\def\PY@tc##1{\textcolor[rgb]{0.00,0.00,1.00}{##1}}}
\expandafter\def\csname PY@tok@na\endcsname{\def\PY@tc##1{\textcolor[rgb]{0.49,0.56,0.16}{##1}}}

\def\PYZbs{\char`\\}
\def\PYZus{\char`\_}
\def\PYZob{\char`\{}
\def\PYZcb{\char`\}}
\def\PYZca{\char`\^}
\def\PYZam{\char`\&}
\def\PYZlt{\char`\<}
\def\PYZgt{\char`\>}
\def\PYZsh{\char`\#}
\def\PYZpc{\char`\%}
\def\PYZdl{\char`\$}
\def\PYZhy{\char`\-}
\def\PYZsq{\char`\'}
\def\PYZdq{\char`\"}
\def\PYZti{\char`\~}
% for compatibility with earlier versions
\def\PYZat{@}
\def\PYZlb{[}
\def\PYZrb{]}
\makeatother


    % Exact colors from NB
    \definecolor{incolor}{rgb}{0.0, 0.0, 0.5}
    \definecolor{outcolor}{rgb}{0.545, 0.0, 0.0}



    
    % Prevent overflowing lines due to hard-to-break entities
    \sloppy 
    % Setup hyperref package
    \hypersetup{
      breaklinks=true,  % so long urls are correctly broken across lines
      colorlinks=true,
      urlcolor=blue,
      linkcolor=darkorange,
      citecolor=darkgreen,
      }
    % Slightly bigger margins than the latex defaults
    
    \geometry{verbose,tmargin=1in,bmargin=1in,lmargin=1in,rmargin=1in}
    
    

    \begin{document}
    
    
    \maketitle
    
    

    
    \begin{Verbatim}[commandchars=\\\{\}]
{\color{incolor}In [{\color{incolor}1}]:} \PY{c+c1}{\PYZsh{} votre code }
        \PY{k+kn}{import} \PY{n+nn}{csv}
        \PY{k}{with} \PY{n+nb}{open}\PY{p}{(}\PY{l+s+s1}{\PYZsq{}}\PY{l+s+s1}{regularite\PYZhy{}mensuelle\PYZhy{}tgv.csv}\PY{l+s+s1}{\PYZsq{}}\PY{p}{,} \PY{l+s+s1}{\PYZsq{}}\PY{l+s+s1}{r}\PY{l+s+s1}{\PYZsq{}}\PY{p}{,} \PY{n}{encoding}\PY{o}{=}\PY{l+s+s1}{\PYZsq{}}\PY{l+s+s1}{utf\PYZhy{}8}\PY{l+s+s1}{\PYZsq{}}\PY{p}{)} \PY{k}{as} \PY{n}{csvfile}\PY{p}{:}
            \PY{n}{reader} \PY{o}{=} \PY{n}{csv}\PY{o}{.}\PY{n}{reader}\PY{p}{(}\PY{n}{csvfile}\PY{p}{,} \PY{n}{delimiter}\PY{o}{=}\PY{l+s+s1}{\PYZsq{}}\PY{l+s+s1}{;}\PY{l+s+s1}{\PYZsq{}}\PY{p}{,} \PY{n}{quotechar}\PY{o}{=}\PY{l+s+s1}{\PYZsq{}}\PY{l+s+s1}{\PYZdq{}}\PY{l+s+s1}{\PYZsq{}}\PY{p}{)}
            \PY{k}{for} \PY{n}{row} \PY{o+ow}{in} \PY{n}{reader}\PY{p}{:}
                \PY{n+nb}{print}\PY{p}{(}\PY{l+s+s1}{\PYZsq{}}\PY{l+s+s1}{; }\PY{l+s+s1}{\PYZsq{}}\PY{o}{.}\PY{n}{join}\PY{p}{(}\PY{n}{row}\PY{p}{)}\PY{p}{)}
\end{Verbatim}

    \begin{Verbatim}[commandchars=\\\{\}]
Date; Axe; Départ; Arrivée; Nombre de trains programmés; Nombre de trains ayant circulé; Nombre de trains annulés; Nombre de trains en retard à l'arrivée; Régularité; Commentaires
2014-03; Nord; PARIS NORD; DOUAI; 196; 196; 0; 11; 94.4; 
2014-03; Sud-Est; PARIS LYON; GRENOBLE; 259; 259; 0; 21; 91.9; 
2014-03; Atlantique; LAVAL; PARIS MONTPARNASSE; 243; 243; 0; 14; 94.2; 
2014-03; Atlantique; PARIS MONTPARNASSE; ANGERS SAINT LAUD; 435; 435; 0; 25; 94.3; 
2014-03; Nord; LYON PART DIEU; LILLE; 247; 247; 0; 50; 79.8; Défaut d'alimentation électrique à Montanay le 01. Agression d'un agent à Lyon impactant les 01 et 02. Vol de câbles aux Mazes le 01 et à Lille le 29. Problème matériel à Arsy le 02. Heurt d'un animal à St Georges d'Espéranche le 13 et à Cluny le 21. Dérangement d'installations au Creusot du 16 au 18 et à Macon le 24. Accident de personne à Aix le 17 et à Vallauris le 21. Incendie dans le tunnel de Marseille le 26. Nombreuses limitations de vitesse pour travaux, notamment à Lapalud les 12, 13 et 14.
2014-03; Nord; LILLE; PARIS NORD; 615; 613; 2; 63; 89.7; 
2014-03; Sud-Est; MONTPELLIER; LYON PART DIEU; 397; 395; 2; 66; 83.3; 
2014-03; Sud-Est; PARIS LYON; LYON PART DIEU; 626; 626; 0; 18; 97.1; 
2014-03; Est; PARIS EST; METZ; 306; 306; 0; 4; 98.7; 
2014-03; Atlantique; ANGOULEME; PARIS MONTPARNASSE; 328; 327; 1; 39; 88.1; 
2014-03; Atlantique; PARIS MONTPARNASSE; ANGOULEME; 329; 329; 0; 21; 93.6; 
2014-03; Sud-Est; PARIS LYON; MULHOUSE VILLE; 310; 310; 0; 15; 95.2; 
2014-03; Sud-Est; VALENCE ALIXAN TGV; PARIS LYON; 273; 273; 0; 39; 85.7; 
2014-03; Sud-Est; PARIS LYON; VALENCE ALIXAN TGV; 268; 268; 0; 21; 92.2; 
2014-03; Nord; ARRAS; PARIS NORD; 339; 338; 1; 38; 88.8; 
2014-03; Nord; PARIS NORD; ARRAS; 338; 338; 0; 11; 96.7; 
2014-03; Atlantique; PARIS MONTPARNASSE; VANNES; 168; 168; 0; 10; 94.0; 
2014-03; Atlantique; PARIS MONTPARNASSE; BORDEAUX ST JEAN; 646; 646; 0; 39; 94.0; 
2014-04; Sud-Est; CHAMBERY CHALLES LES EAUX; PARIS LYON; 229; 229; 0; 30; 86.9; Le TGV 9240 du 25 avril est arrivé à Paris avec 6h27min de retard suite à un accident de personne au Creusot
2014-04; Nord; DUNKERQUE; PARIS NORD; 110; 110; 0; 5; 95.5; 
2014-04; Sud-Est; GRENOBLE; PARIS LYON; 236; 236; 0; 16; 93.2; 
2014-04; Atlantique; LAVAL; PARIS MONTPARNASSE; 236; 236; 0; 7; 97.0; 
2014-04; Sud-Est; PARIS LYON; LE CREUSOT MONTCEAU MONTCHANIN; 199; 199; 0; 22; 88.9; 
2014-04; Atlantique; PARIS MONTPARNASSE; LE MANS; 444; 444; 0; 34; 92.3; 
2014-04; Sud-Est; LYON PART DIEU; MONTPELLIER; 387; 387; 0; 59; 84.8; 
2014-04; Atlantique; LYON PART DIEU; RENNES; 30; 30; 0; 0; 100.0; 
2014-04; Sud-Est; PARIS LYON; MARSEILLE ST CHARLES; 470; 470; 0; 33; 93.0; 
2014-04; Est; PARIS EST; METZ; 296; 296; 0; 14; 95.3; 
2014-04; Sud-Est; MONTPELLIER; PARIS LYON; 357; 357; 0; 31; 91.3; 
2014-04; Sud-Est; MULHOUSE VILLE; PARIS LYON; 312; 311; 1; 17; 94.5; 
2014-04; Sud-Est; PERPIGNAN; PARIS LYON; 154; 154; 0; 10; 93.5; 
2014-04; Sud-Est; PARIS LYON; PERPIGNAN; 157; 157; 0; 12; 92.4; 
2014-04; Atlantique; PARIS MONTPARNASSE; QUIMPER; 144; 144; 0; 3; 97.9; 
2014-04; Atlantique; PARIS MONTPARNASSE; ST MALO; 51; 51; 0; 1; 98.0; 
2014-04; Nord; PARIS NORD; ARRAS; 331; 330; 1; 16; 95.2; 
2014-05; Sud-Est; AIX EN PROVENCE TGV; PARIS LYON; 431; 431; 0; 65; 84.9; 
2014-05; Sud-Est; PARIS LYON; CHAMBERY CHALLES LES EAUX; 209; 209; 0; 28; 86.6; 
2014-05; Nord; DUNKERQUE; PARIS NORD; 114; 114; 0; 3; 97.4; 
2014-05; Sud-Est; PARIS LYON; LE CREUSOT MONTCEAU MONTCHANIN; 202; 202; 0; 14; 93.1; 
2014-05; Atlantique; PARIS MONTPARNASSE; ANGERS SAINT LAUD; 443; 443; 0; 14; 96.8; 
2014-05; Nord; LILLE; LYON PART DIEU; 217; 217; 0; 12; 94.5; 
2014-05; Nord; LYON PART DIEU; LILLE; 252; 252; 0; 48; 81.0; Agression d'un agent à Lyon le 2, dérangement d'installations à Marseille le 5 et à Valence le 16, incident caténaire à Lyon le 6, accident de personne à Upie le 11, à Valence le 21 et à Oignies le 26, heurt d'animal à Allan le 11 et à Bonlieu le 19. Nombreuses limitations de vitesse suite à des travaux, notamment à Cesseins, TGV Haute Picardie et Marseille.
2014-05; Nord; PARIS NORD; LILLE; 620; 619; 1; 52; 91.6; 
2014-05; Sud-Est; LYON PART DIEU; PARIS LYON; 606; 606; 0; 13; 97.9; 
2014-05; Atlantique; LYON PART DIEU; RENNES; 30; 30; 0; 4; 86.7; La liaison a principalement subi des retards liés a des incidents au sud de Lyon qui ont désorganisé les circulations (heurt d'un animal le 19 vers Bonlieu, accident de personne et décontrôle d'aiguille les 16 et 21 vers Valence). Un autre accident de personne a eu lieu le 26 au Creusot. Sur le secteur Atlantique, le mois a été marqué par le heurt d'un chevreuil le 29 à Marcoussis et des problèmes d'Infrastructures (rail cassé le 23 et restitution tardive de travaux à Laval le 13 obligeant les trains à emprunter un autre itinéraire).
2014-05; Est; PARIS EST; METZ; 300; 300; 0; 15; 95.0; 
2014-05; Sud-Est; MONTPELLIER; PARIS LYON; 360; 360; 0; 31; 91.4; 
2014-05; Sud-Est; MULHOUSE VILLE; PARIS LYON; 317; 317; 0; 33; 89.6; 
2014-05; Sud-Est; PARIS LYON; MULHOUSE VILLE; 304; 304; 0; 36; 88.2; 
2014-05; Atlantique; PARIS MONTPARNASSE; POITIERS; 510; 510; 0; 30; 94.1; 
2014-05; Atlantique; PARIS MONTPARNASSE; QUIMPER; 152; 152; 0; 5; 96.7; 
2014-05; Est; REIMS; PARIS EST; 202; 202; 0; 12; 94.1; 
2014-05; Est; PARIS EST; REIMS; 213; 213; 0; 15; 93.0; 
2014-05; Atlantique; RENNES; PARIS MONTPARNASSE; 557; 557; 0; 36; 93.5; 
2014-05; Atlantique; PARIS MONTPARNASSE; ST PIERRE DES CORPS; 452; 452; 0; 33; 92.7; 
2014-05; Est; PARIS EST; STRASBOURG; 430; 430; 0; 30; 93.0; 
2014-05; Atlantique; TOURS; PARIS MONTPARNASSE; 203; 203; 0; 22; 89.2; 
2014-05; Atlantique; PARIS MONTPARNASSE; TOURS; 153; 153; 0; 9; 94.1; 
2014-05; Sud-Est; VALENCE ALIXAN TGV; PARIS LYON; 275; 275; 0; 24; 91.3; 
2014-05; Nord; ARRAS; PARIS NORD; 333; 333; 0; 25; 92.5; 
2014-05; Sud-Est; PARIS LYON; AVIGNON TGV; 526; 526; 0; 37; 93.0; 
2014-05; Atlantique; BORDEAUX ST JEAN; PARIS MONTPARNASSE; 567; 566; 1; 35; 93.8; 
2014-06; Sud-Est; PARIS LYON; CHAMBERY CHALLES LES EAUX; 185; 168; 17; 31; 81.5; 
2014-06; Nord; DUNKERQUE; PARIS NORD; 99; 76; 23; 2; 97.4; 
2014-06; Sud-Est; GRENOBLE; PARIS LYON; 215; 183; 32; 19; 89.6; La circulation des TGV a été fortement perturbée par le mouvement social du 11 au 22 juin.
2014-06; Atlantique; PARIS MONTPARNASSE; LA ROCHELLE VILLE; 216; 184; 32; 21; 88.6; 
2014-06; Nord; LILLE; PARIS NORD; 587; 527; 60; 69; 86.9; 
2014-06; Sud-Est; LYON PART DIEU; PARIS LYON; 582; 520; 62; 62; 88.1; 
2014-06; Atlantique; LYON PART DIEU; RENNES; 32; 23; 9; 3; 87.0; 
2014-06; Sud-Est; PARIS LYON; MACON LOCHE; 177; 147; 30; 10; 93.2; 
2014-06; Sud-Est; PARIS LYON; MARSEILLE ST CHARLES; 428; 381; 47; 33; 91.3; 
2014-06; Est; METZ; PARIS EST; 285; 257; 28; 22; 91.4; 
2014-06; Est; PARIS EST; METZ; 294; 262; 32; 14; 94.7; 
2014-06; Sud-Est; MONTPELLIER; PARIS LYON; 326; 293; 33; 54; 81.6; La circulation des TGV a été fortement perturbée par le mouvement social du 11 au 22 juin.
2014-06; Sud-Est; MULHOUSE VILLE; PARIS LYON; 290; 259; 31; 39; 84.9; 
2014-06; Sud-Est; NICE VILLE; PARIS LYON; 193; 162; 31; 38; 76.5; La circulation des TGV a été fortement perturbée par le mouvement social du 11 au 22 juin.
2014-06; Sud-Est; ANNECY; PARIS LYON; 126; 113; 13; 11; 90.3; 
2014-06; Atlantique; ST MALO; PARIS MONTPARNASSE; 84; 79; 5; 5; 93.7; 
2014-06; Sud-Est; PARIS LYON; TOULON; 192; 171; 21; 48; 71.9; La circulation des TGV a été fortement perturbée par le mouvement social du 11 au 22 juin.
2014-06; Sud-Est; PARIS LYON; VALENCE ALIXAN TGV; 241; 211; 30; 22; 89.6; 
2014-06; Sud-Est; BELLEGARDE (AIN); PARIS LYON; 221; 199; 22; 25; 87.4; 
2014-07; Sud-Est; CHAMBERY CHALLES LES EAUX; PARIS LYON; 177; 177; 0; 35; 80.2; 
2014-07; Nord; DOUAI; PARIS NORD; 189; 189; 0; 14; 92.6; 
2014-07; Atlantique; LA ROCHELLE VILLE; PARIS MONTPARNASSE; 226; 226; 0; 15; 93.4; 
2014-07; Atlantique; PARIS MONTPARNASSE; LA ROCHELLE VILLE; 224; 224; 0; 13; 94.2; 
2014-07; Atlantique; ANGERS SAINT LAUD; PARIS MONTPARNASSE; 439; 439; 0; 35; 92.0; 
2014-07; Sud-Est; MACON LOCHE; PARIS LYON; 189; 189; 0; 29; 84.7; 
2014-07; Sud-Est; PARIS LYON; MARSEILLE ST CHARLES; 488; 488; 0; 34; 93.0; 
2014-07; Est; METZ; PARIS EST; 299; 299; 0; 27; 91.0; 
2014-07; Est; PARIS EST; NANCY; 295; 295; 0; 17; 94.2; 
2014-07; Sud-Est; PARIS LYON; NICE VILLE; 255; 255; 0; 42; 83.5; Plusieurs incidents ont pertubé la circulation des TGV et provoqué des retards importants sur cette liaison : heurt de personnes le 6, le 13, le 29, dérangements d'installations suite à des orages le 6, le 20, le 25.
2014-07; Sud-Est; PERPIGNAN; PARIS LYON; 200; 200; 0; 30; 85.0; Plusieurs incidents ont pertubé la circulation des TGV et provoqué des retards importants sur cette relation (heurt de personnes le 18, le 30, dérangements d'installations suite à des orages le 6, 20 et 25, heurt d'animaux le 23, acte de malveillance le 21).
2014-07; Atlantique; POITIERS; PARIS MONTPARNASSE; 500; 500; 0; 45; 91.0; 
2014-07; Sud-Est; ANNECY; PARIS LYON; 117; 117; 0; 10; 91.5; 
2014-07; Atlantique; PARIS MONTPARNASSE; ST PIERRE DES CORPS; 443; 443; 0; 38; 91.4; 
2014-07; Atlantique; TOURS; PARIS MONTPARNASSE; 197; 197; 0; 16; 91.9; 
2014-07; Sud-Est; AVIGNON TGV; PARIS LYON; 536; 536; 0; 94; 82.5; 
2014-07; Sud-Est; BELLEGARDE (AIN); PARIS LYON; 172; 172; 0; 18; 89.5; 
2014-07; Sud-Est; PARIS LYON; BELLEGARDE (AIN); 163; 163; 0; 21; 87.1; 
2014-07; Sud-Est; PARIS LYON; BESANCON FRANCHE COMTE TGV; 207; 207; 0; 12; 94.2; 
2012-07; Atlantique; BREST; PARIS MONTPARNASSE; 176; 176; 0; 8; 95.5; 
2012-07; Sud-Est; PARIS LYON; DIJON VILLE; 453; 453; 0; 26; 94.3; 
2012-07; Nord; DOUAI; PARIS NORD; 189; 189; 0; 13; 93.1; 
2012-07; Nord; PARIS NORD; DUNKERQUE; 164; 164; 0; 6; 96.3; 
2012-07; Sud-Est; GRENOBLE; PARIS LYON; 212; 212; 0; 6; 97.2; 
2012-07; Atlantique; LE MANS; PARIS MONTPARNASSE; 481; 481; 0; 36; 92.5; 
2012-07; Atlantique; PARIS MONTPARNASSE; LE MANS; 436; 436; 0; 44; 89.9; Le 1er juillet, des problèmes sur le matériel roulant (difficulté pour accrocher deux rames ensemble) imposent de demander aux clients de changer de train pour effectuer le voyage. Avant le départ, un voyageur est pris d'un malaise et nécessite l'intervention des pompiers. Le TGV 8933 partira de Paris Montparnasse avec 115 minutes de retard.
2012-07; Atlantique; ANGERS SAINT LAUD; PARIS MONTPARNASSE; 450; 450; 0; 17; 96.2; 
2012-07; Atlantique; PARIS MONTPARNASSE; ANGERS SAINT LAUD; 394; 394; 0; 22; 94.4; 
2012-07; Nord; LYON PART DIEU; LILLE; 281; 281; 0; 55; 80.4; Liaison touchée par un acte de malveillance (câbles coupés) entre Paris et Lyon les 25 et 26 juillet et un dérangement d'installations dans le Nord suite à orage le 27 juillet.
2012-07; Nord; PARIS NORD; LILLE; 564; 564; 0; 47; 91.7; 
2012-07; Sud-Est; MARSEILLE ST CHARLES; LYON PART DIEU; 596; 596; 0; 109; 81.7; 
2012-07; Sud-Est; MONTPELLIER; LYON PART DIEU; 368; 368; 0; 68; 81.5; 
2012-07; Est; METZ; PARIS EST; 302; 302; 0; 26; 91.4; 
2012-07; Est; NANCY; PARIS EST; 292; 292; 0; 12; 95.9; 
2012-07; Est; NANTES; STRASBOURG; 52; 52; 0; 6; 88.5; 
2012-07; Sud-Est; PARIS LYON; NICE VILLE; 226; 226; 0; 54; 76.1; D'importantes phases de travaux d'amélioration de l'infrastructure sur le tronçon Nice Marseille nécessitent la mise en place de limitations de vitesse qui réduisent la fluidité des circulations.
2012-07; Sud-Est; PARIS LYON; PERPIGNAN; 162; 162; 0; 23; 85.8; 
2012-07; Sud-Est; PARIS LYON; SAINT ETIENNE CHATEAUCREUX; 117; 117; 0; 8; 93.2; 
2012-07; Atlantique; ST PIERRE DES CORPS; PARIS MONTPARNASSE; 445; 445; 0; 49; 89.0; Le 5 juillet, le train 810900 subit une panne et gène la circulation des trains qui le suivent. Le TGV 8300 arrive à Paris Montparnasse avec 30 minutes de retard.
2012-07; Atlantique; PARIS MONTPARNASSE; TOULOUSE MATABIAU; 150; 150; 0; 16; 89.3; Le 6 juillet, le TGV 8501 subit une panne lors de son passage sur la ligne à grande vitesse. Il doit s'arrêter et ne pourra repartir qu'avec une vitesse fortement réduite (160km/h maximum). À Tours. échange de rame pour reprendre une circulation normale jusqu'à la fin du parcours. Le TGV 8501 arrive à son terminus avec 78 minutes de retard.
2012-07; Atlantique; PARIS MONTPARNASSE; TOURS; 187; 187; 0; 21; 88.8; Le 11 juillet, le TGV 8350 heurte un chevreuil près de Rouvray (28). Le TGV 8367 qui le suivait est retardé de 20 minutes.
2012-07; Nord; PARIS NORD; ARRAS; 352; 352; 0; 12; 96.6; 
2012-07; Sud-Est; AVIGNON TGV; PARIS LYON; 418; 418; 0; 65; 84.4; 
2012-07; Sud-Est; PARIS LYON; BELLEGARDE (AIN); 263; 263; 0; 21; 92.0; 
2012-08; Sud-Est; PARIS LYON; CHAMBERY CHALLES LES EAUX; 221; 221; 0; 25; 88.7; 
2012-08; Sud-Est; DIJON VILLE; PARIS LYON; 432; 431; 1; 30; 93.0; 
2012-08; Nord; DOUAI; PARIS NORD; 185; 185; 0; 20; 89.2; 
2012-08; Sud-Est; GRENOBLE; PARIS LYON; 200; 200; 0; 17; 91.5; 
2012-08; Atlantique; PARIS MONTPARNASSE; LA ROCHELLE VILLE; 227; 227; 0; 10; 95.6; 
2012-08; Atlantique; LE MANS; PARIS MONTPARNASSE; 469; 468; 1; 38; 91.9; 
2012-08; Sud-Est; MARSEILLE ST CHARLES; LYON PART DIEU; 599; 599; 0; 105; 82.5; 
2012-08; Sud-Est; MONTPELLIER; LYON PART DIEU; 367; 367; 0; 53; 85.6; 
2012-08; Sud-Est; PARIS LYON; LYON PART DIEU; 531; 531; 0; 26; 95.1; 
2012-08; Atlantique; LYON PART DIEU; RENNES; 31; 31; 0; 2; 93.5; 
2012-08; Atlantique; RENNES; LYON PART DIEU; 84; 84; 0; 4; 95.2; 
2012-08; Sud-Est; NICE VILLE; PARIS LYON; 239; 239; 0; 51; 78.7; D'importantes phases de travaux d'amélioration de l'infrastructure sur le tronçon Nice Marseille nécessitent la mise en place de limitations de vitesse qui réduisent la fluidité des circulations.
2012-08; Sud-Est; PARIS LYON; NICE VILLE; 230; 230; 0; 39; 83.0; 
2012-08; Atlantique; POITIERS; PARIS MONTPARNASSE; 515; 515; 0; 56; 89.1; 
2012-08; Atlantique; PARIS MONTPARNASSE; QUIMPER; 190; 190; 0; 12; 93.7; 
2012-08; Atlantique; RENNES; PARIS MONTPARNASSE; 571; 571; 0; 26; 95.4; 
2012-08; Atlantique; PARIS MONTPARNASSE; RENNES; 560; 560; 0; 21; 96.3; 
2012-08; Atlantique; PARIS MONTPARNASSE; ST MALO; 63; 63; 0; 0; 100.0; 
2012-08; Atlantique; PARIS MONTPARNASSE; ST PIERRE DES CORPS; 460; 460; 0; 18; 96.1; 
2012-08; Sud-Est; TOULON; PARIS LYON; 299; 299; 0; 46; 84.6; 
2012-08; Sud-Est; PARIS LYON; TOULON; 254; 254; 0; 29; 88.6; 
2012-08; Atlantique; PARIS MONTPARNASSE; TOULOUSE MATABIAU; 150; 150; 0; 9; 94.0; 
2012-08; Atlantique; PARIS MONTPARNASSE; VANNES; 213; 213; 0; 9; 95.8; 
2012-08; Sud-Est; AVIGNON TGV; PARIS LYON; 421; 421; 0; 57; 86.5; 
2012-08; Sud-Est; PARIS LYON; AVIGNON TGV; 422; 422; 0; 25; 94.1; 
2012-08; Sud-Est; BELLEGARDE (AIN); PARIS LYON; 256; 256; 0; 18; 93.0; 
2012-09; Sud-Est; PARIS LYON; DIJON VILLE; 441; 441; 0; 19; 95.7; 
2012-09; Nord; DOUAI; PARIS NORD; 198; 198; 0; 19; 90.4; Circulation fortement dégradée le 3 septembre par un incident important qui s'est produit sur la ligne à grande vitesse au niveau de la gare d'Arras (incident touchant les installations caténaires avec de lourdes conséquences en termes de retard sur les trains).
2012-09; Atlantique; PARIS MONTPARNASSE; LA ROCHELLE VILLE; 217; 217; 0; 16; 92.6; 
2012-09; Sud-Est; LE CREUSOT MONTCEAU MONTCHANIN; PARIS LYON; 208; 208; 0; 20; 90.4; 
2012-09; Sud-Est; PARIS LYON; LE CREUSOT MONTCEAU MONTCHANIN; 200; 200; 0; 80; 60.0; Une zone de travaux de renouvellement de la voie fait perdre à l'ensemble des TGV environ 5 min juste avant leur arrivée au Creusot.
2012-09; Atlantique; LYON PART DIEU; RENNES; 46; 46; 0; 2; 95.7; 
2012-09; Est; NANCY; PARIS EST; 288; 288; 0; 16; 94.4; 
2012-09; Sud-Est; PARIS LYON; NIMES; 336; 336; 0; 28; 91.7; 
2012-09; Atlantique; PARIS MONTPARNASSE; QUIMPER; 159; 159; 0; 9; 94.3; 
2012-09; Atlantique; RENNES; PARIS MONTPARNASSE; 558; 558; 0; 45; 91.9; 
2012-09; Atlantique; ST MALO; PARIS MONTPARNASSE; 99; 99; 0; 0; 100.0; 
2012-09; Atlantique; PARIS MONTPARNASSE; ST MALO; 54; 54; 0; 3; 94.4; 
2012-09; Est; STRASBOURG; PARIS EST; 461; 459; 2; 66; 85.6; 
2012-09; Sud-Est; TOULON; PARIS LYON; 267; 267; 0; 34; 87.3; 
2012-09; Atlantique; PARIS MONTPARNASSE; TOULOUSE MATABIAU; 132; 132; 0; 10; 92.4; 
2012-09; Atlantique; PARIS MONTPARNASSE; TOURS; 194; 194; 0; 13; 93.3; 
2012-09; Sud-Est; VALENCE ALIXAN TGV; PARIS LYON; 233; 233; 0; 31; 86.7; 
2012-09; Sud-Est; PARIS LYON; VALENCE ALIXAN TGV; 258; 258; 0; 23; 91.1; 
2012-09; Nord; ARRAS; PARIS NORD; 328; 327; 1; 36; 89.0; Circulation fortement dégradée le 3 septembre par un incident important qui s'est produit sur la ligne à grande vitesse au niveau de la gare d'Arras (incident touchant les installations caténaires avec de lourdes conséquences en termes de retard sur les trains).
2012-09; Sud-Est; AVIGNON TGV; PARIS LYON; 367; 367; 0; 41; 88.8; 
2012-09; Sud-Est; PARIS LYON; AVIGNON TGV; 364; 364; 0; 25; 93.1; 
2012-09; Sud-Est; PARIS LYON; BESANCON FRANCHE COMTE TGV; 228; 228; 0; 12; 94.7; 
2012-10; Sud-Est; PARIS LYON; AIX EN PROVENCE TGV; 397; 397; 0; 65; 83.6; 
2012-10; Sud-Est; PARIS LYON; CHAMBERY CHALLES LES EAUX; 218; 218; 0; 59; 72.9; Des travaux vers Mâcon ont d'abord retardé de quelques minutes les TGV. Ces retards ont ensuite généré des difficultés de circulation sur la voie unique située avant l'arrivée à Chambéry.
2012-10; Atlantique; LA ROCHELLE VILLE; PARIS MONTPARNASSE; 215; 214; 1; 26; 87.9; 
2012-10; Atlantique; PARIS MONTPARNASSE; LE MANS; 460; 455; 5; 43; 90.5; 
2012-10; Nord; MARSEILLE ST CHARLES; LILLE; 127; 124; 3; 25; 79.8; Les circulations ont subi malheureusement une recrudescence d’accident de personnes et une hausse du nombre de dérangements d’installations ferroviaires Les conditions météorologiques notamment dans le Sud-Est sur la fin du mois ont aggravé cette situation difficile.
2012-10; Nord; PARIS NORD; LILLE; 644; 634; 10; 76; 88.0; 
2012-10; Sud-Est; PARIS LYON; MACON LOCHE; 182; 179; 3; 9; 95.0; 
2012-10; Atlantique; ANGOULEME; PARIS MONTPARNASSE; 345; 342; 3; 77; 77.5; Les travaux de raccordement de la ligne à grande vitesse Sud-Europe-Atlantique à la ligne classique entre Bordeaux et Tours ont pu engendrer des retards sur nos trains. Par ailleurs un nombre important d'événements ont eu un fort impact : erreur d'appréciation d'un riverain qui à cru observer un défaut de fermeture des barrières d'un passage à niveau lors du passage d'un train le 5, problème affectant la signalisation ferroviaire près de Massy (91) le 6, alerte à la bombe à Angoulême le 11, ranche tombée sur le fil d'alimentation électrique (caténaire) suite à mauvaise manipulation d'une entreprise d'élagage impliquant la nécessité de stopper la circulations des trains dans les deux sens le 23.
2012-10; Atlantique; PARIS MONTPARNASSE; ANGOULEME; 333; 330; 3; 25; 92.4; 
2012-10; Sud-Est; MONTPELLIER; PARIS LYON; 371; 366; 5; 37; 89.9; 
2012-10; Sud-Est; MULHOUSE VILLE; PARIS LYON; 313; 311; 2; 33; 89.4; 
2012-10; Atlantique; NANTES; PARIS MONTPARNASSE; 603; 597; 6; 26; 95.6; 
2012-10; Est; STRASBOURG; NANTES; 62; 61; 1; 8; 86.9; 
2012-10; Atlantique; PARIS MONTPARNASSE; POITIERS; 467; 463; 4; 20; 95.7; 
2012-10; Est; REIMS; PARIS EST; 241; 240; 1; 17; 92.9; 
2012-10; Sud-Est; TOULON; PARIS LYON; 265; 261; 4; 32; 87.7; 
2012-10; Atlantique; TOULOUSE MATABIAU; PARIS MONTPARNASSE; 99; 98; 1; 19; 80.6; 
2012-10; Atlantique; PARIS MONTPARNASSE; TOURS; 212; 210; 2; 20; 90.5; 
2012-11; Sud-Est; AIX EN PROVENCE TGV; PARIS LYON; 400; 400; 0; 66; 83.5; 
2012-11; Sud-Est; LYON PART DIEU; MONTPELLIER; 362; 362; 0; 82; 77.3; Fragilité de cette liaison liée à la longueur du parcours des trains assurant cette desserte.
2012-11; Atlantique; LYON PART DIEU; RENNES; 30; 30; 0; 3; 90.0; 
2012-11; Atlantique; PARIS MONTPARNASSE; NANTES; 556; 556; 0; 48; 91.4; 
2012-11; Est; NANTES; STRASBOURG; 57; 57; 0; 7; 87.7; 
2012-11; Est; STRASBOURG; NANTES; 57; 57; 0; 10; 82.5; Régularité des trains fortement perturbée par la réduction de la vitesse des TGV en aval d'Angers suite à l'affaisement de la voie causée par des intempéries.
2012-11; Sud-Est; PARIS LYON; ANNECY; 199; 199; 0; 23; 88.4; 
2012-11; Est; PARIS EST; REIMS; 243; 243; 0; 17; 93.0; 
2012-11; Sud-Est; SAINT ETIENNE CHATEAUCREUX; PARIS LYON; 115; 115; 0; 20; 82.6; 
2014-03; Sud-Est; AIX EN PROVENCE TGV; PARIS LYON; 391; 391; 0; 49; 87.5; 
2014-03; Nord; DOUAI; PARIS NORD; 201; 201; 0; 24; 88.1; 
2014-03; Sud-Est; GRENOBLE; PARIS LYON; 249; 249; 0; 16; 93.6; 
2014-03; Atlantique; LE MANS; PARIS MONTPARNASSE; 458; 458; 0; 36; 92.1; 
2014-03; Atlantique; PARIS MONTPARNASSE; LE MANS; 449; 449; 0; 31; 93.1; 
2014-03; Atlantique; LYON PART DIEU; RENNES; 31; 31; 0; 2; 93.5; 
2014-03; Atlantique; RENNES; LYON PART DIEU; 61; 61; 0; 4; 93.4; 
2014-03; Sud-Est; PARIS LYON; MACON LOCHE; 222; 222; 0; 5; 97.7; 
2014-03; Est; METZ; PARIS EST; 299; 299; 0; 9; 97.0; 
2014-03; Est; PARIS EST; NANCY; 294; 294; 0; 7; 97.6; 
2014-03; Est; STRASBOURG; NANTES; 41; 41; 0; 4; 90.2; 
2014-03; Sud-Est; PARIS LYON; NIMES; 326; 326; 0; 28; 91.4; 
2014-03; Atlantique; POITIERS; PARIS MONTPARNASSE; 494; 494; 0; 28; 94.3; 
2014-03; Atlantique; PARIS MONTPARNASSE; POITIERS; 508; 508; 0; 31; 93.9; 
2014-03; Est; REIMS; PARIS EST; 211; 211; 0; 9; 95.7; 
2014-03; Est; PARIS EST; REIMS; 216; 216; 0; 9; 95.8; 
2014-03; Atlantique; PARIS MONTPARNASSE; RENNES; 559; 559; 0; 19; 96.6; 
2014-03; Sud-Est; SAINT ETIENNE CHATEAUCREUX; PARIS LYON; 6; 6; 0; 0; 100.0; 
2014-03; Atlantique; PARIS MONTPARNASSE; ST MALO; 55; 55; 0; 2; 96.4; 
2014-03; Atlantique; ST PIERRE DES CORPS; PARIS MONTPARNASSE; 431; 431; 0; 68; 84.2; 
2014-03; Atlantique; PARIS MONTPARNASSE; ST PIERRE DES CORPS; 459; 459; 0; 27; 94.1; 
2014-03; Est; PARIS EST; STRASBOURG; 445; 445; 0; 24; 94.6; 
2014-03; Sud-Est; TOULON; PARIS LYON; 238; 238; 0; 33; 86.1; Des incidents d'origine externe (accidents de personnes, heurts d'animaux, incendie dans un tunnel) ont provoqué des retards importants sur cette relation en mars.
2014-03; Atlantique; PARIS MONTPARNASSE; TOULOUSE MATABIAU; 153; 153; 0; 19; 87.6; Le 10 : le 8513 heurte des panneaux de signalisation routière posés sur la voie ce qui le retarde de 84 minutes. Le 13 : la panne d'un train de Fret près de Mouthiers entraîne le retard du 8513 de 84 minutes. Le 23 : un incident (présence d'un nid de Pie) affectant la caténaire en gare de Massy TGV à fortement perturbé la circulation des trains et notamment des 8511, 8513 et 8505 arrivés à destination avec respectivement 106, 51 et 43 minutes de retard. Le 29 : un accident de personne en gare de St Jory (personne traversant les voies) a engendré le retard du TGV 8501 de 87 minutes.
2014-03; Atlantique; PARIS MONTPARNASSE; TOURS; 150; 150; 0; 10; 93.3; 
2014-03; Atlantique; VANNES; PARIS MONTPARNASSE; 160; 160; 0; 8; 95.0; 
2014-03; Sud-Est; BESANCON FRANCHE COMTE TGV; PARIS LYON; 226; 226; 0; 12; 94.7; 
2014-03; Sud-Est; PARIS LYON; BESANCON FRANCHE COMTE TGV; 218; 218; 0; 7; 96.8; 
2014-04; Nord; PARIS NORD; DOUAI; 188; 188; 0; 15; 92.0; 
2014-04; Atlantique; LA ROCHELLE VILLE; PARIS MONTPARNASSE; 220; 220; 0; 6; 97.3; 
2014-04; Nord; LILLE; MARSEILLE ST CHARLES; 120; 120; 0; 10; 91.7; 
2014-04; Nord; LILLE; PARIS NORD; 605; 603; 2; 62; 89.7; 
2014-04; Sud-Est; LYON PART DIEU; PARIS LYON; 603; 603; 0; 3; 99.5; 
2014-04; Est; NANCY; PARIS EST; 285; 285; 0; 17; 94.0; 
2014-04; Est; STRASBOURG; NANTES; 56; 56; 0; 2; 96.4; 
2014-04; Sud-Est; PARIS LYON; NIMES; 347; 347; 0; 42; 87.9; 
2014-04; Atlantique; QUIMPER; PARIS MONTPARNASSE; 146; 146; 0; 2; 98.6; 
2014-04; Sud-Est; ANNECY; PARIS LYON; 136; 136; 0; 9; 93.4; 
2014-04; Atlantique; RENNES; PARIS MONTPARNASSE; 561; 560; 1; 22; 96.1; 
2014-04; Atlantique; PARIS MONTPARNASSE; RENNES; 549; 549; 0; 14; 97.4; 
2014-04; Sud-Est; SAINT ETIENNE CHATEAUCREUX; PARIS LYON; 5; 5; 0; 0; 100.0; 
2014-04; Atlantique; PARIS MONTPARNASSE; TOULOUSE MATABIAU; 148; 148; 0; 12; 91.9; 
2014-04; Sud-Est; VALENCE ALIXAN TGV; PARIS LYON; 271; 271; 0; 37; 86.3; 
2014-04; Atlantique; VANNES; PARIS MONTPARNASSE; 155; 155; 0; 3; 98.1; 
2014-04; Sud-Est; PARIS LYON; AVIGNON TGV; 511; 511; 0; 43; 91.6; 
2014-04; Sud-Est; BESANCON FRANCHE COMTE TGV; PARIS LYON; 231; 230; 1; 10; 95.7; 
2014-04; Atlantique; BORDEAUX ST JEAN; PARIS MONTPARNASSE; 642; 642; 0; 35; 94.5; 
2014-05; Sud-Est; CHAMBERY CHALLES LES EAUX; PARIS LYON; 201; 201; 0; 23; 88.6; 
2014-05; Sud-Est; DIJON VILLE; PARIS LYON; 466; 466; 0; 40; 91.4; 
2014-05; Nord; PARIS NORD; DOUAI; 194; 194; 0; 32; 83.5; 
2014-05; Atlantique; PARIS MONTPARNASSE; LAVAL; 230; 230; 0; 6; 97.4; 
2014-05; Sud-Est; LE CREUSOT MONTCEAU MONTCHANIN; PARIS LYON; 218; 218; 0; 14; 93.6; 
2014-05; Nord; LILLE; MARSEILLE ST CHARLES; 124; 124; 0; 23; 81.5; 
2014-05; Nord; LILLE; PARIS NORD; 609; 609; 0; 62; 89.8; 
2014-05; Sud-Est; MACON LOCHE; PARIS LYON; 192; 192; 0; 13; 93.2; 
2014-05; Sud-Est; PARIS LYON; MACON LOCHE; 218; 218; 0; 2; 99.1; 
2014-05; Sud-Est; PARIS LYON; NIMES; 351; 351; 0; 34; 90.3; 
2014-05; Sud-Est; PERPIGNAN; PARIS LYON; 159; 159; 0; 17; 89.3; Cinq heurts d'animaux sauvages et deux accidents de personnes sur lignes à grande vitesse ont pénalisé la régularité de cette liaison.
2014-05; Sud-Est; PARIS LYON; PERPIGNAN; 167; 167; 0; 11; 93.4; 
2014-05; Atlantique; PARIS MONTPARNASSE; RENNES; 556; 556; 0; 17; 96.9; 
2014-05; Sud-Est; PARIS LYON; SAINT ETIENNE CHATEAUCREUX; 6; 6; 0; 0; 100.0; 
2014-05; Atlantique; ST MALO; PARIS MONTPARNASSE; 102; 102; 0; 4; 96.1; 
2014-05; Sud-Est; AVIGNON TGV; PARIS LYON; 465; 465; 0; 79; 83.0; 
2014-06; Atlantique; PARIS MONTPARNASSE; BREST; 193; 170; 23; 20; 88.2; Le 9, des travaux rendus tardivement près de Rennes retardent 9 TGV et de violents orages en région parisienne ont entraîné de multiples dérangements d'installations sur la Ligne à Grande Vitesse. Le 25, un incendie dans une maison à proximité des voies près du Mans retarde 4 TGV. A ces incidents s'ajoutent les retards liés à la période de grève nationale de juin.
2014-06; Sud-Est; CHAMBERY CHALLES LES EAUX; PARIS LYON; 184; 169; 15; 29; 82.8; 
2014-06; Atlantique; LA ROCHELLE VILLE; PARIS MONTPARNASSE; 215; 185; 30; 17; 90.8; 
2014-06; Atlantique; LAVAL; PARIS MONTPARNASSE; 228; 210; 18; 19; 91.0; 
2014-06; Sud-Est; PARIS LYON; LE CREUSOT MONTCEAU MONTCHANIN; 198; 198; 0; 33; 83.3; 
2014-06; Atlantique; PARIS MONTPARNASSE; ANGERS SAINT LAUD; 418; 368; 50; 43; 88.3; 
2014-06; Nord; MARSEILLE ST CHARLES; LILLE; 434; 375; 59; 42; 88.8; Les mouvements sociaux qui ont eu lieu du 11 au 24 juin ont nettement affecté les résultats de ce mois. Des travaux entre Macon et Lyon (Cesseins) depuis le 1er juin, avec des limitations de vitesse ont engendré des pertes de temps de quelques minutes. Vers la fin du mois, une augmentation des interventions à bord, que ce soit des pompiers pour porter assistance à des voyageurs ou bien des forces de l'ordre pour rétablir la sécurité. Quelques dérangement des installations également : problème caténaire le 6 à Montanay, d'aiguillage les 22 et 25 sur LGV Nord ou d'un dispositif de détection à Sathonay le 30.
2014-06; Sud-Est; LYON PART DIEU; MARSEILLE ST CHARLES; 600; 502; 98; 154; 69.3; La circulation des TGV a été fortement perturbée par le mouvement social du 11 au 22 juin.
2014-06; Sud-Est; MACON LOCHE; PARIS LYON; 182; 158; 24; 20; 87.3; 
2014-06; Sud-Est; PERPIGNAN; PARIS LYON; 140; 129; 11; 26; 79.8; La circulation des TGV a été fortement perturbée par le mouvement social du 11 au 22 juin.
2014-06; Atlantique; PARIS MONTPARNASSE; POITIERS; 490; 394; 96; 49; 87.6; 
2014-06; Est; STRASBOURG; PARIS EST; 448; 407; 41; 50; 87.7; 
2014-06; Sud-Est; TOULON; PARIS LYON; 224; 182; 42; 38; 79.1; La circulation des TGV a été fortement perturbée par le mouvement social du 11 au 22 juin.
2014-06; Atlantique; TOULOUSE MATABIAU; PARIS MONTPARNASSE; 85; 84; 1; 22; 73.8; Le 22, la chute d'un véhicule sur les voies près d'Agen retarde 3 TGV. Le 27, la panne d'un train de marchandises et un dérangement d'installation près de Montauban retardent 2 TGV. Le 29, de violents orages sur la région de Bordeaux ont entraîné une fuite de gaz et de multiples dérangements d'installations. Le 30, un dérangement des installations à St-Pierre-des Corps retarde 1 TGV. A ces incidents s'ajoutent les retards liés à la période de grève nationale de juin.
2014-06; Sud-Est; VALENCE ALIXAN TGV; PARIS LYON; 243; 206; 37; 45; 78.2; La circulation des TGV a été fortement perturbée par le mouvement social du 11 au 22 juin.
2014-06; Nord; ARRAS; PARIS NORD; 320; 257; 63; 42; 83.7; 
2014-06; Atlantique; PARIS MONTPARNASSE; VANNES; 156; 131; 25; 12; 90.8; Le 9, des travaux rendus tardivement près de Rennes retardent 9 TGV et de violents orages en région parisienne ont entraîné de multiples dérangements d'installations sur la Ligne à Grande Vitesse. Le 25, un incendie dans une maison à proximité des voies près du Mans retarde 4 TGV. A ces incidents s'ajoutent les retards liés à la période de grève nationale de juin.
2014-06; Sud-Est; PARIS LYON; BELLEGARDE (AIN); 204; 185; 19; 29; 84.3; 
2014-07; Atlantique; BREST; PARIS MONTPARNASSE; 180; 180; 0; 10; 94.4; 
2014-07; Atlantique; PARIS MONTPARNASSE; BREST; 202; 202; 0; 16; 92.1; 
2014-07; Nord; PARIS NORD; DUNKERQUE; 135; 135; 0; 9; 93.3; 
2014-07; Sud-Est; GRENOBLE; PARIS LYON; 189; 188; 1; 17; 91.0; 
2014-07; Atlantique; LAVAL; PARIS MONTPARNASSE; 218; 218; 0; 23; 89.4; Le 9 la panne d'un train à Marcoussis, sur la Ligne à Grande Vitesse, qui retarde 28 TGV de 8 minutes à 1h35, le 10 un acte de malveillance affectant un train à St Léger, sur la Ligne à Grande Vitesse, retarde 43 TGV de 14 minutes à 4h30, du 14 au 24 des ralentissements consécutifs à des travaux en sortie de Laval ont entraîné des retards dans le sens Province/Paris, le 19 un accident de personne à Champagné retarde 26 TGV de 18min à 3h58.

2014-07; Nord; LILLE; LYON PART DIEU; 217; 217; 0; 16; 92.6; 
2014-07; Atlantique; RENNES; LYON PART DIEU; 79; 79; 0; 13; 83.5; Les 3, 17, 18 et 24, une limitation de la vitesse suite aux fortes chaleurs en région parisienne a ralenti entre 11 et 22 TGV avec des retards pouvant atteindre 1h10. Le 9, la panne d'un train à Marcoussis, sur la Ligne à Grande Vitesse, retarde 28 TGV de 8min à 1h35. Le 10, un acte de malveillance affectant un train à St Léger, sur la Ligne à Grande Vitesse, retarde 43 TGV de 14min à 4h30. Le 30, un problème d'aiguillage sur la Ligne à Grande Vitesse retarde 11 TGV de 8 et 20min.
2014-07; Sud-Est; PARIS LYON; MACON LOCHE; 216; 216; 0; 15; 93.1; 
2014-07; Est; PARIS EST; METZ; 303; 303; 0; 20; 93.4; 
2014-07; Atlantique; ANGOULEME; PARIS MONTPARNASSE; 328; 328; 0; 58; 82.3; 
2014-07; Sud-Est; MULHOUSE VILLE; PARIS LYON; 323; 322; 1; 13; 96.0; 
2014-07; Sud-Est; PARIS LYON; MULHOUSE VILLE; 303; 303; 0; 26; 91.4; 
2014-07; Atlantique; PARIS MONTPARNASSE; NANTES; 523; 523; 0; 34; 93.5; 
2014-07; Sud-Est; NIMES; PARIS LYON; 382; 382; 0; 71; 81.4; 
2014-07; Sud-Est; PARIS LYON; NIMES; 373; 373; 0; 42; 88.7; 
2014-07; Sud-Est; PARIS LYON; ANNECY; 152; 152; 0; 6; 96.1; 
2014-07; Atlantique; RENNES; PARIS MONTPARNASSE; 553; 553; 0; 65; 88.2; 
2014-07; Atlantique; ST MALO; PARIS MONTPARNASSE; 108; 108; 0; 7; 93.5; 
2014-07; Atlantique; ST PIERRE DES CORPS; PARIS MONTPARNASSE; 424; 424; 0; 79; 81.4; 
2011-09; Sud-Est; AIX EN PROVENCE TGV; PARIS LYON; 411; 410; 1; 47; 88.5; 
2011-09; Sud-Est; PARIS LYON; CHAMBERY CHALLES LES EAUX; 180; 180; 0; 26; 85.6; 
2011-09; Atlantique; PARIS MONTPARNASSE; LA ROCHELLE VILLE; 188; 188; 0; 14; 92.6; 
2011-09; Atlantique; LAVAL; PARIS MONTPARNASSE; 228; 228; 0; 6; 97.4; 
2011-09; Sud-Est; PARIS LYON; LE CREUSOT MONTCEAU MONTCHANIN; 202; 202; 0; 27; 86.6; 
2011-09; Atlantique; ANGERS SAINT LAUD; PARIS MONTPARNASSE; 475; 475; 0; 16; 96.6; 
2011-09; Nord; LYON PART DIEU; LILLE; 311; 311; 0; 53; 83.0; 
2011-09; Nord; MARSEILLE ST CHARLES; LILLE; 199; 199; 0; 41; 79.4; 
2011-09; Sud-Est; MARSEILLE ST CHARLES; LYON PART DIEU; 540; 540; 0; 77; 85.7; 
2011-09; Sud-Est; MACON LOCHE; PARIS LYON; 180; 180; 0; 16; 91.1; 
2011-09; Sud-Est; PARIS LYON; MARSEILLE ST CHARLES; 505; 505; 0; 46; 90.9; 
2011-09; Est; PARIS EST; METZ; 304; 304; 0; 12; 96.1; 
2011-09; Atlantique; PARIS MONTPARNASSE; ANGOULEME; 325; 325; 0; 39; 88.0; 
2011-09; Est; NANCY; PARIS EST; 289; 288; 1; 9; 96.9; 
2011-09; Atlantique; NANTES; PARIS MONTPARNASSE; 600; 600; 0; 22; 96.3; 
2011-09; Sud-Est; PARIS LYON; NICE VILLE; 182; 182; 0; 29; 84.1; 
2011-09; Sud-Est; PARIS LYON; PERPIGNAN; 121; 121; 0; 22; 81.8; 
2011-09; Sud-Est; PARIS LYON; ANNECY; 202; 202; 0; 16; 92.1; 
2011-09; Sud-Est; PARIS LYON; SAINT ETIENNE CHATEAUCREUX; 116; 116; 0; 19; 83.6; 
2011-09; Atlantique; PARIS MONTPARNASSE; ST MALO; 44; 44; 0; 1; 97.7; 
2011-09; Atlantique; ST PIERRE DES CORPS; PARIS MONTPARNASSE; 459; 459; 0; 40; 91.3; Un chantier long \& complexe de maintenance au Nord de Tours (entrée Sud de la LGV) a généré de nombreux ralentissements pour les trains de l'OD entraînant d'importants retards. A cela s'ajoute 6 restitutions tardive de travaux en début de matinée.
2011-09; Est; PARIS EST; STRASBOURG; 489; 489; 0; 36; 92.6; 
2011-09; Atlantique; TOULOUSE MATABIAU; PARIS MONTPARNASSE; 115; 115; 0; 10; 91.3; L'OD a été touchée par plusieurs dérangements d'installations dont un rail cassé à Poitiers le 29 (23 TGV touchés) et un défaut l'alimentation ERDF vers Châtellerault le 7 (11 TGV). Elle a également été impactée par des problèmes matériels sur un train FRET au nord de Bordeaux le 3 (20 TGV) et sur un TER vers Châtellerault le 20 (8 TGV). Par ailleurs l'OD a subi quelques événements extérieurs dont un accident de personne à Poitiers le 21 (23 TGV) et un autre à Castelsarrasin le 7 (8TGV).
2011-09; Sud-Est; PARIS LYON; BELLEGARDE (AIN); 260; 260; 0; 29; 88.8; 
2011-10; Sud-Est; PARIS LYON; AIX EN PROVENCE TGV; 421; 421; 0; 40; 90.5; 
2011-10; Atlantique; PARIS MONTPARNASSE; BREST; 176; 176; 0; 24; 86.4; 
2011-10; Atlantique; PARIS MONTPARNASSE; LE MANS; 487; 487; 0; 106; 78.2; 
2011-10; Atlantique; ANGERS SAINT LAUD; PARIS MONTPARNASSE; 484; 476; 8; 62; 87.0; 
2011-10; Atlantique; PARIS MONTPARNASSE; ANGERS SAINT LAUD; 452; 452; 0; 58; 87.2; 
2011-10; Sud-Est; MARSEILLE ST CHARLES; LYON PART DIEU; 553; 550; 3; 128; 76.7; 
2011-10; Sud-Est; MACON LOCHE; PARIS LYON; 186; 186; 0; 17; 90.9; 
2011-10; Sud-Est; MARSEILLE ST CHARLES; PARIS LYON; 482; 478; 4; 44; 90.8; 
2011-10; Est; METZ; PARIS EST; 292; 291; 1; 18; 93.8; 
2011-10; Atlantique; NANTES; PARIS MONTPARNASSE; 609; 599; 10; 62; 89.6; 
2011-10; Atlantique; PARIS MONTPARNASSE; NANTES; 594; 594; 0; 55; 90.7; 
2011-10; Sud-Est; NICE VILLE; PARIS LYON; 175; 175; 0; 27; 84.6; 
2011-10; Sud-Est; PARIS LYON; NIMES; 374; 372; 2; 48; 87.1; 
2011-10; Est; PARIS EST; REIMS; 250; 250; 0; 25; 90.0; 
2011-10; Atlantique; PARIS MONTPARNASSE; RENNES; 555; 555; 0; 85; 84.7; 
2011-10; Est; PARIS EST; STRASBOURG; 500; 499; 1; 50; 90.0; 
2011-10; Sud-Est; PARIS LYON; VALENCE ALIXAN TGV; 259; 259; 0; 22; 91.5; 
2011-10; Nord; ARRAS; PARIS NORD; 330; 329; 1; 41; 87.5; 
2011-10; Atlantique; PARIS MONTPARNASSE; VANNES; 194; 194; 0; 25; 87.1; 
2011-10; Sud-Est; BELLEGARDE (AIN); PARIS LYON; 263; 263; 0; 30; 88.6; 
2011-11; Sud-Est; CHAMBERY CHALLES LES EAUX; PARIS LYON; 169; 169; 0; 14; 91.7; 
2011-11; Sud-Est; PARIS LYON; CHAMBERY CHALLES LES EAUX; 190; 190; 0; 20; 89.5; 
2011-11; Sud-Est; PARIS LYON; DIJON VILLE; 393; 393; 0; 16; 95.9; 
2011-11; Nord; DOUAI; PARIS NORD; 128; 128; 0; 8; 93.8; 
2011-11; Nord; PARIS NORD; DUNKERQUE; 166; 166; 0; 6; 96.4; 
2011-11; Sud-Est; LE CREUSOT MONTCEAU MONTCHANIN; PARIS LYON; 213; 213; 0; 19; 91.1; 
2011-11; Sud-Est; LYON PART DIEU; MARSEILLE ST CHARLES; 548; 548; 0; 115; 79.0; 
2011-11; Sud-Est; MONTPELLIER; LYON PART DIEU; 358; 358; 0; 59; 83.5; 
2011-11; Est; PARIS EST; METZ; 304; 304; 0; 16; 94.7; 
2011-11; Atlantique; PARIS MONTPARNASSE; ANGOULEME; 335; 332; 3; 26; 92.2; 
2011-11; Est; MULHOUSE VILLE; PARIS LYON; 198; 198; 0; 28; 85.9; 
2011-11; Est; NANCY; PARIS EST; 282; 282; 0; 11; 96.1; 
2011-11; Sud-Est; PARIS LYON; NICE VILLE; 140; 140; 0; 19; 86.4; 
2011-11; Sud-Est; NIMES; PARIS LYON; 376; 375; 1; 60; 84.0; 
2011-11; Sud-Est; PARIS LYON; PERPIGNAN; 119; 119; 0; 21; 82.4; 
2011-11; Atlantique; QUIMPER; PARIS MONTPARNASSE; 149; 146; 3; 11; 92.5; 
2011-11; Est; PARIS EST; REIMS; 243; 243; 0; 22; 90.9; 
2011-11; Atlantique; PARIS MONTPARNASSE; RENNES; 537; 528; 9; 49; 90.7; 
2011-11; Est; PARIS EST; STRASBOURG; 484; 484; 0; 41; 91.5; 
2011-11; Sud-Est; PARIS LYON; TOULON; 241; 241; 0; 24; 90.0; 
2011-12; Atlantique; PARIS MONTPARNASSE; LA ROCHELLE VILLE; 216; 216; 0; 27; 87.5; 
2011-12; Atlantique; LE MANS; PARIS MONTPARNASSE; 495; 492; 3; 107; 78.3; 
2011-12; Atlantique; RENNES; LYON PART DIEU; 86; 86; 0; 6; 93.0; 
2011-12; Sud-Est; MACON LOCHE; PARIS LYON; 182; 182; 0; 7; 96.2; 
2011-12; Est; PARIS EST; METZ; 311; 311; 0; 11; 96.5; 
2011-12; Sud-Est; PARIS LYON; NICE VILLE; 167; 167; 0; 20; 88.0; 
2011-12; Atlantique; PARIS MONTPARNASSE; QUIMPER; 167; 167; 0; 12; 92.8; 
2011-12; Atlantique; PARIS MONTPARNASSE; ST MALO; 53; 53; 0; 5; 90.6; 
2011-12; Est; STRASBOURG; PARIS EST; 495; 495; 0; 55; 88.9; 
2011-12; Sud-Est; TOULON; PARIS LYON; 247; 247; 0; 35; 85.8; 
2011-12; Atlantique; TOURS; PARIS MONTPARNASSE; 199; 199; 0; 34; 82.9; 
2011-12; Sud-Est; AVIGNON TGV; PARIS LYON; 349; 349; 0; 48; 86.2; 
2011-12; Atlantique; PARIS MONTPARNASSE; BORDEAUX ST JEAN; 662; 662; 0; 36; 94.6; 
2012-01; Sud-Est; AIX EN PROVENCE TGV; PARIS LYON; 419; 419; 0; 32; 92.4; 
2012-01; Sud-Est; DIJON VILLE; PARIS LYON; 470; 470; 0; 37; 92.1; 
2012-01; Nord; PARIS NORD; DOUAI; 212; 212; 0; 20; 90.6; 
2012-01; Sud-Est; PARIS LYON; GRENOBLE; 257; 257; 0; 20; 92.2; 
2012-01; Atlantique; LA ROCHELLE VILLE; PARIS MONTPARNASSE; 226; 226; 0; 24; 89.4; 
2012-01; Atlantique; PARIS MONTPARNASSE; LE MANS; 455; 455; 0; 40; 91.2; 
2012-01; Nord; MARSEILLE ST CHARLES; LILLE; 128; 128; 0; 26; 79.7; 
2012-01; Nord; PARIS NORD; LILLE; 645; 645; 0; 56; 91.3; 
2012-01; Sud-Est; MARSEILLE ST CHARLES; LYON PART DIEU; 597; 597; 0; 79; 86.8; 
2012-01; Sud-Est; LYON PART DIEU; MONTPELLIER; 375; 375; 0; 66; 82.4; 
2012-01; Sud-Est; MONTPELLIER; LYON PART DIEU; 369; 369; 0; 59; 84.0; 
2012-01; Sud-Est; PARIS LYON; LYON PART DIEU; 631; 631; 0; 23; 96.4; 
2012-01; Sud-Est; PARIS LYON; MACON LOCHE; 193; 193; 0; 12; 93.8; 
2012-01; Atlantique; ANGOULEME; PARIS MONTPARNASSE; 353; 353; 0; 52; 85.3; 
2012-01; Sud-Est; MONTPELLIER; PARIS LYON; 372; 372; 0; 26; 93.0; 
2012-01; Est; PARIS LYON; MULHOUSE VILLE; 315; 315; 0; 15; 95.2; 
2012-01; Atlantique; QUIMPER; PARIS MONTPARNASSE; 151; 151; 0; 10; 93.4; 
2012-01; Sud-Est; ANNECY; PARIS LYON; 207; 207; 0; 8; 96.1; 
2012-01; Atlantique; PARIS MONTPARNASSE; RENNES; 569; 569; 0; 32; 94.4; 
2012-01; Atlantique; PARIS MONTPARNASSE; ST PIERRE DES CORPS; 471; 471; 0; 44; 90.7; 
2012-01; Est; PARIS EST; STRASBOURG; 480; 480; 0; 33; 93.1; 
2012-01; Sud-Est; PARIS LYON; TOULON; 214; 214; 0; 19; 91.1; 
2012-01; Sud-Est; BESANCON FRANCHE COMTE TGV; PARIS LYON; 220; 220; 0; 17; 92.3; 
2012-07; Sud-Est; AIX EN PROVENCE TGV; PARIS LYON; 438; 438; 0; 67; 84.7; 
2012-07; Sud-Est; PARIS LYON; AIX EN PROVENCE TGV; 437; 437; 0; 58; 86.7; 
2012-07; Sud-Est; DIJON VILLE; PARIS LYON; 432; 431; 1; 32; 92.6; 
2012-07; Nord; PARIS NORD; DOUAI; 183; 183; 0; 7; 96.2; 
2012-07; Nord; LILLE; LYON PART DIEU; 310; 310; 0; 34; 89.0; 
2012-07; Nord; MARSEILLE ST CHARLES; LILLE; 130; 130; 0; 34; 73.8; Liaison touchée par un acte de malveillance (câbles coupés) entre Paris et Lyon les 25 et 26 juillet et un dérangement d'installations dans le Nord suite à orage le 27 juillet.
2012-07; Sud-Est; LYON PART DIEU; MARSEILLE ST CHARLES; 619; 619; 0; 144; 76.7; D'importantes phases de travaux d'amélioration de l'infrastructure sur le tronçon Nice Marseille nécessitent la mise en place de limitations de vitesse qui réduisent la fluidité des circulations.
2012-07; Sud-Est; LYON PART DIEU; MONTPELLIER; 377; 377; 0; 76; 79.8; D'importantes phases de travaux d'amélioration de l'infrastructure sur le tronçon Nice Marseille nécessitent la mise en place de limitations de vitesse qui réduisent la fluidité des circulations.
2012-07; Sud-Est; MARSEILLE ST CHARLES; PARIS LYON; 503; 503; 0; 27; 94.6; 
2012-07; Atlantique; ANGOULEME; PARIS MONTPARNASSE; 343; 343; 0; 40; 88.3; Le 8 juillet, afin d'assurer la correspondance avec le TER 864527, le TGV 8448 est retenu en gare d'Angoulême environ 15 minutes.
2012-07; Sud-Est; PARIS LYON; MONTPELLIER; 395; 395; 0; 51; 87.1; 
2012-07; Est; PARIS EST; NANCY; 296; 296; 0; 16; 94.6; 
2012-07; Est; PARIS EST; REIMS; 225; 225; 0; 6; 97.3; 
2012-07; Atlantique; RENNES; PARIS MONTPARNASSE; 570; 570; 0; 27; 95.3; 
2012-07; Atlantique; PARIS MONTPARNASSE; RENNES; 553; 553; 0; 42; 92.4; 
2012-07; Sud-Est; SAINT ETIENNE CHATEAUCREUX; PARIS LYON; 119; 119; 0; 15; 87.4; 
2012-07; Est; STRASBOURG; PARIS EST; 477; 476; 1; 45; 90.5; 
2012-07; Atlantique; VANNES; PARIS MONTPARNASSE; 169; 169; 0; 4; 97.6; 
2012-07; Atlantique; BORDEAUX ST JEAN; PARIS MONTPARNASSE; 636; 636; 0; 48; 92.5; 
2012-08; Sud-Est; AIX EN PROVENCE TGV; PARIS LYON; 438; 438; 0; 70; 84.0; 
2012-08; Sud-Est; CHAMBERY CHALLES LES EAUX; PARIS LYON; 216; 216; 0; 15; 93.1; 
2012-08; Atlantique; LA ROCHELLE VILLE; PARIS MONTPARNASSE; 231; 231; 0; 17; 92.6; 
2012-08; Nord; PARIS NORD; LILLE; 573; 572; 1; 53; 90.7; 
2012-08; Sud-Est; PARIS LYON; MACON LOCHE; 184; 184; 0; 15; 91.8; 
2012-08; Sud-Est; MARSEILLE ST CHARLES; PARIS LYON; 505; 505; 0; 32; 93.7; 
2012-08; Atlantique; PARIS MONTPARNASSE; ANGOULEME; 323; 323; 0; 12; 96.3; 
2012-08; Atlantique; PARIS MONTPARNASSE; NANTES; 528; 528; 0; 29; 94.5; 
2012-08; Est; STRASBOURG; NANTES; 52; 52; 0; 5; 90.4; 
2012-08; Sud-Est; NIMES; PARIS LYON; 399; 399; 0; 52; 87.0; 
2012-08; Est; PARIS EST; REIMS; 224; 224; 0; 17; 92.4; La quasi totalité des retards du mois d'août se sont produit le 16 août à cause d'un accident de personne à Vaires.
2012-08; Sud-Est; SAINT ETIENNE CHATEAUCREUX; PARIS LYON; 119; 119; 0; 3; 97.5; 
2012-08; Sud-Est; PARIS LYON; SAINT ETIENNE CHATEAUCREUX; 118; 118; 0; 11; 90.7; 
2012-08; Est; PARIS EST; STRASBOURG; 486; 485; 1; 28; 94.2; La quasi totalité des retards du mois d'août se sont produit le 16 août à cause d'un accident de personne à Vaires.
2012-08; Atlantique; TOULOUSE MATABIAU; PARIS MONTPARNASSE; 89; 89; 0; 8; 91.0; 
2012-08; Atlantique; PARIS MONTPARNASSE; TOURS; 190; 190; 0; 9; 95.3; 
2012-08; Nord; ARRAS; PARIS NORD; 334; 334; 0; 42; 87.4; 
2012-09; Sud-Est; PARIS LYON; CHAMBERY CHALLES LES EAUX; 209; 209; 0; 21; 90.0; 
2012-09; Nord; PARIS NORD; DOUAI; 198; 198; 0; 25; 87.4; Circulation fortement dégradée le 3 septembre par un incident important qui s'est produit sur la ligne à grande vitesse au niveau de la gare d'Arras (incident touchant les installations caténaires avec de lourdes conséquences en termes de retard sur les trains).
2012-09; Atlantique; LAVAL; PARIS MONTPARNASSE; 234; 234; 0; 14; 94.0; 
2012-09; Atlantique; PARIS MONTPARNASSE; LAVAL; 234; 234; 0; 12; 94.9; 
2012-09; Atlantique; LE MANS; PARIS MONTPARNASSE; 451; 451; 0; 69; 84.7; 
2012-09; Atlantique; PARIS MONTPARNASSE; ANGERS SAINT LAUD; 424; 424; 0; 20; 95.3; 
2012-09; Nord; LILLE; MARSEILLE ST CHARLES; 147; 147; 0; 20; 86.4; Circulation fortement dégradée le 3 septembre par un incident important qui s'est produit sur la ligne à grande vitesse au niveau de la gare d'Arras (incident touchant les installations caténaires avec de lourdes conséquences en termes de retard sur les trains).
2012-09; Nord; MARSEILLE ST CHARLES; LILLE; 125; 125; 0; 16; 87.2; Circulation fortement dégradée le 3 septembre par un incident important qui s'est produit sur la ligne à grande vitesse au niveau de la gare d'Arras (incident touchant les installations caténaires avec de lourdes conséquences en termes de retard sur les trains).
2012-09; Sud-Est; LYON PART DIEU; PARIS LYON; 617; 617; 0; 22; 96.4; 
2012-09; Sud-Est; PARIS LYON; LYON PART DIEU; 613; 613; 0; 14; 97.7; 
2012-09; Est; METZ; PARIS EST; 288; 288; 0; 16; 94.4; 
2012-09; Est; PARIS EST; METZ; 304; 304; 0; 11; 96.4; 
2012-09; Sud-Est; NICE VILLE; PARIS LYON; 186; 186; 0; 35; 81.2; 
2012-09; Sud-Est; PARIS LYON; NICE VILLE; 179; 179; 0; 34; 81.0; D'importantes phases de travaux d'amélioration de l'infrastructure sur le tronçon Nice Marseille nécessitent la mise en place de limitations de vitesse qui réduisent la fluidité des circulations.
2012-09; Sud-Est; ANNECY; PARIS LYON; 194; 193; 1; 11; 94.3; 
2012-09; Sud-Est; PARIS LYON; ANNECY; 200; 200; 0; 8; 96.0; 
2012-09; Est; PARIS EST; REIMS; 242; 241; 1; 12; 95.0; 
2012-09; Atlantique; ST PIERRE DES CORPS; PARIS MONTPARNASSE; 454; 454; 0; 58; 87.2; 
2012-09; Atlantique; PARIS MONTPARNASSE; ST PIERRE DES CORPS; 452; 452; 0; 20; 95.6; 
2012-09; Est; PARIS EST; STRASBOURG; 457; 456; 1; 28; 93.9; 
2012-09; Atlantique; TOURS; PARIS MONTPARNASSE; 197; 197; 0; 12; 93.9; 
2012-09; Sud-Est; BESANCON FRANCHE COMTE TGV; PARIS LYON; 207; 207; 0; 7; 96.6; 
2012-10; Atlantique; BREST; PARIS MONTPARNASSE; 181; 180; 1; 8; 95.6; 
2012-10; Sud-Est; PARIS LYON; DIJON VILLE; 465; 462; 3; 16; 96.5; 
2012-10; Nord; PARIS NORD; DOUAI; 213; 211; 2; 16; 92.4; 
2012-10; Nord; DUNKERQUE; PARIS NORD; 103; 101; 2; 5; 95.0; 
2012-10; Sud-Est; PARIS LYON; LE CREUSOT MONTCEAU MONTCHANIN; 208; 208; 0; 116; 44.2; Une zone de travaux de renouvellement de la voie fait perdre à l'ensemble des TGV environ 5 min juste avant leur arrivée au Creusot.
2012-10; Nord; LILLE; LYON PART DIEU; 309; 305; 4; 41; 86.6; 
2012-10; Nord; LYON PART DIEU; LILLE; 279; 273; 6; 49; 82.1; 
2012-10; Nord; LILLE; MARSEILLE ST CHARLES; 150; 147; 3; 35; 76.2; Recrudescence d’accidents de personnes et hausse du nombre de dérangements d’installations ferroviaires. Les conditions météorologiques notamment dans le Sud-Est sur la fin du mois ont aggravé cette situation difficile.
2012-10; Nord; LILLE; PARIS NORD; 651; 641; 10; 45; 93.0; 
2012-10; Sud-Est; MARSEILLE ST CHARLES; LYON PART DIEU; 592; 587; 5; 101; 82.8; 
2012-10; Sud-Est; MONTPELLIER; LYON PART DIEU; 371; 363; 8; 74; 79.6; Intempéries en fin de mois et trois accidents de personne.
2012-10; Atlantique; LYON PART DIEU; RENNES; 34; 34; 0; 3; 91.2; 
2012-10; Sud-Est; MACON LOCHE; PARIS LYON; 178; 176; 2; 9; 94.9; 
2012-10; Sud-Est; MARSEILLE ST CHARLES; PARIS LYON; 520; 509; 11; 30; 94.1; 
2012-10; Atlantique; ST MALO; PARIS MONTPARNASSE; 102; 102; 0; 5; 95.1; 
2012-10; Atlantique; PARIS MONTPARNASSE; ST MALO; 59; 59; 0; 3; 94.9; 
2012-10; Sud-Est; VALENCE ALIXAN TGV; PARIS LYON; 245; 241; 4; 29; 88.0; 
2012-11; Sud-Est; GRENOBLE; PARIS LYON; 238; 237; 1; 20; 91.6; 
2012-11; Atlantique; LA ROCHELLE VILLE; PARIS MONTPARNASSE; 215; 215; 0; 34; 84.2; 
2012-11; Sud-Est; LE CREUSOT MONTCEAU MONTCHANIN; PARIS LYON; 208; 208; 0; 41; 80.3; 
2012-11; Atlantique; PARIS MONTPARNASSE; LE MANS; 442; 442; 0; 35; 92.1; 
2012-11; Atlantique; PARIS MONTPARNASSE; ANGERS SAINT LAUD; 432; 432; 0; 60; 86.1; 
2012-11; Nord; LYON PART DIEU; LILLE; 267; 267; 0; 54; 79.8; Liaison impactée principalement par des événements externes (vols de câble et accidents de personne) mais aussi par des dérangements des installations au sol et quelques difficultés liées au matériel.
2012-11; Sud-Est; LYON PART DIEU; MARSEILLE ST CHARLES; 600; 600; 0; 151; 74.8; Fragilité de cette liaison liée à la longueur du parcours des trains assurant cette desserte.
2012-11; Atlantique; RENNES; LYON PART DIEU; 80; 80; 0; 14; 82.5; 
2012-11; Sud-Est; MARSEILLE ST CHARLES; PARIS LYON; 500; 500; 0; 56; 88.8; 
2012-11; Sud-Est; PARIS LYON; MARSEILLE ST CHARLES; 500; 500; 0; 42; 91.6; 
2012-11; Est; METZ; PARIS EST; 289; 289; 0; 24; 91.7; 
2012-11; Sud-Est; MONTPELLIER; PARIS LYON; 355; 354; 1; 45; 87.3; 
2012-11; Sud-Est; PARIS LYON; MONTPELLIER; 342; 342; 0; 42; 87.7; 
2012-11; Sud-Est; MULHOUSE VILLE; PARIS LYON; 302; 302; 0; 25; 91.7; 
2012-11; Sud-Est; NICE VILLE; PARIS LYON; 152; 152; 0; 24; 84.2; 
2012-11; Sud-Est; NIMES; PARIS LYON; 355; 354; 1; 56; 84.2; 
2012-11; Atlantique; POITIERS; PARIS MONTPARNASSE; 494; 493; 1; 100; 79.7; Importants travaux de raccordement de la ligne à grande vitesse Sud-Europe-Atlantique à la ligne classique entre Bordeaux et Tours ainsi que des travaux de régénération des voies de la ligne à grande vitesse atlantique entre Courtalain et Vendôme.
2012-11; Atlantique; PARIS MONTPARNASSE; POITIERS; 453; 453; 0; 22; 95.1; 
2012-11; Atlantique; PARIS MONTPARNASSE; QUIMPER; 157; 157; 0; 11; 93.0; 
2012-11; Est; REIMS; PARIS EST; 231; 231; 0; 29; 87.4; 
2012-11; Est; STRASBOURG; PARIS EST; 448; 447; 1; 38; 91.5; 
2012-11; Atlantique; TOULOUSE MATABIAU; PARIS MONTPARNASSE; 86; 86; 0; 20; 76.7; Importants travaux de raccordement de la ligne à grande vitesse Sud-Europe-Atlantique à la ligne classique entre Bordeaux et Tours ainsi que des travaux de régénération des voies de la ligne à grande vitesse atlantique entre Courtalain et Vendôme et des travaux caténaire entre Montbarbier et Montauban.
2012-11; Sud-Est; PARIS LYON; VALENCE ALIXAN TGV; 261; 261; 0; 43; 83.5; 
2012-11; Nord; PARIS NORD; ARRAS; 364; 364; 0; 35; 90.4; 
2012-11; Atlantique; PARIS MONTPARNASSE; VANNES; 180; 180; 0; 13; 92.8; 
2014-08; Atlantique; ANGERS SAINT LAUD; PARIS MONTPARNASSE; 440; 440; 0; 21; 95.2; 
2014-08; Atlantique; ANGOULEME; PARIS MONTPARNASSE; 327; 327; 0; 31; 90.5; 
2014-08; Sud-Est; ANNECY; PARIS LYON; 124; 124; 0; 5; 96.0; 
2014-08; Sud-Est; AVIGNON TGV; PARIS LYON; 524; 524; 0; 89; 83.0; En août, la régularité des TGV entre Paris et la Méditerranée a été fortement impactée par 8 accidents de personnes et par un acte de malveillance près du Creusot.
2014-08; Sud-Est; BESANCON FRANCHE COMTE TGV; PARIS LYON; 214; 214; 0; 9; 95.8; 
2014-08; Nord; LILLE; MARSEILLE ST CHARLES; 124; 124; 0; 13; 89.5; 
2014-08; Sud-Est; LYON PART DIEU; MONTPELLIER; 397; 397; 0; 75; 81.1; 
2014-08; Sud-Est; MACON LOCHE; PARIS LYON; 188; 188; 0; 25; 86.7; 
2014-08; Sud-Est; MULHOUSE VILLE; PARIS LYON; 318; 318; 0; 21; 93.4; 
2014-08; Est; PARIS EST; STRASBOURG; 440; 440; 0; 28; 93.6; 
2014-08; Sud-Est; PARIS LYON; BELLEGARDE (AIN); 118; 118; 0; 11; 90.7; 
2014-08; Sud-Est; PARIS LYON; TOULON; 246; 246; 0; 44; 82.1; 
2014-08; Sud-Est; PARIS LYON; VALENCE ALIXAN TGV; 274; 274; 0; 33; 88.0; 
2014-08; Atlantique; PARIS MONTPARNASSE; BREST; 209; 209; 0; 9; 95.7; 
2014-08; Atlantique; PARIS MONTPARNASSE; LE MANS; 442; 442; 0; 48; 89.1; 
2014-08; Atlantique; PARIS MONTPARNASSE; QUIMPER; 175; 175; 0; 10; 94.3; 
2014-08; Atlantique; POITIERS; PARIS MONTPARNASSE; 505; 505; 0; 29; 94.3; 
2014-08; Atlantique; QUIMPER; PARIS MONTPARNASSE; 134; 134; 0; 13; 90.3; 
2014-08; Est; REIMS; PARIS EST; 209; 207; 2; 9; 95.7; 
2014-08; Sud-Est; SAINT ETIENNE CHATEAUCREUX; PARIS LYON; 65; 65; 0; 3; 95.4; 
2014-08; Sud-Est; TOULON; PARIS LYON; 321; 321; 0; 58; 81.9; En août, la régularité des TGV entre Paris et la Méditerranée a été fortement impactée par 8 accidents de personnes et par un acte de malveillance près du Creusot.
2014-08; Atlantique; TOURS; PARIS MONTPARNASSE; 188; 188; 0; 15; 92.0; 
2014-08; Atlantique; VANNES; PARIS MONTPARNASSE; 160; 160; 0; 14; 91.3; 
2014-09; Atlantique; ANGERS SAINT LAUD; PARIS MONTPARNASSE; 457; 457; 0; 32; 93.0; 
2014-09; Atlantique; ANGOULEME; PARIS MONTPARNASSE; 316; 315; 1; 38; 87.9; 
2014-09; Sud-Est; BELLEGARDE (AIN); PARIS LYON; 196; 196; 0; 17; 91.3; 
2014-09; Atlantique; BORDEAUX ST JEAN; PARIS MONTPARNASSE; 635; 632; 3; 53; 91.6; 
2014-09; Sud-Est; CHAMBERY CHALLES LES EAUX; PARIS LYON; 200; 200; 0; 27; 86.5; 
2014-09; Nord; DOUAI; PARIS NORD; 193; 193; 0; 32; 83.4; 
2014-09; Nord; LILLE; MARSEILLE ST CHARLES; 119; 119; 0; 20; 83.2; 
2014-09; Nord; LILLE; PARIS NORD; 606; 606; 0; 53; 91.3; 
2014-09; Sud-Est; MARSEILLE ST CHARLES; LYON PART DIEU; 493; 493; 0; 105; 78.7; Plusieurs incidents liés à la présence de personnes ou d'animaux sur les voies ont fortement perturbé la régularité des TGV sur cette liaison en septembre.
2014-09; Sud-Est; MARSEILLE ST CHARLES; PARIS LYON; 442; 439; 3; 33; 92.5; 
2014-09; Est; NANCY; PARIS EST; 285; 285; 0; 12; 95.8; 
2014-09; Est; NANTES; STRASBOURG; 42; 42; 0; 4; 90.5; 
2014-09; Sud-Est; NICE VILLE; PARIS LYON; 190; 189; 1; 37; 80.4; Les intempéries dans le Var, ainsi que plusieurs incidents liés à la présence de personnes ou d'animaux sur les voies, ont fortement perturbé la régularité des TGV sur cette liaison en septembre.
2014-09; Est; PARIS EST; METZ; 297; 297; 0; 10; 96.6; 
2014-09; Sud-Est; PARIS LYON; CHAMBERY CHALLES LES EAUX; 189; 189; 0; 18; 90.5; 
2014-09; Sud-Est; PARIS LYON; LE CREUSOT MONTCEAU MONTCHANIN; 200; 200; 0; 18; 91.0; 
2014-09; Sud-Est; PARIS LYON; MARSEILLE ST CHARLES; 462; 462; 0; 24; 94.8; 
2014-09; Atlantique; PARIS MONTPARNASSE; BREST; 192; 192; 0; 7; 96.4; 
2014-09; Atlantique; PARIS MONTPARNASSE; VANNES; 168; 168; 0; 15; 91.1; 
2014-09; Sud-Est; PERPIGNAN; PARIS LYON; 149; 148; 1; 28; 81.1; Les intempéries sur les départements du Gard et de l'Hérault, ainsi que plusieurs incidents liés à la présence de personnes ou d'animaux sur les voies, ont fortement perturbé la régularité des TGV sur cette liaison en septembre.
2014-09; Atlantique; ST MALO; PARIS MONTPARNASSE; 91; 91; 0; 6; 93.4; 
2014-09; Sud-Est; TOULON; PARIS LYON; 241; 241; 0; 40; 83.4; Les intempéries dans le Var, ainsi que plusieurs incidents liés à la présence de personnes ou d'animaux sur les voies, ont fortement perturbé la régularité des TGV sur cette liaison en septembre.
2014-10; Nord; DUNKERQUE; PARIS NORD; 101; 101; 0; 7; 93.1; 
2014-10; Atlantique; LA ROCHELLE VILLE; PARIS MONTPARNASSE; 227; 227; 0; 13; 94.3; 
2014-10; Sud-Est; LYON PART DIEU; MARSEILLE ST CHARLES; 597; 597; 0; 147; 75.4; En octobre, la régularité de cette relation a été affectée par plusieurs accidents de personne et heurts d'animaux sur lignes à grande vitesse et par les intempéries en région PACA (inondations)
2014-10; Est; NANCY; PARIS EST; 299; 299; 0; 16; 94.6; 
2014-10; Atlantique; NANTES; PARIS MONTPARNASSE; 572; 570; 2; 53; 90.7; 
2014-10; Est; PARIS EST; NANCY; 299; 299; 0; 11; 96.3; 
2014-10; Sud-Est; PARIS LYON; PERPIGNAN; 161; 161; 0; 29; 82.0; 
2014-10; Nord; PARIS NORD; DOUAI; 164; 164; 0; 7; 95.7; 
2014-10; Atlantique; ST MALO; PARIS MONTPARNASSE; 100; 100; 0; 3; 97.0; 
2014-10; Sud-Est; TOULON; PARIS LYON; 256; 255; 1; 43; 83.1; 
2014-10; Atlantique; VANNES; PARIS MONTPARNASSE; 171; 171; 0; 11; 93.6; 
2014-11; Sud-Est; AIX EN PROVENCE TGV; PARIS LYON; 412; 412; 0; 69; 83.3; 
2014-11; Sud-Est; AVIGNON TGV; PARIS LYON; 426; 422; 4; 77; 81.8; 
2014-11; Nord; DOUAI; PARIS NORD; 192; 192; 0; 39; 79.7; 
2014-11; Sud-Est; GRENOBLE; PARIS LYON; 226; 225; 1; 23; 89.8; 
2014-11; Atlantique; LA ROCHELLE VILLE; PARIS MONTPARNASSE; 218; 218; 0; 10; 95.4; 
2014-11; Nord; MARSEILLE ST CHARLES; LILLE; 550; 548; 2; 155; 71.7; Importantes précipitations sur Montpellier en fin de mois, heurts d'animal à St Georges d'Espéranches le 21 et à Vestric et Crisenoy le 28, plusieurs dérangements des installations à Vianges le 06 et à Upie le 20, colis suspects à Marseille le 3, défaillances Matériel à Aubagne le 4, Cesseins le 5 et Marolles le 14, des défauts d'alimentation à Montpellier le 3, Marseille le 4 et surtout à Lambersart du 20 au 24, heurt de branchages à Claveyson le 25, plusieurs ralentissements causes travaux, essentiellements sur Aubagne.
2014-11; Sud-Est; MONTPELLIER; LYON PART DIEU; 381; 378; 3; 79; 79.1; La régularité des TGV de cette relation a été perturbée en novembre par 2 accidents de personne, 3 heurts d'animaux et par les intempéries en région PACA.
2014-11; Sud-Est; NICE VILLE; PARIS LYON; 172; 169; 3; 44; 74.0; La régularité des TGV de cette relation a été perturbée en novembre par 2 accidents de personne, 3 heurts d'animaux et par les intempéries en région PACA.
2014-11; Sud-Est; NIMES; PARIS LYON; 329; 328; 1; 49; 85.1; 
2014-11; Sud-Est; PARIS LYON; AVIGNON TGV; 470; 466; 4; 38; 91.8; 
2014-11; Sud-Est; PARIS LYON; LE CREUSOT MONTCEAU MONTCHANIN; 197; 197; 0; 20; 89.8; 
2014-11; Atlantique; PARIS MONTPARNASSE; BORDEAUX ST JEAN; 621; 621; 0; 34; 94.5; 
2014-11; Atlantique; PARIS MONTPARNASSE; NANTES; 545; 545; 0; 14; 97.4; 
2014-11; Atlantique; PARIS MONTPARNASSE; TOURS; 142; 142; 0; 18; 87.3; 
2014-11; Est; REIMS; PARIS EST; 203; 203; 0; 9; 95.6; 
2014-11; Atlantique; ST PIERRE DES CORPS; PARIS MONTPARNASSE; 429; 429; 0; 73; 83.0; 
2014-11; Est; STRASBOURG; NANTES; 44; 44; 0; 4; 90.9; 
2014-11; Sud-Est; VALENCE ALIXAN TGV; PARIS LYON; 266; 265; 1; 38; 85.7; 
2014-12; Sud-Est; CHAMBERY CHALLES LES EAUX; PARIS LYON; 243; 243; 0; 45; 81.5; D'importantes chutes de neige ont fortement perturbé le trafic TGV dans les Alpes le week end du 27 et du 28 décembre.
2014-12; Sud-Est; DIJON VILLE; PARIS LYON; 470; 470; 0; 37; 92.1; 
2014-12; Sud-Est; MACON LOCHE; PARIS LYON; 198; 198; 0; 17; 91.4; 
2014-12; Sud-Est; MARSEILLE ST CHARLES; PARIS LYON; 432; 432; 0; 18; 95.8; 
2014-12; Est; METZ; PARIS EST; 298; 298; 0; 27; 90.9; 
2014-12; Est; NANCY; PARIS EST; 296; 295; 1; 13; 95.6; 
2014-12; Sud-Est; PARIS LYON; CHAMBERY CHALLES LES EAUX; 230; 230; 0; 34; 85.2; 
2014-12; Sud-Est; PARIS LYON; PERPIGNAN; 164; 164; 0; 25; 84.8; Les intempéries en région Languedoc Roussillon ont perturbé la régularité de cette relation en décembre.
2014-12; Sud-Est; PARIS LYON; SAINT ETIENNE CHATEAUCREUX; 21; 21; 0; 0; 100.0; 
2014-12; Atlantique; PARIS MONTPARNASSE; BREST; 213; 213; 0; 22; 89.7; La ligne a été affectée par des incidents externes dont un accident de personne à Laval le 01/12 (20 TGV retardés de 16min à 3h19) \& le heurt d'un sanglier sur la Ligne à Grande Vitesse près de St Arnoult le 22/12 (30 TGV retardés de 11min à 1h18);
Des dérangements des installations ont également entraîné des retards: le 25/12 notamment entre Paris Montparnasse \& Massy (32 TGV de 25' à 2h38) \& un rail fissuré à Neau (près de Laval) le 29/12 (18 TGV de 12' à 1h15). A noter également, la défaillance Matériel d'un train d'une autre compagnie ferroviaire en sortie du Mans le 26/12 (20 TGV retardés de 14' à 2h35). 
2014-12; Atlantique; PARIS MONTPARNASSE; LA ROCHELLE VILLE; 223; 222; 1; 17; 92.3; 
2014-12; Atlantique; PARIS MONTPARNASSE; QUIMPER; 148; 148; 0; 19; 87.2; 
2014-12; Atlantique; PARIS MONTPARNASSE; ST MALO; 58; 58; 0; 7; 87.9; 
2011-09; Atlantique; BREST; PARIS MONTPARNASSE; 195; 195; 0; 5; 97.4; 
2011-09; Atlantique; PARIS MONTPARNASSE; BREST; 177; 177; 0; 11; 93.8; 
2011-09; Atlantique; PARIS MONTPARNASSE; LAVAL; 242; 242; 0; 24; 90.1; 
2011-09; Nord; LILLE; LYON PART DIEU; 339; 339; 0; 29; 91.4; 
2011-09; Sud-Est; LYON PART DIEU; MONTPELLIER; 359; 359; 0; 72; 79.9; 
2011-09; Atlantique; RENNES; LYON PART DIEU; 84; 84; 0; 10; 88.1; 
2011-09; Atlantique; ANGOULEME; PARIS MONTPARNASSE; 358; 358; 0; 23; 93.6; 
2011-09; Est; STRASBOURG; NANTES; 52; 52; 0; 4; 92.3; 
2011-09; Sud-Est; PERPIGNAN; PARIS LYON; 129; 129; 0; 15; 88.4; 
2011-09; Atlantique; PARIS MONTPARNASSE; QUIMPER; 150; 150; 0; 7; 95.3; 
2011-09; Sud-Est; ANNECY; PARIS LYON; 196; 196; 0; 19; 90.3; 
2011-09; Est; REIMS; PARIS EST; 233; 233; 0; 20; 91.4; 
2011-09; Est; PARIS EST; REIMS; 245; 245; 0; 16; 93.5; 
2012-11; Atlantique; ST PIERRE DES CORPS; PARIS MONTPARNASSE; 458; 458; 0; 86; 81.2; 
2012-11; Atlantique; PARIS MONTPARNASSE; ST PIERRE DES CORPS; 459; 459; 0; 29; 93.7; 
2012-11; Atlantique; PARIS MONTPARNASSE; TOULOUSE MATABIAU; 132; 132; 0; 12; 90.9; 
2012-11; Atlantique; TOURS; PARIS MONTPARNASSE; 204; 204; 0; 21; 89.7; 
2012-12; Sud-Est; AIX EN PROVENCE TGV; PARIS LYON; 398; 398; 0; 61; 84.7; 
2012-12; Atlantique; PARIS MONTPARNASSE; BREST; 178; 178; 0; 7; 96.1; 
2012-12; Nord; DUNKERQUE; PARIS NORD; 87; 87; 0; 1; 98.9; 
2012-12; Atlantique; PARIS MONTPARNASSE; LA ROCHELLE VILLE; 225; 225; 0; 6; 97.3; 
2012-12; Sud-Est; PARIS LYON; LE CREUSOT MONTCEAU MONTCHANIN; 202; 202; 0; 27; 86.6; 
2012-12; Nord; LYON PART DIEU; LILLE; 255; 255; 0; 63; 75.3; 
2012-12; Sud-Est; MONTPELLIER; LYON PART DIEU; 356; 356; 0; 68; 80.9; 
2012-12; Sud-Est; PARIS LYON; MARSEILLE ST CHARLES; 514; 513; 1; 43; 91.6; 
2012-12; Est; METZ; PARIS EST; 298; 298; 0; 44; 85.2; 
2012-12; Est; PARIS EST; METZ; 314; 314; 0; 43; 86.3; 
2012-12; Atlantique; ANGOULEME; PARIS MONTPARNASSE; 335; 335; 0; 33; 90.1; 
2012-12; Sud-Est; PARIS LYON; MONTPELLIER; 345; 345; 0; 37; 89.3; 
2012-12; Est; PARIS EST; NANCY; 293; 293; 0; 42; 85.7; 
2012-12; Atlantique; PARIS MONTPARNASSE; QUIMPER; 161; 161; 0; 7; 95.7; 
2012-12; Sud-Est; PARIS LYON; ANNECY; 182; 182; 0; 21; 88.5; 
2012-12; Sud-Est; SAINT ETIENNE CHATEAUCREUX; PARIS LYON; 114; 114; 0; 13; 88.6; 
2012-12; Est; STRASBOURG; PARIS EST; 478; 476; 2; 99; 79.2; Problème sur les rames TGV suite aux episodes neigeux des 3, 6, 7, 8 et 9 décembre.
2012-12; Atlantique; TOURS; PARIS MONTPARNASSE; 185; 185; 0; 20; 89.2; 
2012-12; Atlantique; PARIS MONTPARNASSE; TOURS; 159; 159; 0; 23; 85.5; 
2012-12; Sud-Est; BESANCON FRANCHE COMTE TGV; PARIS LYON; 213; 213; 0; 20; 90.6; 
2012-12; Atlantique; BORDEAUX ST JEAN; PARIS MONTPARNASSE; 673; 673; 0; 65; 90.3; 
2013-01; Sud-Est; CHAMBERY CHALLES LES EAUX; PARIS LYON; 264; 264; 0; 49; 81.4; 
2013-01; Sud-Est; PARIS LYON; CHAMBERY CHALLES LES EAUX; 281; 281; 0; 55; 80.4; 
2013-01; Sud-Est; PARIS LYON; DIJON VILLE; 469; 469; 0; 27; 94.2; 
2013-01; Atlantique; PARIS MONTPARNASSE; LE MANS; 453; 453; 0; 79; 82.6; 
2013-01; Atlantique; ANGERS SAINT LAUD; PARIS MONTPARNASSE; 473; 473; 0; 52; 89.0; 
2013-01; Atlantique; PARIS MONTPARNASSE; ANGERS SAINT LAUD; 446; 446; 0; 40; 91.0; 
2013-01; Nord; LYON PART DIEU; LILLE; 247; 247; 0; 79; 68.0; Circulation touchée principalement par les chutes de neige du 14 au 23 janvier sur le Nord et du 15 au 21 janvier sur le Sud Est. Durant cette période, pour éviter les dangers liés aux projections de glace, la vitesse des trains sur la ligne à grande vitesse du Nord ou du Sud Est a été limitée à 230 km/h voire 170 km/h . Les installations au sol ainsi que les rames TGV ont également connu des difficultés liées à cet épisode neigeux.
2013-01; Sud-Est; MARSEILLE ST CHARLES; PARIS LYON; 481; 481; 0; 53; 89.0; 
2013-01; Est; METZ; PARIS EST; 301; 301; 0; 64; 78.7; Plusieures journées d'intempéries ont contraint à réduire la vitesse des TGV sur leur parcours.
2013-01; Sud-Est; PARIS LYON; MONTPELLIER; 349; 349; 0; 38; 89.1; 
2013-01; Sud-Est; PARIS LYON; MULHOUSE VILLE; 307; 307; 0; 28; 90.9; 
2013-01; Sud-Est; PERPIGNAN; PARIS LYON; 155; 155; 0; 16; 89.7; 
2013-01; Sud-Est; SAINT ETIENNE CHATEAUCREUX; PARIS LYON; 112; 112; 0; 17; 84.8; 
2013-01; Atlantique; PARIS MONTPARNASSE; ST MALO; 56; 56; 0; 3; 94.6; 
2013-01; Atlantique; PARIS MONTPARNASSE; ST PIERRE DES CORPS; 457; 457; 0; 69; 84.9; 
2013-01; Sud-Est; PARIS LYON; TOULON; 215; 215; 0; 37; 82.8; 
2013-01; Sud-Est; VALENCE ALIXAN TGV; PARIS LYON; 262; 262; 0; 37; 85.9; 
2013-01; Sud-Est; PARIS LYON; AVIGNON TGV; 425; 424; 1; 61; 85.6; 
2013-01; Atlantique; PARIS MONTPARNASSE; BORDEAUX ST JEAN; 625; 625; 0; 47; 92.5; 
2013-02; Atlantique; PARIS MONTPARNASSE; BREST; 64; 64; 0; 3; 95.3; 
2013-02; Sud-Est; DIJON VILLE; PARIS LYON; 413; 412; 1; 50; 87.9; 
2013-02; Sud-Est; GRENOBLE; PARIS LYON; 220; 220; 0; 33; 85.0; 
2013-02; Atlantique; PARIS MONTPARNASSE; LE MANS; 415; 415; 0; 27; 93.5; 
2013-02; Nord; LYON PART DIEU; LILLE; 223; 223; 0; 67; 70.0; Les chutes de neige des 10 et 11 février nécessitent de réduire la vitesse des trains sur la ligne à grande vitesse à 230 km/h voire 170 km/h pour éviter les dangers liés aux projections de glace. Les rames TGV ont également souffert de ces intempéries ainsi que du froid associé avec des pannes de train en cours de circulation. Les installations caténaires ont subi également des avaries. A noter aussi des heurts d'animaux dans le Sud-Est et des colis abandonnés à Marseille et Marne.
2013-02; Nord; MARSEILLE ST CHARLES; LILLE; 115; 115; 0; 45; 60.9; Les chutes de neige des 10 et 11 février nécessitent de réduire la vitesse des trains sur la ligne à grande vitesse à 230 km/h voire 170 km/h pour éviter les dangers liés aux projections de glace. Les rames TGV ont également souffert de ces intempéries ainsi que du froid associé avec des pannes de train en cours de circulation. Les installations caténaires ont subi également des avaries. A noter aussi des heurts d'animaux dans le Sud-Est et des colis abandonnés à Marseille et Marne.
2013-02; Sud-Est; LYON PART DIEU; MONTPELLIER; 367; 367; 0; 74; 79.8; Conditions météorologiques difficiles (en particulier la neige et le froid).
2013-02; Sud-Est; PARIS LYON; MACON LOCHE; 186; 186; 0; 22; 88.2; 
2013-02; Sud-Est; PARIS LYON; MARSEILLE ST CHARLES; 460; 460; 0; 48; 89.6; 
2013-02; Atlantique; NANTES; PARIS MONTPARNASSE; 519; 518; 1; 28; 94.6; Quelques événements ayant de fortes répercussions sur la régularité des trains ont eu lieu (panne d'un train de travaux le 12, accident de personne le 20, rail cassé à l'entrée de la ligne à grande vitesse le 21, heurt d'une vache le 26).
2013-02; Atlantique; PARIS MONTPARNASSE; NANTES; 515; 515; 0; 30; 94.2; Quelques événements ayant de fortes répercussions sur la régularité des trains ont eu lieu (panne d'un train de travaux le 12, accident de personne le 20, rail cassé à l'entrée de la ligne à grande vitesse le 21, heurt d'une vache le 26).
2013-02; Sud-Est; NICE VILLE; PARIS LYON; 146; 146; 0; 34; 76.7; Conditions météorologiques difficiles (en particulier la neige et le froid).
2013-02; Est; REIMS; PARIS EST; 192; 192; 0; 22; 88.5; 
2013-02; Atlantique; PARIS MONTPARNASSE; ST PIERRE DES CORPS; 414; 414; 0; 26; 93.7; 
2013-02; Sud-Est; PARIS LYON; BELLEGARDE (AIN); 248; 248; 0; 72; 71.0; Conditions météorologiques difficiles (en particulier la neige et le froid).
2013-02; Sud-Est; BESANCON FRANCHE COMTE TGV; PARIS LYON; 197; 197; 0; 18; 90.9; 
2013-02; Sud-Est; PARIS LYON; BESANCON FRANCHE COMTE TGV; 217; 217; 0; 22; 89.9; 
2013-03; Sud-Est; AIX EN PROVENCE TGV; PARIS LYON; 404; 404; 0; 54; 86.6; 
2013-03; Atlantique; PARIS MONTPARNASSE; BREST; 87; 87; 0; 6; 93.1; 
2013-03; Sud-Est; PARIS LYON; DIJON VILLE; 482; 481; 1; 29; 94.0; La journée du 12 mars a été très perturbée par les intempéries (neige et pluies verglassantes) entrainant de nombreux et importants retards avoisinants les 2 heures.
2013-03; Nord; DUNKERQUE; PARIS NORD; 88; 81; 7; 6; 92.6; 
2013-03; Nord; PARIS NORD; DUNKERQUE; 127; 121; 6; 13; 89.3; 
2013-03; Atlantique; LA ROCHELLE VILLE; PARIS MONTPARNASSE; 222; 222; 0; 13; 94.1; Épisode neigeux avec fortes répercussions sur la production les 12 et 13 mars.
2013-03; Nord; LYON PART DIEU; LILLE; 250; 240; 10; 58; 75.8; Les très importantes chutes de neige avec formation de congères survenues dans le nord à partir du lundi 11 mars dans la soirée jusqu'au vendredi 15 mars et également dans le sud-est le 12 mars ont fortement perturbé la production. Elles ont imposé des limitations de vitesse pour éviter les projections de glace. Les conditions climatiques difficiles de ces derniers mois ont par ailleurs perturbé les différents programmes travaux sur les voies prolongeant ainsi certaines limitations de vitesse au-delà des délais prévus. notamment sur Arras et la périphérie de Lyon.
2013-03; Nord; MARSEILLE ST CHARLES; LILLE; 127; 124; 3; 34; 72.6; 
2013-03; Sud-Est; LYON PART DIEU; MARSEILLE ST CHARLES; 597; 591; 6; 160; 72.9; Malgré un temps de parcours relativement court entre ces deux gares, l'essentiel des trains assurant cette desserte sont origine de l'ouest ou du Nord de la France et ont donc parcouru une très longue distance et souvent déjà accumulé du retard avant même d'assurer cette desserte située en fin de parcours du train.
2013-03; Atlantique; PARIS MONTPARNASSE; ANGOULEME; 304; 304; 0; 25; 91.8; 
2013-03; Atlantique; PARIS MONTPARNASSE; POITIERS; 510; 510; 0; 28; 94.5; 
2013-03; Sud-Est; ANNECY; PARIS LYON; 137; 137; 0; 9; 93.4; La journée du 12 mars a été très perturbée par les intempéries (neige et pluies verglassantes) entrainant de nombreux et importants retards avoisinants les 2 heures.
2013-03; Atlantique; RENNES; PARIS MONTPARNASSE; 565; 565; 0; 70; 87.6; 
2013-03; Atlantique; PARIS MONTPARNASSE; ST PIERRE DES CORPS; 455; 455; 0; 42; 90.8; 
2013-03; Atlantique; PARIS MONTPARNASSE; TOULOUSE MATABIAU; 134; 134; 0; 22; 83.6; 
2013-03; Sud-Est; PARIS LYON; VALENCE ALIXAN TGV; 255; 254; 1; 30; 88.2; 
2013-03; Sud-Est; BELLEGARDE (AIN); PARIS LYON; 285; 285; 0; 51; 82.1; 
2013-03; Sud-Est; BESANCON FRANCHE COMTE TGV; PARIS LYON; 223; 222; 1; 17; 92.3; La journée du 12 mars a été très perturbée par les intempéries (neige et pluies verglassantes) entrainant de nombreux et importants retards avoisinants les 2 heures.
2013-04; Sud-Est; AIX EN PROVENCE TGV; PARIS LYON; 425; 425; 0; 61; 85.6; 
2013-04; Atlantique; BREST; PARIS MONTPARNASSE; 168; 168; 0; 10; 94.0; Heurt d'un sanglier à l'entrée de la ligne à grande vitesse, présence d'une vache sur les voies près de Plouaret Trégor, accident de personne entre Rennes et Vitré, problème d'alimentation électrique près de Morlaix.
2013-04; Sud-Est; PARIS LYON; CHAMBERY CHALLES LES EAUX; 238; 238; 0; 33; 86.1; 
2013-04; Nord; PARIS NORD; DOUAI; 204; 204; 0; 25; 87.7; 
2013-04; Atlantique; LA ROCHELLE VILLE; PARIS MONTPARNASSE; 215; 215; 0; 7; 96.7; Panne d'un train de travaux en ligne près de la Rochelle, accident de personne à Poitiers, retard dans la mise à quai du TGV 8384.
2013-04; Atlantique; LE MANS; PARIS MONTPARNASSE; 489; 489; 0; 34; 93.0; 
2013-04; Atlantique; PARIS MONTPARNASSE; LE MANS; 440; 440; 0; 40; 90.9; 
2013-04; Atlantique; ANGERS SAINT LAUD; PARIS MONTPARNASSE; 457; 456; 1; 21; 95.4; 
2013-04; Sud-Est; MARSEILLE ST CHARLES; LYON PART DIEU; 572; 572; 0; 99; 82.7; 
2013-04; Sud-Est; LYON PART DIEU; PARIS LYON; 625; 621; 4; 39; 93.7; Plusieurs incidents techniques et des actes de malveillance ont eu lieu les 17, 19, 21, 25 et 26 Avril sur l'Axe Paris-Lyon entrainant de nombreux retards.
2013-04; Sud-Est; PARIS LYON; LYON PART DIEU; 612; 612; 0; 46; 92.5; 
2013-04; Atlantique; RENNES; LYON PART DIEU; 80; 80; 0; 6; 92.5; Accident de personne près de Lyon, rail cassé entre Nantes et Le Mans retardant l'une des deux parties du TGV 5346 pour Lyon (l'autre étant en provenance de Rennes), incident caténaire (fil d'alimentation électrique) à Lyon, problème technique ne permettant pas de tracer le parcours des itinéraires des TGV.
2013-04; Sud-Est; MACON LOCHE; PARIS LYON; 194; 193; 1; 25; 87.0; 
2013-04; Est; STRASBOURG; NANTES; 59; 59; 0; 7; 88.1; 
2013-04; Sud-Est; PARIS LYON; PERPIGNAN; 153; 153; 0; 17; 88.9; Plusieurs incidents techniques et des actes de malveillance ont eu lieu les 17, 19, 21, 25 et 26 Avril sur l'Axe Paris-Lyon entrainant de nombreux retards.
2013-04; Est; PARIS EST; STRASBOURG; 466; 466; 0; 22; 95.3; 
2013-04; Atlantique; TOULOUSE MATABIAU; PARIS MONTPARNASSE; 93; 93; 0; 5; 94.6; Accidents de personne les 15, 17 et 18. Présomption d'avarie sur le fil d'alimentation électrique (caténaire) le 14 près de Bordeaux.
2013-04; Sud-Est; PARIS LYON; AVIGNON TGV; 375; 375; 0; 35; 90.7; 
2013-04; Sud-Est; BELLEGARDE (AIN); PARIS LYON; 274; 274; 0; 25; 90.9; 
2013-04; Sud-Est; PARIS LYON; BESANCON FRANCHE COMTE TGV; 237; 237; 0; 22; 90.7; 
2014-03; Sud-Est; CHAMBERY CHALLES LES EAUX; PARIS LYON; 297; 297; 0; 24; 91.9; 
2014-07; Sud-Est; PARIS LYON; AVIGNON TGV; 583; 583; 0; 50; 91.4; 
2014-07; Atlantique; BORDEAUX ST JEAN; PARIS MONTPARNASSE; 646; 642; 4; 113; 82.4; Le 9 la panne d'un train à Marcoussis, sur la Ligne à Grande Vitesse, retarde 28 TGV de 8 minutes à 1h35, le 10 un acte de malveillance affectant un train à St Léger, sur la Ligne à Grande Vitesse, retarde 43 TGV de 14 minutes à 4h30, le 20 un arbre sur la voie suite aux intempéries entre Angoulême et Mouthiers retarde 16 TGV de 14 minutes à 3h25.
2014-08; Sud-Est; AIX EN PROVENCE TGV; PARIS LYON; 424; 424; 0; 63; 85.1; 
2014-08; Atlantique; BREST; PARIS MONTPARNASSE; 179; 179; 0; 7; 96.1; 
2014-08; Atlantique; LE MANS; PARIS MONTPARNASSE; 456; 456; 0; 48; 89.5; 
2014-08; Atlantique; LYON PART DIEU; RENNES; 31; 31; 0; 6; 80.6; Cette liaison a été particulièrement touchée ce mois-ci par une augmentation des accidents de personne : 6 au total dont 4 sur le réseau Sud-Est. Une limitation temporaire de vitesse pour travaux sur la Ligne à Grande Vitesse vers Mâcon accentue également les retards. Le 27, deux défaillances Matériel (l'une vers Laval, l'autre sur la Ligne à Grande Vitesse Sud-Est) retardent une dizaine de TGV entre 12 minutes et 2h08. Le 31, un dérangement d'aiguille en région parisienne retarde 23 TGV de 12 minutes à 1h12.
2014-08; Est; METZ; PARIS EST; 295; 295; 0; 22; 92.5; 
2014-08; Est; NANTES; STRASBOURG; 50; 50; 0; 4; 92.0; 
2014-08; Est; PARIS EST; METZ; 304; 304; 0; 10; 96.7; 
2014-08; Est; PARIS EST; REIMS; 214; 213; 1; 9; 95.8; 
2014-08; Sud-Est; PARIS LYON; AIX EN PROVENCE TGV; 461; 461; 0; 53; 88.5; 
2014-08; Sud-Est; PARIS LYON; ANNECY; 151; 151; 0; 9; 94.0; 
2014-08; Sud-Est; PARIS LYON; BESANCON FRANCHE COMTE TGV; 203; 203; 0; 12; 94.1; 
2014-08; Sud-Est; PARIS LYON; DIJON VILLE; 485; 485; 0; 19; 96.1; 
2014-08; Sud-Est; PARIS LYON; LE CREUSOT MONTCEAU MONTCHANIN; 206; 206; 0; 16; 92.2; 
2014-08; Sud-Est; PARIS LYON; NIMES; 366; 366; 0; 61; 83.3; 
2014-08; Sud-Est; PARIS LYON; SAINT ETIENNE CHATEAUCREUX; 83; 83; 0; 3; 96.4; 
2014-08; Atlantique; PARIS MONTPARNASSE; ANGOULEME; 329; 329; 0; 21; 93.6; 
2014-08; Atlantique; PARIS MONTPARNASSE; BORDEAUX ST JEAN; 654; 654; 0; 48; 92.7; 
2014-08; Atlantique; PARIS MONTPARNASSE; LA ROCHELLE VILLE; 225; 225; 0; 8; 96.4; 
2014-08; Nord; PARIS NORD; DOUAI; 117; 116; 1; 8; 93.1; 
2014-08; Atlantique; RENNES; LYON PART DIEU; 78; 78; 0; 9; 88.5; Cette liaison a été particulièrement touchée ce mois-ci par une augmentation des accidents de personne : 6 au total dont 4 sur le réseau Sud-Est. Une limitation temporaire de vitesse pour travaux sur la Ligne à Grande Vitesse (vers Mâcon) accentue également les retards. Le 27, deux défaillances Matériel (l'une vers Laval, l'autre sur la Ligne à Grande Vitesse Sud-Est) retardent une dizaine de TGV entre 12 minutes et 2h08. Le 31, un dérangement d'aiguille en région parisienne retarde 23 TGV de 12 minutes à 1h12. 
2014-08; Atlantique; RENNES; PARIS MONTPARNASSE; 563; 563; 0; 47; 91.7; 
2014-09; Nord; DUNKERQUE; PARIS NORD; 110; 110; 0; 5; 95.5; 
2014-09; Atlantique; LA ROCHELLE VILLE; PARIS MONTPARNASSE; 215; 214; 1; 11; 94.9; 
2014-09; Sud-Est; LE CREUSOT MONTCEAU MONTCHANIN; PARIS LYON; 213; 213; 0; 36; 83.1; 
2014-09; Sud-Est; LYON PART DIEU; PARIS LYON; 617; 617; 0; 40; 93.5; 
2014-09; Sud-Est; MONTPELLIER; LYON PART DIEU; 348; 344; 4; 84; 75.6; Les intempéries sur les départements du Gard et de l'Hérault, ainsi que plusieurs incidents liés à la présence de personnes ou d'animaux sur les voies, ont fortement perturbé la régularité des TGV sur cette liaison en septembre.
2014-09; Sud-Est; MULHOUSE VILLE; PARIS LYON; 311; 311; 0; 18; 94.2; 
2014-09; Atlantique; NANTES; PARIS MONTPARNASSE; 577; 577; 0; 34; 94.1; 
2014-09; Sud-Est; PARIS LYON; NICE VILLE; 173; 173; 0; 39; 77.5; Les intempéries dans le Var, ainsi que plusieurs incidents liés à la présence de personnes ou d'animaux sur les voies, ont fortement perturbé la régularité des TGV sur cette liaison en septembre.
2014-09; Atlantique; PARIS MONTPARNASSE; LAVAL; 214; 214; 0; 10; 95.3; 
2014-09; Atlantique; PARIS MONTPARNASSE; ST PIERRE DES CORPS; 443; 443; 0; 30; 93.2; 
2014-09; Nord; PARIS NORD; DOUAI; 160; 159; 1; 11; 93.1; 
2014-09; Est; REIMS; PARIS EST; 203; 203; 0; 5; 97.5; 
2014-09; Atlantique; ST PIERRE DES CORPS; PARIS MONTPARNASSE; 420; 420; 0; 81; 80.7; 
2014-09; Est; STRASBOURG; NANTES; 43; 43; 0; 4; 90.7; 
2014-10; Sud-Est; AIX EN PROVENCE TGV; PARIS LYON; 448; 448; 0; 67; 85.0; 
2014-10; Atlantique; ANGERS SAINT LAUD; PARIS MONTPARNASSE; 466; 465; 1; 72; 84.5; 
2014-10; Sud-Est; ANNECY; PARIS LYON; 142; 142; 0; 12; 91.5; 
2014-10; Sud-Est; BESANCON FRANCHE COMTE TGV; PARIS LYON; 218; 218; 0; 14; 93.6; 
2014-10; Sud-Est; CHAMBERY CHALLES LES EAUX; PARIS LYON; 207; 207; 0; 22; 89.4; 
2014-10; Sud-Est; GRENOBLE; PARIS LYON; 250; 250; 0; 27; 89.2; 
2014-10; Nord; LILLE; LYON PART DIEU; 211; 211; 0; 14; 93.4; 
2014-10; Nord; LILLE; PARIS NORD; 632; 632; 0; 71; 88.8; 
2014-10; Sud-Est; LYON PART DIEU; MONTPELLIER; 390; 390; 0; 108; 72.3; Plusieurs accidents de personne et heurts d'animaux sur lignes à grande vitesse et en région Languedoc-Roussillon ont perturbé la régularité de cette relation en octobre
2014-10; Sud-Est; MARSEILLE ST CHARLES; LYON PART DIEU; 549; 549; 0; 130; 76.3; En octobre, la régularité de cette relation a été affectée par plusieurs accidents de personne et heurts d'animaux sur lignes à grande vitesse et par les intempéries en région PACA (inondations)
2014-10; Sud-Est; PARIS LYON; BELLEGARDE (AIN); 249; 247; 2; 28; 88.7; 
2014-10; Sud-Est; PARIS LYON; BESANCON FRANCHE COMTE TGV; 212; 212; 0; 12; 94.3; 
2014-10; Sud-Est; PARIS LYON; GRENOBLE; 259; 259; 0; 27; 89.6; 
2014-10; Sud-Est; PARIS LYON; LYON PART DIEU; 643; 643; 0; 30; 95.3; 
2014-10; Sud-Est; PARIS LYON; MARSEILLE ST CHARLES; 484; 484; 0; 28; 94.2; 
2014-10; Sud-Est; PARIS LYON; MONTPELLIER; 324; 324; 0; 40; 87.7; 
2014-10; Sud-Est; PARIS LYON; NICE VILLE; 181; 181; 0; 34; 81.2; En octobre, la régularité de cette relation a été affectée par plusieurs accidents de personne et heurts d'animaux sur lignes à grande vitesse et par les intempéries en région PACA (inondations)
2014-10; Sud-Est; PARIS LYON; SAINT ETIENNE CHATEAUCREUX; ; ; 0; 0; ; 
2014-10; Atlantique; PARIS MONTPARNASSE; LE MANS; 459; 459; 0; 53; 88.5; 
2014-10; Atlantique; PARIS MONTPARNASSE; ST PIERRE DES CORPS; 458; 458; 0; 49; 89.3; 
2014-10; Atlantique; PARIS MONTPARNASSE; TOULOUSE MATABIAU; 149; 149; 0; 22; 85.2; 
2014-10; Atlantique; PARIS MONTPARNASSE; VANNES; 161; 161; 0; 16; 90.1; 
2014-10; Nord; PARIS NORD; LILLE; 638; 638; 0; 34; 94.7; 
2014-10; Atlantique; QUIMPER; PARIS MONTPARNASSE; 148; 148; 0; 10; 93.2; 
2014-10; Est; STRASBOURG; PARIS EST; 486; 485; 1; 37; 92.4; 
2014-11; Atlantique; ANGOULEME; PARIS MONTPARNASSE; 327; 327; 0; 39; 88.1; 
2014-11; Sud-Est; ANNECY; PARIS LYON; 133; 133; 0; 7; 94.7; 
2014-11; Nord; ARRAS; PARIS NORD; 321; 320; 1; 60; 81.3; 
2014-11; Sud-Est; BESANCON FRANCHE COMTE TGV; PARIS LYON; 219; 219; 0; 16; 92.7; 
2014-11; Nord; LILLE; MARSEILLE ST CHARLES; 118; 118; 0; 17; 85.6; 
2014-11; Sud-Est; MULHOUSE VILLE; PARIS LYON; 310; 310; 0; 25; 91.9; 
2014-11; Atlantique; NANTES; PARIS MONTPARNASSE; 554; 554; 0; 24; 95.7; 
2014-11; Sud-Est; PARIS LYON; AIX EN PROVENCE TGV; 427; 427; 0; 38; 91.1; 
2014-11; Sud-Est; PARIS LYON; ANNECY; 144; 144; 0; 13; 91.0; 
2014-11; Sud-Est; PARIS LYON; BELLEGARDE (AIN); 234; 234; 0; 26; 88.9; 
2014-11; Sud-Est; PARIS LYON; BESANCON FRANCHE COMTE TGV; 208; 208; 0; 11; 94.7; 
2014-11; Sud-Est; PARIS LYON; GRENOBLE; 235; 234; 1; 26; 88.9; 
2014-11; Sud-Est; PARIS LYON; MACON LOCHE; 187; 186; 1; 17; 90.9; 
2014-11; Sud-Est; PARIS LYON; MONTPELLIER; 310; 310; 0; 33; 89.4; 
2014-11; Sud-Est; PARIS LYON; MULHOUSE VILLE; 302; 302; 0; 19; 93.7; 
2014-11; Sud-Est; PARIS LYON; TOULON; 169; 168; 1; 25; 85.1; La régularité des TGV de cette relation a été perturbée en novembre par 2 accidents de personne, 3 heurts d'animaux et par les intempéries en région PACA.
2014-11; Atlantique; PARIS MONTPARNASSE; ANGERS SAINT LAUD; 425; 425; 0; 7; 98.4; 
2014-11; Atlantique; PARIS MONTPARNASSE; BREST; 202; 202; 0; 8; 96.0; 
2014-11; Atlantique; PARIS MONTPARNASSE; LA ROCHELLE VILLE; 219; 219; 0; 6; 97.3; 
2014-11; Atlantique; PARIS MONTPARNASSE; LAVAL; 437; 437; 0; 21; 95.2; 
2014-11; Atlantique; PARIS MONTPARNASSE; ST PIERRE DES CORPS; 436; 436; 0; 39; 91.1; 
2014-11; Atlantique; PARIS MONTPARNASSE; VANNES; 148; 148; 0; 7; 95.3; 
2014-11; Nord; PARIS NORD; ARRAS; 325; 325; 0; 32; 90.2; 
2014-11; Sud-Est; PERPIGNAN; PARIS LYON; 155; 155; 0; 25; 83.9; 
2014-11; Atlantique; RENNES; LYON PART DIEU; 60; 60; 0; 4; 93.3; 
2014-11; Atlantique; ST MALO; PARIS MONTPARNASSE; 99; 99; 0; 7; 92.9; 
2014-12; Sud-Est; ANNECY; PARIS LYON; 139; 139; 0; 11; 92.1; 
2014-12; Nord; ARRAS; PARIS NORD; 336; 336; 0; 54; 83.9; 
2014-12; Atlantique; LE MANS; PARIS MONTPARNASSE; 467; 467; 0; 58; 87.6; 
2014-12; Nord; LYON PART DIEU; LILLE; 264; 262; 2; 58; 77.9; Les principaux évènements survenus au mois de décembre sont :
Importantes chutes de neige dans le sud est le 27
Accident de personne à Lunel le 11
Heurts d'animaux : Vic Mireval le 04, Cesseins le 08, La palud le 12 et à Lacour le 30
Un dérangement des installations à Chevry le 03, 
Plusieurs colis suspects dont essentiellement à Marne la Vallée et Lyon le 27 
Des défauts d'alimentation à St Cézaire le 09 et Lyon Part Dieu le 11
Une aggression sur Grenoble le 16 occasionnant des droits de retrait sur plusieurs jours 
2014-12; Sud-Est; LYON PART DIEU; MARSEILLE ST CHARLES; 609; 609; 0; 125; 79.5; La régularité de cette relation a été perturbée en décembre par un heurt d'animal et 3 pannes de TGV sur la ligne à grande vitesse.
2014-12; Nord; MARSEILLE ST CHARLES; LILLE; 141; 141; 0; 42; 70.2; Les principaux évènements survenus au mois de décembre sont :
Importantes chutes de neige dans le sud est le 27
Accident de personne à Lunel le 11
Heurts d'animaux : Vic Mireval le 04, Cesseins le 08, La palud le 12 et à Lacour le 30
Un dérangement des installations à Chevry le 03, 
Plusieurs colis suspects dont essentiellement à Marne la Vallée et Lyon le 27 
Des défauts d'alimentation à St Cézaire le 09 et Lyon Part Dieu le 11
Une aggression sur Grenoble le 16 occasionnant des droits de retrait sur plusieurs jours 
2014-12; Atlantique; NANTES; PARIS MONTPARNASSE; 579; 579; 0; 37; 93.6; 
2014-12; Est; PARIS EST; METZ; 310; 310; 0; 10; 96.8; 
2014-12; Est; PARIS EST; NANCY; 291; 291; 0; 16; 94.5; 
2014-12; Est; PARIS EST; REIMS; 227; 227; 0; 16; 93.0; 
2014-12; Sud-Est; PARIS LYON; LE CREUSOT MONTCEAU MONTCHANIN; 207; 207; 0; 17; 91.8; 
2014-12; Sud-Est; PARIS LYON; MACON LOCHE; 207; 207; 0; 6; 97.1; 
2012-01; Sud-Est; PARIS LYON; BESANCON FRANCHE COMTE TGV; 244; 244; 0; 17; 93.0; 
2012-01; Atlantique; BORDEAUX ST JEAN; PARIS MONTPARNASSE; 646; 646; 0; 56; 91.3; 
2012-02; Atlantique; PARIS MONTPARNASSE; BREST; 167; 167; 0; 9; 94.6; 
2012-02; Atlantique; LA ROCHELLE VILLE; PARIS MONTPARNASSE; 200; 199; 1; 50; 74.9; 
2012-02; Atlantique; LE MANS; PARIS MONTPARNASSE; 443; 443; 0; 116; 73.8; 
2012-02; Atlantique; ANGERS SAINT LAUD; PARIS MONTPARNASSE; 461; 461; 0; 67; 85.5; 
2012-02; Nord; MARSEILLE ST CHARLES; LILLE; 117; 117; 0; 42; 64.1; 
2012-02; Sud-Est; LYON PART DIEU; MARSEILLE ST CHARLES; 577; 576; 1; 197; 65.8; 
2012-02; Atlantique; RENNES; LYON PART DIEU; 77; 77; 0; 13; 83.1; 
2012-02; Sud-Est; MACON LOCHE; PARIS LYON; 172; 169; 3; 24; 85.8; 
2012-02; Est; PARIS EST; METZ; 294; 294; 0; 22; 92.5; 
2012-02; Sud-Est; MULHOUSE VILLE; PARIS LYON; 296; 295; 1; 49; 83.4; 
2012-02; Sud-Est; ANNECY; PARIS LYON; 194; 191; 3; 22; 88.5; 
2012-02; Est; REIMS; PARIS EST; 224; 223; 1; 16; 92.8; 
2012-02; Sud-Est; PARIS LYON; SAINT ETIENNE CHATEAUCREUX; 99; 98; 1; 17; 82.7; 
2012-02; Atlantique; ST PIERRE DES CORPS; PARIS MONTPARNASSE; 436; 436; 0; 145; 66.7; 
2012-02; Est; STRASBOURG; PARIS EST; 451; 450; 1; 96; 78.7; 
2012-02; Sud-Est; TOULON; PARIS LYON; 228; 228; 0; 43; 81.1; 
2012-02; Sud-Est; PARIS LYON; TOULON; 203; 202; 1; 42; 79.2; 
2012-02; Atlantique; TOULOUSE MATABIAU; PARIS MONTPARNASSE; 109; 109; 0; 36; 67.0; 
2012-02; Nord; PARIS NORD; ARRAS; 357; 356; 1; 53; 85.1; 
2012-02; Sud-Est; BELLEGARDE (AIN); PARIS LYON; 255; 255; 0; 37; 85.5; 
2012-02; Sud-Est; PARIS LYON; BESANCON FRANCHE COMTE TGV; 226; 226; 0; 33; 85.4; 
2012-02; Atlantique; PARIS MONTPARNASSE; BORDEAUX ST JEAN; 583; 582; 1; 106; 81.8; 
2012-03; Atlantique; PARIS MONTPARNASSE; BREST; 183; 183; 0; 4; 97.8; Des ralentissements pour travaux en Région Pays-de-la-Loire retardent les TGV sur leur parcours entre Nantes et Le Mans (tunnel de chantenay, entrée de gare de Nantes, réfection des quai à Champtocé et Savennières, travaux à Montoir).
2012-03; Sud-Est; PARIS LYON; CHAMBERY CHALLES LES EAUX; 260; 260; 0; 35; 86.5; 
2012-03; Atlantique; PARIS MONTPARNASSE; LA ROCHELLE VILLE; 218; 218; 0; 15; 93.1; Des ralentissements pour travaux de renouvellement des voies au nord de Bordeaux retardent les TGV en provenance de Bordeaux et qui desservent Poitiers.
2012-03; Atlantique; LE MANS; PARIS MONTPARNASSE; 473; 473; 0; 70; 85.2; 
2012-03; Atlantique; PARIS MONTPARNASSE; LE MANS; 457; 457; 0; 39; 91.5; Une liaison très proche de l'objectif de 90\% de régularité. Le projet d'amélioration de la liaison Tours-Paris initié dès janvier 2011 porte ses fruits sur les TGV Paris-Tours. Ainsi, leur régularité progresse de 5 points par rapport à février et de 3 points par rapport à janvier.
2012-03; Sud-Est; LYON PART DIEU; MARSEILLE ST CHARLES; 620; 619; 1; 128; 79.3; 60\% des trains assurant cette desserte sont des trains à parcours longs (venant de Metz, Lille, Nantes ou Strasbourg). Ces trains sont fragilisés par la longueur de leur parcours et arrivent au début de cette desserte avec du retard (20.68\%).
2012-03; Sud-Est; PARIS LYON; LYON PART DIEU; 638; 638; 0; 19; 97.0; 
2012-03; Atlantique; RENNES; LYON PART DIEU; 83; 83; 0; 2; 97.6; 
2012-03; Est; PARIS EST; METZ; 310; 310; 0; 16; 94.8; Incendie d'un entrepôt à hauteur de Gagny qui a généré une coupure caténaire les 10 et 11 mars 2012.
2012-03; Atlantique; ANGOULEME; PARIS MONTPARNASSE; 359; 358; 1; 57; 84.1; Irrégularité est due essentiellement aux travaux.
2012-03; Atlantique; NANTES; PARIS MONTPARNASSE; 590; 590; 0; 34; 94.2; 
2012-03; Est; NANTES; STRASBOURG; 62; 62; 0; 7; 88.7; Incendie d'un entrepôt à hauteur de Gagny qui a généré une coupure caténaire les 10 et 11 mars 2012.
2012-03; Sud-Est; NICE VILLE; PARIS LYON; 160; 160; 0; 28; 82.5; 
2012-03; Atlantique; PARIS MONTPARNASSE; POITIERS; 466; 466; 0; 21; 95.5; 
2012-03; Est; REIMS; PARIS EST; 238; 238; 0; 16; 93.3; Incendie d'un entrepôt à hauteur de Gagny qui a généré une coupure caténaire les 10 et 11 mars 2012.
2012-03; Est; PARIS EST; REIMS; 251; 250; 1; 21; 91.6; Incendie d'un entrepôt à hauteur de Gagny qui a généré une coupure caténaire les 10 et 11 mars 2012.
2012-03; Atlantique; ST MALO; PARIS MONTPARNASSE; 102; 102; 0; 2; 98.0; 
2012-03; Atlantique; PARIS MONTPARNASSE; ST MALO; 58; 58; 0; 1; 98.3; 
2012-03; Sud-Est; PARIS LYON; TOULON; 217; 217; 0; 23; 89.4; 
2012-03; Sud-Est; VALENCE ALIXAN TGV; PARIS LYON; 242; 242; 0; 29; 88.0; 
2012-03; Nord; ARRAS; PARIS NORD; 347; 345; 2; 45; 87.0; Circulation fortement dégradée le 5 mars avec des conséquences sur le 6 mars suite aux difficiles conditions climatiques entraînant notamment la chute d'un câble ERDF sur la ligne à grande vitesse. La ponctualité a également été impactée par les travaux importants de rénovation des voies sur la ligne à grande vitesse entre Lille et TGV - Haute Picardie.
2012-03; Nord; PARIS NORD; ARRAS; 380; 379; 1; 32; 91.6; Circulation des trains dégradée le 5 mars avec des conséquences sur le 6 mars suite aux difficiles conditions climatiques entraînant notamment la chute d'un câble ERDF sur la ligne à grande vitesse. La ponctualité a également été impactée par les travaux importants de rénovation des voies sur la ligne à grande vitesse entre Lille et TGV - Haute Picardie.
2012-03; Atlantique; PARIS MONTPARNASSE; VANNES; 170; 170; 0; 9; 94.7; 
2012-03; Sud-Est; BESANCON FRANCHE COMTE TGV; PARIS LYON; 220; 220; 0; 25; 88.6; 
2012-04; Sud-Est; DIJON VILLE; PARIS LYON; 440; 440; 0; 56; 87.3; 
2012-04; Nord; DOUAI; PARIS NORD; 201; 201; 0; 52; 74.1; Travaux importants de rénovation des voies entre Arras et Isbergues. entre Douai et Valenciennes et sur la ligne à grande vitesse entre Lille et TGV - Haute Picardie. Les circulations ont également été impactées par une succession de dérangements d'installations ferroviaires.
2012-04; Nord; LILLE; LYON PART DIEU; 300; 300; 0; 30; 90.0; Conditions difficiles avec des retards liés notamment aux travaux importants sur le réseau Nord et Sud de la France. Les circulations ont également été impactées par une succession de dérangements d'installations ferroviaires.
2012-04; Nord; LILLE; PARIS NORD; 594; 591; 3; 40; 93.2; Travaux importants de rénovation des voies entre Arras et Isbergues. entre Douai et Valenciennes et sur la ligne à grande vitesse entre Lille et TGV - Haute Picardie. Les circulations ont également été impactées par une succession de dérangements d'installations ferroviaires.
2012-04; Nord; PARIS NORD; LILLE; 607; 606; 1; 68; 88.8; Travaux importants de rénovation des voies entre Arras et Isbergues. entre Douai et Valenciennes et sur la ligne à grande vitesse entre Lille et TGV - Haute Picardie. Les circulations ont également été impactées par une succession de dérangements d'installations ferroviaires.
2012-04; Sud-Est; MARSEILLE ST CHARLES; LYON PART DIEU; 576; 576; 0; 106; 81.6; 
2012-04; Sud-Est; LYON PART DIEU; PARIS LYON; 609; 607; 2; 22; 96.4; 
2012-04; Sud-Est; PARIS LYON; MARSEILLE ST CHARLES; 503; 503; 0; 45; 91.1; 
2012-04; Est; NANCY; PARIS EST; 289; 289; 0; 13; 95.5; Une liaison au-dessus de l'objectif malgré les pertes de temps du au jet de pierre et à l'incendie à l'entrée de Paris-Est.
2012-04; Sud-Est; NICE VILLE; PARIS LYON; 194; 194; 0; 41; 78.9; Une baisse de la régularité en avril qui s'explique par une augmentation des incidents d'origine externe (trois accidents de personne, un vol de câbles sur ligne nouvelle) et deux incidents d'exploitation importants dans le noeud Marseillais.
2012-04; Sud-Est; NIMES; PARIS LYON; 382; 381; 1; 38; 90.0; 
2012-04; Sud-Est; PARIS LYON; NIMES; 364; 364; 0; 46; 87.4; 
2012-04; Atlantique; POITIERS; PARIS MONTPARNASSE; 498; 498; 0; 58; 88.4; Le 26 décembre, une personne suicidaire sur les voies sur la Ligne à Grande Vitesse au niveau de St-Léger retarde 17 TGV à partir de 17h00.
2012-04; Sud-Est; ANNECY; PARIS LYON; 195; 195; 0; 15; 92.3; 
2012-04; Atlantique; RENNES; PARIS MONTPARNASSE; 563; 563; 0; 32; 94.3; 
2012-04; Atlantique; PARIS MONTPARNASSE; ST MALO; 54; 54; 0; 0; 100.0; 
2012-04; Sud-Est; TOULON; PARIS LYON; 259; 259; 0; 51; 80.3; 
2012-04; Atlantique; TOURS; PARIS MONTPARNASSE; 203; 203; 0; 22; 89.2; Panne d'un train de travaux à 06h30 affectant les départs du matin, anomalie au passage d'un TGV affectant d'autres trains, disjonction caténaire sur la Ligne Grande Vitesse Atlantique.
2012-04; Nord; PARIS NORD; ARRAS; 357; 355; 2; 34; 90.4; Conditions difficiles avec des retards liés notamment aux travaux importants de rénovation des voies entre Arras et Isbergues, entre Douai et Valenciennes et sur la ligne à grande vitesse entre Lille et TGV - Haute Picardie. Les circulations ont également été impactées par une succession de dérangements d'installations ferroviaires.
2012-05; Sud-Est; CHAMBERY CHALLES LES EAUX; PARIS LYON; 199; 199; 0; 24; 87.9; 
2012-05; Sud-Est; PARIS LYON; GRENOBLE; 241; 241; 0; 21; 91.3; 
2012-05; Atlantique; LA ROCHELLE VILLE; PARIS MONTPARNASSE; 208; 208; 0; 27; 87.0; Travaux de suppression d'un passage à niveau, de suppression d'une passerelle à Mauzé (79) et d'amélioration de la qualité de la voie.
2012-05; Atlantique; ANGERS SAINT LAUD; PARIS MONTPARNASSE; 476; 476; 0; 30; 93.7; 
2012-05; Sud-Est; MARSEILLE ST CHARLES; LYON PART DIEU; 600; 600; 0; 119; 80.2; 
2012-05; Sud-Est; LYON PART DIEU; PARIS LYON; 629; 628; 1; 44; 93.0; 
2012-05; Atlantique; RENNES; LYON PART DIEU; 81; 81; 0; 9; 88.9; Ralentissements dus aux travaux présents sur le parcours (près de Connerré (72) et de Massy (banlieue sud de Paris) pour la régénération de la Ligne à Grande Vitesse Atlantique) mais aussi par un certain nombre d'incidents : difficultés pour atteler deux rames le 1er mai, attente pour réaliser une correspondance le 7, malaise d'un contrôleur le 9.
2012-05; Est; PARIS EST; NANCY; 293; 293; 0; 11; 96.2; En amélioration par rapport aux derniers six mois.
2012-05; Est; STRASBOURG; NANTES; 61; 61; 0; 7; 88.5; Un légère baisse mais un résultat toujours supérieur à l'objectif malgré un baisse de la performance du matériel sur la fin de mois.
2012-05; Sud-Est; PARIS LYON; NICE VILLE; 206; 206; 0; 39; 81.1; D'importantes phases de travaux d'amélioration de l'infrastructure sur le tronçon Nice Marseille nécessitent la mise en place de limitations de vitesse qui réduisent la fluidité des circulations.
2012-05; Sud-Est; PARIS LYON; NIMES; 375; 375; 0; 48; 87.2; 
2012-05; Sud-Est; PARIS LYON; PERPIGNAN; 160; 160; 0; 14; 91.3; 
2012-05; Sud-Est; SAINT ETIENNE CHATEAUCREUX; PARIS LYON; 115; 115; 0; 11; 90.4; 
2012-05; Sud-Est; TOULON; PARIS LYON; 276; 275; 1; 58; 78.9; D'importantes phases de travaux d'amélioration de l'infrastructure sur le tronçon Nice Marseille nécessitent la mise en place de limitations de vitesse qui réduisent la fluidité des circulations.
2012-05; Atlantique; TOURS; PARIS MONTPARNASSE; 205; 205; 0; 30; 85.4; Outre les limitations de vitesse dues aux travaux effectués près de Connerré (72) ainsi que ceux de la régénération des voies de la Ligne à Grande Vitesse Atlantique à Massy (banlieue sud de Paris), quelques incidents ont nui à la régularité des TGV : problème d'aiguille à Massy, panne d'un train les 15 et 16, problème de matériel TGV à Tours ainsi que le dépannage d'un train à Vendôme avec 9 TGV touchés.
2012-05; Sud-Est; VALENCE ALIXAN TGV; PARIS LYON; 241; 241; 0; 32; 86.7; 
2012-05; Atlantique; VANNES; PARIS MONTPARNASSE; 175; 175; 0; 13; 92.6; 
2012-05; Sud-Est; PARIS LYON; AVIGNON TGV; 377; 377; 0; 32; 91.5; 
2012-05; Sud-Est; PARIS LYON; BELLEGARDE (AIN); 263; 263; 0; 34; 87.1; 
2012-05; Sud-Est; BESANCON FRANCHE COMTE TGV; PARIS LYON; 212; 212; 0; 18; 91.5; 
2012-05; Sud-Est; PARIS LYON; BESANCON FRANCHE COMTE TGV; 236; 236; 0; 22; 90.7; 
2012-06; Sud-Est; AIX EN PROVENCE TGV; PARIS LYON; 431; 430; 1; 81; 81.2; 
2012-06; Sud-Est; PARIS LYON; AIX EN PROVENCE TGV; 417; 417; 0; 48; 88.5; 
2012-06; Nord; DUNKERQUE; PARIS NORD; 111; 111; 0; 5; 95.5; 
2012-06; Nord; PARIS NORD; DUNKERQUE; 158; 158; 0; 8; 94.9; 
2012-06; Sud-Est; LE CREUSOT MONTCEAU MONTCHANIN; PARIS LYON; 210; 210; 0; 29; 86.2; 
2012-06; Nord; LILLE; PARIS NORD; 605; 605; 0; 55; 90.9; 
2012-06; Sud-Est; LYON PART DIEU; MARSEILLE ST CHARLES; 598; 598; 0; 120; 79.9; Fragilité de cette liaison liée à la longueur du parcours des trains assurant cette desserte et une baisse de la régularité en Juin suite à plusieurs journées perturbées par des orages .
2012-06; Sud-Est; MARSEILLE ST CHARLES; LYON PART DIEU; 574; 574; 0; 94; 83.6; 
2012-06; Atlantique; LYON PART DIEU; RENNES; 30; 30; 0; 1; 96.7; 
2012-06; Sud-Est; PARIS LYON; MACON LOCHE; 178; 178; 0; 15; 91.6; 
2012-06; Sud-Est; PARIS LYON; MARSEILLE ST CHARLES; 507; 507; 0; 24; 95.3; 
2012-06; Sud-Est; PARIS LYON; MONTPELLIER; 373; 373; 0; 29; 92.2; 
2012-11; Sud-Est; PARIS LYON; AVIGNON TGV; 382; 382; 0; 33; 91.4; 
2012-11; Atlantique; BORDEAUX ST JEAN; PARIS MONTPARNASSE; 640; 639; 1; 107; 83.3; 
2012-12; Atlantique; BREST; PARIS MONTPARNASSE; 181; 181; 0; 10; 94.5; 
2012-12; Sud-Est; CHAMBERY CHALLES LES EAUX; PARIS LYON; 251; 251; 0; 38; 84.9; 
2012-12; Sud-Est; PARIS LYON; DIJON VILLE; 466; 466; 0; 34; 92.7; 
2012-12; Sud-Est; GRENOBLE; PARIS LYON; 230; 230; 0; 19; 91.7; 
2012-12; Sud-Est; PARIS LYON; GRENOBLE; 250; 250; 0; 24; 90.4; 
2012-12; Atlantique; LA ROCHELLE VILLE; PARIS MONTPARNASSE; 224; 224; 0; 14; 93.8; 
2012-12; Sud-Est; LE CREUSOT MONTCEAU MONTCHANIN; PARIS LYON; 220; 220; 0; 29; 86.8; 
2012-12; Atlantique; LE MANS; PARIS MONTPARNASSE; 458; 458; 0; 58; 87.3; 
2012-12; Nord; LILLE; PARIS NORD; 599; 599; 0; 53; 91.2; 
2012-12; Sud-Est; LYON PART DIEU; MARSEILLE ST CHARLES; 636; 636; 0; 170; 73.3; 
2012-12; Atlantique; NANTES; PARIS MONTPARNASSE; 573; 572; 1; 43; 92.5; 
2012-12; Est; STRASBOURG; NANTES; 50; 50; 0; 4; 92.0; 
2012-12; Sud-Est; NIMES; PARIS LYON; 366; 366; 0; 47; 87.2; 
2012-12; Atlantique; QUIMPER; PARIS MONTPARNASSE; 147; 147; 0; 14; 90.5; 
2012-12; Atlantique; ST MALO; PARIS MONTPARNASSE; 101; 101; 0; 7; 93.1; 
2012-12; Atlantique; PARIS MONTPARNASSE; ST MALO; 55; 55; 0; 1; 98.2; 
2012-12; Atlantique; ST PIERRE DES CORPS; PARIS MONTPARNASSE; 444; 444; 0; 76; 82.9; 
2013-01; Sud-Est; PARIS LYON; AIX EN PROVENCE TGV; 399; 399; 0; 72; 82.0; 
2013-01; Nord; DOUAI; PARIS NORD; 208; 208; 0; 60; 71.2; Liaison touchée principalement par les chutes de neige du 14 au 23 janvier. Durant cette période, pour éviter les dangers liés aux projections de glace, la vitesse des trains sur la ligne à grande vitesse a été limitée à 230 km/h voire 170 km/h. Les installations au sol ainsi que les rames TGV ont également connu des difficultés liées à cet épisode neigeux.
2013-01; Nord; PARIS NORD; DOUAI; 208; 208; 0; 56; 73.1; Liaison touchée principalement par les chutes de neige du 14 au 23 janvier. Durant cette période, pour éviter les dangers liés aux projections de glace, la vitesse des trains sur la ligne à grande vitesse a été limitée à 230 km/h voire 170 km/h. Les installations au sol ainsi que les rames TGV ont également connu des difficultés liées à cet épisode neigeux.
2013-01; Nord; PARIS NORD; DUNKERQUE; 132; 132; 0; 27; 79.5; Liaison touchée principalement par les chutes de neige du 14 au 23 janvier. Durant cette période, pour éviter les dangers liés aux projections de glace, la vitesse des trains sur la ligne à grande vitesse a été limitée à 230 km/h voire 170 km/h. Les installations au sol ainsi que les rames TGV ont également connu des difficultés liées à cet épisode neigeux.
2013-01; Sud-Est; PARIS LYON; GRENOBLE; 260; 260; 0; 41; 84.2; 
2013-01; Sud-Est; PARIS LYON; LE CREUSOT MONTCEAU MONTCHANIN; 208; 208; 0; 53; 74.5; Conditions météorologiques difficiles (en particulier la neige et le froid).
2013-01; Atlantique; RENNES; LYON PART DIEU; 63; 63; 0; 13; 79.4; Forts épisodes neigeux les 15, 18, 19, 20, 21 et 22, nécessitant la baisse de vitesse des trains, notamment sur ligne à grande vitesse Atlantique mais également dans le Sud-Est et sur d'autres journées.
2013-01; Sud-Est; MACON LOCHE; PARIS LYON; 211; 211; 0; 30; 85.8; 
2013-01; Sud-Est; PARIS LYON; MACON LOCHE; 212; 212; 0; 28; 86.8; 
2013-01; Est; PARIS EST; METZ; 314; 314; 0; 68; 78.3; Plusieures journées d'intempéries ont contraint à réduire la vitesse des TGV sur leur parcours.
2013-01; Sud-Est; MONTPELLIER; PARIS LYON; 369; 369; 0; 43; 88.3; 
2013-01; Est; PARIS EST; NANCY; 297; 297; 0; 68; 77.1; Plusieures journées d'intempéries ont contraint à réduire la vitesse des TGV sur leur parcours.
2013-01; Atlantique; PARIS MONTPARNASSE; NANTES; 577; 577; 0; 40; 93.1; 
2013-01; Est; NANTES; STRASBOURG; 44; 44; 0; 4; 90.9; 
2013-01; Est; STRASBOURG; NANTES; 44; 44; 0; 6; 86.4; 
2013-01; Sud-Est; NIMES; PARIS LYON; 369; 369; 0; 61; 83.5; 
2013-01; Sud-Est; ANNECY; PARIS LYON; 146; 146; 0; 21; 85.6; 
2013-01; Sud-Est; PARIS LYON; ANNECY; 177; 177; 0; 19; 89.3; 
2013-01; Est; PARIS EST; REIMS; 216; 216; 0; 40; 81.5; 
2013-01; Atlantique; RENNES; PARIS MONTPARNASSE; 576; 576; 0; 89; 84.5; 
2013-01; Atlantique; PARIS MONTPARNASSE; RENNES; 563; 563; 0; 76; 86.5; 
2013-01; Atlantique; ST PIERRE DES CORPS; PARIS MONTPARNASSE; 431; 431; 0; 94; 78.2; Forts épisodes neigeux les 15, 18, 19, 20, 21 et 22 janvier, nécessitant la baisse de vitesse des trains, notamment sur ligne à grande vitesse Atlantique.
2013-01; Est; STRASBOURG; PARIS EST; 482; 477; 5; 124; 74.0; Plusieures journées d'intempéries ont contraint à réduire la vitesse des TGV sur leur parcours.
2013-01; Atlantique; PARIS MONTPARNASSE; TOURS; 149; 149; 0; 31; 79.2; Forts épisodes neigeux les 15, 18, 19, 20, 21 et 22 janvier, nécessitant la baisse de vitesse des trains, notamment sur ligne à grande vitesse Atlantique.
2013-01; Sud-Est; PARIS LYON; VALENCE ALIXAN TGV; 270; 270; 0; 38; 85.9; 
2013-01; Nord; PARIS NORD; ARRAS; 344; 344; 0; 97; 71.8; Chutes de neige du 14 au 23 janvier. Pour éviter les dangers liés aux projections de glace, la vitesse des trains sur la ligne à grande vitesse a été limitée à 230 km/h voire 170 km/h. Les installations au sol ainsi que les rames TGV ont également connues des difficultés liées à cet épisode neigeux.
2013-01; Atlantique; PARIS MONTPARNASSE; VANNES; 185; 185; 0; 25; 86.5; 
2013-01; Sud-Est; BELLEGARDE (AIN); PARIS LYON; 264; 264; 0; 42; 84.1; 
2013-02; Sud-Est; AIX EN PROVENCE TGV; PARIS LYON; 364; 364; 0; 61; 83.2; 
2013-02; Sud-Est; PARIS LYON; DIJON VILLE; 422; 422; 0; 16; 96.2; 
2013-02; Nord; DOUAI; PARIS NORD; 180; 180; 0; 28; 84.4; Les chutes de neige des 10 et 11 février nécessitent de réduire la vitesse des trains sur la ligne à grande vitesse à 230 km/h voire 170 km/h pour éviter les dangers liés aux projections de glace. Les rames TGV ont également souffert de ces intempéries ainsi que du froid associé avec des pannes de train en cours de circulation. Les installations caténaires ont subi également des avaries. A noter aussi un accident de personne qui a totalement perturbé la circulation sur Paris Nord, des travaux de renouvellement de voie sur Arras et des dérangements informatiques dans les postes d'aiguillages de Douai.
2013-02; Nord; DUNKERQUE; PARIS NORD; 80; 80; 0; 9; 88.8; 
2013-02; Nord; PARIS NORD; DUNKERQUE; 120; 120; 0; 16; 86.7; 
2013-02; Atlantique; LA ROCHELLE VILLE; PARIS MONTPARNASSE; 202; 202; 0; 8; 96.0; 
2013-02; Atlantique; LAVAL; PARIS MONTPARNASSE; 222; 222; 0; 10; 95.5; 
2013-02; Sud-Est; MACON LOCHE; PARIS LYON; 181; 181; 0; 30; 83.4; 
2013-02; Est; PARIS EST; METZ; 284; 284; 0; 41; 85.6; 
2013-02; Atlantique; PARIS MONTPARNASSE; ANGOULEME; 297; 297; 0; 13; 95.6; Quelques événements ayant de fortes répercussions sur la régularité des trains : panne d'un train de travaux le 12 février, accident de personne le 16 février, panne d'un train de l'entreprise Euro Cargo Rail le 19 février.
2013-02; Sud-Est; PARIS LYON; MULHOUSE VILLE; 278; 278; 0; 32; 88.5; 
2013-02; Est; PARIS EST; NANCY; 268; 268; 0; 35; 86.9; 
2013-02; Est; STRASBOURG; NANTES; 40; 40; 0; 5; 87.5; Quelques événements ayant de fortes répercussions sur la régularité des trains ont eu lieu (panne d'un train de travaux le 12, accident de personne le 20, rail cassé à l'entrée de la ligne à grande vitesse le 21, heurt d'une vache le 26).
2013-02; Sud-Est; PERPIGNAN; PARIS LYON; 139; 139; 0; 15; 89.2; 
2013-02; Atlantique; PARIS MONTPARNASSE; POITIERS; 461; 461; 0; 17; 96.3; Quelques événements ayant de fortes répercussions sur la régularité de nos trains ont eu lieu (panne d'un train de travaux le 12. accident de personne 0 Vendôme (41) le 16. rail cassé à l'entrée de la ligne à grande vitesse 25).
2013-02; Atlantique; PARIS MONTPARNASSE; QUIMPER; 131; 131; 0; 5; 96.2; Problèmes sur la matériel roulant le 25 (TGV 8717), problème sur la matériel roulant d'un TER gênant la circulation des TGV 8739 et 8747 le 1er février.
2013-02; Sud-Est; PARIS LYON; ANNECY; 157; 157; 0; 15; 90.4; 
2013-02; Est; PARIS EST; REIMS; 195; 195; 0; 18; 90.8; 
2013-02; Atlantique; RENNES; PARIS MONTPARNASSE; 517; 517; 0; 33; 93.6; 
2013-02; Sud-Est; PARIS LYON; SAINT ETIENNE CHATEAUCREUX; 102; 102; 0; 18; 82.4; 
2013-02; Atlantique; ST MALO; PARIS MONTPARNASSE; 92; 92; 0; 3; 96.7; 
2013-02; Atlantique; PARIS MONTPARNASSE; ST MALO; 48; 48; 0; 1; 97.9; La panne d'un train de travaux à l'entrée de la ligne à grande vitesse atlantique a généré des retards pour de nombreux TGV.
2013-02; Est; PARIS EST; STRASBOURG; 434; 434; 0; 95; 78.1; Suite aux différents épisodes de neige et de froid. la circulation des trains a été impactée par des limitations de vitesse (230 km/h à la place de 320 km/h) et par des dérangements de la signalisation et des aiguillages.
2013-02; Sud-Est; PARIS LYON; TOULON; 192; 192; 0; 32; 83.3; Conditions météorologiques difficiles (en particulier la neige et le froid).
2013-02; Sud-Est; VALENCE ALIXAN TGV; PARIS LYON; 230; 230; 0; 39; 83.0; 
2013-02; Nord; ARRAS; PARIS NORD; 312; 312; 0; 52; 83.3; 
2013-02; Atlantique; BORDEAUX ST JEAN; PARIS MONTPARNASSE; 612; 611; 1; 48; 92.1; 
2013-02; Atlantique; PARIS MONTPARNASSE; BORDEAUX ST JEAN; 582; 582; 0; 43; 92.6; 
2013-03; Sud-Est; PARIS LYON; CHAMBERY CHALLES LES EAUX; 352; 351; 1; 41; 88.3; 
2013-03; Nord; DOUAI; PARIS NORD; 204; 189; 15; 65; 65.6; Les très importantes chutes de neige avec formation de congères survenues à partir du lundi 11 mars dans la soirée jusqu'au vendredi 15 mars ont fortement perturbé la production. Elles ont imposé des limitations de vitesse pour éviter les projections de glace, allant même jusqu’à immobiliser presque complètement le trafic le mardi 12 mars. Les conditions climatiques difficiles de ces derniers mois ont par ailleurs perturbé les différents programmes travaux sur les voies prolongeant ainsi certaines limitations de vitesse au-delà des délais prévus, notamment sur Arras.
2013-03; Sud-Est; GRENOBLE; PARIS LYON; 243; 242; 1; 21; 91.3; La journée du 12 mars a été très perturbée par les intempéries (neige et pluies verglassantes) entrainant de nombreux et importants retards avoisinants les 2 heures.
2013-03; Sud-Est; LE CREUSOT MONTCEAU MONTCHANIN; PARIS LYON; 220; 218; 2; 22; 89.9; 
2013-03; Atlantique; PARIS MONTPARNASSE; LE MANS; 453; 453; 0; 50; 89.0; 
2013-03; Nord; LILLE; PARIS NORD; 613; 594; 19; 139; 76.6; Les très importantes chutes de neige avec formation de congères survenues à partir du lundi 11 mars dans la soirée jusqu'au vendredi 15 mars ont fortement perturbé la production. Elles ont imposé des limitations de vitesse pour éviter les projections de glace, allant même jusqu’à immobiliser presque complètement le trafic le mardi 12 mars. Les conditions climatiques difficiles de ces derniers mois ont par ailleurs perturbé les différents programmes travaux sur les voies prolongeant ainsi certaines limitations de vitesse au-delà des délais prévus, notamment sur Arras.
2013-03; Atlantique; RENNES; LYON PART DIEU; 62; 62; 0; 6; 90.3; 
2013-03; Sud-Est; PARIS LYON; MARSEILLE ST CHARLES; 509; 506; 3; 41; 91.9; 
2013-03; Sud-Est; MONTPELLIER; PARIS LYON; 366; 365; 1; 34; 90.7; La journée du 12 mars a été très perturbée par les intempéries (neige et pluies verglassantes) entrainant de nombreux et importants retards avoisinants les 2 heures.
2013-03; Sud-Est; MULHOUSE VILLE; PARIS LYON; 319; 317; 2; 21; 93.4; La journée du 12 mars a été très perturbée par les intempéries (neige et pluies verglassantes) entrainant de nombreux et importants retards avoisinants les 2 heures.
2013-03; Est; NANCY; PARIS EST; 295; 294; 1; 28; 90.5; Les intempéries (neige) du 12, 13 et 14 mars ont entrainé de nombreux et importants retards.
2013-03; Atlantique; NANTES; PARIS MONTPARNASSE; 583; 582; 1; 35; 94.0; 
2013-03; Sud-Est; PARIS LYON; ANNECY; 171; 170; 1; 16; 90.6; 
2013-03; Sud-Est; SAINT ETIENNE CHATEAUCREUX; PARIS LYON; 94; 94; 0; 10; 89.4; 
2013-03; Atlantique; ST MALO; PARIS MONTPARNASSE; 102; 102; 0; 9; 91.2; 
2013-03; Atlantique; PARIS MONTPARNASSE; TOURS; 152; 152; 0; 18; 88.2; 
2013-03; Atlantique; VANNES; PARIS MONTPARNASSE; 157; 157; 0; 17; 89.2; 
2013-03; Sud-Est; PARIS LYON; BELLEGARDE (AIN); 287; 286; 1; 57; 80.1; 
2013-04; Sud-Est; PARIS LYON; AIX EN PROVENCE TGV; 415; 415; 0; 51; 87.7; 
2013-04; Sud-Est; GRENOBLE; PARIS LYON; 220; 220; 0; 35; 84.1; 
2013-04; Atlantique; PARIS MONTPARNASSE; LA ROCHELLE VILLE; 217; 217; 0; 11; 94.9; Incident matériel sur le TGV 8387 sur la ligne à grande vitesse, accident de personne à Poitiers, acte de malveillance à Poitiers (gilet présent sur le fil d'alimentation électrique), accident de personne à Poitiers. Le 23 avril, le conducteur du TGV 8387 ressent un choc anormal sur son train et applique les procédures de sécurité nécessaire pour poursuivre son parcours.
2013-04; Atlantique; LAVAL; PARIS MONTPARNASSE; 237; 237; 0; 14; 94.1; 
2013-04; Atlantique; LYON PART DIEU; RENNES; 30; 30; 0; 1; 96.7; 
2013-04; Sud-Est; PARIS LYON; MACON LOCHE; 214; 214; 0; 16; 92.5; 
2013-04; Sud-Est; MARSEILLE ST CHARLES; PARIS LYON; 461; 460; 1; 44; 90.4; 
2011-09; Est; STRASBOURG; PARIS EST; 506; 506; 0; 43; 91.5; 
2011-09; Sud-Est; PARIS LYON; TOULON; 222; 222; 0; 25; 88.7; 
2011-09; Atlantique; PARIS MONTPARNASSE; TOURS; 166; 166; 0; 48; 71.1; 
2011-09; Sud-Est; VALENCE ALIXAN TGV; PARIS LYON; 234; 234; 0; 29; 87.6; 
2011-09; Sud-Est; PARIS LYON; VALENCE ALIXAN TGV; 256; 256; 0; 37; 85.5; 
2011-09; Nord; ARRAS; PARIS NORD; 327; 327; 0; 42; 87.2; 
2011-09; Sud-Est; AVIGNON TGV; PARIS LYON; 381; 380; 1; 36; 90.5; 
2011-10; Sud-Est; CHAMBERY CHALLES LES EAUX; PARIS LYON; 173; 173; 0; 18; 89.6; 
2011-10; Sud-Est; PARIS LYON; CHAMBERY CHALLES LES EAUX; 203; 202; 1; 30; 85.1; 
2011-10; Nord; DUNKERQUE; PARIS NORD; 205; 205; 0; 18; 91.2; 
2011-10; Sud-Est; PARIS LYON; GRENOBLE; 135; 135; 0; 10; 92.6; 
2011-10; Atlantique; PARIS MONTPARNASSE; LA ROCHELLE VILLE; 167; 167; 0; 27; 83.8; 
2011-10; Atlantique; PARIS MONTPARNASSE; LAVAL; 245; 245; 0; 34; 86.1; 
2011-10; Sud-Est; PARIS LYON; LE CREUSOT MONTCEAU MONTCHANIN; 207; 207; 0; 28; 86.5; 
2011-10; Nord; LILLE; MARSEILLE ST CHARLES; 166; 165; 1; 42; 74.5; 
2011-10; Nord; MARSEILLE ST CHARLES; LILLE; 200; 200; 0; 59; 70.5; 
2011-10; Sud-Est; LYON PART DIEU; MARSEILLE ST CHARLES; 547; 543; 4; 137; 74.8; 
2011-10; Sud-Est; LYON PART DIEU; PARIS LYON; 626; 625; 1; 37; 94.1; 
2011-10; Atlantique; RENNES; LYON PART DIEU; 85; 85; 0; 11; 87.1; 
2011-10; Atlantique; PARIS MONTPARNASSE; ANGOULEME; 323; 323; 0; 43; 86.7; 
2011-10; Sud-Est; MONTPELLIER; PARIS LYON; 394; 393; 1; 44; 88.8; 
2011-10; Sud-Est; PARIS LYON; MONTPELLIER; 380; 378; 2; 36; 90.5; 
2011-10; Est; NANTES; STRASBOURG; 48; 45; 3; 9; 80.0; 
2011-10; Est; REIMS; PARIS EST; 234; 233; 1; 22; 90.6; 
2011-10; Est; STRASBOURG; PARIS EST; 518; 517; 1; 37; 92.8; 
2011-10; Atlantique; PARIS MONTPARNASSE; TOULOUSE MATABIAU; 118; 118; 0; 45; 61.9; 
2011-10; Atlantique; TOURS; PARIS MONTPARNASSE; 162; 162; 0; 22; 86.4; 
2011-10; Nord; PARIS NORD; ARRAS; 365; 365; 0; 31; 91.5; 
2011-10; Atlantique; VANNES; PARIS MONTPARNASSE; 173; 173; 0; 17; 90.2; 
2011-11; Sud-Est; PARIS LYON; AIX EN PROVENCE TGV; 421; 421; 0; 33; 92.2; 
2011-11; Nord; DUNKERQUE; PARIS NORD; 199; 199; 0; 7; 96.5; 
2011-11; Atlantique; PARIS MONTPARNASSE; LA ROCHELLE VILLE; 174; 173; 1; 19; 89.0; 
2011-11; Atlantique; PARIS MONTPARNASSE; LAVAL; 242; 236; 6; 33; 86.0; 
2011-11; Sud-Est; PARIS LYON; LE CREUSOT MONTCEAU MONTCHANIN; 199; 199; 0; 20; 89.9; 
2011-11; Atlantique; LE MANS; PARIS MONTPARNASSE; 454; 448; 6; 105; 76.6; 
2011-11; Sud-Est; PARIS LYON; LYON PART DIEU; 627; 627; 0; 40; 93.6; 
2011-11; Sud-Est; MACON LOCHE; PARIS LYON; 179; 178; 1; 4; 97.8; 
2011-11; Est; PARIS LYON; MULHOUSE VILLE; 216; 216; 0; 25; 88.4; 
2011-11; Est; PARIS EST; NANCY; 284; 284; 0; 8; 97.2; 
2011-11; Est; REIMS; PARIS EST; 228; 228; 0; 35; 84.6; 
2011-11; Sud-Est; SAINT ETIENNE CHATEAUCREUX; PARIS LYON; 112; 112; 0; 17; 84.8; 
2011-11; Sud-Est; TOULON; PARIS LYON; 254; 252; 2; 27; 89.3; 
2011-11; Atlantique; PARIS MONTPARNASSE; VANNES; 190; 186; 4; 16; 91.4; 
2011-11; Atlantique; PARIS MONTPARNASSE; BORDEAUX ST JEAN; 663; 658; 5; 37; 94.4; 
2011-12; Atlantique; PARIS MONTPARNASSE; BREST; 208; 208; 0; 16; 92.3; 
2011-12; Sud-Est; DIJON VILLE; PARIS LYON; 453; 453; 0; 46; 89.8; 
2011-12; Nord; DUNKERQUE; PARIS NORD; 139; 139; 0; 7; 95.0; 
2011-12; Atlantique; LA ROCHELLE VILLE; PARIS MONTPARNASSE; 204; 204; 0; 19; 90.7; 
2011-12; Atlantique; LAVAL; PARIS MONTPARNASSE; 243; 243; 0; 29; 88.1; 
2011-12; Nord; LILLE; LYON PART DIEU; 323; 323; 0; 36; 88.9; 
2011-12; Nord; PARIS NORD; LILLE; 631; 631; 0; 69; 89.1; 
2011-12; Sud-Est; PARIS LYON; MACON LOCHE; 191; 191; 0; 5; 97.4; 
2011-12; Sud-Est; MARSEILLE ST CHARLES; PARIS LYON; 465; 465; 0; 35; 92.5; 
2011-12; Atlantique; PARIS MONTPARNASSE; ANGOULEME; 338; 338; 0; 22; 93.5; 
2011-12; Sud-Est; MONTPELLIER; PARIS LYON; 336; 336; 0; 30; 91.1; 
2011-12; Sud-Est; PARIS LYON; MONTPELLIER; 323; 323; 0; 24; 92.6; 
2011-12; Atlantique; PARIS MONTPARNASSE; NANTES; 606; 604; 2; 47; 92.2; 
2011-12; Est; STRASBOURG; NANTES; 38; 38; 0; 6; 84.2; 
2011-12; Atlantique; QUIMPER; PARIS MONTPARNASSE; 148; 147; 1; 13; 91.2; 
2011-12; Est; PARIS EST; REIMS; 252; 252; 0; 8; 96.8; 
2011-12; Atlantique; PARIS MONTPARNASSE; RENNES; 569; 569; 0; 46; 91.9; 
2011-12; Atlantique; ST PIERRE DES CORPS; PARIS MONTPARNASSE; 460; 460; 0; 78; 83.0; 
2011-12; Est; PARIS EST; STRASBOURG; 485; 485; 0; 56; 88.5; 
2011-12; Atlantique; VANNES; PARIS MONTPARNASSE; 176; 175; 1; 20; 88.6; 
2011-12; Atlantique; PARIS MONTPARNASSE; VANNES; 191; 191; 0; 13; 93.2; 
2012-01; Sud-Est; PARIS LYON; AIX EN PROVENCE TGV; 406; 406; 0; 40; 90.1; 
2012-01; Atlantique; BREST; PARIS MONTPARNASSE; 180; 180; 0; 10; 94.4; 
2012-01; Sud-Est; CHAMBERY CHALLES LES EAUX; PARIS LYON; 266; 264; 2; 33; 87.5; 
2012-01; Sud-Est; PARIS LYON; DIJON VILLE; 471; 471; 0; 18; 96.2; 
2012-01; Sud-Est; GRENOBLE; PARIS LYON; 248; 247; 1; 18; 92.7; 
2012-01; Atlantique; LAVAL; PARIS MONTPARNASSE; 244; 244; 0; 23; 90.6; 
2012-01; Sud-Est; PARIS LYON; LE CREUSOT MONTCEAU MONTCHANIN; 208; 208; 0; 39; 81.3; 
2012-01; Atlantique; ANGERS SAINT LAUD; PARIS MONTPARNASSE; 492; 489; 3; 33; 93.3; 
2012-01; Nord; LILLE; MARSEILLE ST CHARLES; 157; 157; 0; 22; 86.0; 
2012-01; Sud-Est; LYON PART DIEU; MARSEILLE ST CHARLES; 612; 612; 0; 114; 81.4; 
2012-01; Atlantique; NANTES; PARIS MONTPARNASSE; 598; 595; 3; 31; 94.8; 
2012-01; Est; STRASBOURG; NANTES; 55; 55; 0; 9; 83.6; 
2012-01; Sud-Est; PARIS LYON; NIMES; 357; 357; 0; 39; 89.1; 
2012-01; Sud-Est; PERPIGNAN; PARIS LYON; 152; 152; 0; 13; 91.4; 
2012-01; Atlantique; POITIERS; PARIS MONTPARNASSE; 507; 507; 0; 55; 89.2; 
2012-01; Est; PARIS EST; REIMS; 253; 253; 0; 18; 92.9; 
2012-01; Sud-Est; SAINT ETIENNE CHATEAUCREUX; PARIS LYON; 118; 118; 0; 11; 90.7; 
2012-01; Atlantique; TOURS; PARIS MONTPARNASSE; 212; 212; 0; 28; 86.8; 
2012-01; Nord; PARIS NORD; ARRAS; 377; 377; 0; 26; 93.1; 
2012-01; Atlantique; PARIS MONTPARNASSE; BORDEAUX ST JEAN; 630; 630; 0; 43; 93.2; 
2012-02; Sud-Est; PARIS LYON; AIX EN PROVENCE TGV; 381; 381; 0; 60; 84.3; 
2012-02; Nord; PARIS NORD; DOUAI; 199; 199; 0; 30; 84.9; 
2012-02; Nord; PARIS NORD; DUNKERQUE; 151; 150; 1; 14; 90.7; 
2012-02; Sud-Est; PARIS LYON; GRENOBLE; 255; 255; 0; 32; 87.5; 
2012-02; Atlantique; PARIS MONTPARNASSE; LAVAL; 245; 244; 1; 35; 85.7; 
2012-02; Sud-Est; MONTPELLIER; LYON PART DIEU; 343; 342; 1; 84; 75.4; 
2012-02; Atlantique; ANGOULEME; PARIS MONTPARNASSE; 335; 335; 0; 102; 69.6; 
2012-02; Sud-Est; MONTPELLIER; PARIS LYON; 348; 348; 0; 46; 86.8; 
2012-02; Sud-Est; PARIS LYON; MULHOUSE VILLE; 294; 294; 0; 45; 84.7; 
2012-02; Atlantique; NANTES; PARIS MONTPARNASSE; 560; 559; 1; 67; 88.0; 
2012-02; Sud-Est; PARIS LYON; NICE VILLE; 152; 151; 1; 37; 75.5; 
2012-02; Sud-Est; NIMES; PARIS LYON; 347; 347; 0; 58; 83.3; 
2012-02; Atlantique; POITIERS; PARIS MONTPARNASSE; 473; 472; 1; 140; 70.3; 
2012-02; Atlantique; PARIS MONTPARNASSE; POITIERS; 422; 422; 0; 79; 81.3; 
2012-02; Est; PARIS EST; REIMS; 236; 236; 0; 19; 91.9; 
2014-03; Sud-Est; DIJON VILLE; PARIS LYON; 485; 485; 0; 34; 93.0; 
2014-03; Sud-Est; PARIS LYON; DIJON VILLE; 514; 514; 0; 23; 95.5; 
2014-03; Nord; DUNKERQUE; PARIS NORD; 112; 112; 0; 2; 98.2; 
2014-03; Atlantique; PARIS MONTPARNASSE; LA ROCHELLE VILLE; 223; 223; 0; 12; 94.6; 
2014-03; Nord; LILLE; LYON PART DIEU; 217; 217; 0; 12; 94.5; 
2014-03; Nord; LILLE; MARSEILLE ST CHARLES; 124; 124; 0; 13; 89.5; Défaut d'alimentation électrique à Montanay le 01. Agression d'un agent à Lyon impactant les 01 et 02. Vol de câbles à Lille le 29. Problème matériel à Arsy le 02. Heurt d'un animal à St Georges d' Espéranche le 13 et à Cluny le 21. Dérangement d'installations au Creusot du 16 au 18 et à Mâcon le 24. Accident de personne à Aix le 17. Incendie dans le tunnel de Marseille le 26. Nombreuses limitations de vitesse pour travaux, notamment à Marseille pour la construction de la troisième voie.
2014-03; Sud-Est; MARSEILLE ST CHARLES; LYON PART DIEU; 566; 566; 0; 110; 80.6; 
2014-03; Sud-Est; LYON PART DIEU; MONTPELLIER; 403; 402; 1; 61; 84.8; Des incidents d'origine externe (accidents de personnes, heurts d'animaux, agressions d'agents SNCF) ont provoqué des retards importants sur cette relation en mars.
2014-03; Sud-Est; MARSEILLE ST CHARLES; PARIS LYON; 441; 441; 0; 37; 91.6; 
2014-03; Sud-Est; PARIS LYON; MONTPELLIER; 326; 326; 0; 30; 90.8; 
2014-03; Est; NANCY; PARIS EST; 294; 294; 0; 11; 96.3; 
2014-03; Sud-Est; ANNECY; PARIS LYON; 145; 145; 0; 6; 95.9; 
2014-03; Sud-Est; PARIS LYON; SAINT ETIENNE CHATEAUCREUX; 17; 17; 0; 1; 94.1; 
2014-03; Atlantique; ST MALO; PARIS MONTPARNASSE; 103; 103; 0; 5; 95.1; 
2014-03; Est; STRASBOURG; PARIS EST; 479; 479; 0; 34; 92.9; 
2014-03; Sud-Est; BELLEGARDE (AIN); PARIS LYON; 260; 260; 0; 27; 89.6; 
2014-04; Sud-Est; PARIS LYON; CHAMBERY CHALLES LES EAUX; 224; 224; 0; 40; 82.1; 
2014-04; Sud-Est; PARIS LYON; DIJON VILLE; 498; 498; 0; 23; 95.4; 
2014-04; Atlantique; PARIS MONTPARNASSE; LA ROCHELLE VILLE; 221; 221; 0; 6; 97.3; 
2014-04; Atlantique; PARIS MONTPARNASSE; ANGERS SAINT LAUD; 427; 427; 0; 13; 97.0; 
2014-04; Sud-Est; PARIS LYON; LYON PART DIEU; 616; 616; 0; 39; 93.7; 
2014-04; Sud-Est; PARIS LYON; MACON LOCHE; 197; 197; 0; 13; 93.4; 
2014-04; Atlantique; PARIS MONTPARNASSE; ANGOULEME; 321; 321; 0; 9; 97.2; 
2014-04; Sud-Est; PARIS LYON; MULHOUSE VILLE; 307; 307; 0; 23; 92.5; 
2014-04; Est; PARIS EST; NANCY; 284; 284; 0; 9; 96.8; 
2014-04; Atlantique; NANTES; PARIS MONTPARNASSE; 536; 535; 1; 21; 96.1; 
2014-04; Atlantique; PARIS MONTPARNASSE; POITIERS; 504; 504; 0; 9; 98.2; 
2014-04; Sud-Est; PARIS LYON; ANNECY; 153; 153; 0; 10; 93.5; 
2014-04; Est; PARIS EST; REIMS; 209; 209; 0; 10; 95.2; 
2014-04; Atlantique; ST PIERRE DES CORPS; PARIS MONTPARNASSE; 427; 427; 0; 42; 90.2; 
2014-04; Est; STRASBOURG; PARIS EST; 464; 464; 0; 36; 92.2; 
2014-04; Est; PARIS EST; STRASBOURG; 431; 431; 0; 23; 94.7; 
2014-04; Atlantique; TOULOUSE MATABIAU; PARIS MONTPARNASSE; 106; 106; 0; 9; 91.5; 
2014-04; Atlantique; PARIS MONTPARNASSE; TOURS; 146; 146; 0; 7; 95.2; 
2014-04; Sud-Est; PARIS LYON; VALENCE ALIXAN TGV; 261; 261; 0; 26; 90.0; 
2014-04; Atlantique; PARIS MONTPARNASSE; VANNES; 166; 166; 0; 6; 96.4; 
2014-04; Sud-Est; BELLEGARDE (AIN); PARIS LYON; 246; 246; 0; 30; 87.8; 
2014-05; Sud-Est; PARIS LYON; AIX EN PROVENCE TGV; 450; 450; 0; 33; 92.7; 
2014-05; Sud-Est; GRENOBLE; PARIS LYON; 229; 228; 1; 12; 94.7; 
2014-05; Atlantique; LA ROCHELLE VILLE; PARIS MONTPARNASSE; 223; 223; 0; 11; 95.1; 
2014-05; Nord; MARSEILLE ST CHARLES; LILLE; 128; 128; 0; 26; 79.7; Agression d'un agent à Lyon le 2, dérangement d'installations à Marseille le 5 et à Valence le 16, incident caténaire à Lyon le 6, bagage abandonné à Marseille le 9 et début d'incendie le 30, accident de personne à Upie le 11, à Valence le 21 et à Oignies le 26, heurt d'animal à Allan le 11 et à Bonlieu le 19. Nombreuses limitations de vitesse suite à travaux, notamment à Cesseins, TGV Haute Picardie et Marseille.
2014-05; Sud-Est; LYON PART DIEU; MARSEILLE ST CHARLES; 646; 645; 1; 126; 80.5; 
2014-05; Sud-Est; MARSEILLE ST CHARLES; LYON PART DIEU; 571; 571; 0; 102; 82.1; 
2014-05; Sud-Est; PARIS LYON; LYON PART DIEU; 599; 599; 0; 5; 99.2; 
2014-05; Atlantique; RENNES; LYON PART DIEU; 61; 61; 0; 4; 93.4; 
2014-05; Sud-Est; MARSEILLE ST CHARLES; PARIS LYON; 463; 463; 0; 41; 91.1; 
2014-05; Est; METZ; PARIS EST; 280; 280; 0; 19; 93.2; 
2014-05; Atlantique; ANGOULEME; PARIS MONTPARNASSE; 324; 322; 2; 28; 91.3; 
2014-05; Sud-Est; PARIS LYON; MONTPELLIER; 352; 352; 0; 34; 90.3; 
2014-05; Est; STRASBOURG; NANTES; 57; 57; 0; 7; 87.7; Accident de personne le 2 mai,  extinction le 7 mai du tableau optique du poste d'aiguillage qui télécommande Vendenheim. Le 24 mai, une rame duplex inapte occasionne un retard important.
2014-05; Sud-Est; SAINT ETIENNE CHATEAUCREUX; PARIS LYON; 8; 8; 0; 1; 87.5; 
2014-05; Atlantique; PARIS MONTPARNASSE; ST MALO; 56; 56; 0; 2; 96.4; 
2014-05; Est; STRASBOURG; PARIS EST; 463; 463; 0; 64; 86.2; 
2014-05; Sud-Est; PARIS LYON; VALENCE ALIXAN TGV; 258; 258; 0; 17; 93.4; 
2014-05; Atlantique; VANNES; PARIS MONTPARNASSE; 157; 157; 0; 8; 94.9; 
2014-05; Atlantique; PARIS MONTPARNASSE; VANNES; 172; 172; 0; 9; 94.8; 
2014-05; Sud-Est; BESANCON FRANCHE COMTE TGV; PARIS LYON; 224; 224; 0; 18; 92.0; 
2014-05; Sud-Est; PARIS LYON; BESANCON FRANCHE COMTE TGV; 207; 207; 0; 19; 90.8; 
2014-05; Atlantique; PARIS MONTPARNASSE; BORDEAUX ST JEAN; 569; 569; 0; 41; 92.8; 
2014-06; Sud-Est; DIJON VILLE; PARIS LYON; 434; 393; 41; 57; 85.5; 
2014-06; Sud-Est; PARIS LYON; GRENOBLE; 223; 186; 37; 23; 87.6; 
2014-06; Atlantique; PARIS MONTPARNASSE; LE MANS; 423; 362; 61; 54; 85.1; 
2014-06; Nord; LILLE; LYON PART DIEU; 220; 175; 45; 27; 84.6; 
2014-06; Nord; LYON PART DIEU; LILLE; 248; 200; 48; 66; 67.0; Les mouvements sociaux qui ont eu lieu du 11 au 24 juin ont nettement affecté les résultats de ce mois. Des travaux entre Macon et Lyon (Cesseins) depuis le 1er juin, avec des limitations de vitesse ont engendré des pertes de temps de quelques minutes. Vers la fin du mois, une augmentation des interventions à bord, que ce soit des pompiers pour porter assistance à des voyageurs ou bien des forces de l'ordre pour rétablir la sécurité. Quelques dérangement des installations également : problème caténaire le 6 à Montanay, d'aiguillage les 22 et 25 sur LGV Nord ou d'un dispositif de détection à Sathonay le 30.
2014-06; Nord; PARIS NORD; LILLE; 583; 505; 78; 56; 88.9; 
2014-06; Sud-Est; MARSEILLE ST CHARLES; LYON PART DIEU; 552; 458; 94; 98; 78.6; La circulation des TGV a été fortement perturbée par le mouvement social du 11 au 22 juin.
2014-06; Sud-Est; PARIS LYON; LYON PART DIEU; 570; 516; 54; 27; 94.8; 
2014-06; Atlantique; RENNES; LYON PART DIEU; 60; 41; 19; 4; 90.2; 
2014-06; Atlantique; ANGOULEME; PARIS MONTPARNASSE; 299; 225; 74; 42; 81.3; 
2014-06; Atlantique; PARIS MONTPARNASSE; ANGOULEME; 296; 235; 61; 20; 91.5; 
2014-06; Sud-Est; PARIS LYON; MONTPELLIER; 318; 282; 36; 32; 88.7; 
2014-06; Est; PARIS EST; NANCY; 282; 253; 29; 10; 96.0; 
2014-06; Est; NANTES; STRASBOURG; 54; 41; 13; 4; 90.2; 
2014-06; Sud-Est; PARIS LYON; PERPIGNAN; 145; 126; 19; 16; 87.3; La circulation des TGV a été fortement perturbée par le mouvement social du 11 au 22 juin.
2014-06; Atlantique; POITIERS; PARIS MONTPARNASSE; 481; 400; 81; 52; 87.0; 
2014-06; Est; REIMS; PARIS EST; 200; 184; 16; 9; 95.1; 
2014-06; Sud-Est; PARIS LYON; SAINT ETIENNE CHATEAUCREUX; 1; 1; 0; 0; 100.0; 
2014-06; Atlantique; PARIS MONTPARNASSE; ST MALO; 46; 42; 4; 3; 92.9; 
2014-06; Atlantique; TOURS; PARIS MONTPARNASSE; 195; 162; 33; 31; 80.9; 
2014-06; Nord; PARIS NORD; ARRAS; 322; 271; 51; 23; 91.5; 
2014-06; Sud-Est; PARIS LYON; AVIGNON TGV; 478; 427; 51; 55; 87.1; 
2014-06; Sud-Est; BESANCON FRANCHE COMTE TGV; PARIS LYON; 204; 180; 24; 23; 87.2; 
2014-06; Atlantique; PARIS MONTPARNASSE; BORDEAUX ST JEAN; 575; 487; 88; 61; 87.5; 
2014-07; Sud-Est; AIX EN PROVENCE TGV; PARIS LYON; 431; 431; 0; 53; 87.7; 
2014-07; Sud-Est; PARIS LYON; CHAMBERY CHALLES LES EAUX; 183; 183; 0; 23; 87.4; 
2014-07; Sud-Est; PARIS LYON; DIJON VILLE; 487; 487; 0; 25; 94.9; 
2014-07; Nord; PARIS NORD; DOUAI; 168; 168; 0; 20; 88.1; 
2014-07; Sud-Est; LE CREUSOT MONTCEAU MONTCHANIN; PARIS LYON; 217; 216; 1; 35; 83.8; 
2014-07; Sud-Est; PARIS LYON; LE CREUSOT MONTCEAU MONTCHANIN; 207; 206; 1; 23; 88.8; 
2014-07; Atlantique; PARIS MONTPARNASSE; ANGERS SAINT LAUD; 391; 391; 0; 25; 93.6; 
2014-07; Nord; LILLE; MARSEILLE ST CHARLES; 122; 122; 0; 23; 81.1; 
2014-07; Nord; PARIS NORD; LILLE; 538; 538; 0; 41; 92.4; 
2014-07; Sud-Est; MARSEILLE ST CHARLES; LYON PART DIEU; 571; 571; 0; 130; 77.2; Heurt de personnes le 6, le 13, le 29, installations en dérangement suite à des orages le 6, le 20, le 25.
2014-07; Sud-Est; MONTPELLIER; LYON PART DIEU; 426; 426; 0; 83; 80.5; 
2014-07; Sud-Est; PARIS LYON; LYON PART DIEU; 515; 514; 1; 20; 96.1; 
2014-07; Atlantique; LYON PART DIEU; RENNES; 31; 31; 0; 2; 93.5; 
2014-07; Est; NANCY; PARIS EST; 284; 284; 0; 19; 93.3; 
2014-07; Sud-Est; PARIS LYON; PERPIGNAN; 194; 194; 0; 22; 88.7; 
2014-07; Est; REIMS; PARIS EST; 209; 209; 0; 14; 93.3; 
2014-07; Atlantique; PARIS MONTPARNASSE; RENNES; 536; 536; 0; 45; 91.6; 
2014-12; Sud-Est; PARIS LYON; MONTPELLIER; 320; 318; 2; 28; 91.2; 
2014-12; Sud-Est; PARIS LYON; VALENCE ALIXAN TGV; 273; 271; 2; 18; 93.4; Les intempéries en région Languedoc Roussillon ont perturbé la régularité de cette relation en décembre.
2014-12; Atlantique; PARIS MONTPARNASSE; BORDEAUX ST JEAN; 656; 655; 1; 55; 91.6; 
2014-12; Atlantique; PARIS MONTPARNASSE; TOULOUSE MATABIAU; 149; 149; 0; 20; 86.6; Ce ligne a été affectée par un accident entre un TER \& à un camion sur un passage à niveau vers Montauban le 09/12: des travaux ont été nécessaires jusqu'au 12/12 entraînant des retards \& des perturbations. D'autres événements extérieurs ont affecté la régularité des trains: un accident de personne près de Tours le 14/12 (19 TGV retardés de 11' à 2h40), le heurt d'un sanglier sur la Ligne à Grande Vitesse près de St Arnoult le 22/12 (30 TGV retardés de 11min à 1h18). Enfin, un dérangement des installations entre Paris Montparnasse \& Massy le 25/12 (32 TGV retardés de 25min à 2h38).
2014-12; Nord; PARIS NORD; LILLE; 612; 612; 0; 36; 94.1; 
2014-12; Est; REIMS; PARIS EST; 211; 211; 0; 16; 92.4; 
2014-12; Atlantique; RENNES; LYON PART DIEU; 63; 63; 0; 11; 82.5; Cette ligne a été touché par des incidents externes dont un accident de personne à Laval le 01/12 (20 TGV retardés de 16min à 3h19) \& le heurt d'un sanglier le 22/12 sur la Ligne à Grande Vitesse près de St Arnoult (30 TGV retardés de 11min à 1h18).
Des dérangements des installations ont également entraîné de retards : à Valenton (en région parisienne) notamment les 18, 19 \& 21/12 avec près d'une quinzaine de trains retardés par jour \& sur la ligne à Grande Vitesse Sud Est également le 3/12 (21 TGV). Un incident électrique dans la région lyonnaise le 11/12 retarde 32 TGV \& un rail fissuré à Neau (près de Laval) le 29/12 (18 TGV retardés de 12' à 1h15).
2014-12; Atlantique; ST MALO; PARIS MONTPARNASSE; 103; 103; 0; 11; 89.3; La ligne a été affectée par des incidents externes dont un accident de personne à Laval le 01/12 (20 TGV retardés de 16min à 3h19) \& le heurt d'un sanglier sur la Ligne à Grande Vitesse près de St Arnoult le 22/12 (30 TGV retardés de 11min à 1h18);
Des dérangements des installations ont également entraîné des retards: le 25/12 notamment entre Paris Montparnasse \& Massy (32 TGV de 25' à 2h38) \& un rail fissuré à Neau (près de Laval) le 29/12 (18 TGV de 12' à 1h15). A noter également, la défaillance Matériel d'un train d'une autre compagnie ferroviaire en sortie du Mans le 26/12 (20 TGV retardés de 14' à 2h35).
2014-12; Atlantique; ST PIERRE DES CORPS; PARIS MONTPARNASSE; 465; 464; 1; 75; 83.8; 
2014-12; Sud-Est; VALENCE ALIXAN TGV; PARIS LYON; 261; 260; 1; 43; 83.5; Les intempéries en région Languedoc Roussillon ont perturbé la régularité de cette relation en décembre.
2012-07; Sud-Est; CHAMBERY CHALLES LES EAUX; PARIS LYON; 215; 215; 0; 19; 91.2; 
2012-07; Nord; DUNKERQUE; PARIS NORD; 118; 118; 0; 3; 97.5; 
2012-07; Atlantique; LAVAL; PARIS MONTPARNASSE; 227; 227; 0; 9; 96.0; 
2012-07; Sud-Est; LYON PART DIEU; PARIS LYON; 544; 544; 0; 27; 95.0; 
2012-07; Atlantique; RENNES; LYON PART DIEU; 84; 84; 0; 5; 94.0; 
2012-07; Sud-Est; MACON LOCHE; PARIS LYON; 177; 177; 0; 19; 89.3; 
2012-07; Sud-Est; PARIS LYON; MACON LOCHE; 183; 183; 0; 15; 91.8; 
2012-07; Est; PARIS EST; METZ; 311; 311; 0; 13; 95.8; 
2012-07; Atlantique; PARIS MONTPARNASSE; NANTES; 515; 515; 0; 21; 95.9; 
2012-07; Est; STRASBOURG; NANTES; 51; 51; 0; 3; 94.1; 
2012-07; Sud-Est; NICE VILLE; PARIS LYON; 241; 241; 0; 57; 76.3; D'importantes phases de travaux d'amélioration de l'infrastructure sur le tronçon Nice Marseille nécessitent la mise en place de limitations de vitesse qui réduisent la fluidité des circulations.
2012-07; Sud-Est; NIMES; PARIS LYON; 406; 405; 1; 43; 89.4; 
2012-07; Sud-Est; PERPIGNAN; PARIS LYON; 166; 165; 1; 20; 87.9; 
2012-07; Atlantique; POITIERS; PARIS MONTPARNASSE; 487; 487; 0; 48; 90.1; 
2012-07; Atlantique; PARIS MONTPARNASSE; QUIMPER; 192; 192; 0; 19; 90.1; 
2012-07; Atlantique; ST MALO; PARIS MONTPARNASSE; 96; 96; 0; 4; 95.8; 
2012-07; Atlantique; TOULOUSE MATABIAU; PARIS MONTPARNASSE; 110; 110; 0; 8; 92.7; 
2012-07; Atlantique; TOURS; PARIS MONTPARNASSE; 183; 183; 0; 11; 94.0; 
2012-07; Sud-Est; VALENCE ALIXAN TGV; PARIS LYON; 246; 245; 1; 33; 86.5; 
2012-07; Nord; ARRAS; PARIS NORD; 329; 329; 0; 21; 93.6; 
2012-07; Sud-Est; PARIS LYON; AVIGNON TGV; 419; 419; 0; 42; 90.0; 
2012-07; Sud-Est; PARIS LYON; BESANCON FRANCHE COMTE TGV; 249; 249; 0; 25; 90.0; 
2012-08; Sud-Est; PARIS LYON; AIX EN PROVENCE TGV; 439; 439; 0; 46; 89.5; 
2012-08; Atlantique; BREST; PARIS MONTPARNASSE; 177; 177; 0; 3; 98.3; 
2012-08; Nord; PARIS NORD; DOUAI; 182; 182; 0; 13; 92.9; 
2012-08; Atlantique; LAVAL; PARIS MONTPARNASSE; 229; 229; 0; 10; 95.6; 
2012-08; Nord; LILLE; LYON PART DIEU; 309; 309; 0; 42; 86.4; 
2012-08; Nord; LILLE; MARSEILLE ST CHARLES; 158; 158; 0; 42; 73.4; Les circulations ont été impactées par des dérangements des installations ferroviaires et quelques difficultés liées aux rames TGV. Plusieurs accidents de personnes ont également marqué le mois d'août notamment sur la ligne à grande vitesse du Sud-Est les 16 et 30 août.
2012-08; Sud-Est; LYON PART DIEU; MONTPELLIER; 376; 376; 0; 66; 82.4; 
2012-08; Sud-Est; PARIS LYON; MARSEILLE ST CHARLES; 520; 520; 0; 26; 95.0; 
2012-08; Est; PARIS EST; NANCY; 298; 298; 0; 13; 95.6; La quasi totalité des retards du mois d'août se sont produit le 16 août à cause d'un accident de personne à Vaires.
2012-08; Atlantique; NANTES; PARIS MONTPARNASSE; 559; 558; 1; 36; 93.5; 
2012-08; Sud-Est; PERPIGNAN; PARIS LYON; 166; 166; 0; 21; 87.3; 
2012-08; Sud-Est; ANNECY; PARIS LYON; 205; 205; 0; 7; 96.6; 
2012-08; Atlantique; ST MALO; PARIS MONTPARNASSE; 94; 94; 0; 4; 95.7; 
2012-08; Est; STRASBOURG; PARIS EST; 481; 479; 2; 39; 91.9; La quasi totalité des retards du mois d'août se sont produit le 16 août à cause d'un accident de personne à Vaires.
2012-09; Sud-Est; AIX EN PROVENCE TGV; PARIS LYON; 0; 0; 0; 0; ; 
2012-09; Sud-Est; PARIS LYON; AIX EN PROVENCE TGV; 0; 0; 0; 0; ; 
2012-09; Atlantique; BREST; PARIS MONTPARNASSE; 172; 172; 0; 6; 96.5; 
2012-09; Sud-Est; GRENOBLE; PARIS LYON; 235; 234; 1; 17; 92.7; 
2012-09; Atlantique; PARIS MONTPARNASSE; LE MANS; 434; 434; 0; 35; 91.9; 
2012-09; Sud-Est; PARIS LYON; MONTPELLIER; 338; 338; 0; 24; 92.9; 
2012-09; Est; PARIS EST; NANCY; 281; 281; 0; 6; 97.9; 
2012-09; Atlantique; NANTES; PARIS MONTPARNASSE; 565; 565; 0; 24; 95.8; 
2012-09; Est; STRASBOURG; NANTES; 60; 60; 0; 6; 90.0; 
2012-09; Sud-Est; PARIS LYON; PERPIGNAN; 153; 153; 0; 12; 92.2; 
2012-09; Atlantique; POITIERS; PARIS MONTPARNASSE; 493; 492; 1; 55; 88.8; 
2012-09; Sud-Est; PARIS LYON; TOULON; 199; 199; 0; 19; 90.5; 
2012-09; Atlantique; PARIS MONTPARNASSE; BORDEAUX ST JEAN; 641; 641; 0; 31; 95.2; 
2012-10; Atlantique; PARIS MONTPARNASSE; BREST; 158; 157; 1; 5; 96.8; 
2012-10; Nord; DOUAI; PARIS NORD; 205; 204; 1; 8; 96.1; 
2012-10; Atlantique; PARIS MONTPARNASSE; LAVAL; 246; 243; 3; 10; 95.9; 
2012-10; Sud-Est; LYON PART DIEU; MARSEILLE ST CHARLES; 598; 592; 6; 152; 74.3; Fragilité de cette liaison liée à la longueur du parcours des trains assurant cette desserte.
2012-10; Sud-Est; LYON PART DIEU; PARIS LYON; 648; 643; 5; 29; 95.5; 
2012-10; Est; PARIS EST; METZ; 313; 313; 0; 16; 94.9; 
2012-10; Sud-Est; PARIS LYON; MULHOUSE VILLE; 309; 307; 2; 18; 94.1; 
2012-10; Est; NANCY; PARIS EST; 303; 302; 1; 22; 92.7; 
2012-10; Sud-Est; PARIS LYON; NIMES; 356; 351; 5; 47; 86.6; 
2012-10; Sud-Est; PERPIGNAN; PARIS LYON; 152; 152; 0; 24; 84.2; 
2012-10; Atlantique; POITIERS; PARIS MONTPARNASSE; 514; 507; 7; 92; 81.9; 
2012-10; Sud-Est; ANNECY; PARIS LYON; 205; 203; 2; 12; 94.1; 
2012-10; Sud-Est; PARIS LYON; ANNECY; 208; 206; 2; 16; 92.2; 
2012-10; Sud-Est; SAINT ETIENNE CHATEAUCREUX; PARIS LYON; 114; 114; 0; 11; 90.4; 
2012-10; Est; STRASBOURG; PARIS EST; 486; 486; 0; 122; 74.9; La gare de strasbourg a connu des chantiers sur la voie aussi bien vers Colmar que vers Paris. Ces chantiers on beaucoup impacté les trains notamment ceux venant de Colmar.
2012-10; Atlantique; PARIS MONTPARNASSE; TOULOUSE MATABIAU; 127; 127; 0; 9; 92.9; 
2012-10; Nord; PARIS NORD; ARRAS; 383; 377; 6; 45; 88.1; 
2012-10; Atlantique; VANNES; PARIS MONTPARNASSE; 173; 171; 2; 9; 94.7; 
2012-10; Atlantique; PARIS MONTPARNASSE; VANNES; 187; 186; 1; 10; 94.6; 
2012-10; Sud-Est; PARIS LYON; AVIGNON TGV; 383; 378; 5; 33; 91.3; 
2012-11; Sud-Est; PARIS LYON; CHAMBERY CHALLES LES EAUX; 209; 209; 0; 50; 76.1; Des travaux vers Mâcon ont d'abord retardé de quelques minutes les TGV. Ces retards ont ensuite généré des difficultés de circulation sur la voie unique située avant l'arrivée à Chambéry.
2012-11; Nord; DUNKERQUE; PARIS NORD; 109; 109; 0; 1; 99.1; 
2012-11; Nord; PARIS NORD; DUNKERQUE; 153; 153; 0; 6; 96.1; 
2012-11; Atlantique; LE MANS; PARIS MONTPARNASSE; 477; 477; 0; 159; 66.7; Régularité des trains fortement perturbée par les nombreux travaux de renouvellement des voies en Bretagne ainsi que par la réduction de la vitesse des TGV en amont du Mans suite à l'affaisement de la voie causée par des intempéries.
2012-11; Atlantique; ANGERS SAINT LAUD; PARIS MONTPARNASSE; 471; 471; 0; 115; 75.6; Régularité des trains fortement perturbée par la réduction de la vitesse des TGV en aval d'Angers suite à l'affaisement de la voie causée par des intempéries.
2012-11; Nord; LILLE; MARSEILLE ST CHARLES; 154; 154; 0; 40; 74.0; Relation impactée principalement par des événements externes (vols de câble et accidents de personne) mais aussi par des dérangements des installations au sol et quelques difficultés liées au Matériel.
2012-11; Nord; MARSEILLE ST CHARLES; LILLE; 124; 124; 0; 21; 83.1; Liaison impactée principalement par des événements externes (vols de câble et accidents de personne) mais aussi par des dérangements des installations au sol et quelques difficultés liées au matériel.
2012-11; Nord; PARIS NORD; LILLE; 607; 607; 0; 48; 92.1; 
2012-11; Sud-Est; LYON PART DIEU; PARIS LYON; 620; 620; 0; 38; 93.9; 
2012-11; Sud-Est; PARIS LYON; LYON PART DIEU; 615; 615; 0; 56; 90.9; 
2012-11; Sud-Est; MACON LOCHE; PARIS LYON; 169; 169; 0; 17; 89.9; 
2012-11; Est; PARIS EST; METZ; 302; 302; 0; 11; 96.4; 
2012-11; Atlantique; NANTES; PARIS MONTPARNASSE; 566; 566; 0; 80; 85.9; 
2012-11; Sud-Est; ANNECY; PARIS LYON; 195; 195; 0; 19; 90.3; 
2012-11; Atlantique; PARIS MONTPARNASSE; RENNES; 547; 547; 0; 28; 94.9; 
2012-11; Sud-Est; PARIS LYON; SAINT ETIENNE CHATEAUCREUX; 114; 114; 0; 26; 77.2; Des actes de malveillance et de nombreux heurt d'animaux sauvages sur la ligne à Grande Vitesse Paris Lyon ont perturbé la circulation des trains de cette relation.
2012-11; Atlantique; PARIS MONTPARNASSE; ST MALO; 56; 56; 0; 1; 98.2; 
2012-11; Est; PARIS EST; STRASBOURG; 449; 449; 0; 29; 93.5; 
2012-11; Sud-Est; TOULON; PARIS LYON; 244; 244; 0; 39; 84.0; 
2012-11; Sud-Est; AVIGNON TGV; PARIS LYON; 363; 363; 0; 59; 83.7; 
2012-11; Sud-Est; BESANCON FRANCHE COMTE TGV; PARIS LYON; 210; 210; 0; 14; 93.3; 
2012-06; Sud-Est; NICE VILLE; PARIS LYON; 187; 187; 0; 59; 68.4; D'importantes phases de travaux d'amélioration de l'infrastructure sur le tronçon Nice Marseille nécessitent la mise en place de limitations de vitesse qui réduise la fluidité des circulations.
2012-06; Sud-Est; PARIS LYON; NICE VILLE; 189; 189; 0; 42; 77.8; D'importantes phases de travaux d'amélioration de l'infrastructure sur le tronçon Nice Marseille nécessitent la mise en place de limitations de vitesse qui réduise la fluidité des circulations.
2012-06; Sud-Est; PERPIGNAN; PARIS LYON; 149; 149; 0; 16; 89.3; 
2012-06; Sud-Est; PARIS LYON; ANNECY; 201; 201; 0; 10; 95.0; 
2012-06; Sud-Est; PARIS LYON; TOULON; 209; 209; 0; 22; 89.5; 
2012-06; Sud-Est; VALENCE ALIXAN TGV; PARIS LYON; 235; 235; 0; 28; 88.1; 
2012-06; Sud-Est; PARIS LYON; VALENCE ALIXAN TGV; 262; 262; 0; 19; 92.7; 
2012-06; Sud-Est; PARIS LYON; AVIGNON TGV; 354; 354; 0; 26; 92.7; 
2014-03; Sud-Est; PARIS LYON; AIX EN PROVENCE TGV; 444; 444; 0; 22; 95.0; 
2014-03; Atlantique; PARIS MONTPARNASSE; BREST; 186; 186; 0; 4; 97.8; 
2014-03; Sud-Est; PARIS LYON; CHAMBERY CHALLES LES EAUX; 268; 268; 0; 31; 88.4; 
2014-03; Atlantique; PARIS MONTPARNASSE; LAVAL; 234; 234; 0; 9; 96.2; 
2014-03; Sud-Est; LE CREUSOT MONTCEAU MONTCHANIN; PARIS LYON; 220; 220; 0; 25; 88.6; 
2014-03; Sud-Est; PARIS LYON; LE CREUSOT MONTCEAU MONTCHANIN; 206; 206; 0; 22; 89.3; 
2014-03; Nord; PARIS NORD; LILLE; 625; 624; 1; 29; 95.4; 
2014-03; Sud-Est; LYON PART DIEU; MARSEILLE ST CHARLES; 645; 645; 0; 100; 84.5; 
2014-03; Sud-Est; LYON PART DIEU; PARIS LYON; 629; 629; 0; 25; 96.0; 
2014-03; Sud-Est; MACON LOCHE; PARIS LYON; 205; 205; 0; 10; 95.1; 
2014-03; Sud-Est; MULHOUSE VILLE; PARIS LYON; 320; 320; 0; 24; 92.5; 
2014-03; Atlantique; PARIS MONTPARNASSE; NANTES; 541; 541; 0; 23; 95.7; 
2014-03; Est; NANTES; STRASBOURG; 41; 41; 0; 3; 92.7; 
2014-03; Sud-Est; NICE VILLE; PARIS LYON; 182; 182; 0; 26; 85.7; 
2014-03; Sud-Est; NIMES; PARIS LYON; 336; 336; 0; 45; 86.6; 
2014-03; Atlantique; QUIMPER; PARIS MONTPARNASSE; 150; 150; 0; 8; 94.7; 
2014-03; Atlantique; RENNES; PARIS MONTPARNASSE; 574; 574; 0; 30; 94.8; 
2014-03; Sud-Est; PARIS LYON; TOULON; 186; 186; 0; 14; 92.5; 
2014-03; Atlantique; TOULOUSE MATABIAU; PARIS MONTPARNASSE; 89; 89; 0; 7; 92.1; 
2014-03; Sud-Est; PARIS LYON; AVIGNON TGV; 485; 485; 0; 36; 92.6; 
2014-04; Sud-Est; PARIS LYON; AIX EN PROVENCE TGV; 445; 445; 0; 42; 90.6; 
2014-04; Atlantique; BREST; PARIS MONTPARNASSE; 156; 156; 0; 4; 97.4; 
2014-04; Sud-Est; DIJON VILLE; PARIS LYON; 462; 461; 1; 29; 93.7; 
2014-04; Atlantique; PARIS MONTPARNASSE; LAVAL; 233; 233; 0; 2; 99.1; 
2014-04; Atlantique; ANGERS SAINT LAUD; PARIS MONTPARNASSE; 445; 445; 0; 25; 94.4; 
2014-04; Nord; LYON PART DIEU; LILLE; 243; 243; 0; 43; 82.3; 
2014-04; Sud-Est; LYON PART DIEU; MARSEILLE ST CHARLES; 630; 630; 0; 113; 82.1; 
2014-04; Sud-Est; MARSEILLE ST CHARLES; LYON PART DIEU; 552; 551; 1; 92; 83.3; 
2014-04; Sud-Est; MACON LOCHE; PARIS LYON; 197; 197; 0; 17; 91.4; 
2014-04; Sud-Est; MARSEILLE ST CHARLES; PARIS LYON; 477; 477; 0; 43; 91.0; 
2014-04; Est; METZ; PARIS EST; 290; 290; 0; 21; 92.8; 
2014-04; Atlantique; ANGOULEME; PARIS MONTPARNASSE; 320; 320; 0; 21; 93.4; 
2014-04; Sud-Est; PARIS LYON; MONTPELLIER; 345; 345; 0; 37; 89.3; 
2014-04; Atlantique; PARIS MONTPARNASSE; NANTES; 537; 537; 0; 27; 95.0; 
2014-04; Est; NANTES; STRASBOURG; 55; 55; 0; 2; 96.4; 
2014-04; Sud-Est; PARIS LYON; NICE VILLE; 195; 195; 0; 29; 85.1; 
2014-04; Sud-Est; NIMES; PARIS LYON; 357; 357; 0; 44; 87.7; 
2014-04; Est; REIMS; PARIS EST; 205; 205; 0; 13; 93.7; 
2014-04; Sud-Est; PARIS LYON; SAINT ETIENNE CHATEAUCREUX; 8; 8; 0; 1; 87.5; 
2014-04; Atlantique; PARIS MONTPARNASSE; ST PIERRE DES CORPS; 448; 448; 0; 16; 96.4; 
2014-04; Sud-Est; TOULON; PARIS LYON; 235; 235; 0; 37; 84.3; 
2014-04; Sud-Est; PARIS LYON; TOULON; 193; 193; 0; 32; 83.4; 
2014-04; Nord; ARRAS; PARIS NORD; 329; 329; 0; 36; 89.1; 
2014-05; Atlantique; PARIS MONTPARNASSE; BREST; 216; 216; 0; 5; 97.7; 
2014-05; Sud-Est; PARIS LYON; DIJON VILLE; 492; 492; 0; 20; 95.9; 
2014-05; Sud-Est; PARIS LYON; GRENOBLE; 229; 229; 0; 13; 94.3; 
2014-05; Atlantique; LAVAL; PARIS MONTPARNASSE; 235; 235; 0; 16; 93.2; 
2014-05; Sud-Est; PARIS LYON; MARSEILLE ST CHARLES; 470; 470; 0; 30; 93.6; 
2014-05; Atlantique; PARIS MONTPARNASSE; ANGOULEME; 330; 330; 0; 26; 92.1; 
2014-05; Est; NANCY; PARIS EST; 287; 287; 0; 18; 93.7; 
2014-05; Est; PARIS EST; NANCY; 273; 273; 0; 15; 94.5; 
2014-05; Atlantique; PARIS MONTPARNASSE; NANTES; 571; 571; 0; 12; 97.9; 
2014-05; Sud-Est; PARIS LYON; NICE VILLE; 204; 204; 0; 25; 87.7; Cinq heurts d'animaux sauvages et deux accidents de personnes sur lignes à grande vitesse ont pénalisé la régularité de cette liaison.
2014-05; Atlantique; POITIERS; PARIS MONTPARNASSE; 499; 497; 2; 32; 93.6; 
2014-05; Sud-Est; PARIS LYON; ANNECY; 169; 169; 0; 7; 95.9; 
2014-05; Atlantique; ST PIERRE DES CORPS; PARIS MONTPARNASSE; 434; 433; 1; 47; 89.1; 
2014-05; Sud-Est; TOULON; PARIS LYON; 249; 249; 0; 46; 81.5; Cinq heurts d'animaux sauvages et deux accidents de personnes sur lignes à grande vitesse ont pénalisé la régularité de cette liaison.
2014-05; Nord; PARIS NORD; ARRAS; 333; 333; 0; 40; 88.0; 
2014-05; Sud-Est; PARIS LYON; BELLEGARDE (AIN); 206; 206; 0; 16; 92.2; 
2014-06; Sud-Est; AIX EN PROVENCE TGV; PARIS LYON; 352; 352; 0; 63; 82.1; 
2014-06; Sud-Est; PARIS LYON; DIJON VILLE; 469; 425; 44; 24; 94.4; 
2014-06; Nord; DOUAI; PARIS NORD; 200; 173; 27; 34; 80.3; 
2014-06; Atlantique; PARIS MONTPARNASSE; LAVAL; 231; 189; 42; 11; 94.2; 
2014-06; Atlantique; LE MANS; PARIS MONTPARNASSE; 443; 406; 37; 64; 84.2; 
2014-06; Atlantique; ANGERS SAINT LAUD; PARIS MONTPARNASSE; 447; 386; 61; 40; 89.6; 
2014-06; Nord; LILLE; MARSEILLE ST CHARLES; 116; 101; 15; 28; 72.3; Les mouvements sociaux qui ont eu lieu du 11 au 24 juin ont nettement affecté les résultats de ce mois. Des travaux entre Macon et Lyon (Cesseins) depuis le 1er juin, avec des limitations de vitesse ont engendré des pertes de temps de quelques minutes. Vers la fin du mois, une augmentation des interventions à bord, que ce soit des pompiers pour porter assistance à des voyageurs ou bien des forces de l'ordre pour rétablir la sécurité. Quelques dérangement des installations également : problème caténaire le 6 à Montanay, d'aiguillage les 22 et 25 sur LGV Nord ou d'un dispositif de détection à Sathonay le 30.
2014-06; Sud-Est; LYON PART DIEU; MONTPELLIER; 357; 283; 74; 75; 73.5; La circulation des TGV a été fortement perturbée par le mouvement social du 11 au 22 juin.
2014-06; Est; NANCY; PARIS EST; 280; 238; 42; 12; 95.0; 
2014-06; Atlantique; NANTES; PARIS MONTPARNASSE; 575; 489; 86; 40; 91.8; 
2014-06; Atlantique; PARIS MONTPARNASSE; NANTES; 555; 470; 85; 36; 92.3; 
2014-06; Est; STRASBOURG; NANTES; 54; 42; 12; 4; 90.5; 
2014-06; Sud-Est; PARIS LYON; NICE VILLE; 185; 161; 24; 44; 72.7; La circulation des TGV a été fortement perturbée par le mouvement social du 11 au 22 juin.
2014-06; Sud-Est; NIMES; PARIS LYON; 327; 294; 33; 73; 75.2; La circulation des TGV a été fortement perturbée par le mouvement social du 11 au 22 juin.
2014-06; Atlantique; PARIS MONTPARNASSE; QUIMPER; 134; 117; 17; 7; 94.0; 
2014-06; Sud-Est; PARIS LYON; ANNECY; 148; 131; 17; 10; 92.4; 
2014-06; Sud-Est; SAINT ETIENNE CHATEAUCREUX; PARIS LYON; 3; 3; 0; 0; 100.0; 
2014-06; Atlantique; ST PIERRE DES CORPS; PARIS MONTPARNASSE; 417; 322; 95; 73; 77.3; Le 9, de violents orages en région parisienne ont entraîné de multiples dérangements d'installations sur la Ligne à Grande Vitesse. Le 30, un dérangement des installations à St-Pierre-des Corps retarde cinq TGV. A ces incidents s'ajoutent les retards liés à la période de grève nationale de juin.
2014-06; Est; PARIS EST; STRASBOURG; 426; 395; 31; 38; 90.4; 
2014-06; Atlantique; PARIS MONTPARNASSE; TOULOUSE MATABIAU; 122; 113; 9; 36; 68.1; Le 22, la chute d'un véhicule sur les voies près d'Agen retarde 3 TGV. Le 27, la panne d'un train de marchandises et un dérangement d'installation près de Montauban retardent 2 TGV. Le 29, de violents orages sur la région de Bordeaux ont entraîné une fuite de gaz et de multiples dérangements. Le 30, un dérangement des installations à St-Pierre-des Corps retarde 1 TGV. A ces incidents s'ajoutent les retards liés à la période de grève nationale de juin.
2014-06; Atlantique; PARIS MONTPARNASSE; TOURS; 140; 113; 27; 22; 80.5; 
2014-06; Atlantique; VANNES; PARIS MONTPARNASSE; 142; 115; 27; 16; 86.1; Le 9, des travaux rendus tardivement près de Rennes retardent 14 TGV de 1h24 à 2h13 et de violents orages en région parisienne ont entraîné de multiples dérangements d'installations sur la Ligne à Grande Vitesse. Le 25, un incendie dans une maison à proximité des voies près du Mans retarde 3 TGV. A ces incidents s'ajoutent les retards liés à la période de grève nationale de juin.
2014-06; Sud-Est; PARIS LYON; BESANCON FRANCHE COMTE TGV; 197; 172; 25; 23; 86.6; La circulation des TGV a été fortement perturbée par le mouvement social du 11 au 22 juin.
2014-07; Sud-Est; PARIS LYON; AIX EN PROVENCE TGV; 464; 464; 0; 42; 90.9; 
2014-07; Atlantique; PARIS MONTPARNASSE; LAVAL; 224; 224; 0; 21; 90.6; 
2014-07; Atlantique; LE MANS; PARIS MONTPARNASSE; 456; 456; 0; 77; 83.1; 
2014-07; Nord; LYON PART DIEU; LILLE; 249; 249; 0; 77; 69.1; Accident de personne à Nimes le 18, découverte d'un corps à Lyon le 2 et présence d'une personne suicidaire à Montanay le 16, heurt d'un animal au Creusot le 23, déraillement d'un wagon de fret à Lyon le 18 impactant également le 19, intempéries de type orage à Piolenc le 20 et à Avignon le 25, quelques dérangements d'installations notament à Valence le 3, Macon le 6, Vianges le 8, Lille le 11 et à Vemars le 12, incident matériel le 19 à Hattencourt, plusieurs limitations de vitesse suite à travaux essentiellement à Cesseins, Montanay et Marseille.
2014-07; Nord; LILLE; PARIS NORD; 564; 564; 0; 56; 90.1; 
2014-07; Sud-Est; LYON PART DIEU; MARSEILLE ST CHARLES; 628; 626; 2; 147; 76.5; Heurt de personnes le 6, le 13, le 29, installations en dérangement suite à des orages le 6, le 20, le 25.
2014-07; Sud-Est; LYON PART DIEU; MONTPELLIER; 402; 402; 0; 90; 77.6; Heurt de personnes le 18, le 30, installations en dérangement suite à des orages le 6, le 20, le 25, heurt d'animaux le 23, acte de malveillance le 21.
2014-07; Sud-Est; LYON PART DIEU; PARIS LYON; 531; 530; 1; 28; 94.7; 
2014-07; Sud-Est; MARSEILLE ST CHARLES; PARIS LYON; 455; 455; 0; 28; 93.8; 
2014-07; Atlantique; PARIS MONTPARNASSE; ANGOULEME; 329; 329; 0; 21; 93.6; 
2013-04; Est; METZ; PARIS EST; 292; 292; 0; 25; 91.4; 
2013-04; Sud-Est; PARIS LYON; MONTPELLIER; 342; 342; 0; 37; 89.2; Plusieurs incidents techniques et des actes de malveillance ont eu lieu les 17, 19, 21, 25 et 26 Avril sur l'Axe Paris-Lyon entrainant de nombreux retards.
2013-04; Est; PARIS EST; NANCY; 286; 286; 0; 14; 95.1; Deux gros accidents de personnes et détresse d'un train de fret.
2013-04; Atlantique; PARIS MONTPARNASSE; NANTES; 552; 552; 0; 24; 95.7; 
2013-04; Est; NANTES; STRASBOURG; 59; 59; 0; 3; 94.9; 
2013-04; Sud-Est; NICE VILLE; PARIS LYON; 216; 216; 0; 31; 85.6; 
2013-04; Sud-Est; PARIS LYON; NICE VILLE; 197; 197; 0; 38; 80.7; Plusieurs incidents techniques et des actes de malveillance ont eu lieu les 17, 19, 21, 25 et 26 Avril sur l'Axe Paris-Lyon entrainant de nombreux retards.
2013-04; Sud-Est; ANNECY; PARIS LYON; 115; 115; 0; 8; 93.0; 
2013-04; Sud-Est; PARIS LYON; ANNECY; 150; 150; 0; 15; 90.0; 
2013-04; Est; PARIS EST; REIMS; 210; 210; 0; 6; 97.1; 
2013-04; Sud-Est; TOULON; PARIS LYON; 252; 252; 0; 31; 87.7; 
2013-04; Nord; ARRAS; PARIS NORD; 331; 329; 2; 42; 87.2; 
2014-10; Sud-Est; PARIS LYON; LE CREUSOT MONTCEAU MONTCHANIN; 209; 209; 0; 35; 83.3; 
2014-10; Sud-Est; PARIS LYON; MACON LOCHE; 195; 195; 0; 12; 93.8; 
2014-10; Sud-Est; PARIS LYON; NIMES; 328; 328; 0; 39; 88.1; 
2014-10; Atlantique; PARIS MONTPARNASSE; ANGERS SAINT LAUD; 444; 444; 0; 36; 91.9; 
2014-10; Atlantique; PARIS MONTPARNASSE; NANTES; 573; 573; 0; 40; 93.0; 
2014-10; Atlantique; PARIS MONTPARNASSE; ST MALO; 58; 58; 0; 2; 96.6; 
2014-10; Atlantique; PARIS MONTPARNASSE; TOURS; 154; 154; 0; 28; 81.8; 
2014-10; Est; REIMS; PARIS EST; 213; 213; 0; 8; 96.2; 
2014-10; Sud-Est; SAINT ETIENNE CHATEAUCREUX; PARIS LYON; ; ; 0; 0; ; 
2014-10; Atlantique; ST PIERRE DES CORPS; PARIS MONTPARNASSE; 442; 442; 0; 105; 76.2; Un chantier long \& complexe de maintenance au Nord de Tours (entrée Sud de la LGV) a généré de nombreux ralentissements pour les trains de l'OD entraînant d'importants retards. A cela s'ajoute 6 restitutions tardive de travaux en début de matinée.
2014-11; Atlantique; ANGERS SAINT LAUD; PARIS MONTPARNASSE; 448; 448; 0; 21; 95.3; 
2014-11; Sud-Est; BELLEGARDE (AIN); PARIS LYON; 232; 232; 0; 31; 86.6; 
2014-11; Sud-Est; CHAMBERY CHALLES LES EAUX; PARIS LYON; 199; 199; 0; 27; 86.4; 
2014-11; Sud-Est; DIJON VILLE; PARIS LYON; 457; 456; 1; 45; 90.1; 
2014-11; Nord; DUNKERQUE; PARIS NORD; 102; 101; 1; 4; 96.0; 
2014-11; Nord; LYON PART DIEU; LILLE; 237; 237; 0; 47; 80.2; 
2014-11; Sud-Est; LYON PART DIEU; PARIS LYON; 608; 605; 3; 16; 97.4; 
2014-11; Atlantique; LYON PART DIEU; RENNES; 30; 30; 0; 2; 93.3; 
2014-11; Sud-Est; MACON LOCHE; PARIS LYON; 193; 192; 1; 21; 89.1; 
2014-11; Sud-Est; MARSEILLE ST CHARLES; LYON PART DIEU; 550; 548; 2; 155; 71.7; La régularité des TGV de cette relation a été perturbée en novembre par 2 accidents de personne, 3 heurts d'animaux et par les intempéries en région PACA.
2014-11; Sud-Est; MARSEILLE ST CHARLES; PARIS LYON; 441; 439; 2; 34; 92.3; 
2014-11; Est; NANCY; PARIS EST; 280; 280; 0; 9; 96.8; 
2014-11; Est; NANTES; STRASBOURG; 44; 44; 0; 5; 88.6; 
2014-11; Est; PARIS EST; NANCY; 281; 281; 0; 10; 96.4; 
2014-11; Sud-Est; PARIS LYON; CHAMBERY CHALLES LES EAUX; 183; 183; 0; 24; 86.9; 
2014-11; Sud-Est; PARIS LYON; DIJON VILLE; 492; 492; 0; 27; 94.5; 
2014-11; Sud-Est; PARIS LYON; LYON PART DIEU; 597; 596; 1; 26; 95.6; 
2014-11; Atlantique; PARIS MONTPARNASSE; ANGOULEME; 315; 315; 0; 8; 97.5; 
2014-11; Atlantique; PARIS MONTPARNASSE; POITIERS; 489; 489; 0; 17; 96.5; 
2014-11; Est; STRASBOURG; PARIS EST; 454; 454; 0; 36; 92.1; 
2014-11; Sud-Est; TOULON; PARIS LYON; 217; 215; 2; 53; 75.3; La régularité des TGV de cette relation a été perturbée en novembre par 2 accidents de personne, 3 heurts d'animaux et par les intempéries en région PACA.
2014-11; Atlantique; TOURS; PARIS MONTPARNASSE; 200; 200; 0; 38; 81.0; 
2014-12; Sud-Est; BESANCON FRANCHE COMTE TGV; PARIS LYON; 227; 227; 0; 11; 95.2; 
2014-12; Nord; DOUAI; PARIS NORD; 198; 198; 0; 33; 83.3; 
2014-12; Nord; DUNKERQUE; PARIS NORD; 116; 116; 0; 4; 96.6; 
2014-12; Atlantique; LA ROCHELLE VILLE; PARIS MONTPARNASSE; 223; 223; 0; 9; 96.0; 
2014-12; Sud-Est; LE CREUSOT MONTCEAU MONTCHANIN; PARIS LYON; 220; 220; 0; 28; 87.3; 
2014-12; Sud-Est; MARSEILLE ST CHARLES; LYON PART DIEU; 557; 557; 0; 121; 78.3; La régularité de cette relation a été perturbée en décembre par un heurt d'animal et 3 pannes de TGV sur la ligne à grande vitesse.
2014-12; Sud-Est; MONTPELLIER; PARIS LYON; 326; 324; 2; 49; 84.9; Les intempéries en région Languedoc Roussillon ont perturbé la régularité de cette relation en décembre.
2014-12; Est; PARIS EST; STRASBOURG; 441; 441; 0; 32; 92.7; 
2014-12; Sud-Est; PARIS LYON; ANNECY; 160; 160; 0; 9; 94.4; 
2014-12; Sud-Est; PARIS LYON; BELLEGARDE (AIN); 252; 252; 0; 27; 89.3; 
2014-12; Sud-Est; PARIS LYON; BESANCON FRANCHE COMTE TGV; 219; 219; 0; 9; 95.9; 
2014-12; Sud-Est; PARIS LYON; NICE VILLE; 185; 185; 0; 17; 90.8; 
2014-12; Sud-Est; PARIS LYON; NIMES; 321; 319; 2; 24; 92.5; 
2014-12; Atlantique; PARIS MONTPARNASSE; ANGERS SAINT LAUD; 449; 448; 1; 19; 95.8; 
2014-12; Atlantique; PARIS MONTPARNASSE; ANGOULEME; 335; 334; 1; 21; 93.7; 
2014-12; Atlantique; PARIS MONTPARNASSE; LAVAL; 240; 240; 0; 20; 91.7; 
2014-12; Atlantique; PARIS MONTPARNASSE; RENNES; 567; 567; 0; 62; 89.1; La ligne a été affectée par des incidents externes dont un accident de personne à Laval le 01/12 (20 TGV retardés de 16min à 3h19) \& le heurt d'un sanglier sur la Ligne à Grande Vitesse près de St Arnoult le 22/12 (30 TGV retardés de 11min à 1h18);
Des dérangements des installations ont également entraîné des retards: le 25/12 notamment entre Paris Montparnasse \& Massy (32 TGV de 25' à 2h38) \& un rail fissuré à Neau (près de Laval) le 29/12 (18 TGV de 12' à 1h15). A noter également, la défaillance Matériel d'un train d'une autre compagnie ferroviaire en sortie du Mans le 26/12 (20 TGV retardés de 14' à 2h35). 
2014-12; Nord; PARIS NORD; DUNKERQUE; 115; 115; 0; 7; 93.9; 
2014-12; Nord; PARIS NORD; ARRAS; 321; 321; 0; 19; 94.1; 
2014-12; Sud-Est; PERPIGNAN; PARIS LYON; 158; 158; 0; 33; 79.1; Les intempéries en région Languedoc Roussillon ont perturbé la régularité de cette relation en décembre.
2014-12; Atlantique; POITIERS; PARIS MONTPARNASSE; 507; 506; 1; 34; 93.3; 
2014-12; Atlantique; RENNES; PARIS MONTPARNASSE; 571; 571; 0; 52; 90.9; 
2014-12; Atlantique; VANNES; PARIS MONTPARNASSE; 172; 172; 0; 19; 89.0; La ligne a été affectée par des incidents externes dont un accident de personne à Laval le 01/12 (20 TGV retardés de 16min à 3h19) \& le heurt d'un sanglier sur la Ligne à Grande Vitesse près de St Arnoult le 22/12 (30 TGV retardés de 11min à 1h18);
Des dérangements des installations ont également entraîné des retards: le 25/12 notamment entre Paris Montparnasse \& Massy (32 TGV de 25' à 2h38) \& un rail fissuré à Neau (près de Laval) le 29/12 (18 TGV de 12' à 1h15). A noter également, la défaillance Matériel d'un train d'une autre compagnie ferroviaire en sortie du Mans le 26/12 (20 TGV retardés de 14' à 2h35).
2015-01; Atlantique; ANGOULEME; PARIS MONTPARNASSE; 335; 334; 1; 36; 89.2; La ligne a été impactée par des incidents externes dont un accident de personne à VIVONNE le 04/01 (3 TGV retardés de 1h34 à 3h40) ainsi que la divagation d’un veau le 07/01 entre ANGOULÊME et POITIERS (5 TGV de 24’ à 1h04). Un dérangement des installations au nord de CHATELLERAULT le 15/01 a également entraîné des retards (24 TGV de 12’ à 1h50) ainsi que la panne d’un TGV sur la Ligne Grande Vitesse le 23/01 à DANGEAU (41 TGV retardés de 11’ à 3h18) et la panne d’un TER le 28/01 au sud de POITIERS (10 TGV retardés de 18’ à 3h23).
2015-01; Sud-Est; ANNECY; PARIS LYON; 192; 191; 1; 23; 88.0; 
2015-01; Nord; ARRAS; PARIS NORD; 322; 322; 0; 51; 84.2; 
2015-01; Sud-Est; BESANCON FRANCHE COMTE TGV; PARIS LYON; 208; 206; 2; 13; 93.7; 
2015-01; Sud-Est; GRENOBLE; PARIS LYON; 228; 223; 5; 16; 92.8; 
2015-01; Nord; LILLE; MARSEILLE ST CHARLES; 196; 196; 0; 39; 80.1; Les principaux évènements survenus au mois de janvier sont :
Chutes de neige dans le sud-est le 20 
Accident de personne : Cesseins le 01 et Vennissieux le 03
Heurts d'animaux  : Chevry le 05, Neufmoutiers le 06, Cluny le 18 et Moussy et Lapalud le 29
Plusieurs colis suspects : à Marseille le 08, Marne la Vallée le 08, Roissy les 08,10 et 11 et à Lyon les 09 et 15
Un défaut d'alimentation à Fresnoy le 06
Obstacle dans la voie à Macon le 13
Des travaux importants sur la LGV Nord entrainant des limitations de vitesse
2015-01; Nord; LILLE; PARIS NORD; 574; 573; 1; 86; 85.0; 
2015-01; Sud-Est; LYON PART DIEU; MONTPELLIER; 348; 346; 2; 57; 83.5; 
2015-01; Atlantique; LYON PART DIEU; RENNES; 82; 82; 0; 16; 80.5; 
2015-01; Sud-Est; MONTPELLIER; LYON PART DIEU; 317; 315; 2; 44; 86.0; 
2015-01; Sud-Est; PARIS LYON; CHAMBERY CHALLES LES EAUX; 242; 242; 0; 22; 90.9; 
2015-01; Sud-Est; PARIS LYON; DIJON VILLE; 445; 445; 0; 26; 94.2; 
2015-01; Sud-Est; PARIS LYON; MONTPELLIER; 320; 319; 1; 35; 89.0; 
2015-01; Sud-Est; PARIS LYON; NICE VILLE; 178; 177; 1; 26; 85.3; 5 collisions avec des animaux sauvages sur la ligne à grande vitesse ont perturbé la régularité de cette relation en janvier 2015
2015-01; Sud-Est; PARIS LYON; PERPIGNAN; 157; 157; 0; 18; 88.5; 5 collisions avec des animaux sauvages sur la ligne à grande vitesse ont perturbé la régularité de cette relation en janvier 2015
2015-01; Sud-Est; PARIS LYON; VALENCE ALIXAN TGV; 288; 287; 1; 26; 90.9; 
2015-01; Atlantique; PARIS MONTPARNASSE; LAVAL; 217; 217; 0; 6; 97.2; 
2015-01; Atlantique; PARIS MONTPARNASSE; RENNES; 508; 508; 0; 34; 93.3; 
2015-01; Nord; PARIS NORD; DOUAI; 184; 184; 0; 21; 88.6; 
2015-01; Atlantique; POITIERS; PARIS MONTPARNASSE; 472; 471; 1; 35; 92.6; La ligne a été impactée par des incidents externes dont un accident de personne au sud de POITIERS le 04/01 (3 TGV retardés de 1h34 à 3h40) ainsi que la divagation d’un veau le 07/01 entre ANGOULÊME et POITIERS (5 TGV de 24’ à 1h04). Un dérangement des installations au nord de CHATELLERAULT le 15/01 a également entraîné des retards (24 TGV de 12’ à 1h50) ainsi que la panne d’un TGV sur la Ligne Grande Vitesse le 23/01 (41 TGV retardés de 11’ à 3h18) et la panne d’un TER le 28/01 au sud de POITIERS (10 TGV retardés de 18’ à 3h23).
2015-02; Sud-Est; BESANCON FRANCHE COMTE TGV; PARIS LYON; 206; 205; 1; 7; 96.6; 
2015-02; Sud-Est; CHAMBERY CHALLES LES EAUX; PARIS LYON; 219; 218; 1; 37; 83.0; 
2015-02; Sud-Est; GRENOBLE; PARIS LYON; 231; 230; 1; 20; 91.3; La panne d'un TGV entre Lyon et Grenoble le 20 février a retardé 5 TGV de plus de 30 min
2015-02; Nord; LYON PART DIEU; LILLE; 249; 249; 0; 51; 79.5; Les principaux évènements survenus au mois de janvier sont :
Chutes de neige dans le sud-est les 04 et 21
Accident de personne à Miribel le 02 et Avignon le 10
Heurts d'animaux  : Sangliers à Macon et Combles le 04
Plusieurs colis suspects à Roissy les 06 et 23
Des défauts d'alimentations : Grenay le 03 et Lyon le 26
Obstacle sur la voie à Toussieu le 14
Aggression d'un contrôleur occasionnant des droits de retrait les 07 et 08
Des travaux importants sur la LGV Nord entrainant des limitations de vitesse
2015-02; Sud-Est; LYON PART DIEU; MARSEILLE ST CHARLES; 551; 550; 1; 121; 78.0; 
2015-02; Nord; MARSEILLE ST CHARLES; LILLE; 194; 194; 0; 38; 80.4; 
2015-02; Sud-Est; MARSEILLE ST CHARLES; PARIS LYON; 391; 391; 0; 22; 94.4; 
2015-02; Est; NANCY; PARIS EST; 264; 261; 3; 17; 93.5; 
2015-02; Sud-Est; NICE VILLE; PARIS LYON; 169; 168; 1; 26; 84.5; 
2015-02; Est; PARIS EST; METZ; 276; 271; 5; 23; 91.5; 
2015-02; Est; PARIS EST; NANCY; 265; 262; 3; 16; 93.9; 
2015-02; Sud-Est; PARIS LYON; BELLEGARDE (AIN); 223; 223; 0; 47; 78.9; De fortes chutes de neige dans le Jura et les Alpes ont perturbé la circulation des TGV Paris-Genève en février
2015-02; Sud-Est; PARIS LYON; GRENOBLE; 242; 242; 0; 17; 93.0; La panne d'un TGV entre Lyon et Grenoble le 20 février a retardé 5 TGV de plus de 30 min
2012-02; Atlantique; PARIS MONTPARNASSE; ST MALO; 53; 53; 0; 2; 96.2; 
2012-02; Atlantique; PARIS MONTPARNASSE; VANNES; 170; 170; 0; 24; 85.9; 
2012-02; Sud-Est; PARIS LYON; BELLEGARDE (AIN); 271; 271; 0; 64; 76.4; 
2012-03; Sud-Est; AIX EN PROVENCE TGV; PARIS LYON; 418; 418; 0; 53; 87.3; 
2012-03; Sud-Est; PARIS LYON; DIJON VILLE; 465; 465; 0; 23; 95.1; 
2012-03; Atlantique; LA ROCHELLE VILLE; PARIS MONTPARNASSE; 203; 202; 1; 20; 90.1; "Les ralentissements pour travaux de renouvellement des voies au nord de Bordeaux retardent les TGV en provenance de Bordeaux
2012-03; Sud-Est; PARIS LYON; LE CREUSOT MONTCEAU MONTCHANIN; 207; 207; 0; 34; 83.6; 
2012-03; Sud-Est; MARSEILLE ST CHARLES; LYON PART DIEU; 596; 596; 0; 89; 85.1; 
2012-03; Sud-Est; PARIS LYON; MACON LOCHE; 192; 192; 0; 5; 97.4; 
2012-03; Sud-Est; MONTPELLIER; PARIS LYON; 361; 361; 0; 24; 93.4; 
2012-03; Est; NANCY; PARIS EST; 298; 297; 1; 13; 95.6; Incendie d'un entrepôt à hauteur de Gagny qui a généré une coupure caténaire les 10 et 11 mars 2012.
2012-03; Est; STRASBOURG; NANTES; 61; 61; 0; 7; 88.5; Incendie d'un entrepôt à hauteur de Gagny qui a généré une coupure caténaire les 10 et 11 mars 2012.
2012-03; Sud-Est; PERPIGNAN; PARIS LYON; 146; 146; 0; 11; 92.5; 
2012-03; Sud-Est; ANNECY; PARIS LYON; 209; 209; 0; 19; 90.9; 
2012-03; Sud-Est; PARIS LYON; ANNECY; 208; 208; 0; 9; 95.7; 
2012-03; Atlantique; PARIS MONTPARNASSE; RENNES; 571; 571; 0; 31; 94.6; 
2012-03; Sud-Est; PARIS LYON; SAINT ETIENNE CHATEAUCREUX; 112; 112; 0; 9; 92.0; 
2012-03; Atlantique; ST PIERRE DES CORPS; PARIS MONTPARNASSE; 472; 472; 0; 90; 80.9; 
2012-03; Atlantique; PARIS MONTPARNASSE; TOURS; 221; 221; 0; 25; 88.7; 
2012-03; Sud-Est; PARIS LYON; VALENCE ALIXAN TGV; 267; 267; 0; 24; 91.0; 
2012-03; Sud-Est; PARIS LYON; BESANCON FRANCHE COMTE TGV; 241; 241; 0; 21; 91.3; 
2012-04; Sud-Est; CHAMBERY CHALLES LES EAUX; PARIS LYON; 222; 222; 0; 16; 92.8; 
2012-04; Sud-Est; PARIS LYON; DIJON VILLE; 442; 442; 0; 33; 92.5; 
2012-04; Nord; PARIS NORD; DUNKERQUE; 153; 152; 1; 8; 94.7; Travaux importants de rénovation des voies entre Arras et Isbergues. entre Douai et Valenciennes et sur la ligne à grande vitesse entre Lille et TGV - Haute Picardie. Les circulations ont également été impactées par une succession de dérangements d'installations ferroviaires.
2012-04; Atlantique; LAVAL; PARIS MONTPARNASSE; 235; 235; 0; 17; 92.8; 
2012-04; Sud-Est; LE CREUSOT MONTCEAU MONTCHANIN; PARIS LYON; 211; 211; 0; 29; 86.3; 
2012-04; Sud-Est; PARIS LYON; LE CREUSOT MONTCEAU MONTCHANIN; 200; 199; 1; 26; 86.9; 
2012-04; Atlantique; LE MANS; PARIS MONTPARNASSE; 474; 474; 0; 55; 88.4; Dérangement d'installation à Angers, accident de personne sur les voies sur la Ligne à Grande Vitesse au niveau de St-Léger affectant 17 TGV.
2012-04; Atlantique; PARIS MONTPARNASSE; LE MANS; 438; 438; 0; 29; 93.4; 
2012-04; Nord; LYON PART DIEU; LILLE; 271; 271; 0; 63; 76.8; Travaux importants sur le réseau Nord et Sud de la France. Les circulations ont également été impactées par une succession de dérangements d'installations ferroviaires.
2012-04; Nord; LILLE; MARSEILLE ST CHARLES; 153; 153; 0; 27; 82.4; Conditions difficiles avec des retards liés notamment aux travaux importants sur le réseau Nord et Sud de la France. Les circulations ont également été impactées par une succession de dérangements d'installations ferroviaires.
2012-04; Sud-Est; MONTPELLIER; LYON PART DIEU; 358; 358; 0; 63; 82.4; 
2012-04; Atlantique; LYON PART DIEU; RENNES; 29; 29; 0; 2; 93.1; 
2012-04; Sud-Est; PARIS LYON; MACON LOCHE; 182; 182; 0; 13; 92.9; 
2012-04; Sud-Est; PARIS LYON; MONTPELLIER; 363; 363; 0; 29; 92.0; 
2012-04; Atlantique; NANTES; PARIS MONTPARNASSE; 568; 568; 0; 20; 96.5; 
2012-04; Est; NANTES; STRASBOURG; 60; 60; 0; 1; 98.3; Un mois au dessus de l'objectif et en amélioration avec l'arrêt d'une partie des travaux.
2012-04; Sud-Est; PARIS LYON; NICE VILLE; 197; 197; 0; 37; 81.2; Une baisse de la régularité en avril qui s'explique par une augmentation des incidents d'origine externe (trois accidents de personne, un vol de câbles sur ligne nouvelle) et deux incidents d'exploitation importants dans le noeud Marseillais.
2012-04; Atlantique; QUIMPER; PARIS MONTPARNASSE; 146; 146; 0; 9; 93.8; 
2012-04; Est; PARIS EST; REIMS; 241; 241; 0; 9; 96.3; Une liaison au-dessus de l'objectif malgré les pertes de temps du au jet de pierre et à l'incendie à l'entrée de Paris-Est.
2012-04; Sud-Est; PARIS LYON; SAINT ETIENNE CHATEAUCREUX; 114; 114; 0; 11; 90.4; 
2012-04; Atlantique; PARIS MONTPARNASSE; ST PIERRE DES CORPS; 452; 452; 0; 40; 91.2; 
2012-04; Atlantique; TOULOUSE MATABIAU; PARIS MONTPARNASSE; 113; 113; 0; 5; 95.6; 
2012-04; Atlantique; PARIS MONTPARNASSE; TOULOUSE MATABIAU; 141; 141; 0; 13; 90.8; 
2012-04; Sud-Est; PARIS LYON; AVIGNON TGV; 370; 370; 0; 32; 91.4; 
2012-04; Sud-Est; BELLEGARDE (AIN); PARIS LYON; 268; 268; 0; 13; 95.1; 
2012-04; Atlantique; BORDEAUX ST JEAN; PARIS MONTPARNASSE; 626; 626; 0; 50; 92.0; 
2012-05; Atlantique; PARIS MONTPARNASSE; BREST; 177; 177; 0; 13; 92.7; 
2012-05; Sud-Est; PARIS LYON; CHAMBERY CHALLES LES EAUX; 215; 215; 0; 29; 86.5; 
2012-05; Sud-Est; DIJON VILLE; PARIS LYON; 439; 439; 0; 52; 88.2; 
2012-05; Sud-Est; PARIS LYON; DIJON VILLE; 444; 444; 0; 33; 92.6; 
2012-05; Nord; PARIS NORD; DUNKERQUE; 155; 155; 0; 9; 94.2; 
2012-05; Sud-Est; GRENOBLE; PARIS LYON; 234; 233; 1; 14; 94.0; 
2012-05; Atlantique; PARIS MONTPARNASSE; LA ROCHELLE VILLE; 210; 210; 0; 37; 82.4; Travaux de suppression d'un passage à niveau, de suppression d'une passerelle à Mauzé (79) et d'amélioration de la qualité de la voie.
2012-05; Nord; LILLE; MARSEILLE ST CHARLES; 162; 162; 0; 32; 80.2; Les circulations ont été impactées par de nombreux événements : plusieurs accidents de personne ou personnes dans les voies. des heurts d'animaux. Il y a eu également des dérangements des installations ferroviaires et quelques difficultés liées aux rames TGV
2012-05; Nord; MARSEILLE ST CHARLES; LILLE; 136; 136; 0; 34; 75.0; Les circulations ont été impactées par de nombreux événements : plusieurs accidents de personne ou personnes dans les voies. des heurts d'animaux. Il y a eu également des dérangements des installations ferroviaires et quelques difficultés liées aux rames TGV
2012-05; Atlantique; LYON PART DIEU; RENNES; 31; 31; 0; 1; 96.8; 
2012-05; Sud-Est; PERPIGNAN; PARIS LYON; 159; 159; 0; 12; 92.5; 
2012-05; Atlantique; POITIERS; PARIS MONTPARNASSE; 496; 496; 0; 74; 85.1; Résultats du mois de mai légèrement en baisse par rapport au mois d'avril. La principale cause d'irrégularité sur cette liaison est la succession des dérangements liés à l'infrastructure (voies et signalisation). L'incident caténaire (rupture du fil d'alimentation électrique) a eu un lourd impact sur la régularité des trains (24 TGV touchés avec des retards allant jusqu'à plusieurs heures).
2012-05; Est; PARIS EST; REIMS; 252; 252; 0; 14; 94.4; Un légère baisse mais un résultat toujours supérieur à l'objectif malgré un baisse de la performance du matériel sur la fin de mois.
2012-05; Atlantique; ST MALO; PARIS MONTPARNASSE; 103; 103; 0; 4; 96.1; 
2012-05; Atlantique; PARIS MONTPARNASSE; TOULOUSE MATABIAU; 143; 143; 0; 14; 90.2; 
2012-05; Atlantique; PARIS MONTPARNASSE; TOURS; 193; 193; 0; 19; 90.2; 
2012-05; Atlantique; PARIS MONTPARNASSE; BORDEAUX ST JEAN; 606; 606; 0; 50; 91.7; 
2012-06; Atlantique; PARIS MONTPARNASSE; BREST; 175; 175; 0; 7; 96.0; 
2012-06; Sud-Est; CHAMBERY CHALLES LES EAUX; PARIS LYON; 189; 189; 0; 25; 86.8; 
2012-06; Sud-Est; PARIS LYON; CHAMBERY CHALLES LES EAUX; 201; 201; 0; 19; 90.5; 
2012-06; Sud-Est; PARIS LYON; GRENOBLE; 245; 245; 0; 23; 90.6; 
2012-06; Atlantique; PARIS MONTPARNASSE; LAVAL; 233; 233; 0; 13; 94.4; 
2012-06; Sud-Est; PARIS LYON; LE CREUSOT MONTCEAU MONTCHANIN; 200; 200; 0; 22; 89.0; 
2012-06; Nord; MARSEILLE ST CHARLES; LILLE; 127; 127; 0; 37; 70.9; Liaison touchée par des vols de câbles et des accidents de personnes sur le Nord mais aussi par des dérangements d'installations ferroviaires sur le Sud-Est.
2012-06; Nord; PARIS NORD; LILLE; 616; 616; 0; 69; 88.8; 
2012-06; Est; PARIS EST; METZ; 300; 300; 0; 21; 93.0; 
2012-06; Atlantique; ANGOULEME; PARIS MONTPARNASSE; 350; 350; 0; 57; 83.7; Quelques incidents ayant un fort impact sur la régularité des TGV.
2012-06; Sud-Est; MONTPELLIER; PARIS LYON; 384; 384; 0; 38; 90.1; 
2012-06; Sud-Est; PARIS LYON; MULHOUSE VILLE; 172; 172; 0; 15; 91.3; 
2012-06; Est; STRASBOURG; NANTES; 60; 60; 0; 4; 93.3; 
2012-06; Atlantique; QUIMPER; PARIS MONTPARNASSE; 144; 144; 0; 9; 93.8; 
2012-06; Sud-Est; ANNECY; PARIS LYON; 194; 194; 0; 17; 91.2; 
2012-06; Est; PARIS EST; REIMS; 243; 243; 0; 19; 92.2; 
2012-06; Est; PARIS EST; STRASBOURG; 465; 465; 0; 23; 95.1; 
2012-06; Sud-Est; TOULON; PARIS LYON; 265; 265; 0; 57; 78.5; D'importantes phases de travaux d'amélioration de l'infrastructure sur le tronçon Nice Marseille nécessitent la mise en place de limitations de vitesse qui réduise la fluidité des circulations.
2012-06; Atlantique; TOURS; PARIS MONTPARNASSE; 204; 204; 0; 15; 92.6; 
2012-06; Atlantique; PARIS MONTPARNASSE; TOURS; 200; 200; 0; 29; 85.5; Quelques incidents ayant un fort impact sur la régularité des TGV.
2012-06; Nord; PARIS NORD; ARRAS; 369; 369; 0; 44; 88.1; 
2012-06; Atlantique; PARIS MONTPARNASSE; VANNES; 182; 182; 0; 12; 93.4; 
2014-12; Atlantique; PARIS MONTPARNASSE; ST PIERRE DES CORPS; 473; 473; 0; 46; 90.3; 
2014-12; Atlantique; PARIS MONTPARNASSE; TOURS; 154; 154; 0; 22; 85.7; 
2014-12; Atlantique; PARIS MONTPARNASSE; VANNES; 167; 167; 0; 21; 87.4; 
2014-12; Nord; PARIS NORD; DOUAI; 106; 106; 0; 6; 94.3; 
2014-12; Sud-Est; SAINT ETIENNE CHATEAUCREUX; PARIS LYON; 10; 10; 0; 0; 100.0; 
2014-12; Est; STRASBOURG; NANTES; 43; 43; 0; 3; 93.0; 
2014-12; Atlantique; TOULOUSE MATABIAU; PARIS MONTPARNASSE; 127; 127; 0; 13; 89.8; 
2014-12; Atlantique; TOURS; PARIS MONTPARNASSE; 196; 196; 0; 26; 86.7; 
2015-01; Nord; DOUAI; PARIS NORD; 190; 190; 0; 32; 83.2; 
2015-01; Sud-Est; MACON LOCHE; PARIS LYON; 187; 187; 0; 17; 90.9; 
2015-01; Sud-Est; MARSEILLE ST CHARLES; LYON PART DIEU; 527; 527; 0; 79; 85.0; 
2015-01; Est; NANCY; PARIS EST; 288; 287; 1; 18; 93.7; 
2015-01; Sud-Est; NICE VILLE; PARIS LYON; 178; 176; 2; 23; 86.9; 5 collisions avec des animaux sauvages sur la ligne à grande vitesse ont perturbé la régularité de cette relation en janvier 2015
2015-01; Sud-Est; PARIS LYON; BESANCON FRANCHE COMTE TGV; 192; 192; 0; 12; 93.8; 
2014-07; Atlantique; PARIS MONTPARNASSE; BORDEAUX ST JEAN; 640; 640; 0; 67; 89.5; 
2014-08; Atlantique; BORDEAUX ST JEAN; PARIS MONTPARNASSE; 656; 656; 0; 68; 89.6; 
2014-08; Sud-Est; DIJON VILLE; PARIS LYON; 468; 468; 0; 39; 91.7; 
2014-08; Nord; DUNKERQUE; PARIS NORD; 117; 116; 1; 4; 96.6; 
2014-08; Sud-Est; GRENOBLE; PARIS LYON; 183; 183; 0; 16; 91.3; 
2014-08; Atlantique; LA ROCHELLE VILLE; PARIS MONTPARNASSE; 228; 228; 0; 9; 96.1; 
2014-08; Nord; LILLE; LYON PART DIEU; 217; 217; 0; 15; 93.1; 
2014-08; Nord; LYON PART DIEU; LILLE; 248; 248; 0; 86; 65.3; De nombreux accidents de personne, à Chevry-Cossigny le 1er, Montpellier les 12 et 22 et Allan le 16. Découverte d'un corps à St Cézaire le 14. Plusieurs dérangement des installations essentiellement à Montanay et Cuy le 6 et à Tonerre le 14. Défaut d'alimentation au Creusot le 25 en raison d'un acte de malveillance et à Toutry le 27. Divaguation de bestiaux à Cesseins le 11. Quelques détresses matériel à Avignon le 4, Lyon le 5 et à Chatelet le 7. Des bagages abandonnés à Marne la Vallée le 14. Plusieurs limitations de vitesse suite à travaux essentiellement à Macon, Montanay et Marseille.
2014-08; Nord; MARSEILLE ST CHARLES; LILLE; 128; 128; 0; 50; 60.9; De nombreux accidents de personne, à Chevry-Cossigny le 1er, Montpellier les 12 et 22 et Allan le 16. Découverte d'un corps à St Cézaire le 14.
Plusieurs dérangement des installations essentiellement à Montanay et Cuy le 6 et à Tonerre le 14. Défaut d'alimentation au Creusot le 25 pour acte de malveillance et à Toutry le 27. Divaguation de bestiaux à Cesseins le 11. Quelques détresses matériel à Avignon le 4, Lyon le 5 et à Chatelet le 7. Des bagages abandonnés à Marne la Vallée le 14. Plusieurs limitations de vitesse suite à travaux essentiellement à Macon, Montanay et Marseille.
2014-08; Sud-Est; MONTPELLIER; PARIS LYON; 371; 371; 0; 48; 87.1; En août, la régularité des TGV entre Paris et la Méditerranée a été fortement impactée par 8 accidents de personnes et par un acte de malveillance près du Creusot.
2014-08; Sud-Est; NIMES; PARIS LYON; 371; 371; 0; 60; 83.8; En août, la régularité des TGV entre Paris et la Méditerranée a été fortement impactée par 8 accidents de personnes et par un acte de malveillance près du Creusot.
2014-08; Sud-Est; PARIS LYON; CHAMBERY CHALLES LES EAUX; 199; 199; 0; 27; 86.4; 
2014-08; Sud-Est; PARIS LYON; MACON LOCHE; 221; 221; 0; 13; 94.1; 
2014-08; Sud-Est; PARIS LYON; MONTPELLIER; 366; 366; 0; 50; 86.3; 
2014-08; Sud-Est; PARIS LYON; PERPIGNAN; 194; 194; 0; 29; 85.1; En août, la régularité des TGV entre Paris et la Méditerranée a été fortement impactée par 8 accidents de personnes et par un acte de malveillance près du Creusot.
2014-08; Atlantique; PARIS MONTPARNASSE; ANGERS SAINT LAUD; 396; 396; 0; 22; 94.4; 
2014-08; Atlantique; PARIS MONTPARNASSE; LAVAL; 236; 236; 0; 12; 94.9; 
2014-08; Atlantique; PARIS MONTPARNASSE; TOULOUSE MATABIAU; 154; 154; 0; 24; 84.4; 
2014-08; Nord; PARIS NORD; DUNKERQUE; 129; 128; 1; 3; 97.7; 
2014-08; Atlantique; ST MALO; PARIS MONTPARNASSE; 106; 106; 0; 3; 97.2; 
2014-08; Atlantique; ST PIERRE DES CORPS; PARIS MONTPARNASSE; 418; 418; 0; 56; 86.6; 
2014-08; Est; STRASBOURG; PARIS EST; 463; 462; 1; 31; 93.3; 
2014-08; Sud-Est; VALENCE ALIXAN TGV; PARIS LYON; 261; 261; 0; 40; 84.7; En août, la régularité des TGV entre Paris et la Méditerranée a été fortement impactée par 8 accidents de personnes et par un acte de malveillance près du Creusot.
2014-09; Nord; ARRAS; PARIS NORD; 334; 334; 0; 44; 86.8; 
2014-09; Sud-Est; AVIGNON TGV; PARIS LYON; 445; 443; 2; 85; 80.8; 
2014-09; Sud-Est; DIJON VILLE; PARIS LYON; 457; 457; 0; 30; 93.4; 
2014-09; Atlantique; LAVAL; PARIS MONTPARNASSE; 219; 219; 0; 17; 92.2; 
2014-09; Nord; LILLE; LYON PART DIEU; 190; 189; 1; 13; 93.1; 
2014-09; Sud-Est; LYON PART DIEU; MARSEILLE ST CHARLES; 539; 539; 0; 129; 76.1; Plusieurs incidents liés à la présence de personnes ou d'animaux sur les voies ont fortement perturbé la régularité des TGV sur cette liaison en septembre.
2014-09; Sud-Est; LYON PART DIEU; MONTPELLIER; 359; 354; 5; 100; 71.8; Les intempéries sur les départements du Gard et de l'Hérault, ainsi que plusieurs incidents liés à la présence de personnes ou d'animaux sur les voies, ont fortement perturbé la régularité des TGV sur cette liaison en septembre.
2014-09; Sud-Est; MACON LOCHE; PARIS LYON; 196; 196; 0; 34; 82.7; 
2014-09; Nord; MARSEILLE ST CHARLES; LILLE; 119; 119; 0; 34; 71.4; Accidents de personne à Toulon le 1er et Roeux le 16, tentative de suicide à Avignon le 8, heurts d'animal à Haute Picardie le 3 et à Piolenc le 29, dérangement des installations à Nimes le 21, Croisilles le 22, Marne le 25 et Roeux le 26, violents orages à Piolenc le 15 et Toulon le 19, défaillance matériel à Macon le 23, incendie en gare de Manduel le 21, défaut d'alimentation à Nimes le 25 et Lyon le 26, très importantes innondations sur Montpellier les 29 et 30.
2014-09; Est; METZ; PARIS EST; 289; 289; 0; 12; 95.8; 
2014-09; Sud-Est; NIMES; PARIS LYON; 324; 318; 6; 77; 75.8; Les intempéries sur les départements du Gard et de l'Hérault, ainsi que plusieurs incidents liés à la présence de personnes ou d'animaux sur les voies, ont fortement perturbé la régularité des TGV sur cette liaison en septembre.
2014-09; Est; PARIS EST; REIMS; 206; 206; 0; 8; 96.1; 
2014-09; Sud-Est; PARIS LYON; ANNECY; 154; 154; 0; 5; 96.8; 
2014-09; Sud-Est; PARIS LYON; LYON PART DIEU; 612; 612; 0; 14; 97.7; 
2014-09; Sud-Est; PARIS LYON; PERPIGNAN; 153; 151; 2; 17; 88.7; Les intempéries sur les départements du Gard et de l'Hérault, ainsi que plusieurs incidents liés à la présence de personnes ou d'animaux sur les voies, ont fortement perturbé la régularité des TGV sur cette liaison en septembre.
2014-09; Atlantique; PARIS MONTPARNASSE; BORDEAUX ST JEAN; 618; 618; 0; 40; 93.5; 
2014-09; Atlantique; PARIS MONTPARNASSE; POITIERS; 496; 496; 0; 24; 95.2; 
2014-09; Atlantique; PARIS MONTPARNASSE; QUIMPER; 146; 146; 0; 15; 89.7; Le 5, deux bagages abandonnés en gare de Paris-Montparnasse ont entrainé le retard de plusieurs train au départ. Le 7, une personne assise dans les voies près de Rosporden engendre un retard de 50 minutes. Le 9, la panne d'un train de Fret à Redon à retardé un TGV de 134 minutes. Le 15, un dysfonctionnement de la signalisation près de Rosporden, sans impact sur la sécurité des trains, a retardé un TGV de 45 minutes. Le 24, la panne d'un TER à Questembert a retardé un TGV de 55 minutes. 
2014-09; Atlantique; PARIS MONTPARNASSE; TOULOUSE MATABIAU; 141; 141; 0; 12; 91.5; 
2014-09; Atlantique; PARIS MONTPARNASSE; TOURS; 145; 145; 0; 11; 92.4; 
2014-09; Atlantique; QUIMPER; PARIS MONTPARNASSE; 124; 124; 0; 11; 91.1; 
2014-09; Est; STRASBOURG; PARIS EST; 465; 465; 0; 41; 91.2; 
2014-09; Atlantique; TOULOUSE MATABIAU; PARIS MONTPARNASSE; 100; 100; 0; 10; 90.0; Le 8, le dysfonctionnement d'un passage à niveau a entrainé le retard du TGV 8516 de 80 minutes à son terminus. Le 17, un accident de personne en gare de Tonneins (47) retarde le TGV 8500. Le 19, des difficultés de circulation près de Monts retardent les TGV 8516 et 8510 respectivement de 33 et 39 minutes.
2014-09; Atlantique; VANNES; PARIS MONTPARNASSE; 147; 147; 0; 14; 90.5; 
2014-10; Atlantique; ANGOULEME; PARIS MONTPARNASSE; 332; 331; 1; 36; 89.1; 
2014-10; Nord; ARRAS; PARIS NORD; 348; 348; 0; 47; 86.5; 
2014-10; Sud-Est; BELLEGARDE (AIN); PARIS LYON; 242; 240; 2; 40; 83.3; 
2014-10; Atlantique; BREST; PARIS MONTPARNASSE; 182; 182; 0; 6; 96.7; 
2014-10; Sud-Est; MACON LOCHE; PARIS LYON; 201; 201; 0; 29; 85.6; 
2014-10; Nord; MARSEILLE ST CHARLES; LILLE; 124; 124; 0; 36; 71.0; Les principaux évènements survenus au mois d'octobre sont :
Accidents de personne à Lunel le 23
Découverte d'un corps à Arsy le 20
Trois heurts d'animal : à Vaux en Pré le 09, Solers le 10 et Grenay le 12
Plusieurs dérangement des installations : à Marolles le 01, Lyon les 01 et 10, à valence le 17, à Vennissieux et Grenay le 23 et Lyon le 24
Des colis suspects à Lyon le 03 et Marseille les 20 et 28 
Défaillance matériel à Vennissieux le 09
Défaut d'alimentation à Flamengries le 24
Feu de talus à Lyon le 06
2014-10; Est; METZ; PARIS EST; 293; 293; 0; 25; 91.5; 
2014-10; Sud-Est; MONTPELLIER; PARIS LYON; 389; 388; 1; 110; 71.6; 
2014-10; Est; NANTES; STRASBOURG; 42; 42; 0; 4; 90.5; 
2014-10; Est; PARIS EST; METZ; 305; 305; 0; 21; 93.1; 
2014-10; Est; PARIS EST; REIMS; 218; 218; 0; 6; 97.2; 
2014-10; Sud-Est; PARIS LYON; CHAMBERY CHALLES LES EAUX; 186; 186; 0; 22; 88.2; 
2014-10; Sud-Est; PARIS LYON; MULHOUSE VILLE; 303; 303; 0; 25; 91.7; 
2014-10; Sud-Est; PARIS LYON; TOULON; 199; 199; 0; 26; 86.9; En octobre, la régularité de cette relation a été affectée par plusieurs accidents de personne et heurts d'animaux sur lignes à grande vitesse et par les intempéries en région PACA (inondations)
2014-10; Atlantique; PARIS MONTPARNASSE; BORDEAUX ST JEAN; 652; 652; 0; 50; 92.3; 
2014-10; Atlantique; PARIS MONTPARNASSE; BREST; 209; 209; 0; 12; 94.3; 
2014-10; Atlantique; PARIS MONTPARNASSE; QUIMPER; 143; 143; 0; 13; 90.9; 
2014-10; Sud-Est; PERPIGNAN; PARIS LYON; 158; 157; 1; 21; 86.6; Plusieurs accidents de personne et heurts d'animaux sur lignes à grande vitesse et en région Languedoc-Roussillon ont perturbé la régularité de cette relation en octobre
2014-10; Atlantique; POITIERS; PARIS MONTPARNASSE; 505; 504; 1; 36; 92.9; 
2014-10; Atlantique; RENNES; LYON PART DIEU; 60; 60; 0; 8; 86.7; 
2014-10; Atlantique; RENNES; PARIS MONTPARNASSE; 584; 584; 0; 32; 94.5; 
2014-10; Est; STRASBOURG; NANTES; 42; 42; 0; 2; 95.2; Seulement 2 circulations ont été impactées pendant la période dont 1 suite à un accident de personne où le retard du train a été de plus de 3 heures
2014-10; Atlantique; TOULOUSE MATABIAU; PARIS MONTPARNASSE; 107; 107; 0; 15; 86.0; L'OD a été touchée par plusieurs dérangements d'installations dont un rail cassé à Poitiers le 29 (23 TGV touchés) et un défaut l'alimentation ERDF vers Châtellerault le 7 (11 TGV). Elle a également été impactée par des problèmes matériels sur un train FRET au nord de Bordeaux le 3 (20 TGV) et sur un TER vers Châtellerault le 20 (8 TGV). Par ailleurs l'OD a subi quelques événements extérieurs dont un accident de personne à Poitiers le 21 (23 TGV) et un autre à Castelsarrasin le 7 (8TGV).
2014-10; Sud-Est; VALENCE ALIXAN TGV; PARIS LYON; 279; 276; 3; 54; 80.4; 
2014-11; Atlantique; BORDEAUX ST JEAN; PARIS MONTPARNASSE; 645; 644; 1; 60; 90.7; 
2014-11; Atlantique; LE MANS; PARIS MONTPARNASSE; 443; 443; 0; 63; 85.8; 
2014-11; Nord; LILLE; LYON PART DIEU; 30; 30; 0; 2; 93.3; 
2014-11; Nord; LILLE; PARIS NORD; 585; 584; 1; 73; 87.5; 
2014-11; Sud-Est; LYON PART DIEU; MARSEILLE ST CHARLES; 593; 592; 1; 116; 80.4; 
2014-11; Sud-Est; LYON PART DIEU; MONTPELLIER; 388; 387; 1; 74; 80.9; 
2014-11; Sud-Est; PARIS LYON; MARSEILLE ST CHARLES; 457; 454; 3; 28; 93.8; 
2014-11; Sud-Est; PARIS LYON; PERPIGNAN; 156; 156; 0; 31; 80.1; 
2014-11; Sud-Est; PARIS LYON; SAINT ETIENNE CHATEAUCREUX; 1; 1; 0; 0; 100.0; 
2014-11; Sud-Est; PARIS LYON; VALENCE ALIXAN TGV; 251; 251; 0; 19; 92.4; 
2014-11; Atlantique; PARIS MONTPARNASSE; QUIMPER; 133; 133; 0; 8; 94.0; 
2014-11; Atlantique; PARIS MONTPARNASSE; RENNES; 536; 536; 0; 17; 96.8; 
2014-11; Atlantique; PARIS MONTPARNASSE; ST MALO; 57; 57; 0; 1; 98.2; 
2014-11; Nord; PARIS NORD; DUNKERQUE; 112; 112; 0; 14; 87.5; 
2014-11; Atlantique; QUIMPER; PARIS MONTPARNASSE; 140; 140; 0; 8; 94.3; 
2014-11; Atlantique; TOULOUSE MATABIAU; PARIS MONTPARNASSE; 94; 93; 1; 16; 82.8; 
2014-11; Atlantique; VANNES; PARIS MONTPARNASSE; 158; 158; 0; 10; 93.7; 
2014-12; Sud-Est; AIX EN PROVENCE TGV; PARIS LYON; 443; 443; 0; 36; 91.9; 
2014-12; Atlantique; ANGERS SAINT LAUD; PARIS MONTPARNASSE; 475; 475; 0; 34; 92.8; 
2014-12; Sud-Est; AVIGNON TGV; PARIS LYON; 506; 506; 0; 45; 91.1; 
2014-12; Sud-Est; BELLEGARDE (AIN); PARIS LYON; 249; 248; 1; 30; 87.9; 
2014-12; Atlantique; BREST; PARIS MONTPARNASSE; 179; 179; 0; 14; 92.2; 
2014-12; Sud-Est; GRENOBLE; PARIS LYON; 243; 241; 2; 29; 88.0; 
2014-12; Nord; LILLE; LYON PART DIEU; 218; 216; 2; 19; 91.2; 
2014-12; Nord; LILLE; PARIS NORD; 599; 599; 0; 66; 89.0; 
2013-05; Sud-Est; AIX EN PROVENCE TGV; PARIS LYON; 431; 431; 0; 69; 84.0; 
2013-05; Sud-Est; PARIS LYON; CHAMBERY CHALLES LES EAUX; 212; 212; 0; 22; 89.6; 
2013-05; Sud-Est; LYON PART DIEU; MARSEILLE ST CHARLES; 633; 633; 0; 124; 80.4; 
2013-05; Sud-Est; MARSEILLE ST CHARLES; LYON PART DIEU; 592; 592; 0; 143; 75.8; La rénovation des voies sur le secteur Lyon Grenoble affecte les trains de cette relation à l'entrée de la zone Lyonnaise.
2012-12; Sud-Est; PARIS LYON; AIX EN PROVENCE TGV; 386; 386; 0; 60; 84.5; 
2012-12; Nord; DOUAI; PARIS NORD; 204; 204; 0; 18; 91.2; 
2012-12; Atlantique; LAVAL; PARIS MONTPARNASSE; 243; 243; 0; 21; 91.4; 
2012-12; Atlantique; PARIS MONTPARNASSE; LE MANS; 449; 449; 0; 41; 90.9; 
2012-12; Atlantique; PARIS MONTPARNASSE; ANGERS SAINT LAUD; 439; 439; 0; 32; 92.7; 
2012-12; Nord; LILLE; LYON PART DIEU; 309; 309; 0; 56; 81.9; 
2012-12; Sud-Est; LYON PART DIEU; MONTPELLIER; 399; 399; 0; 95; 76.2; 
2012-12; Sud-Est; MULHOUSE VILLE; PARIS LYON; 315; 315; 0; 31; 90.2; 
2012-12; Sud-Est; PARIS LYON; MULHOUSE VILLE; 307; 307; 0; 25; 91.9; 
2012-12; Atlantique; PARIS MONTPARNASSE; NANTES; 558; 558; 0; 32; 94.3; 
2012-12; Sud-Est; PARIS LYON; NICE VILLE; 152; 152; 0; 32; 78.9; D'importantes phases de travaux d'amélioration de l'infrastructure sur le tronçon Marseille Nice nécessitent la mise en place de limitations de vitesse qui réduisent la fluidité des circulations.
2012-12; Atlantique; PARIS MONTPARNASSE; POITIERS; 497; 497; 0; 16; 96.8; 
2012-12; Atlantique; RENNES; PARIS MONTPARNASSE; 568; 568; 0; 51; 91.0; 
2012-12; Est; PARIS EST; STRASBOURG; 478; 478; 0; 93; 80.5; 
2012-12; Sud-Est; PARIS LYON; TOULON; 196; 196; 0; 37; 81.1; D'importantes phases de travaux d'amélioration de l'infrastructure sur le tronçon Marseille Toulon nécessitent la mise en place de limitations de vitesse qui réduisent la fluidité des circulations.
2012-12; Atlantique; TOULOUSE MATABIAU; PARIS MONTPARNASSE; 87; 87; 0; 18; 79.3; Travaux sur LGV et incidents matériel.
2012-12; Atlantique; PARIS MONTPARNASSE; TOULOUSE MATABIAU; 151; 151; 0; 18; 88.1; 
2012-12; Sud-Est; PARIS LYON; VALENCE ALIXAN TGV; 267; 266; 1; 35; 86.8; 
2012-12; Nord; ARRAS; PARIS NORD; 331; 331; 0; 28; 91.5; 
2012-12; Sud-Est; PARIS LYON; BELLEGARDE (AIN); 275; 275; 0; 98; 64.4; La ligne du Haut Bugey a été particulièrement impactée par les intempéries.
2012-12; Sud-Est; PARIS LYON; BESANCON FRANCHE COMTE TGV; 236; 236; 0; 26; 89.0; 
2012-12; Atlantique; PARIS MONTPARNASSE; BORDEAUX ST JEAN; 643; 643; 0; 24; 96.3; 
2013-01; Sud-Est; AIX EN PROVENCE TGV; PARIS LYON; 411; 411; 0; 78; 81.0; 
2013-01; Atlantique; PARIS MONTPARNASSE; BREST; 132; 132; 0; 8; 93.9; 
2013-01; Nord; DUNKERQUE; PARIS NORD; 87; 87; 0; 13; 85.1; Liaison touchée principalement par les chutes de neige du 14 au 23 janvier. Durant cette période, pour éviter les dangers liés aux projections de glace, la vitesse des trains sur la ligne à grande vitesse a été limitée à 230 km/h voire 170 km/h. Les installations au sol ainsi que les rames TGV ont également connu des difficultés liées à cet épisode neigeux.
2013-01; Atlantique; LA ROCHELLE VILLE; PARIS MONTPARNASSE; 216; 216; 0; 20; 90.7; 
2013-01; Atlantique; PARIS MONTPARNASSE; LA ROCHELLE VILLE; 216; 216; 0; 15; 93.1; 
2013-01; Atlantique; LAVAL; PARIS MONTPARNASSE; 239; 239; 0; 32; 86.6; 
2013-01; Atlantique; PARIS MONTPARNASSE; LAVAL; 236; 236; 0; 36; 84.7; 
2013-01; Sud-Est; LE CREUSOT MONTCEAU MONTCHANIN; PARIS LYON; 238; 238; 0; 45; 81.1; 
2013-01; Nord; MARSEILLE ST CHARLES; LILLE; 126; 126; 0; 43; 65.9; Circulation touchée principalement par les chutes de neige du 14 au 23 janvier sur le Nord et du 15 au 21 janvier sur le Sud Est. Durant cette période, pour éviter les dangers liés aux projections de glace, la vitesse des trains sur la ligne à grande vitesse du Nord ou du Sud Est a été limitée à 230 km/h voire 170 km/h . Les installations au sol ainsi que les rames TGV ont également connu des difficultés liées à cet épisode neigeux.
2013-01; Sud-Est; LYON PART DIEU; MARSEILLE ST CHARLES; 648; 648; 0; 177; 72.7; Conditions météorologiques difficiles (en particulier la neige et le froid).
2013-01; Sud-Est; MARSEILLE ST CHARLES; LYON PART DIEU; 597; 597; 0; 117; 80.4; 
2013-01; Sud-Est; MONTPELLIER; LYON PART DIEU; 350; 350; 0; 74; 78.9; Conditions météorologiques difficiles (en particulier la neige et le froid).
2013-01; Atlantique; ANGOULEME; PARIS MONTPARNASSE; 349; 349; 0; 50; 85.7; 
2013-01; Atlantique; PARIS MONTPARNASSE; ANGOULEME; 320; 320; 0; 21; 93.4; 
2013-01; Est; NANCY; PARIS EST; 296; 296; 0; 57; 80.7; 
2013-01; Sud-Est; PARIS LYON; NICE VILLE; 157; 157; 0; 27; 82.8; 
2013-01; Atlantique; POITIERS; PARIS MONTPARNASSE; 492; 491; 1; 61; 87.6; 
2013-01; Atlantique; PARIS MONTPARNASSE; POITIERS; 504; 504; 0; 33; 93.5; 
2013-01; Sud-Est; PARIS LYON; SAINT ETIENNE CHATEAUCREUX; 117; 117; 0; 28; 76.1; Conditions météorologiques difficiles (en particulier la neige et le froid).
2013-01; Sud-Est; TOULON; PARIS LYON; 235; 235; 0; 44; 81.3; 
2013-01; Atlantique; PARIS MONTPARNASSE; TOULOUSE MATABIAU; 144; 144; 0; 18; 87.5; 
2013-01; Atlantique; TOURS; PARIS MONTPARNASSE; 181; 181; 0; 34; 81.2; Forts épisodes neigeux les 15, 18, 19, 20, 21 et 22 janvier, nécessitant la baisse de vitesse des trains, notamment sur ligne à grande vitesse Atlantique.
2013-01; Nord; ARRAS; PARIS NORD; 341; 341; 0; 100; 70.7; Liaison touchée principalement par les chutes de neige du 14 au 23 janvier. Durant cette période, pour éviter les dangers liés aux projections de glace, la vitesse des trains sur la ligne à grande vitesse a été limitée à 230 km/h voire 170 km/h. Les installations au sol ainsi que les rames TGV ont également connu des difficultés liées à cet épisode neigeux.
2013-01; Atlantique; BORDEAUX ST JEAN; PARIS MONTPARNASSE; 667; 665; 2; 78; 88.3; 
2013-02; Atlantique; BREST; PARIS MONTPARNASSE; 35; 35; 0; 2; 94.3; 
2013-02; Sud-Est; PARIS LYON; GRENOBLE; 234; 234; 0; 23; 90.2; 
2013-02; Atlantique; PARIS MONTPARNASSE; LAVAL; 219; 219; 0; 9; 95.9; 
2013-02; Atlantique; PARIS MONTPARNASSE; ANGERS SAINT LAUD; 404; 404; 0; 24; 94.1; 
2013-02; Nord; LILLE; LYON PART DIEU; 279; 279; 0; 45; 83.9; 
2013-02; Nord; LILLE; MARSEILLE ST CHARLES; 138; 138; 0; 48; 65.2; Les chutes de neige des 10 et 11 février nécessitent de réduire la vitesse des trains sur la ligne à grande vitesse à 230 km/h voire 170 km/h pour éviter les dangers liés aux projections de glace. Les rames TGV ont également souffert de ces intempéries ainsi que du froid associé avec des pannes de train en cours de circulation. Les installations caténaires ont subi également des avaries. A noter aussi des heurts d'animaux dans le Sud-Est et des colis abandonnés à Marseille et Marne.
2013-02; Nord; LILLE; PARIS NORD; 566; 566; 0; 115; 79.7; Les chutes de neige des 10 et 11 février nécessitent de réduire la vitesse des trains sur la ligne à grande vitesse à 230 km/h voire 170 km/h pour éviter les dangers liés aux projections de glace. Les rames TGV ont également souffert de ces intempéries ainsi que du froid associé avec des pannes de train en cours de circulation. Les installations caténaires ont subi également des avaries. A noter aussi un accident de personne qui a totalement perturbé la circulation sur Paris Nord, des travaux de renouvellement de voie sur Arras et des dérangements informatiques dans les postes d'aiguillages de Douai.
2013-02; Nord; PARIS NORD; LILLE; 574; 574; 0; 81; 85.9; 
2013-02; Sud-Est; LYON PART DIEU; MARSEILLE ST CHARLES; 579; 578; 1; 160; 72.3; Conditions météorologiques difficiles (en particulier la neige et le froid).
2013-02; Sud-Est; LYON PART DIEU; PARIS LYON; 570; 568; 2; 44; 92.3; 
2013-02; Sud-Est; PARIS LYON; LYON PART DIEU; 577; 576; 1; 38; 93.4; 
2013-02; Atlantique; RENNES; LYON PART DIEU; 60; 60; 0; 5; 91.7; 
2013-02; Sud-Est; MONTPELLIER; PARIS LYON; 329; 329; 0; 37; 88.8; 
2013-02; Sud-Est; PARIS LYON; MONTPELLIER; 315; 315; 0; 42; 86.7; Conditions météorologiques difficiles (en particulier la neige et le froid).
2013-02; Sud-Est; MULHOUSE VILLE; PARIS LYON; 282; 282; 0; 32; 88.7; 
2013-02; Est; NANCY; PARIS EST; 267; 267; 0; 31; 88.4; 
2013-02; Sud-Est; PARIS LYON; NICE VILLE; 146; 146; 0; 32; 78.1; Conditions météorologiques difficiles (en particulier la neige et le froid).
2013-02; Atlantique; PARIS MONTPARNASSE; RENNES; 507; 507; 0; 29; 94.3; 
2013-02; Est; STRASBOURG; PARIS EST; 433; 432; 1; 103; 76.2; Suite aux différents épisodes de neige et de froid. la circulation des trains a été impactée par des limitations de vitesse (230 km/h à la place de 320 km/h) et par des dérangements de la signalisation et des aiguillages.
2013-02; Atlantique; PARIS MONTPARNASSE; TOURS; 135; 135; 0; 11; 91.9; 
2013-02; Atlantique; PARIS MONTPARNASSE; VANNES; 167; 167; 0; 7; 95.8; 
2013-02; Sud-Est; AVIGNON TGV; PARIS LYON; 295; 295; 0; 54; 81.7; 
2013-03; Sud-Est; CHAMBERY CHALLES LES EAUX; PARIS LYON; 321; 321; 0; 38; 88.2; 
2013-03; Sud-Est; DIJON VILLE; PARIS LYON; 478; 476; 2; 37; 92.2; La journée du 12 mars a été très perturbée par les intempéries (neige et pluies verglassantes) entrainant de nombreux et importants retards avoisinants les 2 heures.
2013-03; Atlantique; PARIS MONTPARNASSE; LA ROCHELLE VILLE; 225; 225; 0; 12; 94.7; 
2013-03; Atlantique; LAVAL; PARIS MONTPARNASSE; 229; 229; 0; 22; 90.4; 
2013-03; Sud-Est; PARIS LYON; LE CREUSOT MONTCEAU MONTCHANIN; 208; 207; 1; 30; 85.5; 
2013-03; Atlantique; LE MANS; PARIS MONTPARNASSE; 465; 465; 0; 70; 84.9; 
2013-03; Sud-Est; PARIS LYON; LYON PART DIEU; 631; 628; 3; 34; 94.6; La journée du 12 mars a été très perturbée par les intempéries (neige et pluies verglassantes) entrainant de nombreux et importants retards avoisinants les 2 heures.
2013-03; Atlantique; LYON PART DIEU; RENNES; 30; 30; 0; 1; 96.7; 
2013-03; Sud-Est; MACON LOCHE; PARIS LYON; 201; 201; 0; 24; 88.1; 
2013-03; Sud-Est; MARSEILLE ST CHARLES; PARIS LYON; 478; 477; 1; 34; 92.9; La journée du 12 mars a été très perturbée par les intempéries (neige et pluies verglassantes) entrainant de nombreux et importants retards avoisinants les 2 heures.
2013-03; Est; METZ; PARIS EST; 300; 300; 0; 34; 88.7; 
2013-03; Est; PARIS EST; METZ; 314; 314; 0; 28; 91.1; 
2013-03; Sud-Est; PARIS LYON; MONTPELLIER; 347; 347; 0; 47; 86.5; 
2013-03; Est; PARIS EST; NANCY; 295; 294; 1; 23; 92.2; 
2013-03; Atlantique; PARIS MONTPARNASSE; NANTES; 581; 581; 0; 31; 94.7; 
2013-03; Sud-Est; PARIS LYON; PERPIGNAN; 159; 159; 0; 20; 87.4; La journée du 12 mars a été très perturbée par les intempéries (neige et pluies verglassantes) entrainant de nombreux et importants retards avoisinants les 2 heures.
2013-03; Atlantique; POITIERS; PARIS MONTPARNASSE; 499; 499; 0; 45; 91.0; 
2013-03; Est; REIMS; PARIS EST; 212; 212; 0; 14; 93.4; 
2013-03; Atlantique; PARIS MONTPARNASSE; RENNES; 554; 554; 0; 47; 91.5; Incident sur la caténaire en raison de l'épisode neigeux (accumulation de givre sur le fil d'alimentation), panne d'un train de travaux sur la ligne à grande vitesse.
2013-03; Sud-Est; PARIS LYON; SAINT ETIENNE CHATEAUCREUX; 102; 101; 1; 12; 88.1; 
2013-03; Atlantique; PARIS MONTPARNASSE; ST MALO; 54; 54; 0; 5; 90.7; 
2013-03; Est; PARIS EST; STRASBOURG; 478; 478; 0; 52; 89.1; 
2013-03; Nord; ARRAS; PARIS NORD; 339; 317; 22; 109; 65.6; Les très importantes chutes de neige avec formation de congères survenues à partir du lundi 11 mars dans la soirée jusqu'au vendredi 15 mars ont fortement perturbé la production. Elles ont imposé des limitations de vitesse pour éviter les projections de glace, allant même jusqu’à immobiliser presque complètement le trafic le mardi 12 mars. Les conditions climatiques difficiles de ces derniers mois ont par ailleurs perturbé les différents programmes travaux sur les voies prolongeant ainsi certaines limitations de vitesse au-delà des délais prévus, notamment sur Arras.
2013-03; Sud-Est; PARIS LYON; BESANCON FRANCHE COMTE TGV; 237; 236; 1; 24; 89.8; 
2013-03; Atlantique; BORDEAUX ST JEAN; PARIS MONTPARNASSE; 683; 683; 0; 100; 85.4; 
2013-03; Atlantique; PARIS MONTPARNASSE; BORDEAUX ST JEAN; 647; 647; 0; 71; 89.0; Épisode neigeux avec fortes répercussions sur la circulation des trains les 12 et 13 mars. Incident affectant la caténaire le 7 mars.
2013-04; Sud-Est; CHAMBERY CHALLES LES EAUX; PARIS LYON; 229; 229; 0; 34; 85.2; 
2013-04; Sud-Est; PARIS LYON; DIJON VILLE; 462; 462; 0; 22; 95.2; 
2013-04; Sud-Est; PARIS LYON; GRENOBLE; 241; 241; 0; 38; 84.2; 
2013-04; Sud-Est; PARIS LYON; LE CREUSOT MONTCEAU MONTCHANIN; 201; 201; 0; 27; 86.6; 
2013-04; Atlantique; PARIS MONTPARNASSE; ANGERS SAINT LAUD; 431; 431; 0; 13; 97.0; 
2013-04; Nord; LILLE; LYON PART DIEU; 299; 299; 0; 32; 89.3; Nombreux accidents de personnes et incidents d'installations techniques et électriques) répartis sur l'ensemble du parcours.
2014-07; Atlantique; NANTES; PARIS MONTPARNASSE; 563; 563; 0; 32; 94.3; 
2014-07; Est; STRASBOURG; NANTES; 50; 50; 0; 8; 84.0; 
2014-07; Sud-Est; NICE VILLE; PARIS LYON; 274; 274; 0; 49; 82.1; Plusieurs incidents ont pertubé la circulation des TGV et provoqué des retards importants sur cette liaison (heurt de personnes le 6, le 13, le 29, dérangements d'installations suite à des orages le 6, 20 et 25).
2014-07; Atlantique; PARIS MONTPARNASSE; POITIERS; 497; 497; 0; 28; 94.4; 
2014-07; Atlantique; QUIMPER; PARIS MONTPARNASSE; 125; 125; 0; 12; 90.4; 
2014-07; Atlantique; PARIS MONTPARNASSE; QUIMPER; 171; 171; 0; 11; 93.6; 
2014-07; Est; PARIS EST; REIMS; 214; 214; 0; 16; 92.5; 
2014-07; Sud-Est; TOULON; PARIS LYON; 327; 327; 0; 50; 84.7; Plusieurs incidents ont pertubé la circulation des TGV et provoqué des retards importants sur cette relation (heurt de personnes le 6, le 13, le 29, dérangements d'installations suite à des orages le 6, le 20, le 25).
2014-07; Sud-Est; PARIS LYON; TOULON; 246; 246; 0; 38; 84.6; 
2014-07; Sud-Est; VALENCE ALIXAN TGV; PARIS LYON; 259; 259; 0; 51; 80.3; 
2014-07; Nord; ARRAS; PARIS NORD; 324; 324; 0; 32; 90.1; 
2014-07; Nord; PARIS NORD; ARRAS; 321; 321; 0; 27; 91.6; 
2014-07; Atlantique; VANNES; PARIS MONTPARNASSE; 151; 151; 0; 15; 90.1; 
2014-08; Sud-Est; BELLEGARDE (AIN); PARIS LYON; 111; 111; 0; 8; 92.8; 
2014-08; Sud-Est; CHAMBERY CHALLES LES EAUX; PARIS LYON; 195; 195; 0; 37; 81.0; 
2014-08; Atlantique; LAVAL; PARIS MONTPARNASSE; 225; 225; 0; 9; 96.0; 
2014-08; Sud-Est; LE CREUSOT MONTCEAU MONTCHANIN; PARIS LYON; 215; 215; 0; 32; 85.1; 
2014-08; Nord; LILLE; PARIS NORD; 541; 541; 0; 40; 92.6; 
2014-08; Sud-Est; LYON PART DIEU; MARSEILLE ST CHARLES; 616; 616; 0; 107; 82.6; 
2014-08; Sud-Est; LYON PART DIEU; PARIS LYON; 531; 531; 0; 26; 95.1; 
2014-08; Sud-Est; MONTPELLIER; LYON PART DIEU; 418; 418; 0; 105; 74.9; En août, la régularité des TGV entre Paris et la Méditerranée a été fortement impactée par 8 accidents de personnes et par un acte de malveillance près du Creusot.
2014-08; Atlantique; NANTES; PARIS MONTPARNASSE; 549; 549; 0; 16; 97.1; 
2014-08; Est; PARIS EST; NANCY; 290; 290; 0; 16; 94.5; 
2014-08; Sud-Est; PARIS LYON; LYON PART DIEU; 518; 517; 1; 21; 95.9; 
2014-08; Sud-Est; PARIS LYON; MULHOUSE VILLE; 279; 279; 0; 17; 93.9; 
2014-08; Atlantique; PARIS MONTPARNASSE; NANTES; 514; 514; 0; 23; 95.5; 
2014-08; Atlantique; PARIS MONTPARNASSE; ST MALO; 60; 60; 0; 1; 98.3; 
2014-08; Atlantique; PARIS MONTPARNASSE; TOURS; 144; 144; 0; 10; 93.1; 
2014-08; Nord; PARIS NORD; ARRAS; 318; 316; 2; 34; 89.2; 
2014-08; Sud-Est; PERPIGNAN; PARIS LYON; 196; 196; 0; 32; 83.7; En août, la régularité des TGV entre Paris et la Méditerranée a été fortement impactée par 8 accidents de personnes et par un acte de malveillance près du Creusot.
2014-09; Sud-Est; ANNECY; PARIS LYON; 136; 136; 0; 8; 94.1; 
2014-09; Atlantique; BREST; PARIS MONTPARNASSE; 174; 174; 0; 8; 95.4; 
2014-09; Atlantique; LE MANS; PARIS MONTPARNASSE; 449; 449; 0; 67; 85.1; 
2014-09; Atlantique; LYON PART DIEU; RENNES; 28; 28; 0; 2; 92.9; 
2014-09; Sud-Est; MONTPELLIER; PARIS LYON; 322; 318; 4; 52; 83.6; Les intempéries sur les départements du Gard et de l'Hérault, ainsi que plusieurs incidents liés à la présence de personnes ou d'animaux sur les voies, ont fortement perturbé la régularité des TGV sur cette liaison en septembre.
2014-09; Est; PARIS EST; STRASBOURG; 431; 431; 0; 17; 96.1; 
2014-09; Sud-Est; PARIS LYON; AIX EN PROVENCE TGV; 437; 437; 0; 39; 91.1; 
2014-09; Sud-Est; PARIS LYON; BESANCON FRANCHE COMTE TGV; 212; 212; 0; 9; 95.8; 
2014-09; Sud-Est; PARIS LYON; DIJON VILLE; 492; 492; 0; 19; 96.1; 
2014-09; Sud-Est; PARIS LYON; GRENOBLE; 245; 245; 0; 25; 89.8; 
2014-09; Sud-Est; PARIS LYON; NIMES; 310; 305; 5; 49; 83.9; 
2014-09; Sud-Est; PARIS LYON; SAINT ETIENNE CHATEAUCREUX; 5; 5; 0; 0; 100.0; 
2014-09; Sud-Est; PARIS LYON; VALENCE ALIXAN TGV; 253; 250; 3; 15; 94.0; 
2014-09; Atlantique; PARIS MONTPARNASSE; ANGERS SAINT LAUD; 428; 428; 0; 18; 95.8; 
2014-09; Atlantique; PARIS MONTPARNASSE; ANGOULEME; 313; 313; 0; 19; 93.9; 
2014-09; Atlantique; PARIS MONTPARNASSE; LA ROCHELLE VILLE; 212; 212; 0; 10; 95.3; 
2014-09; Atlantique; PARIS MONTPARNASSE; NANTES; 562; 562; 0; 24; 95.7; 
2014-09; Atlantique; PARIS MONTPARNASSE; RENNES; 519; 519; 0; 51; 90.2; 
2014-09; Nord; PARIS NORD; DUNKERQUE; 124; 124; 0; 9; 92.7; 
2014-09; Nord; PARIS NORD; LILLE; 612; 612; 0; 47; 92.3; 
2014-09; Nord; PARIS NORD; ARRAS; 327; 326; 1; 19; 94.2; 
2014-09; Atlantique; POITIERS; PARIS MONTPARNASSE; 483; 481; 2; 38; 92.1; 
2014-09; Atlantique; RENNES; LYON PART DIEU; 51; 51; 0; 7; 86.3; 
2014-09; Atlantique; RENNES; PARIS MONTPARNASSE; 539; 539; 0; 44; 91.8; 
2014-09; Atlantique; TOURS; PARIS MONTPARNASSE; 199; 199; 0; 34; 82.9; 
2014-10; Sud-Est; AVIGNON TGV; PARIS LYON; 461; 460; 1; 72; 84.3; 
2014-10; Sud-Est; DIJON VILLE; PARIS LYON; 444; 443; 1; 50; 88.7; 
2014-10; Atlantique; LAVAL; PARIS MONTPARNASSE; 244; 244; 0; 10; 95.9; 
2014-10; Nord; LYON PART DIEU; LILLE; 241; 241; 0; 60; 75.1; Les principaux évènements survenus au mois d'octobre sont :
Accidents de personne à Lunel le 23
Découverte d'un corps à Arsy le 20
Trois heurts d'animal : à Vaux en Pré le 09, Solers le 10 et Grenay le 12
Plusieurs dérangement des installations : à Marolles le 01, Lyon les 01 et 10, à valence le 17, à Vennissieux et Grenay le 23 et Lyon le 24
Des colis suspects à Lyon le 03 et Marseille les 20 et 28 
Défaillance matériel à Vennissieux le 09
Défaut d'alimentation à Flamengries le 24
Feu de talus à Lyon le 06
2014-10; Sud-Est; LYON PART DIEU; PARIS LYON; 648; 648; 0; 38; 94.1; 
2014-10; Atlantique; LYON PART DIEU; RENNES; 31; 31; 0; 1; 96.8; 
2014-10; Sud-Est; NICE VILLE; PARIS LYON; 197; 197; 0; 37; 81.2; 
2014-10; Sud-Est; NIMES; PARIS LYON; 339; 336; 3; 64; 81.0; 
2014-10; Est; PARIS EST; STRASBOURG; 449; 449; 0; 25; 94.4; 
2014-10; Sud-Est; PARIS LYON; DIJON VILLE; 476; 476; 0; 27; 94.3; 
2014-10; Sud-Est; PARIS LYON; VALENCE ALIXAN TGV; 265; 265; 0; 23; 91.3; 
2014-10; Atlantique; PARIS MONTPARNASSE; ANGOULEME; 332; 332; 0; 18; 94.6; 
2014-10; Atlantique; PARIS MONTPARNASSE; LA ROCHELLE VILLE; 226; 226; 0; 10; 95.6; 
2014-10; Atlantique; PARIS MONTPARNASSE; LAVAL; 241; 241; 0; 8; 96.7; 
2014-10; Atlantique; PARIS MONTPARNASSE; POITIERS; 520; 520; 0; 33; 93.7; 
2014-10; Atlantique; PARIS MONTPARNASSE; RENNES; 571; 571; 0; 28; 95.1; 
2014-10; Nord; PARIS NORD; DUNKERQUE; 120; 120; 0; 5; 95.8; 
2014-10; Nord; PARIS NORD; ARRAS; 346; 346; 0; 26; 92.5; 
2014-10; Atlantique; TOURS; PARIS MONTPARNASSE; 215; 215; 0; 54; 74.9; Un chantier long \& complexe de maintenance au Nord de Tours (entrée Sud de la LGV) a généré de nombreux ralentissements pour les trains de l'OD entraînant d'importants retards. A cela s'ajoute 6 restitutions tardive de travaux en début de matinée.
2014-11; Atlantique; BREST; PARIS MONTPARNASSE; 177; 177; 0; 6; 96.6; 
2014-11; Atlantique; LAVAL; PARIS MONTPARNASSE; 228; 228; 0; 17; 92.5; 
2014-11; Sud-Est; LE CREUSOT MONTCEAU MONTCHANIN; PARIS LYON; 211; 211; 0; 19; 91.0; 
2014-11; Est; METZ; PARIS EST; 282; 282; 0; 14; 95.0; 
2014-11; Sud-Est; MONTPELLIER; PARIS LYON; 328; 327; 1; 39; 88.1; 
2014-11; Est; PARIS EST; METZ; 292; 292; 0; 10; 96.6; 
2014-11; Est; PARIS EST; REIMS; 213; 213; 0; 14; 93.4; 
2014-11; Est; PARIS EST; STRASBOURG; 425; 425; 0; 15; 96.5; 
2014-11; Sud-Est; PARIS LYON; NICE VILLE; 157; 155; 2; 38; 75.5; La régularité des TGV de cette relation a été perturbée en novembre par 2 accidents de personne, 3 heurts d'animaux et par les intempéries en région PACA.
2014-11; Sud-Est; PARIS LYON; NIMES; 310; 310; 0; 36; 88.4; 
2014-11; Atlantique; PARIS MONTPARNASSE; LE MANS; 227; 227; 0; 1; 99.6; 
2014-11; Atlantique; PARIS MONTPARNASSE; TOULOUSE MATABIAU; 139; 139; 0; 16; 88.5; 
2014-11; Nord; PARIS NORD; DOUAI; 128; 128; 0; 16; 87.5; 
2014-11; Nord; PARIS NORD; LILLE; 596; 595; 1; 50; 91.6; 
2014-11; Atlantique; POITIERS; PARIS MONTPARNASSE; 483; 483; 0; 32; 93.4; 
2014-11; Atlantique; RENNES; PARIS MONTPARNASSE; 538; 538; 0; 30; 94.4; 
2014-11; Sud-Est; SAINT ETIENNE CHATEAUCREUX; PARIS LYON; 1; 1; 0; 0; 100.0; 
2014-12; Atlantique; ANGOULEME; PARIS MONTPARNASSE; 357; 357; 0; 37; 89.6; La ligne a été affectée par des incidents externes dont un accident de personne d'abord près de Tours le 14/12 retarde près d'une vingtaine de TGV de 11' à 2h40; le 22/12 le heurt d'un sanglier sur la Ligne à Grande Vitesse près de St Arnoult retarde une trentaine de TGV (de 11min à 1h18).
Enfin, le 25/12: un dérangement des installations entre Paris Montparnasse \& Massy retarde 32 TGV (25min à 2h38).
2014-12; Atlantique; BORDEAUX ST JEAN; PARIS MONTPARNASSE; 684; 683; 1; 64; 90.6; 
2014-12; Atlantique; LAVAL; PARIS MONTPARNASSE; 246; 246; 0; 20; 91.9; 
2014-12; Nord; LILLE; MARSEILLE ST CHARLES; 122; 122; 0; 21; 82.8; 
2014-12; Sud-Est; LYON PART DIEU; MONTPELLIER; 370; 367; 3; 94; 74.4; Les intempéries en région Languedoc Roussillon ont perturbé la régularité de cette relation en décembre.
2014-12; Sud-Est; MULHOUSE VILLE; PARIS LYON; 320; 320; 0; 17; 94.7; 
2014-12; Sud-Est; PARIS LYON; DIJON VILLE; 494; 492; 2; 15; 97.0; 
2014-12; Sud-Est; PARIS LYON; GRENOBLE; 246; 246; 0; 29; 88.2; 
2014-12; Sud-Est; PARIS LYON; MARSEILLE ST CHARLES; 455; 455; 0; 16; 96.5; 
2015-02; Sud-Est; PARIS LYON; NICE VILLE; 173; 172; 1; 22; 87.2; 
2015-02; Sud-Est; PARIS LYON; NIMES; 295; 294; 1; 30; 89.8; 
2015-02; Atlantique; PARIS MONTPARNASSE; ANGOULEME; 300; 299; 1; 20; 93.3; 
2015-02; Atlantique; PARIS MONTPARNASSE; LA ROCHELLE VILLE; 204; 204; 0; 16; 92.2; 
2015-02; Atlantique; PARIS MONTPARNASSE; LE MANS; 411; 411; 0; 48; 88.3; 
2015-02; Atlantique; PARIS MONTPARNASSE; NANTES; 503; 503; 0; 44; 91.3; Cette ligne a été touchée par un incident majeur le 13/02 : un problème caténaire à Laval a entraîné de nombreuses modifications d’itinéraires via Nantes/Le Mans avec un total de 83 TGV retardés de 7min à 5h59. On compte également 4 incidents externes dont deux survenus sur la Ligne à Grande Vitesse : un accident de personne à Rouvray le 7/02 (19 TGV retardés de 30min à 4h04), un près de Paris le 18/02 (26 TGV retardés de 12 à 48min) \& un autre à La Suze le 27/02 (13 TGV de 22min à 4h02) ; le heurt d’un chevreuil à Dollon le 01/02 retarde 33 TGV (11min à 2h34). Par ailleurs, un dérangement des installations à Tiercé le 7/02 retarde 7 TGV (14 à 28min) Massy  le 11/02 touche 16 TGV (11 à 38min) \& on compte également deux incidents caténaires les 16 \& 18/02 : le premier sur la Ligne à Grande Vitesse (37 TGV de 16min à 1h41) \& le second à Massy-Verrières en région parisienne (22 TGV de 13min à 3h37). Enfin, une défaillance de Matériel à Massy le 19/02 pénalise 16 TGV (15min \& 1h47) \& les travaux en cours entre Nantes \& Angers entraînent des ralentissements.
2015-02; Atlantique; PARIS MONTPARNASSE; ST PIERRE DES CORPS; 416; 416; 0; 61; 85.3; 
2015-02; Atlantique; PARIS MONTPARNASSE; TOURS; 174; 174; 0; 31; 82.2; 
2015-02; Atlantique; PARIS MONTPARNASSE; VANNES; 158; 158; 0; 14; 91.1; Cette ligne a été touchée par un incident majeur le 13/02 : un problème caténaire à Laval retarde 83 TGV de 7min à 5h59. On compte également 4 incidents externes dont deux survenus sur la Ligne à Grande Vitesse : un accident de personne à Rouvray le 7/02 (19 TGV retardés de 30min à 4h04), un à Questembert le 14/02 (3 TGV 16min \& 1h50) \& un autre le 18/02 près de Paris (26 TGV retardés de 12 à 48min) ; le heurt d’un chevreuil à Dollon le 01/02 retarde 33 TGV (11min à 2h34). Par ailleurs, un dérangement des installations à Massy  le 11/02 touche 16 TGV (11 à 38min) \& on compte également deux incidents caténaires les 16 \& 18/02 : le premier sur la Ligne à Grande Vitesse (37 TGV de 16min à 1h41) \& le second à Massy-Verrières en région parisienne (22 TGV de 13min à 3h37). Enfin une défaillance de Matériel à Massy le 19/02 pénalise 16 TGV (15min \& 1h47).
2015-02; Nord; PARIS NORD; LILLE; 569; 568; 1; 47; 91.7; 
2015-02; Atlantique; POITIERS; PARIS MONTPARNASSE; 455; 452; 3; 45; 90.0; 
2015-02; Atlantique; QUIMPER; PARIS MONTPARNASSE; 185; 184; 1; 19; 89.7; Cette ligne a été touchée par un incident majeur le 13/02 : un problème caténaire à Laval retarde 83 TGV de 7min à 5h59. On compte également 6 incidents externes dont deux survenus sur la Ligne à Grande Vitesse : un accident de personne à Rouvray le 7/02 (19 de 30min à 4h04), deux le 14/02 à Questembert (3 TGV 16min \& 1h50) et à Auray (1 TGV à 1h05), un autre le 18/02 près de Paris (26 TGV retardés de 12 à 48min) \& à Quimperlé le 27/02 (1 TGV à 1h46) ; le heurt d’un chevreuil à Dollon le 01/02 retarde 33 TGV (11min à 2h34). Par ailleurs, un dérangement des installations à Massy  le 11/02 touche 16 TGV (11 à 38min) \& on compte également deux incidents caténaires les 16 \& 18/02 : le premier sur la Ligne à Grande Vitesse (37 TGV de 16min à 1h41) \& le second à Massy-Verrières en région parisienne (22 TGV de 13min à 3h37). Enfin une défaillance de Matériel à Massy le 19/02 pénalise 16 TGV (15min \& 1h47).
2015-02; Sud-Est; SAINT ETIENNE CHATEAUCREUX; PARIS LYON; 98; 98; 0; 11; 88.8; 
2015-02; Est; STRASBOURG; PARIS EST; 429; 424; 5; 40; 90.6; 
2015-03; Atlantique; ANGOULEME; PARIS MONTPARNASSE; 356; 356; 0; 18; 94.9; 
2015-03; Sud-Est; BESANCON FRANCHE COMTE TGV; PARIS LYON; 205; 205; 0; 2; 99.0; 
2015-03; Atlantique; BORDEAUX ST JEAN; PARIS MONTPARNASSE; 682; 682; 0; 44; 93.5; Cette OD a été touchée par les événements suivants:  un heurt de chevreuil sur la ligne grande vitesse à Vendôme le 2 (28 TGV; de 13' à 58'), un accident de personne à Poitiers le 8 (6 TGV; de 2h06 à 3h28), un feu de traverses à l'entrée de Paris Montparnasse le 14 (6 TGV; 12' à 33') ainsi que la divagation d'un chien sur la ligne grande vitesse à St Arnoult le 31 (13 TGV; de 12' à 36'). On compte par ailleurs 4 dérangements d'installations sur la ligne grande vitesse les 1, 2, 16 et 23 touchant jusqu'à 7 trains par jour avec des retards allant de 12' à 1h05 ainsi qu'un autre dérangement au niveau de  Poitiers le 23 (6 TGV; 18' à 2h10).
2015-03; Sud-Est; CHAMBERY CHALLES LES EAUX; PARIS LYON; 220; 220; 0; 12; 94.5; 
2015-03; Atlantique; LE MANS; PARIS MONTPARNASSE; 468; 468; 0; 47; 90.0; 
2015-03; Nord; LILLE; PARIS NORD; 602; 601; 1; 38; 93.7; 
2015-03; Sud-Est; LYON PART DIEU; MONTPELLIER; 352; 352; 0; 91; 74.1; Les travaux de raccordement du réseau actuel à la nouvelle ligne à grande vitesse (qui contournera Nîmes et Montpellier à partir de 2017), ainsi que des travaux sur ligne classique près de Montpellier ont généré des ralentissements en mars
2015-03; Atlantique; LYON PART DIEU; RENNES; 89; 89; 0; 4; 95.5; Cette OD a été touchée par les événements suivants:  un heurt de chevreuil sur la ligne grande vitesse à Vendôme le 2 (28 TGV; de 13' à 58'), un accident de personne à Piolenc dans le Sud-Est le 28  (3 TGV; 31' à 92') ainsi que la divagation d'un chien sur la ligne grande vitesse à St Arnoult le 31 (13 TGV; de 12' à 36'). On compte par ailleurs 4 dérangements d'installations sur la ligne grande vitesse les 1, 2, 16 et 23 touchant jusqu'à 7 trains par jour avec des retards allant de 12' à 1h05 ainsi qu'un incident d'origine électrique sur la région parisienne (6 TGV; 8' à 15').
2015-03; Nord; MARSEILLE ST CHARLES; LILLE; 214; 213; 1; 31; 85.4; 
2015-03; Sud-Est; MARSEILLE ST CHARLES; LYON PART DIEU; 550; 549; 1; 85; 84.5; 
2015-03; Sud-Est; PARIS LYON; NICE VILLE; 165; 163; 2; 24; 85.3; 
2015-03; Sud-Est; PARIS LYON; NIMES; 292; 292; 0; 20; 93.2; 
2015-03; Atlantique; PARIS MONTPARNASSE; RENNES; 556; 556; 0; 37; 93.3; 
2015-03; Nord; PARIS NORD; DOUAI; 209; 208; 1; 15; 92.8; 
2015-03; Atlantique; QUIMPER; PARIS MONTPARNASSE; 193; 193; 0; 16; 91.7; Cette OD a été touchée par les événements suivants:  un heurt de chevreuil sur la ligne grande vitesse le 2 (28 TGV; de 13' à 58'), 2 accidents de personne à Vannes le 6 (6 TGV; de 13' à 1h51) et à Cesson le 26 (11 TGV; 12' à 2h24'), un feu de traverses à l'entrée de Paris Montparnasse le 14 (6 TGV; 12' à 33'), un incendie aux abords des voies vers Laval le 23  (7 TGV; 22' à 2h05) ainsi que la divagation d'un chien sur la ligne grande vitesse à St Arnoult le 31 (13 TGV; de 12' à 36'). On compte par ailleurs 4 dérangements d'installations sur la ligne grande vitesse les 1, 2, 16 et 23 touchant jusqu'à 7 trains par jour avec des retards allant de 12' à 1h05 et un incident caténaire le 22 vers Redon (9 TGV; de 12' à 80').
2015-04; Atlantique; ANGOULEME; PARIS MONTPARNASSE; 345; 343; 2; 26; 92.4; 
2015-04; Sud-Est; CHAMBERY CHALLES LES EAUX; PARIS LYON; 218; 217; 1; 26; 88.0; 
2015-04; Sud-Est; LYON PART DIEU; MARSEILLE ST CHARLES; 590; 589; 1; 123; 79.1; 
2015-04; Sud-Est; LYON PART DIEU; PARIS LYON; 634; 634; 0; 57; 91.0; 
2015-04; Sud-Est; MARSEILLE ST CHARLES; LYON PART DIEU; 561; 559; 2; 152; 72.8; Des travaux de modernisation de l'infrastructure ont perturbé la régularité de cette relation en Avril.
2015-04; Sud-Est; MONTPELLIER; LYON PART DIEU; 324; 324; 0; 60; 81.5; 
2015-04; Est; PARIS EST; NANCY; 284; 284; 0; 9; 96.8; 
2015-04; Sud-Est; PARIS LYON; AIX EN PROVENCE TGV; 442; 442; 0; 42; 90.5; 
2015-04; Sud-Est; PARIS LYON; ANNECY; 148; 147; 1; 10; 93.2; 
2015-04; Sud-Est; PARIS LYON; AVIGNON TGV; 526; 526; 0; 69; 86.9; 
2015-04; Atlantique; PARIS MONTPARNASSE; ANGOULEME; 321; 321; 0; 19; 94.1; 
2015-04; Atlantique; PARIS MONTPARNASSE; LE MANS; 441; 441; 0; 38; 91.4; 
2015-04; Atlantique; PARIS MONTPARNASSE; NANTES; 548; 547; 1; 21; 96.2; 
2015-04; Atlantique; PARIS MONTPARNASSE; TOURS; 187; 187; 0; 21; 88.8; 
2015-04; Nord; PARIS NORD; DOUAI; 196; 196; 0; 20; 89.8; 
2015-04; Sud-Est; PERPIGNAN; PARIS LYON; 143; 143; 0; 26; 81.8; 
2015-04; Sud-Est; TOULON; PARIS LYON; 240; 238; 2; 75; 68.5; Des travaux de modernisation de l'infrastructure ont perturbé la régularité de cette relation en Avril.
2015-04; Atlantique; TOULOUSE MATABIAU; PARIS MONTPARNASSE; 170; 170; 0; 14; 91.8; 
2015-05; Atlantique; BORDEAUX ST JEAN; PARIS MONTPARNASSE; 663; 663; 0; 39; 94.1; 
2015-05; Sud-Est; CHAMBERY CHALLES LES EAUX; PARIS LYON; 146; 146; 0; 18; 87.7; 
2015-05; Nord; LILLE; MARSEILLE ST CHARLES; 213; 213; 0; 38; 82.2; Toujours des travaux importants sur la LGV  entrainant des limitations de vitesse, essentiellement à Lapalud. Autres événements :
03/05 : dérangement d'aiguille lors de travaux au Creusot
05 et 28/05 : panne d'une rame sur LGV Nord
08/05 : accident de personne à Lapalud
20/05 : acte de malveillance (jets de pierres) à Vémars
25/05 : occupation des voies à l'escale de Lille Europe
2015-05; Sud-Est; LYON PART DIEU; MARSEILLE ST CHARLES; 629; 628; 1; 131; 79.1; 
2015-05; Sud-Est; LYON PART DIEU; MONTPELLIER; 353; 352; 1; 79; 77.6; 
2015-05; Atlantique; LYON PART DIEU; RENNES; 87; 87; 0; 21; 75.9; 
2015-05; Atlantique; NANTES; PARIS MONTPARNASSE; 547; 545; 2; 38; 93.0; 
2015-05; Est; NANTES; STRASBOURG; 56; 56; 0; 7; 87.5; 
2015-05; Sud-Est; NICE VILLE; PARIS LYON; 217; 217; 0; 44; 79.7; 
2015-05; Est; PARIS EST; METZ; 289; 289; 0; 28; 90.3; 
2015-05; Est; PARIS EST; STRASBOURG; 423; 423; 0; 23; 94.6; 
2015-05; Sud-Est; PARIS LYON; LYON PART DIEU; 594; 594; 0; 32; 94.6; 
2015-05; Sud-Est; PARIS LYON; MARSEILLE ST CHARLES; 457; 456; 1; 40; 91.2; 
2015-05; Atlantique; PARIS MONTPARNASSE; BORDEAUX ST JEAN; 637; 636; 1; 34; 94.7; Cette ligne a été touchée par : un bruit en toiture entrainant un arrêt à St Léger d'un TGV le 05/05 (retarde 4 TGV de 121 minutes), heurt d'un animal sur la LGV le 12/05 (30 TGV soit 1799 minutes), une défaillance matérielle vers Massy le 18/05 (9 TGV soit 239 minutes), la fuite d'huile sur un transformateur d'un TGV vers Port de Piles le 27/05 ( 10 TGV soit 934 minutes), une disjonction sur la LGV le 25/05 (6 TGV soit 87 minutes) et un incident caténaire à Montmoreau le 29/05 (3 TGV , 119 minutes). 
2015-05; Atlantique; PARIS MONTPARNASSE; ST MALO; 102; 102; 0; 2; 98.0; Cette ligne a été retardée par : un accident de personne à Yvre-l'Evêque le 14/05 (15 TGV soit 1456 minutes), une disjonction sur la LGV le 25/05 (6 TGV soit 87 minutes), heurt d'un animal sur la LGV le 12/05 (30 TGV soit 1799 minutes) et heurt d'une voiture le 06/05 à Vitré (18 TGV soit 1704 minutes)
2015-05; Atlantique; PARIS MONTPARNASSE; TOULOUSE MATABIAU; 154; 154; 0; 15; 90.3; 
2015-05; Nord; PARIS NORD; LILLE; 594; 594; 0; 31; 94.8; 
2015-05; Atlantique; TOULOUSE MATABIAU; PARIS MONTPARNASSE; 185; 185; 0; 10; 94.6; 
2015-06; Atlantique; ANGERS SAINT LAUD; PARIS MONTPARNASSE; 458; 458; 0; 53; 88.4; 
2015-06; Sud-Est; ANNECY; PARIS LYON; 193; 193; 0; 12; 93.8; 
2015-06; Sud-Est; AVIGNON TGV; PARIS LYON; 573; 573; 0; 118; 79.4; Des travaux de modernisation de l'infrastructure ont perturbé la régularité de cette relation en Juin
2015-06; Sud-Est; DIJON VILLE; PARIS LYON; 460; 460; 0; 25; 94.6; 
2015-06; Nord; LILLE; PARIS NORD; 586; 583; 3; 76; 87.0; 
2015-06; Sud-Est; LYON PART DIEU; PARIS LYON; 635; 634; 1; 42; 93.4; 
2015-06; Atlantique; LYON PART DIEU; RENNES; 88; 88; 0; 27; 69.3; Cette OD a été touchée le 11 par la panne d'un TGV sur le réseau Sud-Est vers Chevry Cossigny (60TGV; 1478mn), le 6 par un dérangement d'installation au Creusot (49 TGv; 1662mn), 2 accidents de personne le 2 à Lyon (44TGV; 1314mn) et le 22 à Pasilly (27TGV; 1347mn), le 24 par le heurt d'un animal sur le réseau Sud-Est vers Sully (26TGV; 660mn), le 26 par la rupture d'un appareil de voie en sortie de ligne grande vitesse avant Le Mans (52TGV; 923mn), le 3 par l'arrêt de sécurité d'un train dans les tunnels avant Paris (22TGV; 1035mn), le 9 par un dérangement d'installation sur la branche ouest de la ligne grande vitesse (18TGV; 169mn)
2015-06; Sud-Est; MONTPELLIER; PARIS LYON; 336; 336; 0; 62; 81.5; Des travaux de modernisation de l'infrastructure ont perturbé la régularité de cette relation en Juin
2015-06; Atlantique; NANTES; PARIS MONTPARNASSE; 563; 563; 0; 74; 86.9; 
2015-06; Sud-Est; PARIS LYON; GRENOBLE; 241; 241; 0; 10; 95.9; 
2015-06; Sud-Est; PARIS LYON; PERPIGNAN; 154; 154; 0; 44; 71.4; Des travaux de modernisation de l'infrastructure ont perturbé la régularité de cette relation en Juin
2015-06; Atlantique; PARIS MONTPARNASSE; LE MANS; 435; 432; 3; 74; 82.9; 
2015-06; Atlantique; PARIS MONTPARNASSE; NANTES; 554; 551; 3; 73; 86.8; 
2015-06; Atlantique; PARIS MONTPARNASSE; ST PIERRE DES CORPS; 446; 445; 1; 72; 83.8; 
2015-06; Nord; PARIS NORD; DOUAI; 205; 205; 0; 19; 90.7; 
2015-06; Nord; PARIS NORD; LILLE; 611; 610; 1; 58; 90.5; 
2015-06; Atlantique; RENNES; LYON PART DIEU; 94; 94; 0; 21; 77.7; Cette OD a été touchée le 11 par la panne d'un TGV sur le réseau Sud-Est vers Chevry Cossigny (60TGV; 1478mn), le 6 par un dérangement d'installation au Creusot (49 TGv; 1662mn), 2 accidents de personne le 2 à Lyon (44TGV; 1314mn) et le 22 à Pasilly (27TGV; 1347mn), le 24 par le heurt d'un animal sur le réseau Sud-Est vers Sully (26TGV; 660mn), le 26 par la rupture d'un appareil de voie en sortie de ligne grande vitesse avant Le Mans (52TGV; 923mn), le 3 par l'arrêt de sécurité d'un train dans les tunnels avant Paris (22TGV; 1035mn), le 9 par un dérangement d'installation sur la branche ouest de la ligne grande vitesse (18TGV; 169mn)
2015-06; Atlantique; RENNES; PARIS MONTPARNASSE; 560; 560; 0; 59; 89.5; 
2015-06; Atlantique; ST PIERRE DES CORPS; PARIS MONTPARNASSE; 407; 405; 2; 75; 81.5; 
2015-06; Atlantique; TOULOUSE MATABIAU; PARIS MONTPARNASSE; 177; 177; 0; 29; 83.6; Cette OD a été touchée le 4 par un rail cassé couplé à un affaissement caténaire à la sortie de Paris (172 TGV; 7664mn), le 5 par diverses limitations de vitesses liées aux fortes chaleurs dont une à la sortie de Paris reduisant fortement le débit des trains (82TGV; 985mn), 2 incendies aux abords des voies entre Droué et Vendôme le 25 (48TGV; 1826mn) et le 30 (30TGV; 1773mn), le 3 par l'arrêt de sécurité d'un train dans les tunnels avant Paris (22TGV; 1035mn) lié à une presence dans les voies en sortie de gare Montparnasse (16TGV; 419mn),  le 22 par le heurt de 2 chevreuils sur la branche sud de la ligne grande vitesse (15TGV; 328mn)
2015-06; Sud-Est; VALENCE ALIXAN TGV; PARIS LYON; 249; 249; 0; 59; 76.3; Des travaux de modernisation de l'infrastructure ont perturbé la régularité de cette relation en Juin
2013-05; Sud-Est; PARIS LYON; AIX EN PROVENCE TGV; 422; 422; 0; 63; 85.1; 
2013-05; Nord; DUNKERQUE; PARIS NORD; 88; 88; 0; 2; 97.7; 
2013-05; Nord; PARIS NORD; DUNKERQUE; 129; 129; 0; 6; 95.3; 
2013-05; Sud-Est; GRENOBLE; PARIS LYON; 225; 224; 1; 22; 90.2; 
2013-05; Sud-Est; PARIS LYON; GRENOBLE; 237; 237; 0; 25; 89.5; 
2013-05; Atlantique; PARIS MONTPARNASSE; ANGERS SAINT LAUD; 446; 445; 1; 21; 95.3; 
2013-05; Nord; MARSEILLE ST CHARLES; LILLE; 127; 127; 0; 31; 75.6; Plusieurs actes de malveillance (vols de câbles, pose de plaques de béton sur les voies de la LGV Nord), inondations sur Avignon du 19 au 22 mai.
2013-05; Atlantique; LYON PART DIEU; RENNES; 31; 31; 0; 2; 93.5; 
2013-05; Sud-Est; PARIS LYON; MACON LOCHE; 216; 216; 0; 15; 93.1; 
2013-05; Sud-Est; PARIS LYON; MARSEILLE ST CHARLES; 486; 486; 0; 40; 91.8; 
2013-05; Sud-Est; PARIS LYON; MONTPELLIER; 348; 348; 0; 26; 92.5; 
2013-05; Atlantique; PARIS MONTPARNASSE; TOURS; 150; 150; 0; 16; 89.3; 
2013-06; Atlantique; PARIS MONTPARNASSE; LAVAL; 237; 232; 5; 11; 95.3; 
2013-06; Atlantique; LE MANS; PARIS MONTPARNASSE; 445; 440; 5; 46; 89.5; 
2015-01; Sud-Est; PARIS LYON; SAINT ETIENNE CHATEAUCREUX; 113; 112; 1; 15; 86.6; 
2015-01; Atlantique; PARIS MONTPARNASSE; ANGERS SAINT LAUD; 406; 406; 0; 19; 95.3; 
2015-01; Atlantique; PARIS MONTPARNASSE; NANTES; 506; 506; 0; 41; 91.9; 
2015-01; Atlantique; QUIMPER; PARIS MONTPARNASSE; 191; 191; 0; 10; 94.8; 
2015-01; Atlantique; TOURS; PARIS MONTPARNASSE; 189; 189; 0; 25; 86.8; 
2015-02; Sud-Est; ANNECY; PARIS LYON; 184; 184; 0; 21; 88.6; 
2015-02; Sud-Est; BELLEGARDE (AIN); PARIS LYON; 233; 233; 0; 45; 80.7; 
2015-02; Sud-Est; DIJON VILLE; PARIS LYON; 419; 419; 0; 33; 92.1; De fortes chutes de neige dans le Jura ont provoqué des retards importants sur certains TGV Lausanne-Paris
2015-02; Nord; LILLE; LYON PART DIEU; 250; 250; 0; 26; 89.6; 
2015-02; Nord; LILLE; PARIS NORD; 547; 546; 1; 44; 91.9; 
2015-02; Sud-Est; LYON PART DIEU; PARIS LYON; 577; 575; 2; 44; 92.3; 
2015-02; Atlantique; LYON PART DIEU; RENNES; 84; 84; 0; 15; 82.1; Cette ligne a été touchée par deux incidents majeurs : un problème caténaire à Laval le 13/02 (83 TGV de 7min à 5h59) \& une absence d’alimentation électrique à Lyon Part Dieu le 26-02 (72 TGV jusqu’à 1h58 de retard). On compte également 5 incidents externes dont quatre survenus sur les Lignes à Grande Vitesse : un accident de personne à Rouvray le 7/02 (19 TGV retardés de 30min à 4h04) \& un autre le 18/02 près de Paris (26 TGV retardés de 12 à 48min) ; le heurt d’un chevreuil à Dollon le 01/02 retarde 33 TGV (11min à 2h34) \& celui d’un sanglier vers Mâcon touche 27 TGV (6 à 59min) ; les limitations de vitesse sur la ligne à Grande Vitesse Sud-Est le 21-02 suite à la présence de neige retardent 81 TGV (5 à 27min). Par ailleurs, un dérangement des installations à Massy  le 11/02 touche 16 TGV (11 à 38min) \& on compte également trois incidents caténaires les 3, 16 \& 18/02 : l’un près de Lyon (Grenay) le 3/02 (32 TGV de 5min à 1h28), le suivant sur la Ligne à Grande Vitesse Atlantique (37 TGV de 16min à 1h41) \& le troisième à Massy-Verrières en région parisienne (22 TGV de 13min à 3h37). Enfin une défaillance de Matériel à Massy le 19/02 pénalise 16 TGV (15min \& 1h47). 
2015-02; Sud-Est; MACON LOCHE; PARIS LYON; 181; 181; 0; 21; 88.4; 
2015-02; Sud-Est; NIMES; PARIS LYON; 270; 270; 0; 28; 89.6; 
2015-02; Sud-Est; PARIS LYON; AIX EN PROVENCE TGV; 400; 400; 0; 20; 95.0; 
2015-02; Sud-Est; PARIS LYON; ANNECY; 160; 160; 0; 11; 93.1; 
2015-02; Sud-Est; PARIS LYON; AVIGNON TGV; 461; 461; 0; 28; 93.9; 
2015-02; Sud-Est; PARIS LYON; DIJON VILLE; 429; 429; 0; 21; 95.1; 
2015-02; Sud-Est; PARIS LYON; MACON LOCHE; 197; 197; 0; 10; 94.9; 
2015-02; Sud-Est; PARIS LYON; MARSEILLE ST CHARLES; 412; 412; 0; 21; 94.9; 
2015-02; Sud-Est; PARIS LYON; TOULON; 227; 227; 0; 15; 93.4; 
2015-02; Sud-Est; PARIS LYON; VALENCE ALIXAN TGV; 264; 264; 0; 18; 93.2; 
2015-02; Atlantique; PARIS MONTPARNASSE; TOULOUSE MATABIAU; 126; 126; 0; 9; 92.9; 
2015-02; Nord; PARIS NORD; DOUAI; 189; 189; 0; 16; 91.5; 
2015-02; Atlantique; ST MALO; PARIS MONTPARNASSE; 92; 91; 1; 9; 90.1; 
2015-02; Atlantique; ST PIERRE DES CORPS; PARIS MONTPARNASSE; 396; 395; 1; 62; 84.3; 
2015-02; Atlantique; TOULOUSE MATABIAU; PARIS MONTPARNASSE; 155; 155; 0; 17; 89.0; Cette destination a été touchée par 6 incidents externes dont deux survenus sur la Ligne à Grande Vitesse : un accident de personne à Rouvray le 7/02 (19 TGV retardés de 30min à 4h04), un autre le 18/02 près de Paris (26 TGV retardés de 12 à 48min) \& à Marmande le 26/02 (2 TGV de 1h51 \& 1h54); le heurt d’un chevreuil à Dollon le 01/02 retarde 33 TGV (11min à 2h34), celui d’un sanglier le 20/02 vers Angoulême touche 5 TGV (14min à 4h) \& les intempéries sur la région Midi-Pyrénées le 3/02 touche 3 TGV (28min à 1h06). Par ailleurs, un dérangement lié à une autre entreprise ferroviaire de transport de marchandises le 4/02 au Nord de Poitiers ralentit 6 TGV (13 à 54min), un autre à Massy  le 11/02 touche 16 TGV (11 à 38min) \& on compte également deux incidents caténaires les 16 \& 18/02 : le premier sur la Ligne à Grande Vitesse (37 TGV de 16min à 1h41) \& le second à Massy-Verrières en région parisienne (22 TGV de 13min à 3h37). Enfin une défaillance de Matériel à Massy le 19/02 pénalise 16 TGV (15min \& 1h47).
2015-02; Atlantique; VANNES; PARIS MONTPARNASSE; 206; 205; 1; 19; 90.7; Cette ligne a été touchée par un incident majeur le 13/02 : un problème caténaire à Laval retarde 83 TGV de 7min à 5h59. On compte également 4 incidents externes dont deux survenus sur la Ligne à Grande Vitesse : un accident de personne à Rouvray le 7/02 (19 TGV retardés de 30min à 4h04), un à Questembert le 14/02 (3 TGV 16min \& 1h50) \& un autre le 18/02 près de Paris (26 TGV retardés de 12 à 48min) ; le heurt d’un chevreuil à Dollon le 01/02 retarde 33 TGV (11min à 2h34). Par ailleurs, un dérangement des installations à Massy  le 11/02 touche 16 TGV (11 à 38min) \& on compte également deux incidents caténaires les 16 \& 18/02 : le premier sur la Ligne à Grande Vitesse (37 TGV de 16min à 1h41) \& le second à Massy-Verrières en région parisienne (22 TGV de 13min à 3h37). Enfin une défaillance de Matériel à Massy le 19/02 pénalise 16 TGV (15min \& 1h47).
2015-03; Atlantique; BREST; PARIS MONTPARNASSE; 231; 231; 0; 13; 94.4; 
2015-03; Atlantique; LA ROCHELLE VILLE; PARIS MONTPARNASSE; 222; 222; 0; 13; 94.1; Cette OD a été touchée par les événements suivants:  un heurt de chevreuil sur la ligne grande vitesse à Vendôme le 2 (28 TGV; de 13' à 58'), un accident de personne à Poitiers le 8 (6 TGV; de 2h06 à 3h28), un feu de traverses à l'entrée de Paris Montparnasse le 14 (6 TGV; 12' à 33') ainsi que la divagation d'un chien sur la ligne grande vitesse à St Arnoult le 31 (13 TGV; de 12' à 36'). On compte par ailleurs 4 dérangements d'installations sur la ligne grande vitesse les 1, 2, 16 et 23 touchant jusqu'à 7 trains par jour avec des retards allant de 12' à 1h05 ainsi qu'un dérangement à Lusignan le 19 (3 TGV; de 12' à 1h38) et un autre au niveau de  Poitiers le 23 (6 TGV; 18' à 2h10). Par ailleurs, des limitations inopinées de vitesse entre Poitiers et Niort ont touché tous les trains du 29 au 31.
2015-03; Atlantique; LAVAL; PARIS MONTPARNASSE; 241; 241; 0; 14; 94.2; 
2015-03; Sud-Est; LE CREUSOT MONTCEAU MONTCHANIN; PARIS LYON; 197; 197; 0; 10; 94.9; 
2015-03; Sud-Est; LYON PART DIEU; MARSEILLE ST CHARLES; 597; 596; 1; 94; 84.2; 
2015-03; Sud-Est; MONTPELLIER; PARIS LYON; 273; 273; 0; 39; 85.7; 
2015-03; Sud-Est; MULHOUSE VILLE; PARIS LYON; 293; 293; 0; 10; 96.6; 
2015-03; Atlantique; NANTES; PARIS MONTPARNASSE; 562; 562; 0; 42; 92.5; 
2015-03; Sud-Est; NICE VILLE; PARIS LYON; 164; 162; 2; 29; 82.1; 
2015-03; Sud-Est; NIMES; PARIS LYON; 273; 273; 0; 45; 83.5; 
2015-03; Sud-Est; PARIS LYON; BELLEGARDE (AIN); 219; 219; 0; 14; 93.6; 
2015-03; Sud-Est; PARIS LYON; MARSEILLE ST CHARLES; 411; 410; 1; 19; 95.4; 
2015-03; Atlantique; PARIS MONTPARNASSE; LAVAL; 238; 238; 0; 12; 95.0; 
2015-03; Atlantique; PARIS MONTPARNASSE; LE MANS; 455; 455; 0; 36; 92.1; 
2015-03; Atlantique; PARIS MONTPARNASSE; ST MALO; 102; 102; 0; 5; 95.1; Cette OD a été touchée par les événements suivants:  un heurt de chevreuil sur la ligne grande vitesse à Vendôme le 2 (28 TGV; de 13' à 58'), un accident de personne à Cesson le 26 (11 TGV; 12' à 2h24'), un feu de traverses à l'entrée de Paris Montparnasse le 14 (6 TGV; 12' à 33'), un incendie aux abords des voies vers Laval le 23  (7 TGV; 22' à 2h05) ainsi que la divagation d'un chien sur la ligne grande vitesse à St Arnoult le 31 (13 TGV; de 12' à 36'). On compte par ailleurs 4 dérangements d'installations sur la ligne grande vitesse les 1, 2, 16 et 23 touchant jusqu'à 7 trains par jour avec des retards allant de 12' à 1h05.
2015-03; Atlantique; PARIS MONTPARNASSE; ST PIERRE DES CORPS; 460; 460; 0; 23; 95.0; 
2015-03; Nord; PARIS NORD; DUNKERQUE; 262; 261; 1; 17; 93.5; 
2015-03; Atlantique; POITIERS; PARIS MONTPARNASSE; 495; 495; 0; 37; 92.5; 
2015-03; Sud-Est; SAINT ETIENNE CHATEAUCREUX; PARIS LYON; 102; 102; 0; 6; 94.1; 
2015-03; Atlantique; ST PIERRE DES CORPS; PARIS MONTPARNASSE; 439; 439; 0; 35; 92.0; 
2015-03; Est; STRASBOURG; NANTES; 41; 41; 0; 3; 92.7; 
2015-03; Atlantique; VANNES; PARIS MONTPARNASSE; 216; 216; 0; 16; 92.6; Cette OD a été touchée par les événements suivants:  un heurt de chevreuil sur la ligne grande vitesse le 2 (28 TGV; de 13' à 58'), 2 accidents de personne à Vannes le 6 (6 TGV; de 13' à 1h51) et à Cesson le 26 (11 TGV; 12' à 2h24'), un feu de traverses à l'entrée de Paris Montparnasse le 14 (6 TGV; 12' à 33'), un incendie aux abords des voies vers Laval le 23  (7 TGV; 22' à 2h05) ainsi que la divagation d'un chien sur la ligne grande vitesse à St Arnoult le 31 (13 TGV; de 12' à 36'). On compte par ailleurs 4 dérangements d'installations sur la ligne grande vitesse les 1, 2, 16 et 23 touchant jusqu'à 7 trains par jour avec des retards allant de 12' à 1h05 et un incident caténaire le 22 vers Redon (9 TGV; de 12' à 80').
2015-04; Sud-Est; AIX EN PROVENCE TGV; PARIS LYON; 450; 450; 0; 52; 88.4; 
2015-04; Atlantique; BREST; PARIS MONTPARNASSE; 221; 221; 0; 7; 96.8; Cette ligne a été touchée notamment par des causes externes: un accident  de personne à Cesson-Sévigné près de Rennes le 04-04 (7 TGV retardés pour 583min), en région parisienne le 05/04  (7 TGV retardés pour 333min) mais également un colis suspect en gare de Paris Montparnasse le 08/04 (8 TGV touchés pour 119min), un incendie aux abords des voies à Port Brillet (Laval) le 13/04 (3 TGV retardés pour 77min) \& le heurt d'une biche à Voutré - entre Le Mans \& Laval - le 23/04 avec 5 TGV retardés pour 151min.  Des dérangements des installations ont également entraîné des retards notamment sur l'ensemble de la Ligne à Grande Vitesse le 07-04 (12 circulations impactées pour 150min), à Laval le 28/04 (9 TGV retardés pour 302min) \& à Rennes le 29/04 (16 TGV impactés pour 314min). Par ailleurs, la panne d'un train assurant les travaux à St Pierre-la-Cour - entre Laval \& Rennes- le 09-04 a retardé 9 TGV pour 361min.
2015-04; Sud-Est; DIJON VILLE; PARIS LYON; 450; 449; 1; 28; 93.8; 
2015-04; Sud-Est; GRENOBLE; PARIS LYON; 220; 220; 0; 11; 95.0; 3 heurts d'animaux sur ligne à grande vitesse et un accident de personne au Vert de Maisons ont généré des retards importants en Avril 
2015-04; Nord; LILLE; LYON PART DIEU; 269; 269; 0; 23; 91.4; Les principaux évènements survenus au mois d'avril sont :
2 accidents de personne : Marne la Vallée et Choisy le Roi le 05
Heurt d'animaux  : Fresnoy le 14
Des colis suspects à Roissy les 01 et 30
Défaut d'alimentation à Cesseins le 07
Plusieurs dérangement d'installations : Oignies le 03, Montanay le 08 et Sathonay le 24
Des travaux importants sur la LGV  entrainant des limitations de vitesse, essentiellement à Ressons
2015-04; Atlantique; LYON PART DIEU; RENNES; 85; 85; 0; 17; 80.0; 
2015-04; Sud-Est; MONTPELLIER; PARIS LYON; 348; 344; 4; 52; 84.9; 
2015-04; Sud-Est; NIMES; PARIS LYON; 347; 344; 3; 65; 81.1; 
2015-04; Est; PARIS EST; REIMS; 207; 207; 0; 9; 95.7; 
2015-04; Sud-Est; PARIS LYON; BELLEGARDE (AIN); 241; 240; 1; 22; 90.8; 
2015-04; Sud-Est; PARIS LYON; NICE VILLE; 217; 217; 0; 72; 66.8; Des travaux de modernisation de l'infrastructure ont perturbé la régularité de cette relation en Avril.
2015-04; Atlantique; PARIS MONTPARNASSE; ST MALO; 100; 100; 0; 2; 98.0; 
2015-04; Atlantique; PARIS MONTPARNASSE; VANNES; 219; 219; 0; 8; 96.3; 
2015-04; Nord; PARIS NORD; DUNKERQUE; 254; 254; 0; 18; 92.9; 
2015-04; Nord; PARIS NORD; LILLE; 602; 600; 2; 36; 94.0; 
2015-05; Atlantique; ANGOULEME; PARIS MONTPARNASSE; 340; 340; 0; 14; 95.9; 
2015-05; Sud-Est; AVIGNON TGV; PARIS LYON; 567; 567; 0; 102; 82.0; 
2015-05; Sud-Est; BESANCON FRANCHE COMTE TGV; PARIS LYON; 215; 215; 0; 2; 99.1; 
2015-05; Atlantique; BREST; PARIS MONTPARNASSE; 213; 213; 0; 11; 94.8; Cette ligne a été retardée par : un accident de personne à Yvre-l'Evêque le 14/05 (15 TGV soit 1456 minutes), une disjonction sur la LGV le 25/05 (6 TGV soit 87 minutes), heurt d'un animal sur la LGV le 12/05 (30 TGV soit 1799 minutes) et heurt d'une voiture le 06/05 à Vitré (18 TGV soit 1704 minutes)
2015-05; Atlantique; LA ROCHELLE VILLE; PARIS MONTPARNASSE; 189; 189; 0; 6; 96.8; Cette ligne a été touché par des causes Infra : un bruit en toiture entrainant un arrêt à St Léger d'un TGV le 05/05 (retarde 4 TGV de 121 minutes), heurt d'un animal sur la LGV le 12/05 (30 TGV soit 1799 minutes), une défaillance matérielle vers Massy le 18/05 (9 TGV soit 239 minutes), la fuite d'huile sur un transformateur d'un TGV vers Port de Piles le 27/05 ( 10 TGV soit 934 minutes), une disjonction sur la LGV le 25/05 (6 TGV soit 87 minutes) et un incident caténaire à Montmoreau le 29/05 (3 TGV , 119 minutes). 
2015-05; Atlantique; LAVAL; PARIS MONTPARNASSE; 236; 236; 0; 9; 96.2; Cette ligne a été retardée par : un accident de personne à Yvre-l'Evêque le 14/05 (15 TGV soit 1456 minutes), une disjonction sur la LGV le 25/05 (6 TGV soit 87 minutes), heurt d'un animal sur la LGV le 12/05 (30 TGV soit 1799 minutes) et heurt d'une voiture le 06/05 à Vitré (18 TGV soit 1704 minutes)
2015-05; Atlantique; LE MANS; PARIS MONTPARNASSE; 445; 445; 0; 46; 89.7; 
2015-05; Nord; LILLE; PARIS NORD; 576; 576; 0; 37; 93.6; 
2015-05; Sud-Est; LYON PART DIEU; PARIS LYON; 620; 620; 0; 66; 89.4; 
2015-05; Sud-Est; MARSEILLE ST CHARLES; LYON PART DIEU; 579; 579; 0; 155; 73.2; 
2015-05; Est; NANCY; PARIS EST; 281; 280; 1; 8; 97.1; 
2015-05; Sud-Est; NIMES; PARIS LYON; 350; 349; 1; 66; 81.1; 
2015-05; Sud-Est; PARIS LYON; DIJON VILLE; 460; 455; 5; 9; 98.0; 
2015-05; Sud-Est; PARIS LYON; LE CREUSOT MONTCEAU MONTCHANIN; 192; 192; 0; 24; 87.5; 
2015-05; Sud-Est; PARIS LYON; MACON LOCHE; 192; 192; 0; 13; 93.2; 
2015-05; Sud-Est; PARIS LYON; MULHOUSE VILLE; 289; 288; 1; 8; 97.2; 
2015-05; Sud-Est; PARIS LYON; TOULON; 294; 293; 1; 44; 85.0; 
2015-05; Nord; PARIS NORD; DOUAI; 205; 205; 0; 22; 89.3; 
2015-05; Atlantique; RENNES; LYON PART DIEU; 99; 99; 0; 13; 86.9; 
2015-05; Est; STRASBOURG; NANTES; 56; 56; 0; 1; 98.2; 
2015-06; Sud-Est; BELLEGARDE (AIN); PARIS LYON; 236; 235; 1; 20; 91.5; 
2015-06; Sud-Est; BESANCON FRANCHE COMTE TGV; PARIS LYON; 224; 224; 0; 8; 96.4; 
2015-06; Atlantique; BORDEAUX ST JEAN; PARIS MONTPARNASSE; 643; 641; 2; 89; 86.1; Cette OD a été touchée le 4 par un rail cassé couplé à un affaissement caténaire à la sortie de Paris (172 TGV; 7664mn), le 5 par diverses limitations de vitesses liées aux fortes chaleurs dont une à la sortie de Paris reduisant fortement le débit des trains (82TGV; 985mn), 2 incendies aux abords des voies entre Droué et Vendôme le 25 (48TGV; 1826mn) et le 30 (30TGV; 1773mn), le 3 par l'arrêt de sécurité d'un train dans les tunnels avant Paris (22TGV; 1035mn) lié à une presence dans les voies en sortie de gare Montparnasse (16TGV; 419mn),  le 22 par le heurt de 2 chevreuils sur la branche sud de la ligne grande vitesse (15TGV; 328mn)
2015-06; Nord; DUNKERQUE; PARIS NORD; 257; 256; 1; 23; 91.0; 
2015-06; Atlantique; LE MANS; PARIS MONTPARNASSE; 461; 461; 0; 75; 83.7; 
2015-06; Sud-Est; LYON PART DIEU; MARSEILLE ST CHARLES; 611; 609; 2; 183; 70.0; Des travaux de modernisation de l'infrastructure ont perturbé la régularité de cette relation en Juin
2015-06; Sud-Est; LYON PART DIEU; MONTPELLIER; 340; 338; 2; 122; 63.9; Des travaux de modernisation de l'infrastructure ont perturbé la régularité de cette relation en Juin
2015-06; Est; NANCY; PARIS EST; 287; 287; 0; 14; 95.1; 
2015-06; Sud-Est; NICE VILLE; PARIS LYON; 184; 184; 0; 44; 76.1; Des travaux de modernisation de l'infrastructure ont perturbé la régularité de cette relation en Juin
2015-06; Est; PARIS EST; STRASBOURG; 430; 430; 0; 42; 90.2; 
2015-06; Sud-Est; PARIS LYON; AIX EN PROVENCE TGV; 441; 440; 1; 60; 86.4; 
2015-06; Sud-Est; PARIS LYON; ANNECY; 162; 162; 0; 11; 93.2; 
2015-06; Sud-Est; PARIS LYON; AVIGNON TGV; 531; 530; 1; 77; 85.5; 
2015-06; Sud-Est; PARIS LYON; DIJON VILLE; 468; 468; 0; 20; 95.7; 
2015-06; Sud-Est; PARIS LYON; MARSEILLE ST CHARLES; 463; 462; 1; 53; 88.5; 
2013-05; Sud-Est; LYON PART DIEU; MONTPELLIER; 406; 406; 0; 57; 86.0; 
2013-05; Sud-Est; MONTPELLIER; LYON PART DIEU; 351; 351; 0; 49; 86.0; 
2013-05; Sud-Est; MACON LOCHE; PARIS LYON; 190; 189; 1; 24; 87.3; 
2013-05; Est; PARIS EST; METZ; 315; 315; 0; 16; 94.9; 
2013-05; Sud-Est; MONTPELLIER; PARIS LYON; 363; 362; 1; 27; 92.5; 
2013-05; Sud-Est; MULHOUSE VILLE; PARIS LYON; 318; 318; 0; 22; 93.1; 
2013-05; Est; NANTES; STRASBOURG; 62; 62; 0; 2; 96.8; 
2013-05; Sud-Est; PARIS LYON; NIMES; 346; 346; 0; 33; 90.5; 
2013-05; Sud-Est; PERPIGNAN; PARIS LYON; 154; 154; 0; 11; 92.9; 
2013-05; Atlantique; PARIS MONTPARNASSE; POITIERS; 518; 518; 0; 18; 96.5; 
2013-05; Sud-Est; TOULON; PARIS LYON; 250; 250; 0; 38; 84.8; 
2013-05; Nord; PARIS NORD; ARRAS; 346; 346; 0; 45; 87.0; 
2013-05; Sud-Est; PARIS LYON; AVIGNON TGV; 382; 382; 0; 33; 91.4; 
2013-05; Sud-Est; BELLEGARDE (AIN); PARIS LYON; 269; 269; 0; 24; 91.1; 
2013-05; Sud-Est; BESANCON FRANCHE COMTE TGV; PARIS LYON; 220; 220; 0; 18; 91.8; 
2013-05; Sud-Est; PARIS LYON; BESANCON FRANCHE COMTE TGV; 236; 236; 0; 19; 91.9; 
2013-06; Atlantique; BREST; PARIS MONTPARNASSE; 172; 170; 2; 5; 97.1; 
2013-06; Sud-Est; DIJON VILLE; PARIS LYON; 460; 451; 9; 35; 92.2; 
2013-06; Sud-Est; GRENOBLE; PARIS LYON; 234; 227; 7; 18; 92.1; 
2013-06; Atlantique; LAVAL; PARIS MONTPARNASSE; 241; 238; 3; 18; 92.4; 
2013-06; Atlantique; PARIS MONTPARNASSE; LE MANS; 433; 427; 6; 26; 93.9; 
2013-06; Atlantique; PARIS MONTPARNASSE; ANGERS SAINT LAUD; 426; 420; 6; 23; 94.5; 
2013-06; Atlantique; LYON PART DIEU; RENNES; 31; 30; 1; 1; 96.7; 
2013-06; Sud-Est; PARIS LYON; MACON LOCHE; 209; 203; 6; 9; 95.6; 
2013-06; Sud-Est; PARIS LYON; NICE VILLE; 173; 169; 4; 31; 81.7; Plusieurs limitations de vitesse sur des zones de modernisation de l'infrastructure ont fragilisé la circulation des trains sur l'axe Marseille-Nice.
2013-06; Sud-Est; PERPIGNAN; PARIS LYON; 149; 148; 1; 11; 92.6; 
2013-06; Sud-Est; PARIS LYON; PERPIGNAN; 152; 150; 2; 13; 91.3; 
2013-06; Atlantique; PARIS MONTPARNASSE; POITIERS; 494; 482; 12; 16; 96.7; 
2013-06; Est; STRASBOURG; PARIS EST; 462; 455; 7; 73; 84.0; 
2013-06; Sud-Est; VALENCE ALIXAN TGV; PARIS LYON; 249; 244; 5; 28; 88.5; 
2013-06; Sud-Est; PARIS LYON; BELLEGARDE (AIN); 261; 258; 3; 49; 81.0; 
2013-07; Sud-Est; CHAMBERY CHALLES LES EAUX; PARIS LYON; 213; 213; 0; 44; 79.3; Des orages et fortes chaleurs, des vols de câbles et accidents de personne perturbent la circulation des TGV.
2013-07; Sud-Est; PARIS LYON; GRENOBLE; 204; 204; 0; 19; 90.7; 
2013-07; Atlantique; PARIS MONTPARNASSE; LA ROCHELLE VILLE; 224; 224; 0; 17; 92.4; 
2013-07; Atlantique; PARIS MONTPARNASSE; LAVAL; 231; 231; 0; 21; 90.9; 
2013-07; Atlantique; PARIS MONTPARNASSE; LE MANS; 434; 434; 0; 69; 84.1; 
2013-07; Atlantique; ANGERS SAINT LAUD; PARIS MONTPARNASSE; 443; 442; 1; 69; 84.4; 
2013-07; Nord; LILLE; LYON PART DIEU; 310; 310; 0; 48; 84.5; Fortes chaleurs, orages, incendies et coupures de tension.
2013-07; Nord; LILLE; PARIS NORD; 578; 576; 2; 99; 82.8; 
2013-07; Sud-Est; LYON PART DIEU; PARIS LYON; 540; 540; 0; 36; 93.3; 
2013-07; Sud-Est; PARIS LYON; LYON PART DIEU; 537; 537; 0; 21; 96.1; 
2013-07; Sud-Est; MARSEILLE ST CHARLES; PARIS LYON; 464; 464; 0; 65; 86.0; Des orages et fortes chaleurs, des vols de câbles et accidents de personne perturbent la circulation des TGV.
2013-07; Atlantique; ANGOULEME; PARIS MONTPARNASSE; 357; 357; 0; 63; 82.4; Fortes chaleurs les 7 et 8. Fortes intempéries les 26 et 27. Fuite de gazole d'un train de fret. Pannes de train. Heurt d'animal.
2013-07; Sud-Est; PARIS LYON; PERPIGNAN; 162; 162; 0; 26; 84.0; Des orages et fortes chaleurs, des vols de câbles et accidents de personne perturbent la circulation des TGV.
2013-07; Atlantique; PARIS MONTPARNASSE; POITIERS; 503; 502; 1; 31; 93.8; 
2013-07; Atlantique; PARIS MONTPARNASSE; RENNES; 543; 543; 0; 64; 88.2; 
2013-07; Sud-Est; SAINT ETIENNE CHATEAUCREUX; PARIS LYON; 118; 118; 0; 14; 88.1; 
2013-07; Atlantique; ST PIERRE DES CORPS; PARIS MONTPARNASSE; 423; 423; 0; 98; 76.8; Fortes chaleurs les 7 et 8. Fortes intempéries les 26 et 27. Fuite de gazole d'un train de fret. Pannes de train.
2013-07; Atlantique; PARIS MONTPARNASSE; ST PIERRE DES CORPS; 449; 448; 1; 60; 86.6; 
2013-07; Sud-Est; PARIS LYON; TOULON; 248; 248; 0; 45; 81.9; Des orages et fortes chaleurs, des vols de câbles et accidents de personne perturbent la circulation des TGV.
2013-07; Nord; ARRAS; PARIS NORD; 321; 320; 1; 55; 82.8; 
2013-07; Sud-Est; BESANCON FRANCHE COMTE TGV; PARIS LYON; 231; 231; 0; 9; 96.1; 
2013-07; Sud-Est; PARIS LYON; BESANCON FRANCHE COMTE TGV; 235; 235; 0; 14; 94.0; 
2013-07; Atlantique; BORDEAUX ST JEAN; PARIS MONTPARNASSE; 687; 686; 1; 109; 84.1; Fortes chaleurs les 7 et 8. Fortes intempéries les 26 et 27. Fuite de gazole d'un train de fret. Pannes de train. Heurt d'animal.
2013-08; Sud-Est; PARIS LYON; AIX EN PROVENCE TGV; 437; 437; 0; 82; 81.2; 
2013-08; Sud-Est; CHAMBERY CHALLES LES EAUX; PARIS LYON; 202; 202; 0; 26; 87.1; 
2013-08; Sud-Est; DIJON VILLE; PARIS LYON; 467; 467; 0; 25; 94.6; 
2013-08; Sud-Est; LE CREUSOT MONTCEAU MONTCHANIN; PARIS LYON; 216; 216; 0; 24; 88.9; 
2013-08; Nord; LILLE; LYON PART DIEU; 308; 308; 0; 26; 91.6; 
2013-08; Nord; LYON PART DIEU; LILLE; 250; 250; 0; 70; 72.0; Défauts d'alimentation électrique à Lille et Lyon, accident de personne à Lille, heurts de Chevreuil du côté de Compiègne et d'Auxerre.
2013-08; Sud-Est; MONTPELLIER; LYON PART DIEU; 347; 346; 1; 84; 75.7; Deux incidents caténaires, des vols de câbles, des heurts d'animaux sauvages ont fortement perturbé la circulation des TGV en août.
2013-08; Est; STRASBOURG; NANTES; 52; 52; 0; 4; 92.3; 
2013-08; Sud-Est; PERPIGNAN; PARIS LYON; 167; 167; 0; 22; 86.8; 
2013-08; Sud-Est; PARIS LYON; SAINT ETIENNE CHATEAUCREUX; 119; 119; 0; 10; 91.6; 
2013-08; Nord; ARRAS; PARIS NORD; 324; 324; 0; 30; 90.7; 
2013-08; Atlantique; VANNES; PARIS MONTPARNASSE; 171; 171; 0; 9; 94.7; 
2013-08; Atlantique; PARIS MONTPARNASSE; VANNES; 212; 212; 0; 18; 91.5; 
2013-09; Sud-Est; AIX EN PROVENCE TGV; PARIS LYON; 365; 365; 0; 53; 85.5; 
2013-09; Sud-Est; PARIS LYON; AIX EN PROVENCE TGV; 361; 361; 0; 55; 84.8; 
2013-09; Atlantique; PARIS MONTPARNASSE; BREST; 143; 143; 0; 10; 93.0; 
2013-09; Nord; PARIS NORD; DOUAI; 178; 178; 0; 14; 92.1; 
2013-09; Nord; DUNKERQUE; PARIS NORD; 98; 98; 0; 7; 92.9; 
2013-09; Sud-Est; GRENOBLE; PARIS LYON; 210; 210; 0; 14; 93.3; 
2013-09; Atlantique; LA ROCHELLE VILLE; PARIS MONTPARNASSE; 194; 194; 0; 9; 95.4; 
2013-09; Atlantique; PARIS MONTPARNASSE; LAVAL; 200; 199; 1; 16; 92.0; 
2013-09; Sud-Est; LE CREUSOT MONTCEAU MONTCHANIN; PARIS LYON; 190; 190; 0; 21; 88.9; 
2013-09; Atlantique; PARIS MONTPARNASSE; LE MANS; 388; 387; 1; 44; 88.6; 
2013-09; Atlantique; ANGERS SAINT LAUD; PARIS MONTPARNASSE; 402; 402; 0; 31; 92.3; 
2013-09; Atlantique; PARIS MONTPARNASSE; ANGERS SAINT LAUD; 383; 383; 0; 32; 91.6; 
2013-09; Sud-Est; MONTPELLIER; PARIS LYON; 296; 296; 0; 31; 89.5; 
2013-09; Sud-Est; MULHOUSE VILLE; PARIS LYON; 272; 272; 0; 31; 88.6; Plusieurs incidents ont provoqué des retards importants (incidents techniques, panne d'un TER, vol de câbles).
2013-09; Est; PARIS EST; NANCY; 257; 257; 0; 5; 98.1; 
2013-09; Sud-Est; PARIS LYON; NIMES; 281; 281; 0; 37; 86.8; 
2013-09; Sud-Est; ANNECY; PARIS LYON; 122; 122; 0; 6; 95.1; 
2013-09; Sud-Est; PARIS LYON; ANNECY; 153; 153; 0; 12; 92.2; 
2013-09; Atlantique; RENNES; PARIS MONTPARNASSE; 487; 487; 0; 42; 91.4; 
2013-09; Atlantique; ST PIERRE DES CORPS; PARIS MONTPARNASSE; 372; 372; 0; 72; 80.6; 
2013-09; Atlantique; PARIS MONTPARNASSE; ST PIERRE DES CORPS; 393; 393; 0; 29; 92.6; 
2013-09; Est; PARIS EST; STRASBOURG; 417; 417; 0; 20; 95.2; 
2013-09; Atlantique; TOULOUSE MATABIAU; PARIS MONTPARNASSE; 92; 92; 0; 16; 82.6; 
2013-09; Nord; PARIS NORD; ARRAS; 291; 291; 0; 29; 90.0; 
2013-09; Sud-Est; PARIS LYON; BESANCON FRANCHE COMTE TGV; 206; 206; 0; 18; 91.3; 
2013-09; Atlantique; BORDEAUX ST JEAN; PARIS MONTPARNASSE; 582; 582; 0; 69; 88.1; Bagages abandonnés, dérangements de signalisation, panne d'un train travaux et d'un TER, heurt d'un sanglier, accident de personne.
2013-10; Sud-Est; CHAMBERY CHALLES LES EAUX; PARIS LYON; 208; 207; 1; 33; 84.1; 
2013-10; Sud-Est; PARIS LYON; CHAMBERY CHALLES LES EAUX; 216; 216; 0; 36; 83.3; 
2013-10; Nord; DOUAI; PARIS NORD; 186; 186; 0; 16; 91.4; 
2013-10; Atlantique; LAVAL; PARIS MONTPARNASSE; 247; 246; 1; 37; 85.0; 
2013-10; Sud-Est; PARIS LYON; LE CREUSOT MONTCEAU MONTCHANIN; 209; 209; 0; 32; 84.7; 
2013-10; Atlantique; ANGERS SAINT LAUD; PARIS MONTPARNASSE; 473; 472; 1; 63; 86.7; 
2013-10; Sud-Est; LYON PART DIEU; MARSEILLE ST CHARLES; 632; 632; 0; 153; 75.8; Fragilité de cette liaison liée à la longueur du parcours des trains assurant cette desserte.
2013-04; Nord; MARSEILLE ST CHARLES; LILLE; 124; 124; 0; 38; 69.4; Nombreux accidents de personnes et incidents d'installations techniques et électriques) répartis sur l'ensemble du parcours.
2013-04; Nord; PARIS NORD; LILLE; 613; 611; 2; 80; 86.9; 
2013-04; Sud-Est; LYON PART DIEU; MARSEILLE ST CHARLES; 614; 613; 1; 111; 81.9; Malgré un temps de parcours relativement court entre ces deux gares, l'essentiel des trains assurant cette desserte sont origine de l'ouest ou du Nord de la France et ont donc parcouru une très longue distance et souvent déjà accumulé du retard avant même d'assurer cette desserte située en fin de parcours du train.
2013-04; Sud-Est; MONTPELLIER; PARIS LYON; 355; 355; 0; 37; 89.6; 
2013-04; Sud-Est; PARIS LYON; MULHOUSE VILLE; 303; 303; 0; 15; 95.0; 
2013-04; Sud-Est; PERPIGNAN; PARIS LYON; 146; 146; 0; 16; 89.0; 
2013-04; Atlantique; POITIERS; PARIS MONTPARNASSE; 487; 487; 0; 22; 95.5; Accident de personne à Poitiers, problème matériel sur le TGV 8412, acte de malveillance à Poitiers (gilet sur le fil d'alimentation électrique).
2013-04; Atlantique; PARIS MONTPARNASSE; POITIERS; 498; 498; 0; 13; 97.4; Problème matériel moteur sur le TGV 8871 impliquant un retard assez important pour les TGV le suivant, accident de personne à Poitiers et incident matériel sur le TGV 8387, nécessitant le transbordement des clients.
2013-04; Atlantique; QUIMPER; PARIS MONTPARNASSE; 137; 137; 0; 11; 92.0; 
2013-04; Est; REIMS; PARIS EST; 206; 206; 0; 2; 99.0; 
2013-04; Atlantique; PARIS MONTPARNASSE; ST PIERRE DES CORPS; 446; 446; 0; 27; 93.9; 
2013-04; Atlantique; PARIS MONTPARNASSE; TOURS; 145; 145; 0; 12; 91.7; 
2013-04; Nord; PARIS NORD; ARRAS; 328; 328; 0; 37; 88.7; 
2013-04; Atlantique; PARIS MONTPARNASSE; VANNES; 178; 178; 0; 17; 90.4; Le 15, incident matériel sur le TGV 8387 à l'entrée de la ligne à grande vitesse. Le 16, accident de personne entre Rennes et Vitré. Le 24, incident matériel sur le train de fret de l'entreprise Colas Rail. Le 27, incident matériel sur TER générant des retards sur les TGV. Le 30, incident caténaire en banlieue parisienne.
2013-04; Sud-Est; PARIS LYON; BELLEGARDE (AIN); 274; 274; 0; 32; 88.3; 
2013-04; Atlantique; BORDEAUX ST JEAN; PARIS MONTPARNASSE; 665; 665; 0; 49; 92.6; Fin des travaux de nuit à Bordeaux retardant les TGV du matin, accident de personne à Poitiers, problème matériel sur le TGV 8412, acte de malveillance à Poitiers (gilet sur le fil d'alimentation électrique), panne d'un TER en ligne au nord de Bordeaux.
2013-05; Atlantique; BREST; PARIS MONTPARNASSE; 164; 164; 0; 7; 95.7; 
2013-05; Atlantique; PARIS MONTPARNASSE; BREST; 177; 177; 0; 6; 96.6; 
2013-05; Sud-Est; PARIS LYON; DIJON VILLE; 475; 475; 0; 43; 90.9; 
2013-05; Atlantique; PARIS MONTPARNASSE; LA ROCHELLE VILLE; 228; 228; 0; 6; 97.4; 
2013-05; Sud-Est; LE CREUSOT MONTCEAU MONTCHANIN; PARIS LYON; 221; 221; 0; 29; 86.9; 
2013-05; Atlantique; LE MANS; PARIS MONTPARNASSE; 475; 474; 1; 49; 89.7; 
2013-05; Atlantique; ANGERS SAINT LAUD; PARIS MONTPARNASSE; 469; 468; 1; 30; 93.6; 
2013-05; Nord; LILLE; MARSEILLE ST CHARLES; 152; 152; 0; 32; 78.9; Plusieurs actes de malveillance (vols de câbles, pose de plaques de béton sur les voies de la LGV Nord), inondations sur Avignon du 19 au 22 mai.
2013-05; Nord; LILLE; PARIS NORD; 623; 622; 1; 40; 93.6; 
2013-05; Sud-Est; PARIS LYON; MULHOUSE VILLE; 308; 308; 0; 23; 92.5; 
2013-05; Est; PARIS EST; NANCY; 296; 296; 0; 15; 94.9; 
2013-05; Atlantique; NANTES; PARIS MONTPARNASSE; 570; 569; 1; 29; 94.9; 
2013-05; Sud-Est; NICE VILLE; PARIS LYON; 215; 215; 0; 42; 80.5; 
2013-05; Sud-Est; NIMES; PARIS LYON; 362; 361; 1; 40; 88.9; 
2013-05; Sud-Est; SAINT ETIENNE CHATEAUCREUX; PARIS LYON; 114; 114; 0; 12; 89.5; 
2013-05; Atlantique; TOULOUSE MATABIAU; PARIS MONTPARNASSE; 74; 74; 0; 8; 89.2; Le 3 mai, un accident de personne à St Denis de Pile (nord de boredaux) retarde le TGV 8516. Ce dernier arrive avec 1h42 de retard à Paris.
2013-05; Nord; ARRAS; PARIS NORD; 339; 339; 0; 31; 90.9; 
2013-05; Atlantique; VANNES; PARIS MONTPARNASSE; 175; 175; 0; 10; 94.3; 
2013-05; Sud-Est; AVIGNON TGV; PARIS LYON; 384; 384; 0; 69; 82.0; 
2013-06; Sud-Est; AIX EN PROVENCE TGV; PARIS LYON; 404; 404; 0; 58; 85.6; Une partie des trains assurant cette desserte ont pour origine Nice. Ces trains ont donc déjà parcouru l'axe Nice-Marseille où plusieurs limitations de vitesse sur des zones de modernisation de l'infrastructure fragilisent la circulation.
2013-06; Atlantique; PARIS MONTPARNASSE; BREST; 180; 178; 2; 8; 95.5; 
2013-06; Nord; PARIS NORD; DOUAI; 202; 198; 4; 18; 90.9; 
2013-06; Nord; DUNKERQUE; PARIS NORD; 85; 82; 3; 3; 96.3; 
2013-06; Sud-Est; PARIS LYON; GRENOBLE; 237; 231; 6; 20; 91.3; 
2013-06; Atlantique; PARIS MONTPARNASSE; LA ROCHELLE VILLE; 218; 214; 4; 7; 96.7; 
2013-06; Sud-Est; LE CREUSOT MONTCEAU MONTCHANIN; PARIS LYON; 212; 210; 2; 27; 87.1; 
2013-06; Nord; LILLE; LYON PART DIEU; 303; 295; 8; 25; 91.5; 
2013-06; Sud-Est; LYON PART DIEU; MARSEILLE ST CHARLES; 609; 596; 13; 139; 76.7; Malgré un temps de parcours relativement court entre ces 2 gares, l'essentiel des trains assurant cette desserte sont origine de l'Ouest ou du Nord de la France et ont donc parcouru une très longue distance et souvent déjà accumulé du retard avant même d'assurer cette desserte située en fin de parcours du train
2013-06; Atlantique; RENNES; LYON PART DIEU; 81; 78; 3; 6; 92.3; 
2013-06; Atlantique; PARIS MONTPARNASSE; ANGOULEME; 316; 308; 8; 21; 93.2; 
2013-06; Sud-Est; MONTPELLIER; PARIS LYON; 352; 345; 7; 25; 92.8; 
2013-06; Sud-Est; NICE VILLE; PARIS LYON; 188; 183; 5; 43; 76.5; Plusieurs limitations de vitesse sur des zones de modernisation de l'infrastructure ont fragilisé la circulation des trains sur l'axe Marseille-Nice.
2013-06; Sud-Est; NIMES; PARIS LYON; 353; 346; 7; 32; 90.8; 
2013-06; Atlantique; PARIS MONTPARNASSE; QUIMPER; 135; 134; 1; 12; 91.0; 
2013-06; Atlantique; RENNES; PARIS MONTPARNASSE; 552; 541; 11; 39; 92.8; 
2013-06; Sud-Est; PARIS LYON; SAINT ETIENNE CHATEAUCREUX; 111; 110; 1; 19; 82.7; 
2013-06; Atlantique; ST MALO; PARIS MONTPARNASSE; 97; 97; 0; 4; 95.9; 
2013-06; Atlantique; PARIS MONTPARNASSE; ST PIERRE DES CORPS; 451; 440; 11; 35; 92.0; 
2013-06; Atlantique; PARIS MONTPARNASSE; TOURS; 142; 138; 4; 15; 89.1; 
2013-06; Sud-Est; PARIS LYON; VALENCE ALIXAN TGV; 256; 252; 4; 22; 91.3; 
2013-06; Nord; PARIS NORD; ARRAS; 333; 325; 8; 24; 92.6; 
2013-06; Sud-Est; PARIS LYON; AVIGNON TGV; 364; 358; 6; 33; 90.8; 
2013-06; Sud-Est; BESANCON FRANCHE COMTE TGV; PARIS LYON; 220; 215; 5; 17; 92.1; 
2013-06; Atlantique; PARIS MONTPARNASSE; BORDEAUX ST JEAN; 621; 610; 11; 58; 90.5; 
2013-07; Sud-Est; PARIS LYON; AIX EN PROVENCE TGV; 433; 433; 0; 77; 82.2; 
2013-07; Sud-Est; PARIS LYON; CHAMBERY CHALLES LES EAUX; 223; 223; 0; 30; 86.5; 
2013-07; Nord; DUNKERQUE; PARIS NORD; 91; 91; 0; 9; 90.1; 
2013-07; Sud-Est; LE CREUSOT MONTCEAU MONTCHANIN; PARIS LYON; 217; 217; 0; 27; 87.6; 
2013-07; Nord; LILLE; MARSEILLE ST CHARLES; 150; 150; 0; 53; 64.7; Fortes chaleurs, orages, incendies et coupures de tension, incidents de personnes.
2013-07; Sud-Est; MONTPELLIER; LYON PART DIEU; 345; 345; 0; 107; 69.0; Des orages et fortes chaleurs, des vols de câbles et accidents de personne perturbent la circulation des TGV.
2013-07; Sud-Est; PARIS LYON; MACON LOCHE; 207; 207; 0; 12; 94.2; 
2013-07; Est; METZ; PARIS EST; 302; 302; 0; 14; 95.4; 
2013-07; Atlantique; NANTES; PARIS MONTPARNASSE; 572; 571; 1; 72; 87.4; Ralentissements pour fortes chaleurs, problèmes électriques lors des orages du 22 et incidents divers.
2013-07; Est; NANTES; STRASBOURG; 51; 51; 0; 10; 80.4; 
2013-07; Sud-Est; NIMES; PARIS LYON; 375; 375; 0; 72; 80.8; 
2013-07; Atlantique; POITIERS; PARIS MONTPARNASSE; 503; 503; 0; 58; 88.5; Fortes chaleurs les 7 et 8. Fortes intempéries les 26 et 27. Fuite de gazole d'un train de fret. Pannes de train. Heurt d'animal.
2013-07; Est; REIMS; PARIS EST; 213; 212; 1; 14; 93.4; 
2013-07; Atlantique; PARIS MONTPARNASSE; ST MALO; 60; 60; 0; 2; 96.7; 
2013-07; Est; STRASBOURG; PARIS EST; 452; 451; 1; 45; 90.0; 
2013-07; Sud-Est; TOULON; PARIS LYON; 323; 323; 0; 74; 77.1; Des orages et fortes chaleurs, des vols de câbles et accidents de personne perturbent la circulation des TGV.
2013-07; Sud-Est; AVIGNON TGV; PARIS LYON; 437; 437; 0; 109; 75.1; Des orages et fortes chaleurs, des vols de câbles et accidents de personne perturbent la circulation des TGV.
2013-07; Sud-Est; BELLEGARDE (AIN); PARIS LYON; 275; 274; 1; 27; 90.1; 
2013-07; Sud-Est; PARIS LYON; BELLEGARDE (AIN); 264; 264; 0; 24; 90.9; 
2013-08; Nord; PARIS NORD; DOUAI; 183; 183; 0; 10; 94.5; 
2013-08; Nord; PARIS NORD; DUNKERQUE; 139; 139; 0; 10; 92.8; 
2013-08; Atlantique; LE MANS; PARIS MONTPARNASSE; 461; 461; 0; 66; 85.7; 
2013-08; Atlantique; ANGERS SAINT LAUD; PARIS MONTPARNASSE; 439; 439; 0; 27; 93.8; 
2013-08; Sud-Est; LYON PART DIEU; MARSEILLE ST CHARLES; 635; 635; 0; 141; 77.8; Deux incidents caténaires, des vols de câbles, des heurts d'animaux sauvages et la chute d'une voiture sur les voies près de Marseille ont fortement perturbé la circulation des TGV en août.
2013-08; Sud-Est; MARSEILLE ST CHARLES; LYON PART DIEU; 590; 590; 0; 145; 75.4; Deux incidents caténaires, des vols de câbles, des heurts d'animaux sauvages et la chute d'une voiture sur les voies près de Marseille ont fortement perturbé la circulation des TGV en août.
2013-08; Sud-Est; MACON LOCHE; PARIS LYON; 179; 179; 0; 26; 85.5; 
2013-08; Sud-Est; PARIS LYON; MARSEILLE ST CHARLES; 487; 487; 0; 45; 90.8; 
2013-08; Sud-Est; MONTPELLIER; PARIS LYON; 367; 367; 0; 47; 87.2; 
2013-08; Sud-Est; PARIS LYON; MONTPELLIER; 367; 367; 0; 34; 90.7; 
2013-08; Sud-Est; PARIS LYON; NIMES; 362; 362; 0; 37; 89.8; 
2013-08; Atlantique; QUIMPER; PARIS MONTPARNASSE; 144; 144; 0; 7; 95.1; 
2013-08; Sud-Est; ANNECY; PARIS LYON; 138; 138; 0; 12; 91.3; 
2013-08; Atlantique; RENNES; PARIS MONTPARNASSE; 564; 564; 0; 52; 90.8; 
2013-08; Atlantique; ST MALO; PARIS MONTPARNASSE; 93; 93; 0; 7; 92.5; 
2013-08; Sud-Est; PARIS LYON; TOULON; 252; 252; 0; 31; 87.7; 
2013-08; Atlantique; TOULOUSE MATABIAU; PARIS MONTPARNASSE; 88; 88; 0; 11; 87.5; 
2013-08; Atlantique; TOURS; PARIS MONTPARNASSE; 171; 171; 0; 18; 89.5; 
2013-08; Atlantique; PARIS MONTPARNASSE; TOURS; 147; 147; 0; 20; 86.4; 
2013-08; Nord; PARIS NORD; ARRAS; 295; 295; 0; 16; 94.6; 
2013-08; Sud-Est; PARIS LYON; BELLEGARDE (AIN); 263; 263; 0; 21; 92.0; 
2014-12; Sud-Est; PARIS LYON; TOULON; 229; 229; 0; 22; 90.4; 
2014-12; Atlantique; PARIS MONTPARNASSE; LE MANS; 455; 455; 0; 39; 91.4; 
2014-12; Atlantique; PARIS MONTPARNASSE; NANTES; 577; 576; 1; 24; 95.8; 
2014-12; Atlantique; PARIS MONTPARNASSE; POITIERS; 521; 519; 2; 18; 96.5; 
2014-12; Sud-Est; TOULON; PARIS LYON; 228; 228; 0; 21; 90.8; 
2012-07; Atlantique; PARIS MONTPARNASSE; BREST; 206; 206; 0; 11; 94.7; 
2012-07; Atlantique; LA ROCHELLE VILLE; PARIS MONTPARNASSE; 229; 229; 0; 22; 90.4; 
2012-07; Atlantique; PARIS MONTPARNASSE; LA ROCHELLE VILLE; 226; 226; 0; 34; 85.0; Le 13 juillet, le TGV 8317 heurte un chevreuil près de Saint Arnoult (78). Le train s'arrête et repart avec 32 minutes de retard.
2012-07; Atlantique; PARIS MONTPARNASSE; LAVAL; 235; 235; 0; 13; 94.5; 
2012-07; Sud-Est; LE CREUSOT MONTCEAU MONTCHANIN; PARIS LYON; 215; 215; 0; 38; 82.3; 
2012-07; Nord; LILLE; MARSEILLE ST CHARLES; 159; 159; 0; 32; 79.9; Liaison touchée par un acte de malveillance (câbles coupés) entre Paris et Lyon les 25 et 26 juillet et un dérangement d'installations dans le Nord suite à orage le 27 juillet.
2012-07; Atlantique; LYON PART DIEU; RENNES; 31; 31; 0; 1; 96.8; 
2012-07; Sud-Est; MULHOUSE VILLE; PARIS LYON; 280; 280; 0; 18; 93.6; 
2012-07; Atlantique; NANTES; PARIS MONTPARNASSE; 560; 560; 0; 18; 96.8; 
2012-07; Sud-Est; PARIS LYON; NIMES; 397; 397; 0; 65; 83.6; 
2012-07; Atlantique; QUIMPER; PARIS MONTPARNASSE; 145; 145; 0; 4; 97.2; 
2012-07; Sud-Est; PARIS LYON; ANNECY; 208; 208; 0; 17; 91.8; 
2012-07; Est; REIMS; PARIS EST; 218; 218; 0; 12; 94.5; 
2012-07; Atlantique; PARIS MONTPARNASSE; ST MALO; 60; 60; 0; 6; 90.0; Le 27 juillet, les fortes intempéries ayant eu lieu sur l'ouest de la France ont fortement gêné la circulation de nos TGV sur la ligne à grande vitesse (inondation des gares de Vitré et Port Brillet, arbres sur les voies à Le Genest). Le TGV 8087 est retardé de 62 minutes à son arrivée à Saint-Malo.
2012-07; Atlantique; PARIS MONTPARNASSE; VANNES; 216; 216; 0; 19; 91.2; 
2012-07; Atlantique; PARIS MONTPARNASSE; BORDEAUX ST JEAN; 648; 648; 0; 41; 93.7; 
2012-08; Nord; DUNKERQUE; PARIS NORD; 120; 120; 0; 8; 93.3; 
2012-08; Sud-Est; PARIS LYON; GRENOBLE; 196; 196; 0; 15; 92.3; 
2012-08; Atlantique; PARIS MONTPARNASSE; LAVAL; 234; 234; 0; 11; 95.3; 
2012-08; Atlantique; ANGERS SAINT LAUD; PARIS MONTPARNASSE; 446; 445; 1; 33; 92.6; 
2012-08; Nord; MARSEILLE ST CHARLES; LILLE; 128; 128; 0; 39; 69.5; Les circulations ont été impactées par des dérangements des installations ferroviaires et quelques difficultés liées aux rames TGV. Plusieurs accidents de personnes ont également marqué le mois d'août notamment sur la ligne à grande vitesse du Sud-Est les 16 et 30 août.
2012-08; Sud-Est; LYON PART DIEU; PARIS LYON; 539; 539; 0; 11; 98.0; 
2012-08; Est; METZ; PARIS EST; 303; 301; 2; 28; 90.7; La quasi totalité des retards du mois d'août se sont produit le 16 août à cause d'un accident de personne à Vaires.
2012-08; Atlantique; ANGOULEME; PARIS MONTPARNASSE; 343; 343; 0; 50; 85.4; 
2012-08; Sud-Est; PARIS LYON; MONTPELLIER; 398; 398; 0; 25; 93.7; 
2012-08; Sud-Est; MULHOUSE VILLE; PARIS LYON; 279; 279; 0; 8; 97.1; 
2012-08; Est; NANCY; PARIS EST; 291; 289; 2; 10; 96.5; La quasi totalité des retards du mois d'août se sont produit le 16 août à cause d'un accident de personne à Vaires.
2012-08; Est; NANTES; STRASBOURG; 51; 51; 0; 4; 92.2; 
2012-08; Sud-Est; PARIS LYON; PERPIGNAN; 163; 163; 0; 16; 90.2; 
2012-08; Atlantique; PARIS MONTPARNASSE; POITIERS; 452; 452; 0; 12; 97.3; 
2012-08; Atlantique; QUIMPER; PARIS MONTPARNASSE; 143; 143; 0; 7; 95.1; 
2012-08; Atlantique; TOURS; PARIS MONTPARNASSE; 185; 185; 0; 13; 93.0; 
2012-08; Sud-Est; VALENCE ALIXAN TGV; PARIS LYON; 248; 248; 0; 43; 82.7; 
2012-08; Sud-Est; PARIS LYON; BELLEGARDE (AIN); 262; 262; 0; 22; 91.6; 
2012-08; Sud-Est; BESANCON FRANCHE COMTE TGV; PARIS LYON; 187; 187; 0; 10; 94.7; 
2012-08; Atlantique; PARIS MONTPARNASSE; BORDEAUX ST JEAN; 676; 676; 0; 25; 96.3; 
2012-09; Sud-Est; CHAMBERY CHALLES LES EAUX; PARIS LYON; 199; 198; 1; 29; 85.4; 
2012-09; Sud-Est; PARIS LYON; GRENOBLE; 224; 224; 0; 11; 95.1; 
2012-09; Nord; LILLE; LYON PART DIEU; 300; 300; 0; 25; 91.7; Circulation fortement dégradée le 3 septembre par un incident important qui s'est produit sur la ligne à grande vitesse au niveau de la gare d'Arras (incident touchant les installations caténaires avec de lourdes conséquences en termes de retard sur les trains).
2012-09; Nord; LYON PART DIEU; LILLE; 269; 269; 0; 40; 85.1; Circulation fortement dégradée le 3 septembre par un incident important qui s'est produit sur la ligne à grande vitesse au niveau de la gare d'Arras (incident touchant les installations caténaires avec de lourdes conséquences en termes de retard sur les trains).
2012-09; Nord; LILLE; PARIS NORD; 567; 565; 2; 57; 89.9; Circulation fortement dégradée le 3 septembre par un incident important qui s'est produit sur la ligne à grande vitesse au niveau de la gare d'Arras (incident touchant les installations caténaires avec de lourdes conséquences en termes de retard sur les trains).
2012-09; Sud-Est; LYON PART DIEU; MARSEILLE ST CHARLES; 587; 587; 0; 123; 79.0; Fragilité de cette liaison liée à la longueur du parcours des trains assurant cette desserte.
2012-09; Sud-Est; MARSEILLE ST CHARLES; LYON PART DIEU; 577; 576; 1; 71; 87.7; 
2012-09; Atlantique; RENNES; LYON PART DIEU; 81; 81; 0; 6; 92.6; 
2012-09; Sud-Est; PARIS LYON; MACON LOCHE; 175; 175; 0; 7; 96.0; 
2012-09; Sud-Est; PARIS LYON; MARSEILLE ST CHARLES; 498; 498; 0; 25; 95.0; 
2012-09; Sud-Est; PARIS LYON; MULHOUSE VILLE; 297; 297; 0; 15; 94.9; 
2012-09; Est; NANTES; STRASBOURG; 60; 60; 0; 4; 93.3; 
2012-09; Sud-Est; PERPIGNAN; PARIS LYON; 151; 151; 0; 16; 89.4; 
2012-09; Atlantique; PARIS MONTPARNASSE; POITIERS; 447; 447; 0; 17; 96.2; 
2012-09; Atlantique; QUIMPER; PARIS MONTPARNASSE; 144; 144; 0; 11; 92.4; 
2012-09; Atlantique; TOULOUSE MATABIAU; PARIS MONTPARNASSE; 94; 94; 0; 17; 81.9; 
2012-09; Atlantique; PARIS MONTPARNASSE; VANNES; 180; 180; 0; 8; 95.6; 
2012-09; Sud-Est; BELLEGARDE (AIN); PARIS LYON; 251; 250; 1; 25; 90.0; 
2012-09; Atlantique; BORDEAUX ST JEAN; PARIS MONTPARNASSE; 636; 635; 1; 58; 90.9; 
2012-10; Atlantique; PARIS MONTPARNASSE; LA ROCHELLE VILLE; 221; 220; 1; 22; 90.0; 
2012-10; Atlantique; LE MANS; PARIS MONTPARNASSE; 479; 478; 1; 96; 79.9; De nombreux chantiers ont eu lieu en octobre en amont du Mans générant par leur densité des retards sur les trains en raison de la réduction de la vitesse de circulation de ces derniers dans les zones de chantiers. Par ailleurs un nombre important d'événements ont un fort impact sur la régularité : problème de signalisation ferroviaire, accident de personne, branche tombée sur le fil d'alimentation électrique (caténaire) suite à mauvaise manipulation d'une entreprise d'élagage impliquant la nécessité de stopper la circulations des trains.
2012-10; Atlantique; PARIS MONTPARNASSE; ANGERS SAINT LAUD; 447; 443; 4; 30; 93.2; 
2012-10; Sud-Est; PARIS LYON; MARSEILLE ST CHARLES; 504; 495; 9; 35; 92.9; 
2012-10; Est; PARIS EST; NANCY; 299; 298; 1; 20; 93.3; 
2012-10; Sud-Est; PARIS LYON; NICE VILLE; 167; 164; 3; 31; 81.1; D'importantes phases de travaux d'amélioration de l'infrastructure sur le tronçon Nice Marseille nécessitent la mise en place de limitations de vitesse qui réduisent la fluidité des circulations.
2012-10; Est; PARIS EST; REIMS; 251; 251; 0; 20; 92.0; 
2012-10; Atlantique; ST PIERRE DES CORPS; PARIS MONTPARNASSE; 475; 465; 10; 101; 78.3; Les travaux de raccordement de la ligne à grande vitesse Sud-Europe-Atlantique à la ligne classique au Nord de Bordeaux et en amont de Saint-Pierre-des-Corps. la régénration des voies de la ligne à grande vitesse Atlantique ainsi que les travaux concernant l'alimentation électrique des voies. ont pu engendrer des retards sur nos trains. Par ailleurs un nombre important d'événements ont eu un fort impact. Le 1er, le TGV 8308 heurte un animal près de Vendôme, problème de signalisation ferroviaire près de Massy (91) le 6, accident de personne le 13, branche tombée sur le fil d'alimentation électrique (caténaire) suite à mauvaise manipulation d'une entreprise d'élagage impliquant la nécessité de stopper la circulations des trains dans les deux sens le 23, perte de contrôle d'une aiguille à Saint-Pierre nécessitant l'arrêt des circulations pendant sa réparation le 27.
2012-10; Nord; ARRAS; PARIS NORD; 351; 345; 6; 19; 94.5; 
2012-10; Sud-Est; AVIGNON TGV; PARIS LYON; 379; 375; 4; 46; 87.7; 
2012-10; Sud-Est; BELLEGARDE (AIN); PARIS LYON; 265; 263; 2; 25; 90.5; 
2012-10; Sud-Est; PARIS LYON; BESANCON FRANCHE COMTE TGV; 243; 240; 3; 16; 93.3; 
2012-10; Atlantique; BORDEAUX ST JEAN; PARIS MONTPARNASSE; 664; 658; 6; 95; 85.6; 
2012-10; Atlantique; PARIS MONTPARNASSE; BORDEAUX ST JEAN; 666; 661; 5; 37; 94.4; 
2012-11; Atlantique; PARIS MONTPARNASSE; BREST; 154; 154; 0; 9; 94.2; 
2012-11; Sud-Est; CHAMBERY CHALLES LES EAUX; PARIS LYON; 197; 197; 0; 46; 76.6; Liaison assurée en grande partie par des TGV en provenance d'Italie dont la régularité est perfectible.
2012-11; Sud-Est; DIJON VILLE; PARIS LYON; 443; 443; 0; 37; 91.6; 
2012-11; Sud-Est; PARIS LYON; GRENOBLE; 247; 247; 0; 38; 84.6; 
2012-11; Atlantique; PARIS MONTPARNASSE; LAVAL; 235; 235; 0; 12; 94.9; 
2012-11; Sud-Est; MARSEILLE ST CHARLES; LYON PART DIEU; 577; 577; 0; 89; 84.6; 
2012-11; Sud-Est; PARIS LYON; MACON LOCHE; 173; 173; 0; 25; 85.5; 
2012-11; Atlantique; ANGOULEME; PARIS MONTPARNASSE; 349; 348; 1; 91; 73.9; Importants travaux de raccordement de la ligne à grande vitesse Sud-Europe-Atlantique à la ligne classique entre Bordeaux et Tours ainsi que des travaux de régénération des voies de la ligne à grande vitesse atlantique entre Courtalain et Vendôme.
2012-11; Atlantique; PARIS MONTPARNASSE; ANGOULEME; 321; 321; 0; 31; 90.3; 
2012-11; Est; PARIS EST; NANCY; 286; 286; 0; 14; 95.1; 
2012-11; Sud-Est; PARIS LYON; NICE VILLE; 153; 153; 0; 30; 80.4; 
2012-11; Sud-Est; PARIS LYON; NIMES; 340; 340; 0; 50; 85.3; 
2012-11; Sud-Est; PARIS LYON; PERPIGNAN; 154; 154; 0; 15; 90.3; 
2012-11; Atlantique; QUIMPER; PARIS MONTPARNASSE; 142; 142; 0; 13; 90.8; 
2012-11; Atlantique; VANNES; PARIS MONTPARNASSE; 167; 167; 0; 15; 91.0; 
2012-11; Sud-Est; PARIS LYON; BELLEGARDE (AIN); 256; 256; 0; 73; 71.5; La ligne du Haut Bugey a été particulièrement impactée par les intempéries de la fin du mois.
2012-12; Nord; PARIS NORD; DUNKERQUE; 129; 129; 0; 6; 95.3; 
2012-12; Atlantique; PARIS MONTPARNASSE; LAVAL; 238; 238; 0; 10; 95.8; 
2012-12; Atlantique; ANGERS SAINT LAUD; PARIS MONTPARNASSE; 462; 461; 1; 36; 92.2; 
2012-12; Sud-Est; LYON PART DIEU; PARIS LYON; 632; 632; 0; 28; 95.6; 
2012-12; Sud-Est; PARIS LYON; LYON PART DIEU; 618; 618; 0; 34; 94.5; 
2012-12; Atlantique; LYON PART DIEU; RENNES; 31; 31; 0; 1; 96.8; 
2012-12; Atlantique; RENNES; LYON PART DIEU; 71; 71; 0; 7; 90.1; 
2012-12; Sud-Est; MACON LOCHE; PARIS LYON; 196; 195; 1; 17; 91.3; 
2012-12; Sud-Est; PARIS LYON; MACON LOCHE; 218; 218; 0; 18; 91.7; 
2012-12; Atlantique; PARIS MONTPARNASSE; ANGOULEME; 313; 313; 0; 16; 94.9; 
2013-06; Atlantique; ANGERS SAINT LAUD; PARIS MONTPARNASSE; 452; 444; 8; 22; 95.0; 
2013-06; Nord; LILLE; MARSEILLE ST CHARLES; 147; 144; 3; 28; 80.6; Mois de juin marqué par de nombreux orages du 17 au 19 juin sur la région parisienne ainsi que dans le Sud. Il y a également une augmentation du nombre d'interventions des forces de l'ordre dans les trains pour incivilité.
2013-06; Nord; MARSEILLE ST CHARLES; LILLE; 123; 121; 2; 31; 74.4; Mois de juin marqué par de nombreux orages du 17 au 19 juin sur la région parisienne ainsi que dans le Sud. Il y a également une augmentation du nombre d'interventions des forces de l'ordre dans les trains pour incivilité.
2013-06; Nord; PARIS NORD; LILLE; 606; 594; 12; 42; 92.9; 
2013-06; Sud-Est; MARSEILLE ST CHARLES; LYON PART DIEU; 569; 557; 12; 101; 81.9; 
2013-06; Sud-Est; MONTPELLIER; LYON PART DIEU; 344; 331; 13; 55; 83.4; 
2013-06; Sud-Est; LYON PART DIEU; PARIS LYON; 618; 606; 12; 22; 96.4; 
2013-06; Sud-Est; PARIS LYON; LYON PART DIEU; 612; 602; 10; 24; 96.0; 
2013-06; Sud-Est; MARSEILLE ST CHARLES; PARIS LYON; 458; 449; 9; 32; 92.9; 
2013-06; Sud-Est; PARIS LYON; MONTPELLIER; 337; 330; 7; 28; 91.5; 
2013-06; Sud-Est; MULHOUSE VILLE; PARIS LYON; 309; 303; 6; 20; 93.4; 
2013-06; Sud-Est; PARIS LYON; MULHOUSE VILLE; 296; 290; 6; 19; 93.4; 
2013-06; Est; NANCY; PARIS EST; 284; 281; 3; 16; 94.3; 
2013-06; Est; PARIS EST; NANCY; 284; 281; 3; 19; 93.2; 
2013-06; Est; NANTES; STRASBOURG; 58; 57; 1; 10; 82.5; 
2013-06; Atlantique; QUIMPER; PARIS MONTPARNASSE; 143; 140; 3; 8; 94.3; 
2013-06; Sud-Est; ANNECY; PARIS LYON; 115; 113; 2; 6; 94.7; 
2013-06; Sud-Est; PARIS LYON; ANNECY; 150; 147; 3; 15; 89.8; 
2013-06; Est; PARIS EST; STRASBOURG; 462; 456; 6; 35; 92.3; 
2013-06; Sud-Est; PARIS LYON; TOULON; 190; 187; 3; 25; 86.6; 
2013-06; Atlantique; PARIS MONTPARNASSE; TOULOUSE MATABIAU; 148; 146; 2; 26; 82.2; 
2013-06; Atlantique; TOURS; PARIS MONTPARNASSE; 171; 166; 5; 16; 90.4; 
2013-06; Sud-Est; PARIS LYON; BESANCON FRANCHE COMTE TGV; 232; 227; 5; 22; 90.3; 
2013-07; Atlantique; PARIS MONTPARNASSE; BREST; 184; 184; 0; 18; 90.2; 
2013-07; Sud-Est; DIJON VILLE; PARIS LYON; 476; 476; 0; 23; 95.2; 
2013-07; Sud-Est; PARIS LYON; DIJON VILLE; 461; 461; 0; 19; 95.9; 
2013-07; Nord; PARIS NORD; DOUAI; 184; 184; 0; 23; 87.5; Fortes chaleurs, orages, incendies et coupures de tension.
2013-07; Atlantique; LA ROCHELLE VILLE; PARIS MONTPARNASSE; 225; 225; 0; 15; 93.3; 
2013-07; Nord; LYON PART DIEU; LILLE; 248; 248; 0; 69; 72.2; Fortes chaleurs, orages, incendies et coupures de tension.
2013-07; Nord; MARSEILLE ST CHARLES; LILLE; 125; 125; 0; 48; 61.6; Fortes chaleurs, orages, incendies et coupures de tension, incidents de personnes.
2013-07; Sud-Est; LYON PART DIEU; MARSEILLE ST CHARLES; 630; 630; 0; 164; 74.0; Des orages et fortes chaleurs, des vols de câbles et accidents de personne perturbent la circulation des TGV.
2013-07; Sud-Est; MARSEILLE ST CHARLES; LYON PART DIEU; 586; 586; 0; 148; 74.7; Des orages et fortes chaleurs, des vols de câbles et accidents de personne perturbent la circulation des TGV.
2013-07; Sud-Est; MACON LOCHE; PARIS LYON; 186; 186; 0; 27; 85.5; 
2013-07; Est; PARIS EST; METZ; 311; 311; 0; 11; 96.5; 
2013-07; Atlantique; PARIS MONTPARNASSE; ANGOULEME; 348; 347; 1; 44; 87.3; Fortes chaleurs les 7 et 8 juillet et fortes intempéries les 26 et 27 juillet. Fuite de gazole d'un train de fret. Pannes de train. Heurt d'animal.
2013-07; Sud-Est; MULHOUSE VILLE; PARIS LYON; 324; 324; 0; 16; 95.1; 
2013-07; Sud-Est; PARIS LYON; MULHOUSE VILLE; 302; 302; 0; 17; 94.4; 
2013-07; Atlantique; PARIS MONTPARNASSE; NANTES; 531; 531; 0; 71; 86.6; Ralentissements pour fortes chaleurs, problèmes électriques lors des orages du 22 et incidents divers.
2013-07; Sud-Est; PARIS LYON; NIMES; 371; 371; 0; 60; 83.8; 
2013-07; Sud-Est; PERPIGNAN; PARIS LYON; 165; 165; 0; 23; 86.1; Des orages et fortes chaleurs, des vols de câbles et accidents de personne perturbent la circulation des TGV.
2013-07; Atlantique; QUIMPER; PARIS MONTPARNASSE; 146; 146; 0; 20; 86.3; 
2013-07; Sud-Est; PARIS LYON; SAINT ETIENNE CHATEAUCREUX; 120; 120; 0; 19; 84.2; 
2013-07; Atlantique; TOURS; PARIS MONTPARNASSE; 167; 167; 0; 24; 85.6; 
2013-07; Sud-Est; PARIS LYON; VALENCE ALIXAN TGV; 275; 275; 0; 39; 85.8; 
2013-07; Nord; PARIS NORD; ARRAS; 322; 322; 0; 43; 86.6; 
2013-07; Atlantique; VANNES; PARIS MONTPARNASSE; 169; 169; 0; 19; 88.8; 
2013-07; Atlantique; PARIS MONTPARNASSE; VANNES; 210; 210; 0; 28; 86.7; 
2013-08; Atlantique; PARIS MONTPARNASSE; BREST; 187; 187; 0; 19; 89.8; Le 2 août, plusieurs colis suspects en gare de Paris Montparnasse ont entrainé un fort retard au départ de certains TGV ayant pour destination Brest. Le 23 août, une panne sur un TER a retardé de nombreux trains.
2013-08; Sud-Est; PARIS LYON; DIJON VILLE; 457; 457; 0; 15; 96.7; 
2013-08; Sud-Est; GRENOBLE; PARIS LYON; 188; 188; 0; 12; 93.6; 
2013-08; Sud-Est; PARIS LYON; GRENOBLE; 198; 198; 0; 15; 92.4; 
2013-08; Atlantique; LA ROCHELLE VILLE; PARIS MONTPARNASSE; 231; 231; 0; 9; 96.1; 
2013-08; Atlantique; PARIS MONTPARNASSE; LA ROCHELLE VILLE; 227; 227; 0; 12; 94.7; 
2013-08; Atlantique; PARIS MONTPARNASSE; LE MANS; 445; 444; 1; 57; 87.2; 
2013-08; Nord; LILLE; PARIS NORD; 548; 548; 0; 49; 91.1; 
2013-08; Nord; PARIS NORD; LILLE; 515; 515; 0; 35; 93.2; 
2013-08; Atlantique; LYON PART DIEU; RENNES; 31; 31; 0; 3; 90.3; 
2013-08; Sud-Est; MULHOUSE VILLE; PARIS LYON; 319; 319; 0; 10; 96.9; 
2013-08; Sud-Est; PARIS LYON; MULHOUSE VILLE; 293; 293; 0; 10; 96.6; 
2013-08; Atlantique; PARIS MONTPARNASSE; NANTES; 521; 521; 0; 40; 92.3; 
2013-08; Atlantique; PARIS MONTPARNASSE; QUIMPER; 189; 189; 0; 15; 92.1; 
2013-08; Sud-Est; PARIS LYON; ANNECY; 175; 175; 0; 6; 96.6; 
2013-08; Est; REIMS; PARIS EST; 211; 211; 0; 5; 97.6; 
2013-08; Atlantique; PARIS MONTPARNASSE; TOULOUSE MATABIAU; 157; 157; 0; 25; 84.1; 
2013-08; Sud-Est; BELLEGARDE (AIN); PARIS LYON; 273; 273; 0; 23; 91.6; 
2013-09; Sud-Est; PARIS LYON; CHAMBERY CHALLES LES EAUX; 186; 186; 0; 27; 85.5; 
2013-09; Sud-Est; DIJON VILLE; PARIS LYON; 406; 406; 0; 47; 88.4; Plusieurs incidents ont provoqué des retards importants (incidents techniques, panne d'un TER, vol de câbles).
2013-09; Sud-Est; PARIS LYON; DIJON VILLE; 414; 414; 0; 19; 95.4; 
2013-09; Sud-Est; PARIS LYON; GRENOBLE; 221; 221; 0; 19; 91.4; 
2013-09; Sud-Est; PARIS LYON; LE CREUSOT MONTCEAU MONTCHANIN; 180; 180; 0; 15; 91.7; 
2013-09; Nord; PARIS NORD; LILLE; 543; 543; 0; 52; 90.4; 
2013-09; Est; METZ; PARIS EST; 262; 261; 1; 8; 96.9; 
2013-09; Atlantique; PARIS MONTPARNASSE; ANGOULEME; 282; 282; 0; 24; 91.5; 
2013-09; Est; NANCY; PARIS EST; 256; 256; 0; 6; 97.7; 
2013-09; Atlantique; NANTES; PARIS MONTPARNASSE; 508; 508; 0; 28; 94.5; 
2013-09; Est; NANTES; STRASBOURG; 52; 52; 0; 2; 96.2; 
2013-09; Sud-Est; PARIS LYON; NICE VILLE; 168; 167; 1; 21; 87.4; Heurt d'une personne à Valence, panne d'un train Intercité près de Toulon, plusieurs incidents techniques suite à des orages en fin de mois.
2013-09; Sud-Est; PARIS LYON; PERPIGNAN; 138; 138; 0; 14; 89.9; 
2013-09; Atlantique; PARIS MONTPARNASSE; POITIERS; 443; 443; 0; 17; 96.2; 
2013-09; Est; PARIS EST; REIMS; 189; 189; 0; 4; 97.9; 
2013-09; Sud-Est; SAINT ETIENNE CHATEAUCREUX; PARIS LYON; 102; 102; 0; 10; 90.2; 
2013-09; Sud-Est; PARIS LYON; TOULON; 189; 188; 1; 20; 89.4; 
2013-10; Atlantique; PARIS MONTPARNASSE; BREST; 154; 154; 0; 18; 88.3; Événements occasionnant des retards sur des TGV, souvent dans les deux sens de circulations : heurt d’un chevreuil, incident caténaire, détournement suite à la collision avec un camion, accident de personne, défaut d’alimentation caténaire.
2013-10; Sud-Est; DIJON VILLE; PARIS LYON; 472; 472; 0; 46; 90.3; 
2013-10; Nord; DUNKERQUE; PARIS NORD; 110; 110; 0; 6; 94.5; 
2013-10; Atlantique; LE MANS; PARIS MONTPARNASSE; 499; 499; 0; 123; 75.4; Événements occasionnant des retards sur des TGV, souvent dans les deux sens de circulations : le 1er octobre, dérangement de sémaphore à Champagné (72), le 6 octobre, accident de personne à Mauves s/Loire (44).
2013-10; Nord; LILLE; PARIS NORD; 577; 577; 0; 38; 93.4; 
2013-10; Sud-Est; PARIS LYON; MACON LOCHE; 204; 204; 0; 12; 94.1; 
2013-10; Sud-Est; MARSEILLE ST CHARLES; PARIS LYON; 479; 479; 0; 29; 93.9; 
2013-10; Est; METZ; PARIS EST; 303; 303; 0; 14; 95.4; 
2013-10; Sud-Est; MONTPELLIER; PARIS LYON; 342; 342; 0; 48; 86.0; Six accidents de personne ont perturbé cette liaison en octobre.
2013-10; Sud-Est; MULHOUSE VILLE; PARIS LYON; 317; 317; 0; 16; 95.0; 
2013-10; Atlantique; NANTES; PARIS MONTPARNASSE; 569; 567; 2; 67; 88.2; Événements occasionnant des retards sur des TGV, souvent dans les deux sens de circulations : le 1er octobre, dérangement de sémaphore à Champagné (72), le 6 octobre, accident de personne à Mauves s/Loire (44).
2013-10; Sud-Est; PARIS LYON; NIMES; 326; 326; 0; 42; 87.1; 
2013-10; Atlantique; PARIS MONTPARNASSE; QUIMPER; 123; 123; 0; 29; 76.4; Événements occasionnant des retards sur des TGV, souvent dans les deux sens de circulations : heurt d’un chevreuil, incident caténaire, détournement suite à la collision avec un camion, accident de personne, défaut d’alimentation caténaire.
2013-10; Est; PARIS EST; REIMS; 217; 217; 0; 23; 89.4; 
2013-10; Est; STRASBOURG; PARIS EST; 484; 483; 1; 71; 85.3; 
2013-10; Sud-Est; TOULON; PARIS LYON; 255; 254; 1; 42; 83.5; 
2013-10; Atlantique; PARIS MONTPARNASSE; TOULOUSE MATABIAU; 139; 139; 0; 32; 77.0; Événements occasionnant des retards sur des TGV, souvent dans les deux sens de circulations : présence de ballast sur les rails, arbre tombé aux abords de la voie avec endommagement du pantographe d'un train (bras mécanique permettant l'alimentation électrique du train), incident caténaire à l'arrivée sur Paris, accident de personne, heurts d'animaux, problème d'alimentation de la caténaire.
2013-10; Nord; PARIS NORD; ARRAS; 321; 321; 0; 20; 93.8; 
2015-06; Sud-Est; PARIS LYON; MONTPELLIER; 343; 342; 1; 63; 81.6; Des travaux de modernisation de l'infrastructure ont perturbé la régularité de cette relation en Juin
2015-06; Sud-Est; PARIS LYON; NIMES; 343; 343; 0; 71; 79.3; Des travaux de modernisation de l'infrastructure ont perturbé la régularité de cette relation en Juin
2015-06; Sud-Est; PARIS LYON; TOULON; 268; 267; 1; 56; 79.0; Des travaux de modernisation de l'infrastructure ont perturbé la régularité de cette relation en Juin
2015-06; Atlantique; PARIS MONTPARNASSE; ANGERS SAINT LAUD; 431; 429; 2; 48; 88.8; 
2015-06; Atlantique; PARIS MONTPARNASSE; LAVAL; 227; 226; 1; 20; 91.2; 
2015-06; Atlantique; PARIS MONTPARNASSE; POITIERS; 491; 489; 2; 49; 90.0; 
2015-06; Atlantique; PARIS MONTPARNASSE; QUIMPER; 196; 195; 1; 29; 85.1; Cette OD a été touchée le 4 par un rail cassé couplé à un affaissement caténaire à la sortie de Paris (172 TGV; 7664mn), le 5 par diverses limitations de vitesses liées aux fortes chaleurs dont une à la sortie de Paris reduisant fortement le débit des trains (82TGV; 985mn), le 26 par la rupture d'un appareil de voie en sortie de ligne grande vitesse avant Le Mans (52TGV; 923mn), le 3 par l'arrêt de sécurité d'un train dans les tunnels avant Paris (22TGV; 1035mn) lié à une presence dans les voies en sortie de gare Montparnasse (16TGV; 419mn); le 9 par un dérangement d'installation sur la branche ouest de la ligne grande vitesse (18TGV; 169mn)
2015-06; Atlantique; PARIS MONTPARNASSE; TOURS; 189; 189; 0; 32; 83.1; 
2015-06; Atlantique; ST MALO; PARIS MONTPARNASSE; 97; 96; 1; 3; 96.9; Cette OD a été touchée le 4 par un rail cassé couplé à un affaissement caténaire à la sortie de Paris (172 TGV; 7664mn), le 5 par diverses limitations de vitesses liées aux fortes chaleurs dont une à la sortie de Paris reduisant fortement le débit des trains (82TGV; 985mn), le 26 par la rupture d'un appareil de voie en sortie de ligne grande vitesse avant Le Mans (52TGV; 923mn), le 3 par l'arrêt de sécurité d'un train dans les tunnels avant Paris (22TGV; 1035mn) lié à une presence dans les voies en sortie de gare Montparnasse (16TGV; 419mn); un dérangement d'installation sur la branche ouest de la ligne grande vitesse (18TGV; 169mn)
2015-06; Atlantique; VANNES; PARIS MONTPARNASSE; 239; 239; 0; 19; 92.1; Cette OD a été touchée le 4 par un rail cassé couplé à un affaissement caténaire à la sortie de Paris (172 TGV; 7664mn), le 5 par diverses limitations de vitesses liées aux fortes chaleurs dont une à la sortie de Paris reduisant fortement le débit des trains (82TGV; 985mn), le 26 par la rupture d'un appareil de voie en sortie de ligne grande vitesse avant Le Mans (52TGV; 923mn), le 3 par l'arrêt de sécurité d'un train dans les tunnels avant Paris (22TGV; 1035mn) lié à une presence dans les voies en sortie de gare Montparnasse (16TGV; 419mn); le 9 par un dérangement d'installation sur la branche ouest de la ligne grande vitesse (18TGV; 169mn)
2012-07; Sud-Est; PARIS LYON; CHAMBERY CHALLES LES EAUX; 219; 219; 0; 23; 89.5; 
2012-07; Sud-Est; PARIS LYON; GRENOBLE; 215; 215; 0; 15; 93.0; 
2012-07; Sud-Est; PARIS LYON; LE CREUSOT MONTCEAU MONTCHANIN; 207; 207; 0; 26; 87.4; 
2012-07; Nord; LILLE; PARIS NORD; 586; 585; 1; 44; 92.5; 
2012-07; Sud-Est; PARIS LYON; LYON PART DIEU; 536; 536; 0; 24; 95.5; 
2012-07; Sud-Est; PARIS LYON; MARSEILLE ST CHARLES; 523; 522; 1; 46; 91.2; 
2012-07; Atlantique; PARIS MONTPARNASSE; ANGOULEME; 319; 319; 0; 28; 91.2; 
2012-07; Sud-Est; MONTPELLIER; PARIS LYON; 406; 405; 1; 34; 91.6; 
2012-07; Sud-Est; PARIS LYON; MULHOUSE VILLE; 298; 298; 0; 27; 90.9; 
2012-07; Atlantique; PARIS MONTPARNASSE; POITIERS; 402; 402; 0; 18; 95.5; 
2012-07; Sud-Est; ANNECY; PARIS LYON; 204; 204; 0; 8; 96.1; 
2012-07; Atlantique; PARIS MONTPARNASSE; ST PIERRE DES CORPS; 455; 455; 0; 32; 93.0; 
2012-07; Est; PARIS EST; STRASBOURG; 483; 483; 0; 31; 93.6; 
2012-07; Sud-Est; TOULON; PARIS LYON; 298; 298; 0; 58; 80.5; 
2012-07; Sud-Est; PARIS LYON; TOULON; 266; 266; 0; 33; 87.6; 
2012-07; Sud-Est; PARIS LYON; VALENCE ALIXAN TGV; 274; 274; 0; 30; 89.1; 
2012-07; Sud-Est; BELLEGARDE (AIN); PARIS LYON; 263; 263; 0; 22; 91.6; 
2012-07; Sud-Est; BESANCON FRANCHE COMTE TGV; PARIS LYON; 186; 186; 0; 12; 93.5; 
2012-08; Atlantique; PARIS MONTPARNASSE; BREST; 208; 208; 0; 0; 100.0; 
2012-08; Sud-Est; PARIS LYON; DIJON VILLE; 455; 455; 0; 24; 94.7; 
2012-08; Nord; PARIS NORD; DUNKERQUE; 170; 170; 0; 16; 90.6; 
2012-08; Sud-Est; LE CREUSOT MONTCEAU MONTCHANIN; PARIS LYON; 218; 218; 0; 26; 88.1; 
2012-08; Sud-Est; PARIS LYON; LE CREUSOT MONTCEAU MONTCHANIN; 208; 208; 0; 69; 66.8; Une zone de travaux de renouvellement de la voie fait perdre à l'ensemble des TGV environ 5 min juste avant leur arrivée au Creusot.
2012-08; Atlantique; PARIS MONTPARNASSE; LE MANS; 443; 443; 0; 25; 94.4; 
2012-08; Atlantique; PARIS MONTPARNASSE; ANGERS SAINT LAUD; 396; 396; 0; 21; 94.7; 
2012-08; Nord; LYON PART DIEU; LILLE; 274; 274; 0; 56; 79.6; Les circulations ont été impactées par des dérangements des installations ferroviaires et quelques difficultés liées aux rames TGV. Plusieurs accidents de personnes ont également marqué le mois d'août notamment sur la ligne à grande vitesse du Sud-Est les 16 et 30 août.
2012-08; Nord; LILLE; PARIS NORD; 596; 593; 3; 54; 90.9; 
2012-08; Sud-Est; LYON PART DIEU; MARSEILLE ST CHARLES; 621; 621; 0; 130; 79.1; Fragilité de la liaison liée à la longueur du parcours des trains assurant cette desserte (train en provenance du nord-est ou de l'ouest de la France).
2012-08; Sud-Est; MACON LOCHE; PARIS LYON; 178; 178; 0; 9; 94.9; 
2012-08; Est; PARIS EST; METZ; 312; 312; 0; 24; 92.3; La quasi totalité des retards du mois d'août se sont produit le 16 août à cause d'un accident de personne à Vaires.
2012-08; Sud-Est; MONTPELLIER; PARIS LYON; 400; 400; 0; 40; 90.0; 
2012-08; Sud-Est; PARIS LYON; MULHOUSE VILLE; 302; 302; 0; 22; 92.7; 
2012-08; Sud-Est; PARIS LYON; NIMES; 398; 398; 0; 39; 90.2; 
2012-08; Sud-Est; PARIS LYON; ANNECY; 207; 207; 0; 7; 96.6; 
2012-08; Est; REIMS; PARIS EST; 219; 219; 0; 21; 90.4; La quasi totalité des retards du mois d'août se sont produit le 16 août à cause d'un accident de personne à Vaires.
2012-08; Atlantique; ST PIERRE DES CORPS; PARIS MONTPARNASSE; 445; 445; 0; 50; 88.8; 
2012-08; Sud-Est; PARIS LYON; VALENCE ALIXAN TGV; 280; 280; 0; 31; 88.9; 
2012-08; Nord; PARIS NORD; ARRAS; 359; 359; 0; 33; 90.8; 
2012-08; Atlantique; VANNES; PARIS MONTPARNASSE; 170; 170; 0; 8; 95.3; 
2012-08; Sud-Est; PARIS LYON; BESANCON FRANCHE COMTE TGV; 255; 255; 0; 17; 93.3; 
2012-08; Atlantique; BORDEAUX ST JEAN; PARIS MONTPARNASSE; 647; 647; 0; 48; 92.6; 
2012-09; Atlantique; PARIS MONTPARNASSE; BREST; 183; 183; 0; 6; 96.7; 
2012-09; Sud-Est; DIJON VILLE; PARIS LYON; 441; 441; 0; 29; 93.4; 
2012-09; Nord; DUNKERQUE; PARIS NORD; 95; 95; 0; 4; 95.8; Circulation fortement dégradée le 3 septembre par un incident important qui s'est produit sur la ligne à grande vitesse au niveau de la gare d'Arras (incident touchant les installations caténaires avec de lourdes conséquences en termes de retard sur les trains).
2012-09; Nord; PARIS NORD; DUNKERQUE; 139; 139; 0; 10; 92.8; Circulation fortement dégradée le 3 septembre par un incident important qui s'est produit sur la ligne à grande vitesse au niveau de la gare d'Arras (incident touchant les installations caténaires avec de lourdes conséquences en termes de retard sur les trains).
2012-09; Atlantique; LA ROCHELLE VILLE; PARIS MONTPARNASSE; 217; 217; 0; 14; 93.5; 
2012-09; Atlantique; ANGERS SAINT LAUD; PARIS MONTPARNASSE; 467; 467; 0; 23; 95.1; 
2012-09; Nord; PARIS NORD; LILLE; 591; 591; 0; 56; 90.5; Circulation fortement dégradée le 3 septembre par un incident important qui s'est produit sur la ligne à grande vitesse au niveau de la gare d'Arras (incident touchant les installations caténaires avec de lourdes conséquences en termes de retard sur les trains).
2012-09; Sud-Est; LYON PART DIEU; MONTPELLIER; 364; 364; 0; 50; 86.3; 
2012-09; Sud-Est; MONTPELLIER; LYON PART DIEU; 360; 360; 0; 62; 82.8; 
2012-09; Sud-Est; MACON LOCHE; PARIS LYON; 169; 169; 0; 13; 92.3; 
2012-09; Sud-Est; MARSEILLE ST CHARLES; PARIS LYON; 501; 501; 0; 28; 94.4; 
2012-09; Atlantique; ANGOULEME; PARIS MONTPARNASSE; 349; 349; 0; 53; 84.8; 
2012-09; Atlantique; PARIS MONTPARNASSE; ANGOULEME; 318; 318; 0; 14; 95.6; 
2012-09; Sud-Est; MONTPELLIER; PARIS LYON; 358; 358; 0; 32; 91.1; 
2012-09; Sud-Est; MULHOUSE VILLE; PARIS LYON; 299; 299; 0; 19; 93.6; 
2012-09; Atlantique; PARIS MONTPARNASSE; NANTES; 564; 564; 0; 27; 95.2; 
2012-09; Sud-Est; NIMES; PARIS LYON; 357; 357; 0; 44; 87.7; 
2012-09; Est; REIMS; PARIS EST; 227; 227; 0; 15; 93.4; 
2012-09; Atlantique; PARIS MONTPARNASSE; RENNES; 543; 543; 0; 30; 94.5; 
2012-09; Sud-Est; SAINT ETIENNE CHATEAUCREUX; PARIS LYON; 115; 115; 0; 9; 92.2; 
2012-09; Sud-Est; PARIS LYON; SAINT ETIENNE CHATEAUCREUX; 114; 114; 0; 15; 86.8; 
2012-09; Nord; PARIS NORD; ARRAS; 358; 358; 0; 40; 88.8; Circulation fortement dégradée le 3 septembre par un incident important qui s'est produit sur la ligne à grande vitesse au niveau de la gare d'Arras. Il s'agissait d'un incident touchant les installations caténaires avec de lourdes conséquences en termes de retard sur les trains.
2012-09; Atlantique; VANNES; PARIS MONTPARNASSE; 170; 170; 0; 12; 92.9; 
2012-09; Sud-Est; PARIS LYON; BELLEGARDE (AIN); 252; 252; 0; 46; 81.7; 
2012-10; Sud-Est; AIX EN PROVENCE TGV; PARIS LYON; 411; 411; 0; 55; 86.6; 
2012-10; Sud-Est; CHAMBERY CHALLES LES EAUX; PARIS LYON; 208; 208; 0; 39; 81.3; 
2012-10; Sud-Est; DIJON VILLE; PARIS LYON; 465; 462; 3; 45; 90.3; 
2012-10; Nord; PARIS NORD; DUNKERQUE; 162; 159; 3; 11; 93.1; 
2012-10; Sud-Est; GRENOBLE; PARIS LYON; 250; 245; 5; 18; 92.7; 
2012-10; Sud-Est; PARIS LYON; GRENOBLE; 248; 244; 4; 29; 88.1; 
2012-10; Atlantique; LAVAL; PARIS MONTPARNASSE; 247; 246; 1; 14; 94.3; 
2012-10; Sud-Est; LE CREUSOT MONTCEAU MONTCHANIN; PARIS LYON; 218; 217; 1; 38; 82.5; 
2012-10; Atlantique; ANGERS SAINT LAUD; PARIS MONTPARNASSE; 495; 491; 4; 33; 93.3; 
2012-10; Sud-Est; LYON PART DIEU; MONTPELLIER; 374; 366; 8; 78; 78.7; Intempéries en fin de mois et trois accidents de personne.
2012-10; Sud-Est; PARIS LYON; LYON PART DIEU; 633; 627; 6; 30; 95.2; 
2012-10; Atlantique; RENNES; LYON PART DIEU; 84; 82; 2; 7; 91.5; 
2012-10; Est; METZ; PARIS EST; 302; 302; 0; 20; 93.4; 
2012-10; Sud-Est; PARIS LYON; MONTPELLIER; 354; 349; 5; 35; 90.0; 
2012-10; Atlantique; PARIS MONTPARNASSE; NANTES; 583; 578; 5; 27; 95.3; 
2012-10; Est; NANTES; STRASBOURG; 62; 60; 2; 5; 91.7; 
2012-10; Sud-Est; NICE VILLE; PARIS LYON; 168; 165; 3; 25; 84.8; 
2012-10; Sud-Est; NIMES; PARIS LYON; 371; 366; 5; 47; 87.2; 
2012-10; Sud-Est; PARIS LYON; PERPIGNAN; 155; 154; 1; 19; 87.7; 
2012-10; Atlantique; QUIMPER; PARIS MONTPARNASSE; 152; 151; 1; 8; 94.7; 
2012-10; Atlantique; PARIS MONTPARNASSE; QUIMPER; 164; 164; 0; 7; 95.7; 
2012-10; Atlantique; RENNES; PARIS MONTPARNASSE; 589; 583; 6; 42; 92.8; 
2012-10; Atlantique; PARIS MONTPARNASSE; RENNES; 576; 572; 4; 49; 91.4; 
2012-10; Sud-Est; PARIS LYON; SAINT ETIENNE CHATEAUCREUX; 115; 115; 0; 15; 87.0; 
2012-10; Atlantique; PARIS MONTPARNASSE; ST PIERRE DES CORPS; 478; 474; 4; 35; 92.6; 
2013-10; Atlantique; RENNES; LYON PART DIEU; 75; 75; 0; 14; 81.3; 
2013-10; Sud-Est; PARIS LYON; MONTPELLIER; 326; 326; 0; 38; 88.3; Six accidents de personne ont perturbé cette liaison en octobre.
2013-10; Est; PARIS EST; NANCY; 299; 299; 0; 12; 96.0; 
2013-10; Est; STRASBOURG; NANTES; 61; 61; 0; 7; 88.5; Intempéries et quelques pannes matérielles dont le traitement à necessite de réaliser des demandes de secours.
2013-10; Sud-Est; NIMES; PARIS LYON; 342; 342; 0; 55; 83.9; Six accidents de personne ont perturbé cette liaison en octobre.
2013-10; Sud-Est; PARIS LYON; SAINT ETIENNE CHATEAUCREUX; 120; 120; 0; 27; 77.5; Travaux de rénovation de l'infrastructure.
2013-10; Atlantique; ST MALO; PARIS MONTPARNASSE; 99; 99; 0; 10; 89.9; Événements occasionnant des retards sur des TGV, souvent dans les deux sens de circulations : heurt d’un chevreuil, incident caténaire, détournement suite à la collision avec un camion, accident de personne, défaut d’alimentation caténaire.
2013-10; Atlantique; ST PIERRE DES CORPS; PARIS MONTPARNASSE; 446; 446; 0; 108; 75.8; Événements occasionnant des retards sur des TGV, souvent dans les deux sens de circulations : présence de ballast sur les rails, arbre tombé aux abords de la voie avec endommagement du pantographe d'un train (bras mécanique permettant l'alimentation électrique du train), incident caténaire à l'arrivée sur Paris, accident de personne, heurts d'animaux, problème d'alimentation de la caténaire.
2013-10; Sud-Est; PARIS LYON; TOULON; 208; 208; 0; 30; 85.6; Des travaux de rénovation de l'infrastructure sur la ligne à grande Vitesse et sur le tronçon Marseille-Toulon ont perturbé cette relation en octobre
2013-10; Sud-Est; VALENCE ALIXAN TGV; PARIS LYON; 262; 262; 0; 40; 84.7; Six accidents de personne ont perturbé cette liaison en octobre.
2013-10; Sud-Est; PARIS LYON; VALENCE ALIXAN TGV; 268; 268; 0; 38; 85.8; 
2013-10; Sud-Est; AVIGNON TGV; PARIS LYON; 393; 392; 1; 59; 84.9; 
2013-11; Atlantique; LA ROCHELLE VILLE; PARIS MONTPARNASSE; 205; 205; 0; 18; 91.2; 
2013-11; Nord; LYON PART DIEU; LILLE; 209; 209; 0; 55; 73.7; Accidents de personne, alerte à la bombe à Lille, dérangements d'installation à Beugnatre et Croisilles, nombreuses limitations en vitesse pour cause de travaux, important épisode neigeux en Rhône Alpes les 20, 21 et 22 novembre.
2013-11; Atlantique; RENNES; LYON PART DIEU; 76; 76; 0; 12; 84.2; 
2013-11; Sud-Est; PARIS LYON; MARSEILLE ST CHARLES; 463; 463; 0; 56; 87.9; Les intempéries ayant touché la région lyonnaise les 20, 21 et 22 novembre ont provoqué des retards importants sur l'Axe TGV Sud Est.
2013-11; Est; PARIS EST; METZ; 301; 301; 0; 20; 93.4; 
2013-11; Atlantique; ANGOULEME; PARIS MONTPARNASSE; 334; 333; 1; 46; 86.2; 
2013-11; Sud-Est; MONTPELLIER; PARIS LYON; 323; 322; 1; 51; 84.2; Les intempéries ayant touché la région lyonnaise les 20, 21 et 22 novembre ont provoqué des retards importants sur l'Axe TGV Sud Est.
2013-11; Sud-Est; MULHOUSE VILLE; PARIS LYON; 305; 305; 0; 22; 92.8; 
2013-11; Est; PARIS EST; NANCY; 281; 281; 0; 16; 94.3; 
2013-11; Est; NANTES; STRASBOURG; 57; 57; 0; 6; 89.5; 
2013-11; Sud-Est; PARIS LYON; NICE VILLE; 176; 176; 0; 35; 80.1; Les intempéries ayant touché la région lyonnaise les 20, 21 et 22 novembre ont provoqué des retards importants sur l'Axe TGV Sud Est.
2013-11; Atlantique; PARIS MONTPARNASSE; POITIERS; 487; 486; 1; 38; 92.2; 
2013-11; Sud-Est; PARIS LYON; ANNECY; 168; 168; 0; 13; 92.3; 
2013-11; Est; PARIS EST; REIMS; 207; 207; 0; 16; 92.3; 
2013-11; Atlantique; RENNES; PARIS MONTPARNASSE; 547; 546; 1; 41; 92.5; 
2013-11; Atlantique; PARIS MONTPARNASSE; RENNES; 539; 539; 0; 31; 94.2; 
2013-11; Sud-Est; SAINT ETIENNE CHATEAUCREUX; PARIS LYON; 112; 112; 0; 12; 89.3; 
2013-11; Atlantique; PARIS MONTPARNASSE; ST PIERRE DES CORPS; 446; 445; 1; 52; 88.3; 
2013-11; Sud-Est; PARIS LYON; TOULON; 211; 211; 0; 34; 83.9; Les intempéries ayant touché la région lyonnaise les 20, 21 et 22 novembre ont provoqué des retards importants sur l'Axe TGV Sud Est.
2013-11; Atlantique; PARIS MONTPARNASSE; TOURS; 144; 143; 1; 23; 83.9; 
2013-12; Nord; PARIS NORD; DOUAI; 201; 201; 0; 23; 88.6; 
2013-12; Sud-Est; PARIS LYON; GRENOBLE; 250; 250; 0; 31; 87.6; 
2013-12; Atlantique; PARIS MONTPARNASSE; LA ROCHELLE VILLE; 226; 222; 4; 9; 95.9; 
2013-12; Atlantique; PARIS MONTPARNASSE; LE MANS; 453; 448; 5; 37; 91.7; 
2013-12; Nord; LYON PART DIEU; LILLE; 244; 244; 0; 55; 77.5; Mois de décembre principalement marqué par des actes de malveillance (câbles coupés à Sainghin le 2 et individus dans les voies à Roissy les 12 et 21), des défauts d'alimentation électriques (Beugnatre le 3, Avelin le 5, Vemars le 13 et Lyon St Exupéry le 20) et des accidents de personnes ou d'animaux (Montpellier le 3, Chevreuil à Crisenoy le 5 et Lunel le 29). Un mois également marqué par la découverte d'un fontis à Combles le 18 et surtout un épisode venteux juste avant le jour de Noël entraînant principalement des chutes d'abres sur la caténaire.
2013-12; Nord; PARIS NORD; LILLE; 591; 591; 0; 64; 89.2; 
2013-12; Atlantique; LYON PART DIEU; RENNES; 41; 41; 0; 3; 92.7; 
2013-12; Sud-Est; MARSEILLE ST CHARLES; PARIS LYON; 450; 450; 0; 21; 95.3; 
2013-12; Est; METZ; PARIS EST; 299; 299; 0; 15; 95.0; 
2013-12; Atlantique; PARIS MONTPARNASSE; ANGOULEME; 328; 323; 5; 22; 93.2; 
2013-12; Sud-Est; MONTPELLIER; PARIS LYON; 329; 329; 0; 30; 90.9; 
2013-12; Sud-Est; MULHOUSE VILLE; PARIS LYON; 320; 320; 0; 25; 92.2; 
2013-12; Est; NANCY; PARIS EST; 292; 291; 1; 8; 97.3; 
2013-12; Est; STRASBOURG; NANTES; 52; 51; 1; 5; 90.2; 
2013-12; Sud-Est; NIMES; PARIS LYON; 332; 332; 0; 43; 87.0; 
2013-12; Sud-Est; PARIS LYON; NIMES; 319; 319; 0; 37; 88.4; 
2013-12; Atlantique; PARIS MONTPARNASSE; POITIERS; 513; 503; 10; 29; 94.2; 
2013-12; Sud-Est; ANNECY; PARIS LYON; 142; 142; 0; 9; 93.7; 
2013-12; Atlantique; ST MALO; PARIS MONTPARNASSE; 102; 102; 0; 6; 94.1; 
2013-12; Est; STRASBOURG; PARIS EST; 479; 477; 2; 46; 90.4; 
2013-12; Sud-Est; TOULON; PARIS LYON; 249; 249; 0; 24; 90.4; 
2013-12; Nord; PARIS NORD; ARRAS; 322; 322; 0; 37; 88.5; 
2013-12; Atlantique; PARIS MONTPARNASSE; VANNES; 183; 182; 1; 8; 95.6; 
2013-12; Sud-Est; AVIGNON TGV; PARIS LYON; 410; 408; 2; 54; 86.8; 
2013-12; Sud-Est; BELLEGARDE (AIN); PARIS LYON; 256; 256; 0; 38; 85.2; 
2013-12; Sud-Est; PARIS LYON; BESANCON FRANCHE COMTE TGV; 229; 229; 0; 18; 92.1; 
2013-12; Atlantique; BORDEAUX ST JEAN; PARIS MONTPARNASSE; 671; 659; 12; 52; 92.1; 
2014-01; Nord; DUNKERQUE; PARIS NORD; 114; 114; 0; 7; 93.9; 
2014-01; Nord; PARIS NORD; DUNKERQUE; 129; 129; 0; 8; 93.8; 
2014-01; Atlantique; LE MANS; PARIS MONTPARNASSE; 468; 468; 0; 66; 85.9; 
2014-01; Atlantique; ANGERS SAINT LAUD; PARIS MONTPARNASSE; 457; 457; 0; 37; 91.9; 
2014-01; Atlantique; PARIS MONTPARNASSE; ANGERS SAINT LAUD; 446; 446; 0; 22; 95.1; 
2014-01; Nord; LILLE; LYON PART DIEU; 216; 216; 0; 15; 93.1; 
2014-01; Nord; LYON PART DIEU; LILLE; 245; 245; 0; 45; 81.6; 
2014-01; Nord; LILLE; MARSEILLE ST CHARLES; 123; 123; 0; 16; 87.0; 
2014-01; Sud-Est; LYON PART DIEU; MONTPELLIER; 399; 399; 0; 52; 87.0; 
2014-01; Est; PARIS EST; METZ; 308; 308; 0; 11; 96.4; 
2014-01; Atlantique; ANGOULEME; PARIS MONTPARNASSE; 330; 330; 0; 18; 94.5; 
2014-01; Atlantique; PARIS MONTPARNASSE; ANGOULEME; 334; 334; 0; 13; 96.1; 
2014-01; Sud-Est; MONTPELLIER; PARIS LYON; 337; 337; 0; 26; 92.3; 
2014-01; Sud-Est; MULHOUSE VILLE; PARIS LYON; 325; 325; 0; 12; 96.3; 
2014-01; Sud-Est; PERPIGNAN; PARIS LYON; 149; 149; 0; 6; 96.0; 
2014-01; Atlantique; QUIMPER; PARIS MONTPARNASSE; 150; 150; 0; 10; 93.3; 
2014-01; Atlantique; PARIS MONTPARNASSE; QUIMPER; 140; 140; 0; 8; 94.3; 
2014-01; Sud-Est; ANNECY; PARIS LYON; 147; 147; 0; 8; 94.6; 
2014-01; Est; PARIS EST; REIMS; 216; 216; 0; 12; 94.4; 
2014-01; Atlantique; RENNES; PARIS MONTPARNASSE; 584; 584; 0; 51; 91.3; 
2014-01; Sud-Est; PARIS LYON; TOULON; 194; 194; 0; 27; 86.1; 
2014-01; Atlantique; TOULOUSE MATABIAU; PARIS MONTPARNASSE; 86; 86; 0; 15; 82.6; Le 10, les mesures de sécurité prise suite à la suspicion d'une fuite de gaz sur un train de fret ont engendré un fort retard sur le TGV 8516 (+132 minutes au départ et +123 minutes au terminus, 9 minutes récupérées). Le 19 à Bordeaux, une voiture accidentée sur un pont routier surplombant les voies ferrées risque de tomber et empêche la circulation des trains en sécurité; les TGV 8510 et 8516 sont retardés respectivement de 87 et 49 minutes. Le 25, le heurt d'une voiture se trouvant sur les voies, au nord de Coutras (33), par le TGV 5260 retarde fortement le TGV 8500 de 117 minutes.
2014-01; Atlantique; TOURS; PARIS MONTPARNASSE; 212; 212; 0; 14; 93.4; 
2014-01; Sud-Est; VALENCE ALIXAN TGV; PARIS LYON; 276; 276; 0; 29; 89.5; 
2014-01; Sud-Est; PARIS LYON; VALENCE ALIXAN TGV; 262; 262; 0; 25; 90.5; 
2014-01; Sud-Est; PARIS LYON; AVIGNON TGV; 500; 500; 0; 51; 89.8; 
2014-01; Sud-Est; PARIS LYON; BESANCON FRANCHE COMTE TGV; 223; 223; 0; 12; 94.6; 
2014-01; Atlantique; PARIS MONTPARNASSE; BORDEAUX ST JEAN; 654; 654; 0; 43; 93.4; 
2014-02; Sud-Est; PARIS LYON; AIX EN PROVENCE TGV; 409; 409; 0; 14; 96.6; 
2014-02; Atlantique; BREST; PARIS MONTPARNASSE; 156; 156; 0; 12; 92.3; 
2014-02; Nord; DUNKERQUE; PARIS NORD; 103; 103; 0; 2; 98.1; 
2014-02; Atlantique; PARIS MONTPARNASSE; LE MANS; 410; 410; 0; 45; 89.0; 
2014-02; Sud-Est; LYON PART DIEU; MONTPELLIER; 363; 363; 0; 39; 89.3; 
2014-02; Sud-Est; LYON PART DIEU; PARIS LYON; 573; 573; 0; 17; 97.0; 
2014-02; Est; METZ; PARIS EST; 272; 272; 0; 8; 97.1; 
2014-02; Sud-Est; PARIS LYON; MONTPELLIER; 295; 295; 0; 18; 93.9; 
2014-02; Sud-Est; PARIS LYON; MULHOUSE VILLE; 286; 286; 0; 12; 95.8; 
2014-02; Atlantique; PARIS MONTPARNASSE; NANTES; 495; 495; 0; 30; 93.9; 
2014-02; Sud-Est; PARIS LYON; PERPIGNAN; 140; 140; 0; 9; 93.6; 
2014-02; Atlantique; POITIERS; PARIS MONTPARNASSE; 443; 443; 0; 37; 91.6; 
2014-02; Atlantique; PARIS MONTPARNASSE; POITIERS; 453; 453; 0; 27; 94.0; 
2013-08; Sud-Est; PARIS LYON; BESANCON FRANCHE COMTE TGV; 229; 229; 0; 12; 94.8; 
2013-08; Atlantique; PARIS MONTPARNASSE; BORDEAUX ST JEAN; 663; 663; 0; 36; 94.6; 
2013-09; Sud-Est; CHAMBERY CHALLES LES EAUX; PARIS LYON; 179; 179; 0; 23; 87.2; 
2013-09; Nord; DOUAI; PARIS NORD; 180; 180; 0; 24; 86.7; 
2013-09; Atlantique; PARIS MONTPARNASSE; LA ROCHELLE VILLE; 197; 197; 0; 5; 97.5; 
2013-09; Nord; LILLE; LYON PART DIEU; 269; 269; 0; 34; 87.4; 
2013-09; Nord; LILLE; MARSEILLE ST CHARLES; 130; 130; 0; 35; 73.1; Trois accidents de personnes et dérangements d'installations.
2013-09; Sud-Est; LYON PART DIEU; MONTPELLIER; 353; 353; 0; 80; 77.3; Quatre accidents de personne, un vol de câbles et des incidents techniques suite à des orages ont fortement perturbé la circulation des TGV.
2013-09; Sud-Est; PARIS LYON; LYON PART DIEU; 551; 551; 0; 20; 96.4; 
2013-09; Atlantique; LYON PART DIEU; RENNES; 27; 27; 0; 1; 96.3; 
2013-09; Sud-Est; PARIS LYON; MULHOUSE VILLE; 269; 269; 0; 27; 90.0; Plusieurs incidents ont provoqué des retards importants (incidents techniques, panne d'un TER, vol de câbles).
2013-09; Atlantique; POITIERS; PARIS MONTPARNASSE; 428; 428; 0; 33; 92.3; 
2013-09; Atlantique; QUIMPER; PARIS MONTPARNASSE; 121; 121; 0; 8; 93.4; 
2013-09; Atlantique; ST MALO; PARIS MONTPARNASSE; 87; 87; 0; 8; 90.8; 
2013-09; Sud-Est; VALENCE ALIXAN TGV; PARIS LYON; 226; 226; 0; 24; 89.4; 
2013-09; Sud-Est; PARIS LYON; VALENCE ALIXAN TGV; 231; 231; 0; 31; 86.6; 
2013-09; Sud-Est; AVIGNON TGV; PARIS LYON; 338; 338; 0; 36; 89.3; 
2013-09; Sud-Est; BELLEGARDE (AIN); PARIS LYON; 234; 234; 0; 19; 91.9; 
2013-09; Sud-Est; PARIS LYON; BELLEGARDE (AIN); 217; 217; 0; 18; 91.7; 
2013-10; Sud-Est; AIX EN PROVENCE TGV; PARIS LYON; 419; 419; 0; 61; 85.4; 
2013-10; Atlantique; LA ROCHELLE VILLE; PARIS MONTPARNASSE; 225; 223; 2; 22; 90.1; 
2013-10; Atlantique; PARIS MONTPARNASSE; LA ROCHELLE VILLE; 226; 226; 0; 26; 88.5; Événements occasionnant des retards sur des TGV, souvent dans les deux sens de circulations : le 1er octobre, dérangement de sémaphore à Champagné (72), le 6 juillet, a6/10 : Accident de personne à Mauves s/Loire (44).
2013-10; Sud-Est; LE CREUSOT MONTCEAU MONTCHANIN; PARIS LYON; 220; 220; 0; 23; 89.5; 
2013-10; Atlantique; PARIS MONTPARNASSE; LE MANS; 459; 457; 2; 69; 84.9; 
2013-10; Atlantique; PARIS MONTPARNASSE; ANGERS SAINT LAUD; 441; 440; 1; 46; 89.5; 
2013-10; Nord; LILLE; LYON PART DIEU; 299; 299; 0; 36; 88.0; 
2013-10; Nord; MARSEILLE ST CHARLES; LILLE; 63; 63; 0; 9; 85.7; 
2013-10; Nord; PARIS NORD; LILLE; 600; 600; 0; 38; 93.7; 
2013-10; Sud-Est; LYON PART DIEU; MONTPELLIER; 402; 402; 0; 92; 77.1; Six accidents de personne ont perturbé cette liaison en octobre.
2013-10; Sud-Est; PARIS LYON; MULHOUSE VILLE; 316; 316; 0; 14; 95.6; 
2013-10; Sud-Est; PARIS LYON; NICE VILLE; 179; 179; 0; 32; 82.1; Travaux de rénovation de l'infrastructure sur la ligne à grande Vitesse et sur le tronçon Marseille-Nice.
2013-10; Sud-Est; PERPIGNAN; PARIS LYON; 153; 153; 0; 19; 87.6; Six accidents de personne ont perturbé cette liaison en octobre.
2013-10; Atlantique; POITIERS; PARIS MONTPARNASSE; 503; 500; 3; 77; 84.6; 
2013-10; Atlantique; PARIS MONTPARNASSE; POITIERS; 521; 521; 0; 82; 84.3; Événements occasionnant des retards sur des TGV, souvent dans les deux sens de circulations : présence de ballast sur les rails, arbre tombé aux abords de la voie avec endommagement du pantographe d'un train (bras mécanique permettant l'alimentation électrique du train), incident caténaire à l'arrivée sur Paris, accident de personne, heurts d'animaux, problème d'alimentation de la caténaire.
2013-10; Est; REIMS; PARIS EST; 213; 213; 0; 23; 89.2; 
2013-10; Est; PARIS EST; STRASBOURG; 485; 485; 0; 43; 91.1; 
2013-10; Atlantique; PARIS MONTPARNASSE; VANNES; 183; 183; 0; 33; 82.0; Événements occasionnant des retards sur des TGV, souvent dans les deux sens de circulations : heurt d’un chevreuil, incident caténaire, détournement suite à la collision avec un camion, accident de personne, défaut d’alimentation caténaire.
2013-10; Sud-Est; BELLEGARDE (AIN); PARIS LYON; 271; 271; 0; 26; 90.4; 
2013-10; Sud-Est; PARIS LYON; BELLEGARDE (AIN); 253; 253; 0; 26; 89.7; 
2013-10; Sud-Est; PARIS LYON; BESANCON FRANCHE COMTE TGV; 247; 247; 0; 17; 93.1; 
2013-10; Atlantique; BORDEAUX ST JEAN; PARIS MONTPARNASSE; 685; 683; 2; 120; 82.4; Événements occasionnant des retards sur des TGV, souvent dans les deux sens de circulations : présence de ballast sur les rails, arbre tombé aux abords de la voie avec endommagement du pantographe d'un train (bras mécanique permettant l'alimentation électrique du train), incident caténaire à l'arrivée sur Paris, accident de personne, heurts d'animaux, problème d'alimentation de la caténaire.
2013-11; Nord; DOUAI; PARIS NORD; 196; 196; 0; 22; 88.8; 
2013-11; Nord; PARIS NORD; DUNKERQUE; 121; 121; 0; 12; 90.1; 
2013-11; Atlantique; PARIS MONTPARNASSE; ANGERS SAINT LAUD; 412; 412; 0; 46; 88.8; 
2013-11; Nord; LILLE; LYON PART DIEU; 264; 264; 0; 43; 83.7; 
2013-11; Sud-Est; MONTPELLIER; LYON PART DIEU; 299; 299; 0; 95; 68.2; Les intempéries ayant touché la région lyonnaise les 20, 21 et 22 novembre ont provoqué des retards importants sur l'Axe TGV Sud Est.
2013-11; Sud-Est; MACON LOCHE; PARIS LYON; 186; 186; 0; 30; 83.9; Les intempéries ayant touché la région lyonnaise les 20, 21 et 22 novembre ont provoqué des retards importants sur l'Axe TGV Sud Est.
2013-11; Sud-Est; PARIS LYON; PERPIGNAN; 150; 150; 0; 28; 81.3; Les intempéries ayant touché la région lyonnaise les 20, 21 et 22 novembre ont provoqué des retards importants sur l'Axe TGV Sud Est.
2013-11; Atlantique; POITIERS; PARIS MONTPARNASSE; 478; 476; 2; 54; 88.7; Retards en raison de nombreux événements particulièrement perturbants, notamment : incident caténaire à Cérons (33) le 12, heurt de plusieurs sangliers sur ligne à grande vitesse le 19, accident de personne à St Cyr (37) le 23, accident de personne à St Benoit (86) le 26, accident de personne à St Médard (33) le 27.
2013-11; Atlantique; PARIS MONTPARNASSE; QUIMPER; 121; 121; 0; 7; 94.2; 
2013-11; Atlantique; PARIS MONTPARNASSE; ST MALO; 53; 53; 0; 4; 92.5; 
2013-11; Atlantique; ST PIERRE DES CORPS; PARIS MONTPARNASSE; 413; 412; 1; 83; 79.9; Retards en raison de nombreux événements particulièrement perturbants en amont et en aval de St Pierre des Corps, notamment : des câbles sectionnés lors de travaux sur les voies à Monts (37) le 3, incident caténaire à Cérons (33) le 12, heurt de plusieurs sangliers sur ligne à grande vitesse le 19.
2013-11; Est; STRASBOURG; PARIS EST; 445; 445; 0; 62; 86.1; 
2013-11; Sud-Est; PARIS LYON; BESANCON FRANCHE COMTE TGV; 230; 230; 0; 19; 91.7; 
2013-11; Atlantique; BORDEAUX ST JEAN; PARIS MONTPARNASSE; 653; 652; 1; 85; 87.0; Retards en raison de nombreux événements particulièrement perturbants, notamment : incident caténaire à Cérons (33) le 12, heurt de plusieurs sangliers sur ligne à grande vitesse le 19, accident de personne à St Cyr (37) le 23, accident de personne à St Benoit (86) le 26, accident de personne à St Médard (33) le 27.
2013-12; Sud-Est; AIX EN PROVENCE TGV; PARIS LYON; 406; 406; 0; 37; 90.9; 
2013-12; Atlantique; BREST; PARIS MONTPARNASSE; 177; 176; 1; 14; 92.0; 
2013-12; Sud-Est; CHAMBERY CHALLES LES EAUX; PARIS LYON; 237; 237; 0; 27; 88.6; 
2013-12; Atlantique; LAVAL; PARIS MONTPARNASSE; 246; 245; 1; 25; 89.8; Panne d'un train de travaux sur la branche sud Bretagne (près de Lorient) avec un retard de 89 minutes sur le TGV 8780, heurt d'un sanglier par le TER 857605 et nombreux retards pour plusieurs trains dont les TGV 8052 et 8706 avec respectivement 76 et 56 minutes de retard au terminus, fortes intempéries.
2013-12; Atlantique; PARIS MONTPARNASSE; LAVAL; 245; 240; 5; 10; 95.8; 
2013-12; Nord; LILLE; MARSEILLE ST CHARLES; 135; 135; 0; 28; 79.3; Mois de décembre principalement marqué par des actes de malveillance (câbles coupés à Sainghin le 2 et individus dans les voies à Roissy les 12 et 21), des défauts d'alimentation électriques (Beugnatre le 3, Avelin le 5, Vemars le 13 et Lyon St Exupéry le 20) et des accidents de personnes ou d'animaux (Montpellier le 3, Chevreuil à Crisenoy le 5 et Lunel le 29). Un mois également marqué par la découverte d'un fontis à Combles le 18 et surtout un épisode venteux juste avant le jour de Noël entraînant principalement des chutes d'abres sur la caténaire.
2013-12; Nord; MARSEILLE ST CHARLES; LILLE; 123; 123; 0; 36; 70.7; Mois de décembre principalement marqué par des actes de malveillance (câbles coupés à Sainghin le 2 et individus dans les voies à Roissy les 12 et 21), des défauts d'alimentation électriques (Beugnatre le 3, Avelin le 5, Vemars le 13 et Lyon St Exupéry le 20) et des accidents de personnes ou d'animaux (Montpellier le 3, Chevreuil à Crisenoy le 5 et Lunel le 29). Un mois également marqué par la découverte d'un fontis à Combles le 18 et surtout un épisode venteux juste avant le jour de Noël entraînant principalement des chutes d'abres sur la caténaire.
2013-12; Nord; LILLE; PARIS NORD; 581; 580; 1; 92; 84.1; 
2013-12; Sud-Est; LYON PART DIEU; PARIS LYON; 621; 621; 0; 24; 96.1; 
2013-12; Sud-Est; PARIS LYON; MACON LOCHE; 221; 221; 0; 16; 92.8; 
2013-12; Est; PARIS EST; METZ; 308; 308; 0; 13; 95.8; 
2013-12; Sud-Est; PARIS LYON; NICE VILLE; 175; 175; 0; 30; 82.9; Quatre heurts d'animaux sauvages sur la ligne à grande vitesse, ainsi que les intempéries en région PACA, ont dégradé la régularité en décembre.
2013-12; Sud-Est; PERPIGNAN; PARIS LYON; 156; 156; 0; 17; 89.1; Quatre accidents de personne entre Montpellier et Nîmes et quatre heurts d'animaux sauvages sur la ligne à grande vitesse ont dégradé la régularité en décembre.
2013-12; Sud-Est; PARIS LYON; PERPIGNAN; 157; 157; 0; 17; 89.2; Quatre accidents de personne entre Montpellier et Nîmes et quatre heurts d'animaux sauvages sur la ligne à grande vitesse ont dégradé la régularité en décembre.
2013-12; Atlantique; PARIS MONTPARNASSE; QUIMPER; 133; 133; 0; 6; 95.5; 
2013-12; Est; REIMS; PARIS EST; 210; 209; 1; 15; 92.8; 
2013-12; Est; PARIS EST; REIMS; 214; 213; 1; 16; 92.5; 
2013-12; Sud-Est; PARIS LYON; SAINT ETIENNE CHATEAUCREUX; 50; 50; 0; 8; 84.0; 
2013-12; Sud-Est; PARIS LYON; TOULON; 201; 201; 0; 40; 80.1; 
2013-12; Atlantique; TOULOUSE MATABIAU; PARIS MONTPARNASSE; 94; 94; 0; 15; 84.0; 
2013-12; Sud-Est; VALENCE ALIXAN TGV; PARIS LYON; 264; 264; 0; 33; 87.5; 
2013-12; Sud-Est; PARIS LYON; VALENCE ALIXAN TGV; 254; 254; 0; 30; 88.2; 
2014-01; Atlantique; PARIS MONTPARNASSE; BREST; 209; 209; 0; 9; 95.7; 
2014-01; Sud-Est; CHAMBERY CHALLES LES EAUX; PARIS LYON; 261; 261; 0; 21; 92.0; 
2014-01; Sud-Est; PARIS LYON; CHAMBERY CHALLES LES EAUX; 265; 265; 0; 23; 91.3; 
2014-01; Sud-Est; GRENOBLE; PARIS LYON; 247; 247; 0; 14; 94.3; 
2014-01; Atlantique; LAVAL; PARIS MONTPARNASSE; 254; 254; 0; 24; 90.6; 
2014-01; Atlantique; PARIS MONTPARNASSE; LAVAL; 244; 244; 0; 8; 96.7; 
2014-01; Sud-Est; LE CREUSOT MONTCEAU MONTCHANIN; PARIS LYON; 221; 221; 0; 30; 86.4; 
2014-01; Nord; MARSEILLE ST CHARLES; LILLE; 124; 124; 0; 27; 78.2; Deux agressions (Lyon le 04 et Avignon le 24), des manifestants à Marseille le 30, un arbre sur la caténaire suite aux intempéries sur Lille le 04 ainsi que quelques incidents techniques à divers endroits du parcours.
2014-01; Nord; PARIS NORD; LILLE; 629; 629; 0; 25; 96.0; 
2014-01; Sud-Est; MONTPELLIER; LYON PART DIEU; 393; 393; 0; 72; 81.7; 
2014-01; Sud-Est; PARIS LYON; LYON PART DIEU; 636; 636; 0; 24; 96.2; 
2014-01; Atlantique; RENNES; LYON PART DIEU; 63; 63; 0; 10; 84.1; 
2014-01; Sud-Est; PARIS LYON; MACON LOCHE; 226; 226; 0; 11; 95.1; 
2014-01; Sud-Est; MARSEILLE ST CHARLES; PARIS LYON; 443; 443; 0; 30; 93.2; 
2014-01; Sud-Est; PARIS LYON; MARSEILLE ST CHARLES; 477; 476; 1; 31; 93.5; 
2014-01; Sud-Est; PARIS LYON; MULHOUSE VILLE; 320; 320; 0; 19; 94.1; 
2014-01; Atlantique; PARIS MONTPARNASSE; NANTES; 558; 558; 0; 32; 94.3; 
2014-01; Sud-Est; PARIS LYON; NICE VILLE; 176; 176; 0; 32; 81.8; Des travaux de rénovation de l'infrastructure sur la ligne à grande Vitesse et sur le tronçon Marseille-Nice ont perturbé cette relation en janvier.
2014-01; Atlantique; POITIERS; PARIS MONTPARNASSE; 504; 504; 0; 17; 96.6; 
2014-01; Sud-Est; SAINT ETIENNE CHATEAUCREUX; PARIS LYON; 2; 2; 0; 0; 100.0; 
2014-01; Sud-Est; PARIS LYON; SAINT ETIENNE CHATEAUCREUX; 1; 1; 0; 0; 100.0; 
2014-01; Est; STRASBOURG; PARIS EST; 468; 467; 1; 46; 90.1; 
2014-01; Atlantique; PARIS MONTPARNASSE; TOURS; 153; 153; 0; 19; 87.6; 
2012-12; Sud-Est; MONTPELLIER; PARIS LYON; 366; 366; 0; 31; 91.5; 
2012-12; Sud-Est; PARIS LYON; NIMES; 346; 346; 0; 53; 84.7; 
2012-12; Sud-Est; PERPIGNAN; PARIS LYON; 155; 155; 0; 13; 91.6; 
2012-12; Atlantique; POITIERS; PARIS MONTPARNASSE; 503; 502; 1; 34; 93.2; 
2012-12; Sud-Est; ANNECY; PARIS LYON; 154; 153; 1; 13; 91.5; 
2012-12; Est; REIMS; PARIS EST; 218; 218; 0; 41; 81.2; 
2012-12; Sud-Est; PARIS LYON; SAINT ETIENNE CHATEAUCREUX; 117; 117; 0; 18; 84.6; 
2012-12; Atlantique; PARIS MONTPARNASSE; ST PIERRE DES CORPS; 456; 456; 0; 36; 92.1; 
2012-12; Nord; PARIS NORD; ARRAS; 340; 340; 0; 36; 89.4; 
2013-01; Atlantique; BREST; PARIS MONTPARNASSE; 143; 143; 0; 15; 89.5; 
2013-01; Atlantique; LE MANS; PARIS MONTPARNASSE; 463; 463; 0; 93; 79.9; Forts épisodes neigeux les 15, 18, 19, 20, 21 et 22 janvier, nécessitant la baisse de vitesse des trains, notamment sur ligne à grande vitesse Atlantique.
2013-01; Nord; LILLE; LYON PART DIEU; 308; 308; 0; 73; 76.3; Circulation touchée principalement par les chutes de neige du 14 au 23 janvier sur le Nord et du 15 au 21 janvier sur le Sud Est. Durant cette période, pour éviter les dangers liés aux projections de glace, la vitesse des trains sur la ligne à grande vitesse du Nord ou du Sud Est a été limitée à 230 km/h voire 170 km/h . Les installations au sol ainsi que les rames TGV ont également connu des difficultés liées à cet épisode neigeux.
2013-01; Nord; LILLE; PARIS NORD; 623; 622; 1; 210; 66.2; Liaison touchée principalement par les chutes de neige du 14 au 23 janvier. Durant cette période, pour éviter les dangers liés aux projections de glace, la vitesse des trains sur la ligne à grande vitesse a été limitée à 230 km/h voire 170 km/h. Les installations au sol ainsi que les rames TGV ont également connu des difficultés liées à cet épisode neigeux.
2013-01; Nord; PARIS NORD; LILLE; 633; 631; 2; 166; 73.7; Liaison touchée principalement par les chutes de neige du 14 au 23 janvier. Durant cette période, pour éviter les dangers liés aux projections de glace, la vitesse des trains sur la ligne à grande vitesse a été limitée à 230 km/h voire 170 km/h. Les installations au sol ainsi que les rames TGV ont également connu des difficultés liées à cet épisode neigeux.
2013-01; Sud-Est; PARIS LYON; LYON PART DIEU; 641; 640; 1; 63; 90.2; 
2013-01; Atlantique; LYON PART DIEU; RENNES; 30; 30; 0; 4; 86.7; 
2013-01; Sud-Est; MULHOUSE VILLE; PARIS LYON; 316; 316; 0; 42; 86.7; 
2013-01; Sud-Est; NICE VILLE; PARIS LYON; 158; 158; 0; 30; 81.0; Conditions météorologiques difficiles (en particulier la neige et le froid).
2013-01; Sud-Est; PARIS LYON; NIMES; 348; 348; 0; 54; 84.5; 
2013-01; Sud-Est; PARIS LYON; PERPIGNAN; 159; 159; 0; 23; 85.5; 
2013-01; Atlantique; QUIMPER; PARIS MONTPARNASSE; 153; 153; 0; 17; 88.9; 
2013-01; Atlantique; ST MALO; PARIS MONTPARNASSE; 102; 102; 0; 14; 86.3; 
2013-01; Est; PARIS EST; STRASBOURG; 482; 479; 3; 104; 78.3; Plusieures journées d'intempéries ont contraint à réduire la vitesse des TGV sur leur parcours.
2013-01; Sud-Est; AVIGNON TGV; PARIS LYON; 331; 331; 0; 58; 82.5; 
2013-01; Sud-Est; PARIS LYON; BELLEGARDE (AIN); 269; 269; 0; 40; 85.1; 
2013-01; Sud-Est; PARIS LYON; BESANCON FRANCHE COMTE TGV; 241; 241; 0; 20; 91.7; 
2013-02; Nord; PARIS NORD; DOUAI; 189; 189; 0; 41; 78.3; Les chutes de neige des 10 et 11 février nécessitent de réduire la vitesse des trains sur la ligne à grande vitesse à 230 km/h voire 170 km/h pour éviter les dangers liés aux projections de glace. Les rames TGV ont également souffert de ces intempéries ainsi que du froid associé avec des pannes de train en cours de circulation. Les installations caténaires ont subi également des avaries. A noter aussi un accident de personne qui a totalement perturbé la circulation sur Paris Nord, des travaux de renouvellement de voie sur Arras et des dérangements informatiques dans les postes d'aiguillages de Douai.
2013-02; Sud-Est; PARIS LYON; LE CREUSOT MONTCEAU MONTCHANIN; 188; 188; 0; 37; 80.3; 
2013-02; Sud-Est; MARSEILLE ST CHARLES; LYON PART DIEU; 538; 538; 0; 121; 77.5; Conditions météorologiques difficiles (en particulier la neige et le froid).
2013-02; Sud-Est; MARSEILLE ST CHARLES; PARIS LYON; 426; 426; 0; 41; 90.4; 
2013-02; Est; METZ; PARIS EST; 272; 272; 0; 57; 79.0; Suite aux différents épisodes de neige et de froid. la circulation des trains a été impactée par des limitations de vitesse (230 km/h à la place de 320 km/h) et par des dérangements de la signalisation et des aiguillages.
2013-02; Atlantique; ANGOULEME; PARIS MONTPARNASSE; 321; 321; 0; 30; 90.7; 
2013-02; Est; NANTES; STRASBOURG; 40; 40; 0; 5; 87.5; Quelques événements ayant de fortes répercussions sur la régularité des trains ont eu lieu (panne d'un train de travaux le 12, accident de personne le 20, rail cassé à l'entrée de la ligne à grande vitesse le 21, heurt d'une vache le 26).
2013-02; Sud-Est; NIMES; PARIS LYON; 329; 329; 0; 52; 84.2; 
2013-02; Atlantique; POITIERS; PARIS MONTPARNASSE; 450; 450; 0; 25; 94.4; 
2013-02; Sud-Est; ANNECY; PARIS LYON; 130; 130; 0; 10; 92.3; 
2013-02; Atlantique; ST PIERRE DES CORPS; PARIS MONTPARNASSE; 396; 396; 0; 55; 86.1; 
2013-02; Sud-Est; TOULON; PARIS LYON; 216; 216; 0; 41; 81.0; Conditions météorologiques difficiles (en particulier la neige et le froid).
2013-02; Sud-Est; PARIS LYON; VALENCE ALIXAN TGV; 237; 237; 0; 38; 84.0; 
2013-02; Nord; PARIS NORD; ARRAS; 313; 313; 0; 52; 83.4; 
2013-02; Atlantique; VANNES; PARIS MONTPARNASSE; 158; 158; 0; 15; 90.5; 
2013-02; Sud-Est; PARIS LYON; AVIGNON TGV; 381; 381; 0; 54; 85.8; 
2013-02; Sud-Est; BELLEGARDE (AIN); PARIS LYON; 250; 249; 1; 70; 71.9; Conditions météorologiques difficiles (en particulier la neige et le froid).
2013-03; Sud-Est; PARIS LYON; AIX EN PROVENCE TGV; 395; 395; 0; 54; 86.3; 
2013-03; Nord; PARIS NORD; DOUAI; 205; 192; 13; 64; 66.7; Les très importantes chutes de neige avec formation de congères survenues à partir du lundi 11 mars dans la soirée jusqu'au vendredi 15 mars ont fortement perturbé la production. Elles ont imposé des limitations de vitesse pour éviter les projections de glace, allant même jusqu’à immobiliser presque complètement le trafic le mardi 12 mars. Les conditions climatiques difficiles de ces derniers mois ont par ailleurs perturbé les différents programmes travaux sur les voies prolongeant ainsi certaines limitations de vitesse au-delà des délais prévus, notamment sur Arras.
2013-03; Atlantique; ANGERS SAINT LAUD; PARIS MONTPARNASSE; 466; 465; 1; 32; 93.1; 
2013-03; Nord; LILLE; LYON PART DIEU; 268; 259; 9; 38; 85.3; Les très importantes chutes de neige avec formation de congères survenues dans le nord à partir du lundi 11 mars dans la soirée jusqu'au vendredi 15 mars et également dans le sud-est le 12 mars ont fortement perturbé la production. Elles ont imposé des limitations de vitesse pour éviter les projections de glace. Les conditions climatiques difficiles de ces derniers mois ont par ailleurs perturbé les différents programmes travaux sur les voies prolongeant ainsi certaines limitations de vitesse au-delà des délais prévus. notamment sur Arras et la périphérie de Lyon.
2013-03; Sud-Est; LYON PART DIEU; MONTPELLIER; 325; 322; 3; 83; 74.2; Malgré un temps de parcours relativement court entre ces deux gares, l'essentiel des trains assurant cette desserte sont origine de l'ouest ou du Nord de la France et ont donc parcouru une très longue distance et souvent déjà accumulé du retard avant même d'assurer cette desserte située en fin de parcours du train.
2013-03; Sud-Est; MONTPELLIER; LYON PART DIEU; 346; 344; 2; 81; 76.5; Les travaux de rénovation de la voie à l'entrée de Lyon génèrent des retards chroniques sur les TGV empruntant cette zone.
2013-03; Sud-Est; LYON PART DIEU; PARIS LYON; 628; 625; 3; 32; 94.9; La journée du 12 mars a été très perturbée par les intempéries (neige et pluies verglassantes) entrainant de nombreux et importants retards avoisinants les 2 heures.
2013-03; Sud-Est; PARIS LYON; NICE VILLE; 159; 157; 2; 26; 83.4; La journée du 12 mars a été très perturbée par les intempéries (neige et pluies verglassantes) entrainant de nombreux et importants retards avoisinants les 2 heures.
2013-03; Sud-Est; NIMES; PARIS LYON; 366; 365; 1; 45; 87.7; 
2013-03; Atlantique; TOULOUSE MATABIAU; PARIS MONTPARNASSE; 90; 90; 0; 15; 83.3; 
2013-03; Sud-Est; VALENCE ALIXAN TGV; PARIS LYON; 259; 259; 0; 27; 89.6; 
2013-03; Atlantique; PARIS MONTPARNASSE; VANNES; 186; 186; 0; 25; 86.6; 
2013-03; Sud-Est; AVIGNON TGV; PARIS LYON; 329; 328; 1; 46; 86.0; 
2013-04; Atlantique; PARIS MONTPARNASSE; BREST; 184; 184; 0; 10; 94.6; Mauvaise anticipation de l'absence de conducteur sur un TGV, retardant ce dernier au départ, accident de personne, envahissement de la gare de Rennes par une manifestation contre le mariage pour tous, acte de malveillance sur un détecteur de chute de véhicule sur un pont routier au dessus des voies à l'entrée de la ligne à grande vitesse, TGV sont retardés suite à la panne d'un train de la société Colas Rail en ligne.
2013-04; Sud-Est; DIJON VILLE; PARIS LYON; 452; 452; 0; 29; 93.6; 
2013-04; Nord; DOUAI; PARIS NORD; 205; 205; 0; 25; 87.8; 
2013-04; Nord; PARIS NORD; DUNKERQUE; 126; 126; 0; 10; 92.1; Nombreux accidents de personnes et incidents d'installations techniques et électriques) répartis sur l'ensemble du parcours.
2013-04; Atlantique; PARIS MONTPARNASSE; LAVAL; 240; 240; 0; 13; 94.6; 
2013-04; Nord; LYON PART DIEU; LILLE; 244; 244; 0; 54; 77.9; Nombreux accidents de personnes et incidents d'installations techniques et électriques) répartis sur l'ensemble du parcours.
2013-04; Nord; LILLE; PARIS NORD; 589; 587; 2; 83; 85.9; 
2013-04; Sud-Est; PARIS LYON; MARSEILLE ST CHARLES; 474; 474; 0; 41; 91.4; 
2013-04; Est; PARIS EST; METZ; 306; 306; 0; 17; 94.4; 
2013-04; Atlantique; PARIS MONTPARNASSE; ANGOULEME; 320; 320; 0; 11; 96.6; 
2013-04; Est; NANCY; PARIS EST; 285; 285; 0; 12; 95.8; 
2013-04; Sud-Est; PARIS LYON; NIMES; 340; 340; 0; 44; 87.1; 
2013-04; Atlantique; PARIS MONTPARNASSE; QUIMPER; 141; 141; 0; 7; 95.0; Accident de personne sur un TER entre Rennes et Vitré retardant le TGV 8747 de 163 minutes, incident sur la matériel roulant (détection de chaleur anormale sur l'essieu d'un TER) retardant le TGV 8715 de 98 minutes.
2013-04; Atlantique; PARIS MONTPARNASSE; RENNES; 548; 548; 0; 36; 93.4; Problème d'alimentation électrique d'un TGV, accident de personne entre Rennes et Vitré, acte de malveillance sur les installation ferroviaire à l'entrée de la ligne à grande vitesse (câbles sectionnés), problème sur un train de l'entreprise Colas Rail en ligne.
2013-04; Sud-Est; SAINT ETIENNE CHATEAUCREUX; PARIS LYON; 114; 114; 0; 15; 86.8; 
2013-04; Sud-Est; PARIS LYON; SAINT ETIENNE CHATEAUCREUX; 115; 115; 0; 21; 81.7; 
2013-04; Atlantique; PARIS MONTPARNASSE; ST MALO; 54; 54; 0; 2; 96.3; Le 16, le TGV 8095 est retardé par l'accident de personne entre Rennes et Vitré. Le 24, le TGV 8081 est retardé au départ suite à un problème sur le matériel roulant.
2013-04; Sud-Est; PARIS LYON; VALENCE ALIXAN TGV; 253; 253; 0; 27; 89.3; 
2013-04; Atlantique; VANNES; PARIS MONTPARNASSE; 163; 163; 0; 13; 92.0; 
2013-04; Sud-Est; BESANCON FRANCHE COMTE TGV; PARIS LYON; 219; 219; 0; 16; 92.7; 
2013-05; Sud-Est; DIJON VILLE; PARIS LYON; 471; 471; 0; 40; 91.5; 
2013-05; Atlantique; LAVAL; PARIS MONTPARNASSE; 246; 245; 1; 14; 94.3; 
2013-05; Sud-Est; PARIS LYON; LE CREUSOT MONTCEAU MONTCHANIN; 207; 207; 0; 30; 85.5; 
2013-05; Nord; PARIS NORD; LILLE; 635; 635; 0; 63; 90.1; 
2013-05; Est; NANCY; PARIS EST; 294; 293; 1; 19; 93.5; 
2013-05; Atlantique; PARIS MONTPARNASSE; NANTES; 571; 570; 1; 26; 95.4; 
2013-05; Est; STRASBOURG; NANTES; 62; 62; 0; 0; 100.0; 
2013-05; Atlantique; POITIERS; PARIS MONTPARNASSE; 505; 503; 2; 32; 93.6; 
2013-05; Atlantique; PARIS MONTPARNASSE; QUIMPER; 153; 153; 0; 9; 94.1; 
2013-05; Sud-Est; ANNECY; PARIS LYON; 113; 113; 0; 6; 94.7; 
2013-05; Est; REIMS; PARIS EST; 213; 213; 0; 8; 96.2; 
2013-05; Est; PARIS EST; REIMS; 218; 218; 0; 14; 93.6; 
2013-05; Atlantique; PARIS MONTPARNASSE; RENNES; 573; 572; 1; 26; 95.5; 
2013-05; Est; PARIS EST; STRASBOURG; 486; 485; 1; 20; 95.9; 
2013-05; Sud-Est; PARIS LYON; TOULON; 218; 218; 0; 29; 86.7; Un incident caténaire au Creusot le 7 Mai a provoqué des retards très importants notament sur plusieurs trains de la liaison Paris-Toulon.
2013-05; Atlantique; TOURS; PARIS MONTPARNASSE; 181; 181; 0; 15; 91.7; 
2013-05; Sud-Est; VALENCE ALIXAN TGV; PARIS LYON; 261; 261; 0; 26; 90.0; 
2013-05; Sud-Est; PARIS LYON; VALENCE ALIXAN TGV; 269; 269; 0; 21; 92.2; 
2013-05; Atlantique; BORDEAUX ST JEAN; PARIS MONTPARNASSE; 669; 668; 1; 50; 92.5; 
2013-05; Atlantique; PARIS MONTPARNASSE; BORDEAUX ST JEAN; 636; 636; 0; 35; 94.5; 
2013-06; Sud-Est; CHAMBERY CHALLES LES EAUX; PARIS LYON; 200; 196; 4; 27; 86.2; 
2013-06; Sud-Est; PARIS LYON; DIJON VILLE; 458; 449; 9; 27; 94.0; 
2013-06; Nord; DOUAI; PARIS NORD; 201; 198; 3; 24; 87.9; 
2013-06; Sud-Est; PARIS LYON; LE CREUSOT MONTCEAU MONTCHANIN; 199; 199; 0; 22; 88.9; 
2013-10; Sud-Est; BESANCON FRANCHE COMTE TGV; PARIS LYON; 228; 228; 0; 13; 94.3; 
2013-11; Sud-Est; PARIS LYON; AIX EN PROVENCE TGV; 387; 387; 0; 68; 82.4; 
2013-11; Atlantique; BREST; PARIS MONTPARNASSE; 173; 173; 0; 14; 91.9; 
2013-11; Sud-Est; PARIS LYON; CHAMBERY CHALLES LES EAUX; 207; 207; 0; 38; 81.6; 
2013-11; Sud-Est; GRENOBLE; PARIS LYON; 217; 198; 19; 24; 87.9; 
2013-11; Atlantique; PARIS MONTPARNASSE; LA ROCHELLE VILLE; 206; 206; 0; 11; 94.7; 
2013-11; Atlantique; LE MANS; PARIS MONTPARNASSE; 436; 436; 0; 78; 82.1; 
2013-11; Nord; LILLE; MARSEILLE ST CHARLES; 146; 146; 0; 42; 71.2; Accidents de personne, alerte à la bombe à Lille, dérangements d'installation à Beugnatre et Croisilles, nombreuses limitations en vitesse pour cause de travaux, important épisode neigeux en Rhône Alpes les 20, 21 et 22 novembre.
2013-11; Atlantique; LYON PART DIEU; RENNES; 45; 45; 0; 8; 82.2; Retards en raison de nombreux événements particulièrement perturbants, notamment : dérangement de la signalisation à Avignon le 4, perte de contrôle à distance d'un aiguillage sur la ligne à grande vitesse atlantique près de St Pierre des Corps le 10, panne d'un TER en pleine voie le 19, intrusion d'individus dans le tunnel ferroviaire de Marseille le 20.
2013-11; Sud-Est; MARSEILLE ST CHARLES; PARIS LYON; 454; 454; 0; 56; 87.7; Les intempéries ayant touché la région lyonnaise les 20, 21 et 22 novembre ont provoqué des retards importants sur l'Axe TGV Sud Est.
2013-11; Sud-Est; PARIS LYON; MULHOUSE VILLE; 298; 298; 0; 23; 92.3; 
2013-11; Est; NANCY; PARIS EST; 281; 281; 0; 27; 90.4; 
2013-11; Atlantique; NANTES; PARIS MONTPARNASSE; 544; 542; 2; 48; 91.1; 
2013-11; Est; STRASBOURG; NANTES; 58; 58; 0; 7; 87.9; 
2013-11; Sud-Est; NICE VILLE; PARIS LYON; 195; 195; 0; 43; 77.9; Les intempéries ayant touché la région lyonnaise les 20, 21 et 22 novembre ont provoqué des retards importants sur l'Axe TGV Sud Est.
2013-11; Sud-Est; NIMES; PARIS LYON; 325; 324; 1; 71; 78.1; Les intempéries ayant touché la région lyonnaise les 20, 21 et 22 novembre ont provoqué des retards importants sur l'Axe TGV Sud Est.
2013-11; Sud-Est; PARIS LYON; NIMES; 306; 306; 0; 58; 81.0; 
2013-11; Atlantique; QUIMPER; PARIS MONTPARNASSE; 82; 81; 1; 4; 95.1; 
2013-11; Sud-Est; ANNECY; PARIS LYON; 132; 132; 0; 17; 87.1; 
2013-11; Est; REIMS; PARIS EST; 202; 202; 0; 26; 87.1; 
2013-11; Atlantique; PARIS MONTPARNASSE; TOULOUSE MATABIAU; 141; 141; 0; 19; 86.5; Retards en raison de nombreux événements particulièrement perturbants, notamment : heurt d'une voiture sur un passage à niveau à Port St Marie (47) le 1er, incident caténaire à Cérons (33) le 12, 19, heurt de plusieurs sangliers sur ligne à grande vitesse le 19.
2013-11; Atlantique; TOURS; PARIS MONTPARNASSE; 168; 167; 1; 26; 84.4; 
2013-11; Sud-Est; VALENCE ALIXAN TGV; PARIS LYON; 248; 247; 1; 54; 78.1; Les intempéries ayant touché la région lyonnaise les 20, 21 et 22 novembre ont provoqué des retards importants sur l'Axe TGV Sud Est.
2013-11; Sud-Est; PARIS LYON; VALENCE ALIXAN TGV; 249; 249; 0; 34; 86.3; 
2013-11; Nord; PARIS NORD; ARRAS; 328; 328; 0; 34; 89.6; 
2013-11; Atlantique; PARIS MONTPARNASSE; VANNES; 175; 175; 0; 11; 93.7; 
2013-11; Sud-Est; PARIS LYON; BELLEGARDE (AIN); 241; 241; 0; 47; 80.5; 
2013-12; Sud-Est; DIJON VILLE; PARIS LYON; 474; 474; 0; 43; 90.9; 
2013-12; Sud-Est; PARIS LYON; LE CREUSOT MONTCEAU MONTCHANIN; 208; 208; 0; 34; 83.7; 
2013-12; Atlantique; LE MANS; PARIS MONTPARNASSE; 466; 462; 4; 89; 80.7; 
2013-12; Atlantique; ANGERS SAINT LAUD; PARIS MONTPARNASSE; 472; 464; 8; 43; 90.7; 
2013-12; Sud-Est; PARIS LYON; LYON PART DIEU; 624; 624; 0; 29; 95.4; 
2013-12; Sud-Est; MACON LOCHE; PARIS LYON; 194; 194; 0; 25; 87.1; 
2013-12; Atlantique; ANGOULEME; PARIS MONTPARNASSE; 343; 334; 9; 33; 90.1; 
2013-12; Est; PARIS EST; NANCY; 293; 292; 1; 9; 96.9; 
2013-12; Sud-Est; NICE VILLE; PARIS LYON; 194; 193; 1; 24; 87.6; 
2013-12; Atlantique; TOURS; PARIS MONTPARNASSE; 194; 191; 3; 38; 80.1; 
2013-12; Nord; ARRAS; PARIS NORD; 322; 322; 0; 61; 81.1; 
2013-12; Atlantique; VANNES; PARIS MONTPARNASSE; 173; 170; 3; 15; 91.2; 
2013-12; Atlantique; PARIS MONTPARNASSE; BORDEAUX ST JEAN; 642; 633; 9; 29; 95.4; 
2014-01; Sud-Est; PARIS LYON; AIX EN PROVENCE TGV; 456; 456; 0; 46; 89.9; 
2014-01; Atlantique; BREST; PARIS MONTPARNASSE; 182; 182; 0; 14; 92.3; 
2014-01; Sud-Est; DIJON VILLE; PARIS LYON; 481; 481; 0; 23; 95.2; 
2014-01; Nord; DOUAI; PARIS NORD; 207; 207; 0; 27; 87.0; 
2014-01; Sud-Est; PARIS LYON; GRENOBLE; 257; 257; 0; 24; 90.7; 
2014-01; Nord; LILLE; PARIS NORD; 620; 620; 0; 61; 90.2; 
2014-01; Est; NANCY; PARIS EST; 296; 296; 0; 14; 95.3; 
2014-01; Est; PARIS EST; NANCY; 297; 297; 0; 12; 96.0; 
2014-01; Atlantique; NANTES; PARIS MONTPARNASSE; 551; 551; 0; 41; 92.6; 
2014-01; Sud-Est; NICE VILLE; PARIS LYON; 193; 190; 3; 29; 84.7; 
2014-01; Sud-Est; PARIS LYON; ANNECY; 151; 151; 0; 11; 92.7; 
2014-01; Atlantique; PARIS MONTPARNASSE; RENNES; 571; 571; 0; 25; 95.6; 
2014-01; Atlantique; PARIS MONTPARNASSE; ST MALO; 59; 59; 0; 0; 100.0; 
2014-01; Atlantique; ST PIERRE DES CORPS; PARIS MONTPARNASSE; 449; 449; 0; 41; 90.9; 
2014-01; Atlantique; PARIS MONTPARNASSE; ST PIERRE DES CORPS; 476; 476; 0; 44; 90.8; 
2014-01; Est; PARIS EST; STRASBOURG; 431; 431; 0; 45; 89.6; 
2014-01; Sud-Est; TOULON; PARIS LYON; 253; 251; 2; 41; 83.7; 
2014-01; Atlantique; PARIS MONTPARNASSE; VANNES; 170; 170; 0; 9; 94.7; 
2014-01; Sud-Est; BELLEGARDE (AIN); PARIS LYON; 255; 252; 3; 19; 92.5; 
2014-01; Sud-Est; BESANCON FRANCHE COMTE TGV; PARIS LYON; 230; 230; 0; 10; 95.7; 
2014-02; Atlantique; PARIS MONTPARNASSE; BREST; 170; 170; 0; 9; 94.7; 
2014-02; Nord; DOUAI; PARIS NORD; 187; 187; 0; 23; 87.7; 
2014-02; Nord; PARIS NORD; DOUAI; 192; 192; 0; 21; 89.1; 
2014-02; Sud-Est; GRENOBLE; PARIS LYON; 228; 228; 0; 9; 96.1; 
2014-02; Sud-Est; LE CREUSOT MONTCEAU MONTCHANIN; PARIS LYON; 199; 199; 0; 19; 90.5; 
2014-02; Sud-Est; PARIS LYON; LE CREUSOT MONTCEAU MONTCHANIN; 188; 188; 0; 23; 87.8; 
2014-02; Atlantique; ANGERS SAINT LAUD; PARIS MONTPARNASSE; 411; 411; 0; 32; 92.2; 
2014-02; Atlantique; PARIS MONTPARNASSE; ANGERS SAINT LAUD; 397; 397; 0; 22; 94.5; 
2014-02; Sud-Est; PARIS LYON; LYON PART DIEU; 581; 581; 0; 15; 97.4; 
2014-02; Atlantique; NANTES; PARIS MONTPARNASSE; 490; 490; 0; 39; 92.0; 
2014-02; Sud-Est; NIMES; PARIS LYON; 302; 302; 0; 22; 92.7; 
2014-02; Atlantique; PARIS MONTPARNASSE; QUIMPER; 119; 119; 0; 4; 96.6; 
2014-02; Est; REIMS; PARIS EST; 192; 192; 0; 9; 95.3; 
2014-02; Atlantique; ST MALO; PARIS MONTPARNASSE; 91; 91; 0; 6; 93.4; 
2014-02; Sud-Est; PARIS LYON; TOULON; 176; 176; 0; 11; 93.8; 
2014-02; Atlantique; PARIS MONTPARNASSE; TOULOUSE MATABIAU; 125; 125; 0; 19; 84.8; Dans la nuit du 27 au 28, un problème d'alimenation électrique à Langon (33) a engendré un fort retard (1h53) pour le TGV 8513. Le 10, une panne électrique, entrainant l'impossibilité de manœuvrer un aiguillage à Montbartier (82) à entrainé un retard de 88 et 72 minutes respectivement aux TGV 8501 et 8505. Le 26, des tirs de projectiles dans la banlieue sud de Bordeaux ont pertubé la circulation des trains et notamment du TGV 8511, arrivé à Toulouse avec 81 minutes de retard.
2014-02; Sud-Est; PARIS LYON; VALENCE ALIXAN TGV; 243; 243; 0; 11; 95.5; 
2014-02; Nord; ARRAS; PARIS NORD; 311; 310; 1; 36; 88.4; 
2014-02; Sud-Est; PARIS LYON; AVIGNON TGV; 450; 450; 0; 25; 94.4; 
2014-02; Sud-Est; PARIS LYON; BESANCON FRANCHE COMTE TGV; 199; 199; 0; 9; 95.5; 
2014-02; Atlantique; BORDEAUX ST JEAN; PARIS MONTPARNASSE; 580; 580; 0; 82; 85.9; 
2011-09; Sud-Est; PARIS LYON; AIX EN PROVENCE TGV; 414; 414; 0; 48; 88.4; 
2011-09; Sud-Est; PARIS LYON; DIJON VILLE; 411; 411; 0; 25; 93.9; 
2011-09; Sud-Est; GRENOBLE; PARIS LYON; 240; 239; 1; 23; 90.4; 
2011-09; Atlantique; LE MANS; PARIS MONTPARNASSE; 461; 461; 0; 26; 94.4; 
2011-09; Atlantique; PARIS MONTPARNASSE; LE MANS; 475; 475; 0; 99; 79.2; 
2011-09; Atlantique; PARIS MONTPARNASSE; ANGERS SAINT LAUD; 452; 452; 0; 33; 92.7; 
2011-09; Nord; PARIS NORD; LILLE; 613; 612; 1; 53; 91.3; 
2011-09; Sud-Est; MARSEILLE ST CHARLES; PARIS LYON; 475; 475; 0; 34; 92.8; 
2011-09; Est; MULHOUSE VILLE; PARIS LYON; 203; 203; 0; 31; 84.7; 
2011-09; Est; NANTES; STRASBOURG; 52; 52; 0; 5; 90.4; 
2011-09; Sud-Est; NIMES; PARIS LYON; 383; 383; 0; 50; 86.9; 
2011-09; Sud-Est; PARIS LYON; NIMES; 369; 369; 0; 60; 83.7; 
2011-09; Sud-Est; SAINT ETIENNE CHATEAUCREUX; PARIS LYON; 116; 116; 0; 11; 90.5; 
2011-09; Atlantique; ST MALO; PARIS MONTPARNASSE; 70; 70; 0; 0; 100.0; 
2011-09; Sud-Est; TOULON; PARIS LYON; 231; 230; 1; 23; 90.0; 
2011-09; Atlantique; PARIS MONTPARNASSE; VANNES; 173; 173; 0; 8; 95.4; 
2011-09; Sud-Est; BESANCON FRANCHE COMTE TGV; PARIS LYON; 146; 146; 0; 9; 93.8; 
2011-09; Atlantique; BORDEAUX ST JEAN; PARIS MONTPARNASSE; 639; 639; 0; 34; 94.7; 
2011-09; Atlantique; PARIS MONTPARNASSE; BORDEAUX ST JEAN; 656; 656; 0; 38; 94.2; 
2012-10; Est; PARIS EST; STRASBOURG; 484; 484; 0; 59; 87.8; 
2012-10; Sud-Est; PARIS LYON; TOULON; 188; 186; 2; 31; 83.3; 
2012-10; Atlantique; TOURS; PARIS MONTPARNASSE; 215; 211; 4; 29; 86.3; 
2012-10; Sud-Est; PARIS LYON; VALENCE ALIXAN TGV; 269; 264; 5; 43; 83.7; 
2012-10; Sud-Est; PARIS LYON; BELLEGARDE (AIN); 271; 270; 1; 42; 84.4; 
2012-10; Sud-Est; BESANCON FRANCHE COMTE TGV; PARIS LYON; 218; 216; 2; 17; 92.1; 
2012-11; Sud-Est; PARIS LYON; AIX EN PROVENCE TGV; 388; 388; 0; 43; 88.9; 
2012-11; Atlantique; BREST; PARIS MONTPARNASSE; 167; 167; 0; 10; 94.0; 
2012-11; Sud-Est; PARIS LYON; DIJON VILLE; 443; 443; 0; 33; 92.6; 
2012-11; Nord; DOUAI; PARIS NORD; 202; 202; 0; 12; 94.1; 
2012-11; Nord; PARIS NORD; DOUAI; 204; 204; 0; 17; 91.7; 
2012-11; Atlantique; PARIS MONTPARNASSE; LA ROCHELLE VILLE; 217; 217; 0; 30; 86.2; 
2012-11; Atlantique; LAVAL; PARIS MONTPARNASSE; 235; 235; 0; 26; 88.9; 
2012-11; Sud-Est; PARIS LYON; LE CREUSOT MONTCEAU MONTCHANIN; 198; 198; 0; 32; 83.8; 
2012-11; Nord; LILLE; LYON PART DIEU; 300; 300; 0; 51; 83.0; Liaison impactée principalement par des événements externes (vols de câble et accidents de personne) mais aussi par des dérangements des installations au sol et quelques difficultés liées au matériel.
2012-11; Nord; LILLE; PARIS NORD; 602; 601; 1; 48; 92.0; 
2012-11; Sud-Est; MONTPELLIER; LYON PART DIEU; 359; 359; 0; 65; 81.9; 
2012-11; Sud-Est; PARIS LYON; MULHOUSE VILLE; 299; 299; 0; 18; 94.0; 
2012-11; Est; NANCY; PARIS EST; 291; 291; 0; 13; 95.5; 
2012-11; Sud-Est; PERPIGNAN; PARIS LYON; 149; 148; 1; 20; 86.5; 
2012-11; Atlantique; RENNES; PARIS MONTPARNASSE; 556; 556; 0; 59; 89.4; 
2012-11; Atlantique; ST MALO; PARIS MONTPARNASSE; 98; 98; 0; 5; 94.9; 
2012-11; Sud-Est; PARIS LYON; TOULON; 201; 201; 0; 35; 82.6; D'importantes phases de travaux d'amélioration de l'infrastructure sur le tronçon Marseille Toulon nécessitent la mise en place de limitations de vitesse qui réduisent la fluidité des circulations.
2012-11; Atlantique; PARIS MONTPARNASSE; TOURS; 199; 199; 0; 14; 93.0; 
2012-11; Sud-Est; VALENCE ALIXAN TGV; PARIS LYON; 232; 232; 0; 35; 84.9; 
2012-11; Nord; ARRAS; PARIS NORD; 332; 332; 0; 22; 93.4; 
2012-11; Sud-Est; BELLEGARDE (AIN); PARIS LYON; 252; 252; 0; 47; 81.3; 
2012-11; Sud-Est; PARIS LYON; BESANCON FRANCHE COMTE TGV; 230; 230; 0; 22; 90.4; 
2012-11; Atlantique; PARIS MONTPARNASSE; BORDEAUX ST JEAN; 644; 644; 0; 55; 91.5; 
2012-12; Sud-Est; PARIS LYON; CHAMBERY CHALLES LES EAUX; 265; 265; 0; 42; 84.2; 
2012-12; Sud-Est; DIJON VILLE; PARIS LYON; 463; 463; 0; 65; 86.0; 
2012-12; Nord; PARIS NORD; DOUAI; 207; 207; 0; 21; 89.9; 
2012-12; Nord; LILLE; MARSEILLE ST CHARLES; 157; 157; 0; 50; 68.2; 
2012-12; Nord; MARSEILLE ST CHARLES; LILLE; 129; 129; 0; 31; 76.0; 
2012-12; Nord; PARIS NORD; LILLE; 602; 602; 0; 46; 92.4; 
2012-12; Sud-Est; MARSEILLE ST CHARLES; LYON PART DIEU; 594; 594; 0; 117; 80.3; 
2012-12; Sud-Est; MARSEILLE ST CHARLES; PARIS LYON; 485; 485; 0; 33; 93.2; 
2012-12; Est; NANCY; PARIS EST; 293; 293; 0; 44; 85.0; 
2012-12; Est; NANTES; STRASBOURG; 50; 50; 0; 10; 80.0; Problème sur les rames TGV suite aux episodes neigeux des 3, 6, 7, 8 et 9 décembre.
2012-12; Sud-Est; NICE VILLE; PARIS LYON; 152; 152; 0; 29; 80.9; 
2012-12; Sud-Est; PARIS LYON; PERPIGNAN; 159; 159; 0; 19; 88.1; 
2012-12; Est; PARIS EST; REIMS; 225; 225; 0; 29; 87.1; 
2012-12; Atlantique; PARIS MONTPARNASSE; RENNES; 559; 559; 0; 19; 96.6; 
2012-12; Sud-Est; TOULON; PARIS LYON; 204; 204; 0; 28; 86.3; 
2012-12; Sud-Est; VALENCE ALIXAN TGV; PARIS LYON; 247; 247; 0; 37; 85.0; 
2012-12; Atlantique; VANNES; PARIS MONTPARNASSE; 174; 174; 0; 16; 90.8; 
2012-12; Atlantique; PARIS MONTPARNASSE; VANNES; 182; 182; 0; 5; 97.3; 
2012-12; Sud-Est; AVIGNON TGV; PARIS LYON; 332; 332; 0; 45; 86.4; 
2012-12; Sud-Est; PARIS LYON; AVIGNON TGV; 410; 410; 0; 45; 89.0; 
2012-12; Sud-Est; BELLEGARDE (AIN); PARIS LYON; 272; 271; 1; 84; 69.0; La ligne du Haut Bugey a été particulièrement impactée par les intempéries
2013-01; Sud-Est; DIJON VILLE; PARIS LYON; 467; 467; 0; 66; 85.9; 
2013-01; Sud-Est; GRENOBLE; PARIS LYON; 247; 247; 0; 38; 84.6; 
2013-01; Nord; LILLE; MARSEILLE ST CHARLES; 157; 157; 0; 50; 68.2; Circulation touchée principalement par les chutes de neige du 14 au 23 janvier sur le Nord et du 15 au 21 janvier sur le Sud Est. Durant cette période, pour éviter les dangers liés aux projections de glace, la vitesse des trains sur la ligne à grande vitesse du Nord ou du Sud Est a été limitée à 230 km/h voire 170 km/h . Les installations au sol ainsi que les rames TGV ont également connu des difficultés liées à cet épisode neigeux.
2013-01; Sud-Est; LYON PART DIEU; MONTPELLIER; 407; 407; 0; 84; 79.4; Conditions météorologiques difficiles (en particulier la neige et le froid).
2013-01; Sud-Est; LYON PART DIEU; PARIS LYON; 645; 645; 0; 60; 90.7; 
2013-01; Sud-Est; PARIS LYON; MARSEILLE ST CHARLES; 516; 515; 1; 58; 88.7; 
2013-01; Atlantique; NANTES; PARIS MONTPARNASSE; 589; 589; 0; 55; 90.7; 
2013-01; Atlantique; PARIS MONTPARNASSE; QUIMPER; 146; 146; 0; 16; 89.0; 
2013-01; Est; REIMS; PARIS EST; 212; 212; 0; 37; 82.5; 
2013-01; Atlantique; TOULOUSE MATABIAU; PARIS MONTPARNASSE; 95; 95; 0; 17; 82.1; 
2013-01; Atlantique; VANNES; PARIS MONTPARNASSE; 180; 180; 0; 18; 90.0; 
2013-01; Sud-Est; BESANCON FRANCHE COMTE TGV; PARIS LYON; 220; 220; 0; 32; 85.5; 
2013-02; Sud-Est; PARIS LYON; AIX EN PROVENCE TGV; 360; 360; 0; 49; 86.4; 
2013-02; Sud-Est; CHAMBERY CHALLES LES EAUX; PARIS LYON; 236; 236; 0; 37; 84.3; 
2013-02; Sud-Est; PARIS LYON; CHAMBERY CHALLES LES EAUX; 252; 252; 0; 41; 83.7; 
2013-02; Atlantique; PARIS MONTPARNASSE; LA ROCHELLE VILLE; 204; 204; 0; 6; 97.1; 
2013-02; Sud-Est; LE CREUSOT MONTCEAU MONTCHANIN; PARIS LYON; 201; 201; 0; 32; 84.1; 
2013-02; Atlantique; LE MANS; PARIS MONTPARNASSE; 427; 427; 0; 44; 89.7; 
2013-02; Atlantique; ANGERS SAINT LAUD; PARIS MONTPARNASSE; 431; 430; 1; 29; 93.3; 
2013-02; Sud-Est; MONTPELLIER; LYON PART DIEU; 315; 315; 0; 64; 79.7; Conditions météorologiques difficiles (en particulier la neige et le froid).
2013-02; Atlantique; LYON PART DIEU; RENNES; 28; 28; 0; 2; 92.9; 
2013-02; Sud-Est; PARIS LYON; NIMES; 315; 315; 0; 62; 80.3; 
2013-02; Sud-Est; PARIS LYON; PERPIGNAN; 143; 143; 0; 16; 88.8; Conditions météorologiques difficiles (en particulier la neige et le froid).
2013-02; Atlantique; QUIMPER; PARIS MONTPARNASSE; 134; 134; 0; 13; 90.3; 
2013-02; Sud-Est; SAINT ETIENNE CHATEAUCREUX; PARIS LYON; 101; 101; 0; 20; 80.2; 
2013-02; Atlantique; TOULOUSE MATABIAU; PARIS MONTPARNASSE; 95; 95; 0; 12; 87.4; 
2013-02; Atlantique; PARIS MONTPARNASSE; TOULOUSE MATABIAU; 134; 134; 0; 15; 88.8; 
2013-02; Atlantique; TOURS; PARIS MONTPARNASSE; 164; 164; 0; 17; 89.6; 
2013-03; Atlantique; BREST; PARIS MONTPARNASSE; 50; 50; 0; 3; 94.0; Épisode neigeux avec fortes répercussions sur la production les 12 et 13 mars. TGV 8624 du 3 mars en retard au départ suite mouvement social. TGV 8672 du 16 mars décontrôle d'aiguille près de Plounérin.
2013-03; Sud-Est; PARIS LYON; GRENOBLE; 264; 262; 2; 29; 88.9; La journée du 12 mars a été très perturbée par les intempéries (neige et pluies verglassantes) entrainant de nombreux et importants retards avoisinants les 2 heures.
2013-03; Atlantique; PARIS MONTPARNASSE; LAVAL; 228; 228; 0; 12; 94.7; 
2013-03; Atlantique; PARIS MONTPARNASSE; ANGERS SAINT LAUD; 446; 446; 0; 30; 93.3; 
2013-03; Nord; LILLE; MARSEILLE ST CHARLES; 153; 148; 5; 40; 73.0; Les très importantes chutes de neige avec formation de congères survenues dans le nord à partir du lundi 11 mars dans la soirée jusqu'au vendredi 15 mars et également dans le sud-est le 12 mars ont fortement perturbé la production. Elles ont imposé des limitations de vitesse pour éviter les projections de glace. Les conditions climatiques difficiles de ces derniers mois ont par ailleurs perturbé les différents programmes travaux sur les voies prolongeant ainsi certaines limitations de vitesse au-delà des délais prévus. notamment sur Arras et la périphérie de Lyon.
2013-03; Nord; PARIS NORD; LILLE; 617; 599; 18; 98; 83.6; 
2013-03; Sud-Est; MARSEILLE ST CHARLES; LYON PART DIEU; 596; 592; 4; 155; 73.8; Les travaux de rénovation de la voie à l'entrée de Lyon génèrent des retards chroniques sur les TGV empruntant cette zone.
2013-03; Sud-Est; PARIS LYON; MACON LOCHE; 219; 219; 0; 21; 90.4; 
2013-03; Atlantique; ANGOULEME; PARIS MONTPARNASSE; 326; 326; 0; 46; 85.9; 
2013-03; Sud-Est; PARIS LYON; MULHOUSE VILLE; 307; 305; 2; 27; 91.1; 
2013-03; Est; NANTES; STRASBOURG; 46; 46; 0; 6; 87.0; Les intempéries (neige) du 12, 13 et 14 mars ont entrainé de nombreux et importants retards.
2013-03; Est; STRASBOURG; NANTES; 46; 46; 0; 4; 91.3; 
2013-03; Sud-Est; NICE VILLE; PARIS LYON; 156; 154; 2; 23; 85.1; 
2013-03; Sud-Est; PARIS LYON; NIMES; 348; 348; 0; 53; 84.8; 
2013-03; Sud-Est; PERPIGNAN; PARIS LYON; 155; 155; 0; 21; 86.5; La journée du 12 mars a été très perturbée par les intempéries (neige et pluies verglassantes) entrainant de nombreux et importants retards avoisinants les 2 heures.
2013-03; Atlantique; QUIMPER; PARIS MONTPARNASSE; 132; 132; 0; 13; 90.2; 
2013-03; Atlantique; PARIS MONTPARNASSE; QUIMPER; 146; 146; 0; 16; 89.0; 
2013-03; Est; PARIS EST; REIMS; 216; 216; 0; 27; 87.5; 
2013-03; Atlantique; ST PIERRE DES CORPS; PARIS MONTPARNASSE; 425; 425; 0; 87; 79.5; Des événements avec de fortes répercussions sur la circulation : acte de malveillance (dépôt d'objet sur la voie) en amont de St-Pierre-des-Corps le 2 mars, incident sur la caténaire en amont de St-Pierre-des-Corps le 7 mars, panne d'un train de travaux sur la ligne à grande vitesse le 15 mars, rail cassé en amont de St-Pierre-des-Corps le 17 mars, restituation tardive de travaux sur la ligne à grande vitesse le 27 mars.
2013-03; Est; STRASBOURG; PARIS EST; 478; 478; 0; 66; 86.2; 
2013-03; Sud-Est; TOULON; PARIS LYON; 237; 236; 1; 26; 89.0; La journée du 12 mars a été très perturbée par les intempéries (neige et pluies verglassantes) entrainant de nombreux et importants retards avoisinants les 2 heures.
2013-03; Sud-Est; PARIS LYON; TOULON; 211; 211; 0; 34; 83.9; 
2014-02; Sud-Est; ANNECY; PARIS LYON; 133; 133; 0; 10; 92.5; 
2014-02; Atlantique; ST PIERRE DES CORPS; PARIS MONTPARNASSE; 394; 394; 0; 63; 84.0; 
2014-02; Est; STRASBOURG; PARIS EST; 434; 434; 0; 21; 95.2; 
2014-02; Atlantique; TOULOUSE MATABIAU; PARIS MONTPARNASSE; 68; 68; 0; 7; 89.7; 
2014-02; Sud-Est; PARIS LYON; BELLEGARDE (AIN); 220; 220; 0; 15; 93.2; 
2011-09; Sud-Est; DIJON VILLE; PARIS LYON; 417; 417; 0; 28; 93.3; 
2011-09; Nord; PARIS NORD; DOUAI; 156; 156; 0; 24; 84.6; 
2011-09; Nord; DUNKERQUE; PARIS NORD; 200; 200; 0; 14; 93.0; 
2011-09; Nord; PARIS NORD; DUNKERQUE; 171; 171; 0; 18; 89.5; 
2011-09; Sud-Est; PARIS LYON; GRENOBLE; 134; 134; 0; 18; 86.6; 
2011-09; Sud-Est; LE CREUSOT MONTCEAU MONTCHANIN; PARIS LYON; 216; 216; 0; 27; 87.5; 
2011-09; Sud-Est; LYON PART DIEU; MARSEILLE ST CHARLES; 543; 543; 0; 97; 82.1; 
2011-09; Sud-Est; MONTPELLIER; LYON PART DIEU; 353; 353; 0; 61; 82.7; 
2011-09; Sud-Est; PARIS LYON; LYON PART DIEU; 637; 637; 0; 55; 91.4; 
2011-09; Atlantique; LYON PART DIEU; RENNES; 52; 52; 0; 21; 59.6; 
2011-09; Sud-Est; PARIS LYON; MACON LOCHE; 181; 181; 0; 17; 90.6; 
2011-09; Atlantique; PARIS MONTPARNASSE; NANTES; 595; 595; 0; 24; 96.0; 
2011-09; Atlantique; TOURS; PARIS MONTPARNASSE; 139; 139; 0; 5; 96.4; 
2011-09; Nord; PARIS NORD; ARRAS; 361; 361; 0; 29; 92.0; 
2011-09; Atlantique; VANNES; PARIS MONTPARNASSE; 171; 171; 0; 6; 96.5; 
2011-09; Sud-Est; PARIS LYON; AVIGNON TGV; 384; 384; 0; 51; 86.7; 
2011-10; Atlantique; BREST; PARIS MONTPARNASSE; 183; 183; 0; 20; 89.1; 
2011-10; Nord; PARIS NORD; DOUAI; 158; 158; 0; 24; 84.8; 
2011-10; Nord; PARIS NORD; DUNKERQUE; 174; 174; 0; 17; 90.2; 
2011-10; Sud-Est; LE CREUSOT MONTCEAU MONTCHANIN; PARIS LYON; 220; 220; 0; 32; 85.5; 
2011-10; Sud-Est; MONTPELLIER; LYON PART DIEU; 359; 357; 2; 72; 79.8; 
2011-10; Atlantique; LYON PART DIEU; RENNES; 60; 60; 0; 30; 50.0; 
2011-10; Sud-Est; PARIS LYON; MACON LOCHE; 184; 183; 1; 9; 95.1; 
2011-10; Est; NANCY; PARIS EST; 292; 292; 0; 19; 93.5; 
2011-10; Est; PARIS EST; NANCY; 294; 292; 2; 19; 93.5; 
2011-10; Sud-Est; NIMES; PARIS LYON; 394; 393; 1; 51; 87.0; 
2011-10; Atlantique; PARIS MONTPARNASSE; POITIERS; 454; 454; 0; 39; 91.4; 
2011-10; Atlantique; PARIS MONTPARNASSE; QUIMPER; 171; 171; 0; 19; 88.9; 
2011-10; Sud-Est; ANNECY; PARIS LYON; 200; 200; 0; 15; 92.5; 
2011-10; Atlantique; PARIS MONTPARNASSE; ST MALO; 36; 36; 0; 1; 97.2; 
2011-10; Sud-Est; TOULON; PARIS LYON; 248; 248; 0; 29; 88.3; 
2011-10; Sud-Est; PARIS LYON; TOULON; 236; 236; 0; 30; 87.3; 
2011-10; Sud-Est; PARIS LYON; BELLEGARDE (AIN); 261; 261; 0; 37; 85.8; 
2011-11; Sud-Est; DIJON VILLE; PARIS LYON; 411; 410; 1; 20; 95.1; 
2011-11; Sud-Est; GRENOBLE; PARIS LYON; 234; 233; 1; 24; 89.7; 
2011-11; Sud-Est; PARIS LYON; GRENOBLE; 133; 133; 0; 17; 87.2; 
2011-11; Atlantique; LAVAL; PARIS MONTPARNASSE; 229; 226; 3; 32; 85.8; 
2011-11; Nord; MARSEILLE ST CHARLES; LILLE; 197; 197; 0; 34; 82.7; 
2011-11; Sud-Est; MARSEILLE ST CHARLES; LYON PART DIEU; 541; 541; 0; 89; 83.5; 
2011-11; Sud-Est; PARIS LYON; MACON LOCHE; 179; 179; 0; 12; 93.3; 
2011-11; Est; STRASBOURG; NANTES; 49; 49; 0; 6; 87.8; 
2011-11; Sud-Est; NICE VILLE; PARIS LYON; 156; 154; 2; 12; 92.2; 
2011-11; Atlantique; POITIERS; PARIS MONTPARNASSE; 489; 487; 2; 45; 90.8; 
2011-11; Sud-Est; ANNECY; PARIS LYON; 192; 192; 0; 12; 93.8; 
2011-11; Sud-Est; PARIS LYON; ANNECY; 197; 197; 0; 14; 92.9; 
2011-11; Atlantique; RENNES; PARIS MONTPARNASSE; 533; 519; 14; 87; 83.2; 
2011-11; Atlantique; ST MALO; PARIS MONTPARNASSE; 74; 74; 0; 1; 98.6; 
2011-11; Sud-Est; VALENCE ALIXAN TGV; PARIS LYON; 234; 233; 1; 38; 83.7; 
2011-11; Sud-Est; PARIS LYON; VALENCE ALIXAN TGV; 264; 264; 0; 22; 91.7; 
2011-11; Nord; ARRAS; PARIS NORD; 322; 322; 0; 25; 92.2; 
2011-11; Nord; PARIS NORD; ARRAS; 352; 352; 0; 22; 93.8; 
2011-11; Sud-Est; PARIS LYON; AVIGNON TGV; 390; 390; 0; 42; 89.2; 
2011-11; Sud-Est; BESANCON FRANCHE COMTE TGV; PARIS LYON; 140; 140; 0; 10; 92.9; 
2011-12; Sud-Est; PARIS LYON; AIX EN PROVENCE TGV; 417; 417; 0; 29; 93.0; 
2011-12; Sud-Est; PARIS LYON; GRENOBLE; 218; 218; 0; 20; 90.8; 
2011-12; Atlantique; PARIS MONTPARNASSE; ANGERS SAINT LAUD; 449; 447; 2; 40; 91.1; 
2011-12; Nord; LYON PART DIEU; LILLE; 291; 291; 0; 46; 84.2; 
2011-12; Nord; MARSEILLE ST CHARLES; LILLE; 152; 152; 0; 26; 82.9; 
2011-12; Sud-Est; LYON PART DIEU; MARSEILLE ST CHARLES; 515; 515; 0; 139; 73.0; 
2011-12; Sud-Est; LYON PART DIEU; MONTPELLIER; 376; 375; 1; 70; 81.3; 
2011-12; Sud-Est; MONTPELLIER; LYON PART DIEU; 369; 369; 0; 55; 85.1; 
2011-12; Atlantique; LYON PART DIEU; RENNES; 51; 51; 0; 8; 84.3; 
2011-12; Atlantique; ANGOULEME; PARIS MONTPARNASSE; 349; 349; 0; 43; 87.7; 
2011-12; Est; PARIS LYON; MULHOUSE VILLE; 283; 283; 0; 30; 89.4; 
2011-12; Est; NANCY; PARIS EST; 298; 298; 0; 17; 94.3; 
2011-12; Est; NANTES; STRASBOURG; 38; 38; 0; 3; 92.1; 
2011-12; Sud-Est; NIMES; PARIS LYON; 337; 337; 0; 39; 88.4; 
2011-12; Sud-Est; PARIS LYON; NIMES; 324; 324; 0; 30; 90.7; 
2011-12; Atlantique; POITIERS; PARIS MONTPARNASSE; 501; 501; 0; 52; 89.6; 
2011-12; Atlantique; PARIS MONTPARNASSE; POITIERS; 479; 479; 0; 23; 95.2; 
2011-12; Sud-Est; PARIS LYON; ANNECY; 208; 208; 0; 12; 94.2; 
2011-12; Sud-Est; PARIS LYON; TOULON; 228; 228; 0; 20; 91.2; 
2011-12; Atlantique; PARIS MONTPARNASSE; TOULOUSE MATABIAU; 140; 140; 0; 11; 92.1; 
2011-12; Sud-Est; VALENCE ALIXAN TGV; PARIS LYON; 200; 200; 0; 30; 85.0; 
2011-12; Sud-Est; PARIS LYON; AVIGNON TGV; 383; 383; 0; 20; 94.8; 
2011-12; Sud-Est; BELLEGARDE (AIN); PARIS LYON; 270; 270; 0; 37; 86.3; 
2011-12; Sud-Est; PARIS LYON; BELLEGARDE (AIN); 276; 276; 0; 41; 85.1; 
2011-12; Sud-Est; PARIS LYON; BESANCON FRANCHE COMTE TGV; 222; 222; 0; 16; 92.8; 
2011-12; Atlantique; BORDEAUX ST JEAN; PARIS MONTPARNASSE; 656; 656; 0; 51; 92.2; 
2012-01; Atlantique; PARIS MONTPARNASSE; BREST; 192; 192; 0; 6; 96.9; 
2012-01; Atlantique; PARIS MONTPARNASSE; LAVAL; 241; 241; 0; 19; 92.1; 
2012-01; Nord; LILLE; LYON PART DIEU; 309; 309; 0; 29; 90.6; 
2012-01; Nord; LILLE; PARIS NORD; 625; 625; 0; 84; 86.6; 
2012-01; Sud-Est; LYON PART DIEU; PARIS LYON; 639; 639; 0; 29; 95.5; 
2012-01; Sud-Est; MACON LOCHE; PARIS LYON; 185; 185; 0; 13; 93.0; 
2012-01; Sud-Est; MARSEILLE ST CHARLES; PARIS LYON; 480; 480; 0; 15; 96.9; 
2012-01; Est; NANTES; STRASBOURG; 54; 54; 0; 3; 94.4; 
2012-01; Sud-Est; NIMES; PARIS LYON; 373; 373; 0; 32; 91.4; 
2012-01; Atlantique; PARIS MONTPARNASSE; POITIERS; 477; 477; 0; 31; 93.5; 
2012-01; Sud-Est; PARIS LYON; SAINT ETIENNE CHATEAUCREUX; 118; 118; 0; 15; 87.3; 
2012-01; Atlantique; ST MALO; PARIS MONTPARNASSE; 100; 100; 0; 7; 93.0; 
2012-01; Sud-Est; TOULON; PARIS LYON; 242; 242; 0; 11; 95.5; 
2012-01; Atlantique; PARIS MONTPARNASSE; TOULOUSE MATABIAU; 149; 149; 0; 17; 88.6; 
2012-01; Atlantique; PARIS MONTPARNASSE; TOURS; 190; 190; 0; 27; 85.8; 
2012-01; Atlantique; VANNES; PARIS MONTPARNASSE; 176; 176; 0; 12; 93.2; 
2014-01; Sud-Est; AVIGNON TGV; PARIS LYON; 424; 422; 2; 57; 86.5; 
2014-02; Sud-Est; CHAMBERY CHALLES LES EAUX; PARIS LYON; 268; 268; 0; 27; 89.9; 
2014-02; Sud-Est; PARIS LYON; CHAMBERY CHALLES LES EAUX; 291; 291; 0; 24; 91.8; 
2014-02; Atlantique; LA ROCHELLE VILLE; PARIS MONTPARNASSE; 203; 203; 0; 18; 91.1; 
2014-02; Atlantique; PARIS MONTPARNASSE; LAVAL; 219; 219; 0; 19; 91.3; 
2014-02; Atlantique; LE MANS; PARIS MONTPARNASSE; 445; 445; 0; 63; 85.8; 
2014-02; Nord; LILLE; MARSEILLE ST CHARLES; 111; 111; 0; 9; 91.9; 
2014-02; Nord; PARIS NORD; LILLE; 575; 575; 0; 18; 96.9; 
2014-02; Sud-Est; MONTPELLIER; LYON PART DIEU; 359; 358; 1; 61; 83.0; 
2014-02; Est; NANCY; PARIS EST; 268; 268; 0; 6; 97.8; 
2014-02; Est; PARIS EST; NANCY; 267; 267; 0; 5; 98.1; 
2014-02; Est; NANTES; STRASBOURG; 36; 36; 0; 4; 88.9; 
2014-02; Est; STRASBOURG; NANTES; 36; 36; 0; 3; 91.7; 
2014-02; Sud-Est; PARIS LYON; NICE VILLE; 159; 159; 0; 19; 88.1; 
2014-02; Sud-Est; PERPIGNAN; PARIS LYON; 139; 139; 0; 5; 96.4; 
2014-02; Atlantique; PARIS MONTPARNASSE; RENNES; 511; 511; 0; 28; 94.5; 
2014-02; Sud-Est; SAINT ETIENNE CHATEAUCREUX; PARIS LYON; 5; 5; 0; 1; 80.0; 
2011-09; Sud-Est; CHAMBERY CHALLES LES EAUX; PARIS LYON; 167; 167; 0; 20; 88.0; 
2011-09; Nord; DOUAI; PARIS NORD; 132; 132; 0; 11; 91.7; 
2011-09; Atlantique; LA ROCHELLE VILLE; PARIS MONTPARNASSE; 163; 163; 0; 6; 96.3; 
2011-09; Nord; LILLE; MARSEILLE ST CHARLES; 159; 159; 0; 31; 80.5; 
2011-09; Nord; LILLE; PARIS NORD; 628; 628; 0; 49; 92.2; 
2011-09; Sud-Est; LYON PART DIEU; PARIS LYON; 609; 609; 0; 37; 93.9; 
2011-09; Est; METZ; PARIS EST; 289; 288; 1; 15; 94.8; 
2011-09; Sud-Est; MONTPELLIER; PARIS LYON; 383; 383; 0; 33; 91.4; 
2011-09; Sud-Est; PARIS LYON; MONTPELLIER; 369; 369; 0; 49; 86.7; 
2011-09; Est; PARIS LYON; MULHOUSE VILLE; 217; 217; 0; 23; 89.4; 
2011-09; Est; PARIS EST; NANCY; 288; 288; 0; 11; 96.2; 
2011-09; Sud-Est; NICE VILLE; PARIS LYON; 183; 183; 0; 21; 88.5; 
2011-09; Atlantique; POITIERS; PARIS MONTPARNASSE; 495; 495; 0; 29; 94.1; 
2011-09; Atlantique; PARIS MONTPARNASSE; POITIERS; 457; 457; 0; 38; 91.7; 
2011-09; Atlantique; QUIMPER; PARIS MONTPARNASSE; 150; 150; 0; 5; 96.7; 
2011-09; Atlantique; RENNES; PARIS MONTPARNASSE; 560; 560; 0; 16; 97.1; 
2011-09; Atlantique; PARIS MONTPARNASSE; RENNES; 565; 565; 0; 49; 91.3; 
2011-09; Atlantique; PARIS MONTPARNASSE; ST PIERRE DES CORPS; 448; 448; 0; 75; 83.3; 
2011-09; Atlantique; PARIS MONTPARNASSE; TOULOUSE MATABIAU; 121; 121; 0; 9; 92.6; 
2011-09; Sud-Est; BELLEGARDE (AIN); PARIS LYON; 257; 257; 0; 13; 94.9; 
2011-09; Sud-Est; PARIS LYON; BESANCON FRANCHE COMTE TGV; 178; 178; 0; 16; 91.0; 
2011-10; Sud-Est; DIJON VILLE; PARIS LYON; 398; 397; 1; 23; 94.2; 
2011-10; Sud-Est; PARIS LYON; DIJON VILLE; 393; 393; 0; 16; 95.9; 
2011-10; Sud-Est; GRENOBLE; PARIS LYON; 242; 241; 1; 20; 91.7; 
2011-10; Atlantique; LA ROCHELLE VILLE; PARIS MONTPARNASSE; 144; 143; 1; 17; 88.1; 
2011-10; Atlantique; LAVAL; PARIS MONTPARNASSE; 233; 233; 0; 22; 90.6; 
2011-10; Atlantique; LE MANS; PARIS MONTPARNASSE; 461; 459; 2; 96; 79.1; 
2011-10; Nord; LILLE; LYON PART DIEU; 345; 343; 2; 33; 90.4; 
2011-10; Nord; LYON PART DIEU; LILLE; 315; 314; 1; 74; 76.4; 
2011-10; Sud-Est; PARIS LYON; MARSEILLE ST CHARLES; 514; 513; 1; 41; 92.0; 
2011-10; Atlantique; ANGOULEME; PARIS MONTPARNASSE; 358; 358; 0; 67; 81.3; 
2011-10; Est; MULHOUSE VILLE; PARIS LYON; 211; 211; 0; 17; 91.9; 
2011-10; Est; PARIS LYON; MULHOUSE VILLE; 226; 226; 0; 30; 86.7; 
2011-10; Sud-Est; PARIS LYON; NICE VILLE; 156; 156; 0; 27; 82.7; 
2011-10; Sud-Est; PERPIGNAN; PARIS LYON; 135; 135; 0; 18; 86.7; 
2011-10; Atlantique; POITIERS; PARIS MONTPARNASSE; 503; 502; 1; 77; 84.7; 
2011-10; Sud-Est; PARIS LYON; ANNECY; 205; 204; 1; 16; 92.2; 
2011-10; Sud-Est; SAINT ETIENNE CHATEAUCREUX; PARIS LYON; 116; 116; 0; 18; 84.5; 
2011-10; Sud-Est; PARIS LYON; SAINT ETIENNE CHATEAUCREUX; 119; 119; 0; 21; 82.4; 
2011-10; Atlantique; ST PIERRE DES CORPS; PARIS MONTPARNASSE; 461; 460; 1; 96; 79.1; 
2011-10; Atlantique; PARIS MONTPARNASSE; ST PIERRE DES CORPS; 444; 444; 0; 73; 83.6; 
2011-10; Atlantique; PARIS MONTPARNASSE; TOURS; 186; 186; 0; 51; 72.6; 
2011-10; Sud-Est; AVIGNON TGV; PARIS LYON; 388; 386; 2; 56; 85.5; 
2011-10; Sud-Est; PARIS LYON; BESANCON FRANCHE COMTE TGV; 152; 152; 0; 10; 93.4; 
2011-10; Atlantique; PARIS MONTPARNASSE; BORDEAUX ST JEAN; 659; 659; 0; 64; 90.3; 
2011-11; Atlantique; BREST; PARIS MONTPARNASSE; 176; 172; 4; 25; 85.5; 
2011-11; Atlantique; LA ROCHELLE VILLE; PARIS MONTPARNASSE; 149; 149; 0; 12; 91.9; 
2011-11; Atlantique; PARIS MONTPARNASSE; LE MANS; 492; 481; 11; 51; 89.4; 
2011-11; Nord; PARIS NORD; LILLE; 602; 602; 0; 42; 93.0; 
2011-11; Sud-Est; LYON PART DIEU; MONTPELLIER; 359; 358; 1; 66; 81.6; 
2011-11; Atlantique; LYON PART DIEU; RENNES; 65; 65; 0; 21; 67.7; 
2011-11; Sud-Est; MARSEILLE ST CHARLES; PARIS LYON; 474; 474; 0; 31; 93.5; 
2011-11; Atlantique; ANGOULEME; PARIS MONTPARNASSE; 364; 362; 2; 32; 91.2; 
2011-11; Sud-Est; MONTPELLIER; PARIS LYON; 383; 382; 1; 49; 87.2; 
2011-11; Sud-Est; PARIS LYON; MONTPELLIER; 367; 367; 0; 33; 91.0; 
2011-11; Est; NANTES; STRASBOURG; 50; 50; 0; 7; 86.0; 
2011-11; Sud-Est; PARIS LYON; NIMES; 365; 365; 0; 37; 89.9; 
2011-11; Atlantique; PARIS MONTPARNASSE; POITIERS; 441; 436; 5; 37; 91.5; 
2011-11; Sud-Est; PARIS LYON; SAINT ETIENNE CHATEAUCREUX; 106; 106; 0; 12; 88.7; 
2011-11; Atlantique; ST PIERRE DES CORPS; PARIS MONTPARNASSE; 455; 453; 2; 65; 85.7; 
2011-11; Atlantique; PARIS MONTPARNASSE; TOURS; 182; 180; 2; 21; 88.3; 
2011-11; Atlantique; VANNES; PARIS MONTPARNASSE; 174; 169; 5; 15; 91.1; 
2011-11; Sud-Est; PARIS LYON; BESANCON FRANCHE COMTE TGV; 160; 160; 0; 11; 93.1; 
2011-12; Sud-Est; AIX EN PROVENCE TGV; PARIS LYON; 430; 430; 0; 62; 85.6; 
2011-12; Sud-Est; CHAMBERY CHALLES LES EAUX; PARIS LYON; 231; 230; 1; 44; 80.9; 
2011-12; Sud-Est; PARIS LYON; CHAMBERY CHALLES LES EAUX; 248; 248; 0; 41; 83.5; 
2011-12; Nord; DOUAI; PARIS NORD; 184; 183; 1; 18; 90.2; 
2011-12; Nord; PARIS NORD; DOUAI; 194; 194; 0; 26; 86.6; 
2011-12; Nord; PARIS NORD; DUNKERQUE; 166; 166; 0; 8; 95.2; 
2011-12; Sud-Est; GRENOBLE; PARIS LYON; 242; 242; 0; 17; 93.0; 
2011-12; Atlantique; PARIS MONTPARNASSE; LE MANS; 477; 476; 1; 65; 86.3; 
2011-12; Atlantique; ANGERS SAINT LAUD; PARIS MONTPARNASSE; 490; 486; 4; 49; 89.9; 
2011-12; Nord; LILLE; MARSEILLE ST CHARLES; 138; 138; 0; 33; 76.1; 
2011-12; Sud-Est; MARSEILLE ST CHARLES; LYON PART DIEU; 580; 580; 0; 90; 84.5; 
2011-12; Sud-Est; LYON PART DIEU; PARIS LYON; 605; 605; 0; 23; 96.2; 
2011-12; Sud-Est; PARIS LYON; MARSEILLE ST CHARLES; 414; 414; 0; 23; 94.4; 
2011-12; Est; METZ; PARIS EST; 298; 298; 0; 9; 97.0; 
2011-12; Sud-Est; NICE VILLE; PARIS LYON; 167; 167; 0; 31; 81.4; 
2011-12; Sud-Est; PARIS LYON; PERPIGNAN; 105; 105; 0; 7; 93.3; 
2011-12; Sud-Est; PARIS LYON; SAINT ETIENNE CHATEAUCREUX; 118; 118; 0; 12; 89.8; 
2011-12; Atlantique; ST MALO; PARIS MONTPARNASSE; 93; 93; 0; 4; 95.7; 
2011-12; Atlantique; PARIS MONTPARNASSE; ST PIERRE DES CORPS; 470; 470; 0; 59; 87.4; 
2011-12; Nord; PARIS NORD; ARRAS; 373; 373; 0; 44; 88.2; 
2013-06; Nord; LYON PART DIEU; LILLE; 243; 237; 6; 46; 80.6; 
2013-06; Sud-Est; MACON LOCHE; PARIS LYON; 190; 186; 4; 20; 89.2; 
2013-06; Sud-Est; PARIS LYON; MARSEILLE ST CHARLES; 468; 459; 9; 34; 92.6; 
2013-06; Est; METZ; PARIS EST; 289; 285; 4; 26; 90.9; 
2013-06; Atlantique; ANGOULEME; PARIS MONTPARNASSE; 341; 332; 9; 29; 91.3; 
2013-06; Atlantique; NANTES; PARIS MONTPARNASSE; 555; 546; 9; 24; 95.6; 
2013-06; Atlantique; PARIS MONTPARNASSE; NANTES; 550; 540; 10; 34; 93.7; 
2013-06; Est; STRASBOURG; NANTES; 59; 58; 1; 5; 91.4; 
2013-06; Atlantique; POITIERS; PARIS MONTPARNASSE; 481; 472; 9; 25; 94.7; 
2013-06; Atlantique; PARIS MONTPARNASSE; RENNES; 539; 531; 8; 31; 94.2; 
2013-06; Sud-Est; SAINT ETIENNE CHATEAUCREUX; PARIS LYON; 112; 110; 2; 10; 90.9; 
2013-06; Atlantique; PARIS MONTPARNASSE; ST MALO; 54; 53; 1; 4; 92.5; 
2013-06; Atlantique; ST PIERRE DES CORPS; PARIS MONTPARNASSE; 432; 420; 12; 64; 84.8; 
2013-06; Sud-Est; TOULON; PARIS LYON; 222; 216; 6; 38; 82.4; Plusieurs limitations de vitesse sur des zones de modernisation de l'infrastructure ont fragilisé la circulation des trains sur l'axe Marseille-Nice.
2013-06; Atlantique; TOULOUSE MATABIAU; PARIS MONTPARNASSE; 84; 84; 0; 7; 91.7; 
2013-06; Nord; ARRAS; PARIS NORD; 328; 320; 8; 34; 89.4; 
2013-06; Atlantique; VANNES; PARIS MONTPARNASSE; 167; 163; 4; 10; 93.9; 
2013-06; Sud-Est; AVIGNON TGV; PARIS LYON; 365; 359; 6; 54; 85.0; 
2013-06; Sud-Est; BELLEGARDE (AIN); PARIS LYON; 259; 253; 6; 40; 84.2; 
2013-06; Atlantique; BORDEAUX ST JEAN; PARIS MONTPARNASSE; 661; 649; 12; 72; 88.9; 
2013-07; Nord; DOUAI; PARIS NORD; 187; 186; 1; 37; 80.1; 
2013-07; Sud-Est; PARIS LYON; LE CREUSOT MONTCEAU MONTCHANIN; 207; 207; 0; 26; 87.4; 
2013-07; Nord; PARIS NORD; LILLE; 458; 457; 1; 51; 88.8; 
2013-07; Atlantique; LYON PART DIEU; RENNES; 31; 31; 0; 9; 71.0; Ralentissements pour fortes chaleurs et incidents près de Lyon.
2013-07; Atlantique; RENNES; LYON PART DIEU; 81; 81; 0; 12; 85.2; Ralentissements pour fortes chaleurs et incidents près de Lyon.
2013-07; Sud-Est; PARIS LYON; MARSEILLE ST CHARLES; 490; 490; 0; 52; 89.4; Des orages et fortes chaleurs, des vols de câbles et accidents de personne perturbent la circulation des TGV.
2013-07; Sud-Est; MONTPELLIER; PARIS LYON; 375; 375; 0; 57; 84.8; Des orages et fortes chaleurs, des vols de câbles et accidents de personne perturbent la circulation des TGV.
2013-07; Est; NANCY; PARIS EST; 288; 287; 1; 11; 96.2; 
2013-07; Est; PARIS EST; NANCY; 298; 298; 0; 16; 94.6; Le 29 juillet, le TGV 2571 heurte un Lorry (engin permettant la maintenance des installations ferroviaires) et le TGV est limité à Nancy avec un retard de 127 minutes.
2013-07; Est; STRASBOURG; NANTES; 50; 50; 0; 3; 94.0; 
2013-07; Sud-Est; NICE VILLE; PARIS LYON; 269; 269; 0; 75; 72.1; Des orages et fortes chaleurs, des vols de câbles et accidents de personne perturbent la circulation des TGV.
2013-07; Sud-Est; PARIS LYON; NICE VILLE; 251; 251; 0; 66; 73.7; Des orages et fortes chaleurs, des vols de câbles et accidents de personne perturbent la circulation des TGV.
2013-07; Sud-Est; ANNECY; PARIS LYON; 140; 140; 0; 11; 92.1; 
2013-07; Est; PARIS EST; REIMS; 217; 217; 0; 11; 94.9; 
2013-07; Atlantique; RENNES; PARIS MONTPARNASSE; 563; 563; 0; 83; 85.3; 
2013-07; Atlantique; ST MALO; PARIS MONTPARNASSE; 96; 96; 0; 7; 92.7; 
2013-07; Atlantique; TOULOUSE MATABIAU; PARIS MONTPARNASSE; 87; 87; 0; 22; 74.7; Fortes chaleurs les 7 et 8. Fortes intempéries les 26 et 27. Fuite de gazole d'un train de fret. Pannes de train. Heurt d'animal.
2013-07; Sud-Est; VALENCE ALIXAN TGV; PARIS LYON; 258; 258; 0; 53; 79.5; Des orages et fortes chaleurs, des vols de câbles et accidents de personne perturbent la circulation des TGV.
2013-07; Sud-Est; PARIS LYON; AVIGNON TGV; 423; 423; 0; 48; 88.7; 
2013-08; Sud-Est; PARIS LYON; CHAMBERY CHALLES LES EAUX; 219; 219; 0; 26; 88.1; 
2013-08; Nord; DOUAI; PARIS NORD; 188; 188; 0; 16; 91.5; 
2013-08; Atlantique; LAVAL; PARIS MONTPARNASSE; 230; 230; 0; 20; 91.3; 
2013-08; Nord; MARSEILLE ST CHARLES; LILLE; 126; 126; 0; 45; 64.3; Défauts d'alimentation électrique à Lille, Toulon, Lyon et Nice, accident de personne à Lille, heurts de Chevreuil du côté de Compiègne et d'Auxerre, vol de câbles à Avignon et Nimes, voiture sur les voies à Marseille.
2013-08; Sud-Est; LYON PART DIEU; MONTPELLIER; 405; 404; 1; 89; 78.0; Deux incidents caténaires, des vols de câbles, des heurts d'animaux sauvages ont fortement perturbé la circulation des TGV en août.
2013-08; Sud-Est; PARIS LYON; LYON PART DIEU; 532; 532; 0; 20; 96.2; 
2013-08; Est; METZ; PARIS EST; 297; 297; 0; 10; 96.6; 
2013-08; Est; PARIS EST; METZ; 307; 307; 0; 7; 97.7; 
2013-08; Atlantique; ANGOULEME; PARIS MONTPARNASSE; 356; 355; 1; 34; 90.4; 
2013-08; Atlantique; NANTES; PARIS MONTPARNASSE; 548; 548; 0; 29; 94.7; 
2013-08; Sud-Est; NICE VILLE; PARIS LYON; 268; 268; 0; 75; 72.0; Deux incidents caténaires, des vols de câbles, des heurts d'animaux sauvages et la chute d'une voiture sur les voies près de Marseille ont fortement perturbé la circulation des TGV en août.
2013-08; Sud-Est; PARIS LYON; NICE VILLE; 255; 255; 0; 54; 78.8; Deux incidents caténaires, des vols de câbles, des heurts d'animaux sauvages et la chute d'une voiture sur les voies près de Marseille ont fortement perturbé la circulation des TGV en août.
2013-08; Sud-Est; NIMES; PARIS LYON; 367; 367; 0; 62; 83.1; 
2013-08; Sud-Est; PARIS LYON; PERPIGNAN; 164; 164; 0; 17; 89.6; 
2013-08; Est; PARIS EST; REIMS; 214; 214; 0; 5; 97.7; 
2013-08; Atlantique; PARIS MONTPARNASSE; RENNES; 561; 560; 1; 58; 89.6; 
2013-08; Sud-Est; SAINT ETIENNE CHATEAUCREUX; PARIS LYON; 120; 120; 0; 10; 91.7; 
2013-08; Atlantique; PARIS MONTPARNASSE; ST MALO; 62; 62; 0; 7; 88.7; 
2013-08; Atlantique; ST PIERRE DES CORPS; PARIS MONTPARNASSE; 425; 424; 1; 63; 85.1; 
2013-08; Est; STRASBOURG; PARIS EST; 398; 398; 0; 38; 90.5; 
2013-08; Est; PARIS EST; STRASBOURG; 413; 413; 0; 27; 93.5; 
2013-08; Sud-Est; AVIGNON TGV; PARIS LYON; 436; 436; 0; 92; 78.9; Deux incidents caténaires, des vols de câbles, des heurts d'animaux sauvages et la chute d'une voiture sur les voies près de Marseille ont fortement perturbé la circulation des TGV en août.
2013-08; Sud-Est; PARIS LYON; AVIGNON TGV; 426; 426; 0; 48; 88.7; 
2013-08; Sud-Est; BESANCON FRANCHE COMTE TGV; PARIS LYON; 226; 226; 0; 9; 96.0; 
2013-09; Nord; PARIS NORD; DUNKERQUE; 112; 112; 0; 8; 92.9; 
2013-09; Atlantique; LAVAL; PARIS MONTPARNASSE; 202; 202; 0; 21; 89.6; 
2013-09; Nord; MARSEILLE ST CHARLES; LILLE; 108; 108; 0; 29; 73.1; Trois accidents de personnes et dérangements d'installations.
2013-09; Nord; LILLE; PARIS NORD; 530; 530; 0; 72; 86.4; 
2013-09; Sud-Est; LYON PART DIEU; MARSEILLE ST CHARLES; 551; 551; 0; 129; 76.6; Quatre accidents de personne, un vol de câbles et des incidents techniques suite à des orages ont fortement perturbé la circulation des TGV.
2013-09; Sud-Est; MARSEILLE ST CHARLES; LYON PART DIEU; 511; 511; 0; 100; 80.4; 
2013-09; Sud-Est; LYON PART DIEU; PARIS LYON; 549; 548; 1; 28; 94.9; 
2013-09; Sud-Est; MACON LOCHE; PARIS LYON; 168; 167; 1; 19; 88.6; 
2013-09; Sud-Est; MARSEILLE ST CHARLES; PARIS LYON; 412; 410; 2; 27; 93.4; 
2013-09; Sud-Est; PARIS LYON; MARSEILLE ST CHARLES; 420; 419; 1; 25; 94.0; 
2013-09; Atlantique; ANGOULEME; PARIS MONTPARNASSE; 302; 302; 0; 44; 85.4; 
2013-09; Atlantique; PARIS MONTPARNASSE; NANTES; 499; 499; 0; 34; 93.2; 
2013-09; Atlantique; PARIS MONTPARNASSE; ST MALO; 54; 54; 0; 5; 90.7; 
2013-09; Est; STRASBOURG; PARIS EST; 415; 415; 0; 55; 86.7; 
2013-09; Atlantique; TOURS; PARIS MONTPARNASSE; 154; 154; 0; 21; 86.4; 
2013-09; Atlantique; PARIS MONTPARNASSE; TOURS; 125; 125; 0; 14; 88.8; 
2013-09; Atlantique; VANNES; PARIS MONTPARNASSE; 142; 142; 0; 10; 93.0; 
2013-09; Atlantique; PARIS MONTPARNASSE; VANNES; 152; 152; 0; 8; 94.7; 
2013-09; Sud-Est; PARIS LYON; AVIGNON TGV; 337; 337; 0; 34; 89.9; 
2013-09; Sud-Est; BESANCON FRANCHE COMTE TGV; PARIS LYON; 194; 194; 0; 18; 90.7; 
2013-10; Sud-Est; PARIS LYON; AIX EN PROVENCE TGV; 409; 409; 0; 76; 81.4; 
2013-10; Sud-Est; GRENOBLE; PARIS LYON; 245; 245; 0; 20; 91.8; 
2013-10; Sud-Est; PARIS LYON; GRENOBLE; 259; 259; 0; 24; 90.7; 
2013-10; Sud-Est; MARSEILLE ST CHARLES; LYON PART DIEU; 510; 509; 1; 94; 81.5; 
2013-10; Atlantique; LYON PART DIEU; RENNES; 40; 40; 0; 6; 85.0; Événements occasionnant des retards sur des TGV, souvent dans les deux sens de circulations : heurt d’un chevreuil, incident caténaire, détournement suite à la collision avec un camion, accident de personne, défaut d’alimentation caténaire.
2013-10; Sud-Est; PARIS LYON; MARSEILLE ST CHARLES; 484; 483; 1; 30; 93.8; 
2013-10; Est; PARIS EST; METZ; 315; 315; 0; 16; 94.9; 
2013-10; Atlantique; ANGOULEME; PARIS MONTPARNASSE; 359; 357; 2; 69; 80.7; Événements occasionnant des retards sur des TGV, souvent dans les deux sens de circulations : présence de ballast sur les rails, arbre tombé aux abords de la voie avec endommagement du pantographe d'un train (bras mécanique permettant l'alimentation électrique du train), incident caténaire à l'arrivée sur Paris, accident de personne, heurts d'animaux, problème d'alimentation de la caténaire.
2013-10; Atlantique; PARIS MONTPARNASSE; ANGOULEME; 333; 332; 1; 54; 83.7; Événements occasionnant des retards sur des TGV, souvent dans les deux sens de circulations : présence de ballast sur les rails, arbre tombé aux abords de la voie avec endommagement du pantographe d'un train (bras mécanique permettant l'alimentation électrique du train), incident caténaire à l'arrivée sur Paris, accident de personne, heurts d'animaux, problème d'alimentation de la caténaire.
2013-10; Est; NANCY; PARIS EST; 299; 298; 1; 21; 93.0; 
2013-10; Atlantique; PARIS MONTPARNASSE; NANTES; 555; 553; 2; 39; 92.9; 
2013-10; Est; NANTES; STRASBOURG; 60; 59; 1; 9; 84.7; Cette destination a été fortement péjoré par les intempéries mais aussi par quelques pannes matérielles dont le traitement à necessite de réaliser des demandes de secours.
2013-10; Sud-Est; ANNECY; PARIS LYON; 142; 141; 1; 8; 94.3; 
2013-10; Atlantique; RENNES; PARIS MONTPARNASSE; 579; 578; 1; 100; 82.7; Événements occasionnant des retards sur des TGV, souvent dans les deux sens de circulations : présence de ballast sur les rails, arbre tombé aux abords de la voie avec endommagement du pantographe d'un train (bras mécanique permettant l'alimentation électrique du train), incident caténaire à l'arrivée sur Paris, accident de personne, heurts d'animaux, problème d'alimentation de la caténaire.
2013-10; Sud-Est; SAINT ETIENNE CHATEAUCREUX; PARIS LYON; 116; 116; 0; 13; 88.8; Travaux de rénovation de l'infrastructure.
2013-10; Nord; ARRAS; PARIS NORD; 315; 315; 0; 30; 90.5; 
2013-10; Atlantique; PARIS MONTPARNASSE; BORDEAUX ST JEAN; 653; 652; 1; 89; 86.3; Événements occasionnant des retards sur des TGV, souvent dans les deux sens de circulations : présence de ballast sur les rails, arbre tombé aux abords de la voie avec endommagement du pantographe d'un train (bras mécanique permettant l'alimentation électrique du train), incident caténaire à l'arrivée sur Paris, accident de personne, heurts d'animaux, problème d'alimentation de la caténaire.
2011-10; Sud-Est; AIX EN PROVENCE TGV; PARIS LYON; 444; 444; 0; 71; 84.0; 
2011-10; Nord; DOUAI; PARIS NORD; 130; 130; 0; 14; 89.2; 
2011-10; Nord; LILLE; PARIS NORD; 633; 630; 3; 55; 91.3; 
2011-10; Nord; PARIS NORD; LILLE; 617; 616; 1; 65; 89.4; 
2011-10; Sud-Est; LYON PART DIEU; MONTPELLIER; 369; 365; 4; 84; 77.0; 
2011-10; Sud-Est; PARIS LYON; LYON PART DIEU; 647; 646; 1; 40; 93.8; 
2011-10; Est; PARIS EST; METZ; 310; 310; 0; 17; 94.5; 
2011-10; Est; STRASBOURG; NANTES; 52; 49; 3; 10; 79.6; 
2011-10; Sud-Est; PARIS LYON; PERPIGNAN; 127; 127; 0; 33; 74.0; 
2011-10; Atlantique; QUIMPER; PARIS MONTPARNASSE; 151; 151; 0; 11; 92.7; 
2011-10; Atlantique; RENNES; PARIS MONTPARNASSE; 546; 546; 0; 58; 89.4; 
2011-10; Atlantique; ST MALO; PARIS MONTPARNASSE; 73; 73; 0; 4; 94.5; 
2011-10; Atlantique; TOULOUSE MATABIAU; PARIS MONTPARNASSE; 118; 117; 1; 31; 73.5; 
2011-10; Sud-Est; VALENCE ALIXAN TGV; PARIS LYON; 238; 237; 1; 23; 90.3; 
2011-10; Sud-Est; PARIS LYON; AVIGNON TGV; 395; 394; 1; 38; 90.4; 
2011-10; Sud-Est; BESANCON FRANCHE COMTE TGV; PARIS LYON; 129; 129; 0; 9; 93.0; 
2011-10; Atlantique; BORDEAUX ST JEAN; PARIS MONTPARNASSE; 643; 639; 4; 103; 83.9; 
2011-11; Sud-Est; AIX EN PROVENCE TGV; PARIS LYON; 434; 432; 2; 48; 88.9; 
2011-11; Atlantique; PARIS MONTPARNASSE; BREST; 185; 181; 4; 12; 93.4; 
2011-11; Nord; PARIS NORD; DOUAI; 155; 155; 0; 19; 87.7; 
2011-11; Atlantique; ANGERS SAINT LAUD; PARIS MONTPARNASSE; 467; 460; 7; 26; 94.3; 
2011-11; Atlantique; PARIS MONTPARNASSE; ANGERS SAINT LAUD; 442; 437; 5; 25; 94.3; 
2011-11; Nord; LILLE; LYON PART DIEU; 338; 338; 0; 33; 90.2; 
2011-11; Nord; LYON PART DIEU; LILLE; 308; 307; 1; 52; 83.1; 
2011-11; Nord; LILLE; MARSEILLE ST CHARLES; 164; 164; 0; 33; 79.9; 
2011-11; Nord; LILLE; PARIS NORD; 612; 612; 0; 34; 94.4; 
2011-11; Sud-Est; LYON PART DIEU; PARIS LYON; 606; 606; 0; 41; 93.2; 
2011-11; Atlantique; RENNES; LYON PART DIEU; 84; 84; 0; 18; 78.6; 
2011-11; Sud-Est; PARIS LYON; MARSEILLE ST CHARLES; 507; 507; 0; 29; 94.3; 
2011-11; Est; METZ; PARIS EST; 283; 283; 0; 21; 92.6; 
2011-11; Atlantique; NANTES; PARIS MONTPARNASSE; 592; 583; 9; 16; 97.3; 
2011-11; Atlantique; PARIS MONTPARNASSE; NANTES; 591; 584; 7; 28; 95.2; 
2011-11; Sud-Est; PERPIGNAN; PARIS LYON; 130; 130; 0; 25; 80.8; 
2011-11; Atlantique; PARIS MONTPARNASSE; QUIMPER; 169; 166; 3; 10; 94.0; 
2011-11; Atlantique; PARIS MONTPARNASSE; ST MALO; 40; 40; 0; 1; 97.5; 
2011-11; Atlantique; PARIS MONTPARNASSE; ST PIERRE DES CORPS; 436; 431; 5; 42; 90.3; 
2011-11; Est; STRASBOURG; PARIS EST; 497; 497; 0; 50; 89.9; 
2011-11; Atlantique; TOULOUSE MATABIAU; PARIS MONTPARNASSE; 145; 144; 1; 10; 93.1; 
2011-11; Atlantique; PARIS MONTPARNASSE; TOULOUSE MATABIAU; 137; 137; 0; 17; 87.6; 
2011-11; Atlantique; TOURS; PARIS MONTPARNASSE; 158; 157; 1; 11; 93.0; 
2011-11; Sud-Est; AVIGNON TGV; PARIS LYON; 377; 377; 0; 40; 89.4; 
2011-11; Sud-Est; BELLEGARDE (AIN); PARIS LYON; 255; 255; 0; 31; 87.8; 
2011-11; Sud-Est; PARIS LYON; BELLEGARDE (AIN); 252; 252; 0; 36; 85.7; 
2011-11; Atlantique; BORDEAUX ST JEAN; PARIS MONTPARNASSE; 646; 640; 6; 35; 94.5; 
2011-12; Atlantique; BREST; PARIS MONTPARNASSE; 174; 174; 0; 12; 93.1; 
2011-12; Sud-Est; PARIS LYON; DIJON VILLE; 451; 451; 0; 16; 96.5; 
2011-12; Atlantique; PARIS MONTPARNASSE; LAVAL; 245; 245; 0; 26; 89.4; 
2011-12; Sud-Est; LE CREUSOT MONTCEAU MONTCHANIN; PARIS LYON; 219; 219; 0; 31; 85.8; 
2011-12; Sud-Est; PARIS LYON; LE CREUSOT MONTCEAU MONTCHANIN; 207; 207; 0; 34; 83.6; 
2011-12; Nord; LILLE; PARIS NORD; 627; 627; 0; 67; 89.3; 
2011-12; Sud-Est; PARIS LYON; LYON PART DIEU; 615; 615; 0; 21; 96.6; 
2011-12; Est; MULHOUSE VILLE; PARIS LYON; 279; 279; 0; 38; 86.4; 
2011-12; Est; PARIS EST; NANCY; 296; 296; 0; 12; 95.9; 
2011-12; Atlantique; NANTES; PARIS MONTPARNASSE; 600; 596; 4; 48; 91.9; 
2011-12; Sud-Est; PERPIGNAN; PARIS LYON; 106; 106; 0; 7; 93.4; 
2011-12; Sud-Est; ANNECY; PARIS LYON; 206; 206; 0; 11; 94.7; 
2011-12; Est; REIMS; PARIS EST; 239; 239; 0; 16; 93.3; 
2011-12; Atlantique; RENNES; PARIS MONTPARNASSE; 561; 560; 1; 62; 88.9; 
2011-12; Sud-Est; SAINT ETIENNE CHATEAUCREUX; PARIS LYON; 118; 118; 0; 20; 83.1; 
2011-12; Atlantique; TOULOUSE MATABIAU; PARIS MONTPARNASSE; 130; 130; 0; 13; 90.0; 
2011-12; Atlantique; PARIS MONTPARNASSE; TOURS; 193; 193; 0; 36; 81.3; 
2011-12; Sud-Est; PARIS LYON; VALENCE ALIXAN TGV; 247; 247; 0; 18; 92.7; 
2011-12; Nord; ARRAS; PARIS NORD; 340; 339; 1; 37; 89.1; 
2012-01; Nord; PARIS NORD; DUNKERQUE; 153; 153; 0; 8; 94.8; 
2012-01; Atlantique; PARIS MONTPARNASSE; LA ROCHELLE VILLE; 226; 226; 0; 21; 90.7; 
2012-01; Nord; LYON PART DIEU; LILLE; 279; 279; 0; 51; 81.7; 
2012-01; Atlantique; RENNES; LYON PART DIEU; 84; 84; 0; 8; 90.5; 
2012-01; Sud-Est; PARIS LYON; MONTPELLIER; 357; 357; 0; 34; 90.5; 
2012-01; Est; PARIS EST; NANCY; 296; 296; 0; 12; 95.9; 
2012-01; Sud-Est; PARIS LYON; NICE VILLE; 162; 162; 0; 18; 88.9; 
2012-01; Sud-Est; PARIS LYON; PERPIGNAN; 156; 156; 0; 18; 88.5; 
2012-01; Atlantique; PARIS MONTPARNASSE; QUIMPER; 162; 162; 0; 11; 93.2; 
2012-01; Est; REIMS; PARIS EST; 238; 238; 0; 32; 86.6; 
2012-01; Atlantique; PARIS MONTPARNASSE; ST MALO; 57; 57; 0; 5; 91.2; 
2012-01; Atlantique; TOULOUSE MATABIAU; PARIS MONTPARNASSE; 117; 117; 0; 15; 87.2; 
2012-01; Sud-Est; PARIS LYON; VALENCE ALIXAN TGV; 269; 269; 0; 25; 90.7; 
2012-01; Nord; ARRAS; PARIS NORD; 349; 349; 0; 38; 89.1; 
2012-01; Atlantique; PARIS MONTPARNASSE; VANNES; 185; 185; 0; 16; 91.4; 
2012-02; Sud-Est; CHAMBERY CHALLES LES EAUX; PARIS LYON; 277; 276; 1; 37; 86.6; 
2012-02; Nord; DUNKERQUE; PARIS NORD; 102; 102; 0; 11; 89.2; 
2012-02; Sud-Est; PARIS LYON; LE CREUSOT MONTCEAU MONTCHANIN; 197; 197; 0; 57; 71.1; 
2012-02; Nord; LILLE; LYON PART DIEU; 287; 284; 3; 52; 81.7; 
2012-02; Nord; LILLE; MARSEILLE ST CHARLES; 149; 148; 1; 48; 67.6; 
2012-02; Sud-Est; PARIS LYON; LYON PART DIEU; 602; 600; 2; 45; 92.5; 
2012-02; Sud-Est; PARIS LYON; MACON LOCHE; 181; 181; 0; 20; 89.0; 
2012-02; Atlantique; PARIS MONTPARNASSE; ANGOULEME; 306; 306; 0; 69; 77.5; 
2012-02; Est; NANCY; PARIS EST; 279; 271; 8; 26; 90.4; 
2012-02; Sud-Est; NICE VILLE; PARIS LYON; 153; 152; 1; 32; 78.9; 
2012-02; Atlantique; PARIS MONTPARNASSE; QUIMPER; 150; 150; 0; 17; 88.7; 
2012-02; Atlantique; PARIS MONTPARNASSE; ST PIERRE DES CORPS; 436; 436; 0; 89; 79.6; 
2012-02; Est; PARIS EST; STRASBOURG; 452; 452; 0; 59; 86.9; 
2012-02; Atlantique; TOURS; PARIS MONTPARNASSE; 197; 197; 0; 54; 72.6; 
2012-02; Atlantique; PARIS MONTPARNASSE; TOURS; 194; 194; 0; 51; 73.7; 
2012-02; Sud-Est; PARIS LYON; VALENCE ALIXAN TGV; 252; 252; 0; 31; 87.7; 
2012-02; Sud-Est; BESANCON FRANCHE COMTE TGV; PARIS LYON; 207; 206; 1; 30; 85.4; 
2012-02; Atlantique; BORDEAUX ST JEAN; PARIS MONTPARNASSE; 596; 596; 0; 140; 76.5; 
2012-03; Sud-Est; PARIS LYON; AIX EN PROVENCE TGV; 406; 406; 0; 27; 93.3; 
2012-03; Nord; DUNKERQUE; PARIS NORD; 115; 115; 0; 8; 93.0; Circulation fortement dégradée le 5 mars avec des conséquences sur le 6 mars suite aux difficiles conditions climatiques entraînant notamment la chute d'un câble ERDF sur la ligne à grande vitesse. La ponctualité a également été impactée par les travaux importants de rénovation des voies sur la ligne à grande vitesse entre Lille et TGV - Haute Picardie.
2012-03; Sud-Est; PARIS LYON; GRENOBLE; 260; 260; 0; 19; 92.7; 
2012-03; Atlantique; LAVAL; PARIS MONTPARNASSE; 246; 246; 0; 17; 93.1; 
2013-03; Atlantique; TOURS; PARIS MONTPARNASSE; 180; 180; 0; 28; 84.4; 
2013-03; Nord; PARIS NORD; ARRAS; 319; 298; 21; 107; 64.1; Importantes chutes de neige avec formation de congères survenues à partir du 11 mars dans la soirée jusqu'au 15 mars. Elles ont imposé des limitations de vitesse pour éviter les projections de glace, allant même jusqu’à immobiliser presque complètement le trafic le mardi 12 mars. Les conditions climatiques difficiles de ces derniers mois ont par ailleurs perturbé les différents programmes travaux sur les voies prolongeant ainsi certaines limitations de vitesse au-delà des délais prévus. notamment sur Arras.
2013-03; Sud-Est; PARIS LYON; AVIGNON TGV; 425; 422; 3; 37; 91.2; 
2013-04; Nord; DUNKERQUE; PARIS NORD; 85; 85; 0; 9; 89.4; Nombreux accidents de personnes et incidents d'installations techniques et électriques) répartis sur l'ensemble du parcours.
2013-04; Sud-Est; LE CREUSOT MONTCEAU MONTCHANIN; PARIS LYON; 215; 214; 1; 27; 87.4; 
2013-04; Nord; LILLE; MARSEILLE ST CHARLES; 148; 148; 0; 33; 77.7; Nombreux accidents de personnes et incidents d'installations techniques et électriques) répartis sur l'ensemble du parcours.
2013-04; Sud-Est; LYON PART DIEU; MONTPELLIER; 392; 392; 0; 58; 85.2; Malgré un temps de parcours relativement court entre ces deux gares, l'essentiel des trains assurant cette desserte sont origine de l'ouest ou du Nord de la France et ont donc parcouru une très longue distance et souvent déjà accumulé du retard avant même d'assurer cette desserte située en fin de parcours du train.
2013-04; Sud-Est; MONTPELLIER; LYON PART DIEU; 339; 339; 0; 51; 85.0; 
2013-04; Atlantique; ANGOULEME; PARIS MONTPARNASSE; 346; 346; 0; 20; 94.2; Rame d'un TGV en panne au départ, fin des travaux de nuit à Bordeaux retardant les TGV du matin, accident de personne à Poitiers, acte de malveillance à Poitiers (gilet sur le fil d'alimentation électrique), panne d'un TER en ligne au nord de Bordeaux.
2013-04; Sud-Est; MULHOUSE VILLE; PARIS LYON; 308; 308; 0; 13; 95.8; 
2013-04; Atlantique; NANTES; PARIS MONTPARNASSE; 560; 559; 1; 24; 95.7; 
2013-04; Sud-Est; NIMES; PARIS LYON; 356; 356; 0; 48; 86.5; 
2013-04; Atlantique; RENNES; PARIS MONTPARNASSE; 566; 566; 0; 40; 92.9; 
2013-04; Atlantique; ST MALO; PARIS MONTPARNASSE; 99; 99; 0; 3; 97.0; 
2013-04; Atlantique; ST PIERRE DES CORPS; PARIS MONTPARNASSE; 418; 418; 0; 51; 87.8; 
2013-04; Est; STRASBOURG; PARIS EST; 466; 466; 0; 25; 94.6; 
2013-04; Sud-Est; PARIS LYON; TOULON; 221; 221; 0; 25; 88.7; Plusieurs incidents techniques et des actes de malveillance ont eu lieu les 17, 19, 21, 25 et 26 Avril sur l'Axe Paris-Lyon entrainant de nombreux retards.
2013-04; Atlantique; PARIS MONTPARNASSE; TOULOUSE MATABIAU; 145; 145; 0; 15; 89.7; Accidents de personne les 15, 17 et 18. Présomption d'avarie sur le fil d'alimentation électrique (caténaire) le 14 près de Bordeaux.
2013-04; Atlantique; TOURS; PARIS MONTPARNASSE; 172; 172; 0; 15; 91.3; 
2013-04; Sud-Est; VALENCE ALIXAN TGV; PARIS LYON; 251; 251; 0; 31; 87.6; 
2013-04; Sud-Est; AVIGNON TGV; PARIS LYON; 383; 382; 1; 52; 86.4; 
2013-04; Atlantique; PARIS MONTPARNASSE; BORDEAUX ST JEAN; 626; 626; 0; 41; 93.5; Problème matériel d'un train en ligne générant du retard pour les trains à sa suite le 10 avril. Présomption d'avarie sur le fil d'alimentation électrique (caténaire) près de Bordeaux le 14 avril. Accident de personne à Poitiers le 15 avril. Intervention pompier sur le TGV 8503 suite à malaise grave d'un voyageur à bord le 24 avril.
2013-05; Sud-Est; CHAMBERY CHALLES LES EAUX; PARIS LYON; 205; 205; 0; 27; 86.8; 
2013-05; Nord; DOUAI; PARIS NORD; 209; 209; 0; 21; 90.0; 
2013-05; Nord; PARIS NORD; DOUAI; 212; 212; 0; 21; 90.1; 
2013-05; Atlantique; LA ROCHELLE VILLE; PARIS MONTPARNASSE; 227; 226; 1; 9; 96.0; 
2013-05; Atlantique; PARIS MONTPARNASSE; LAVAL; 244; 243; 1; 10; 95.9; 
2013-05; Atlantique; PARIS MONTPARNASSE; LE MANS; 455; 454; 1; 31; 93.2; 
2013-05; Nord; LILLE; LYON PART DIEU; 309; 309; 0; 33; 89.3; Plusieurs actes de malveillance (vols de câbles, pose de plaques de béton sur les voies de la LGV Nord).
2013-05; Nord; LYON PART DIEU; LILLE; 251; 251; 0; 52; 79.3; Plusieurs actes de malveillance (vols de câbles, pose de plaques de béton sur les voies de la LGV Nord).
2013-05; Sud-Est; LYON PART DIEU; PARIS LYON; 634; 634; 0; 38; 94.0; 
2013-05; Sud-Est; PARIS LYON; LYON PART DIEU; 630; 629; 1; 36; 94.3; 
2013-05; Atlantique; RENNES; LYON PART DIEU; 82; 82; 0; 7; 91.5; 
2013-05; Sud-Est; MARSEILLE ST CHARLES; PARIS LYON; 471; 471; 0; 34; 92.8; 
2013-05; Est; METZ; PARIS EST; 301; 301; 0; 24; 92.0; 
2013-05; Atlantique; ANGOULEME; PARIS MONTPARNASSE; 355; 354; 1; 32; 91.0; 
2013-05; Atlantique; PARIS MONTPARNASSE; ANGOULEME; 332; 332; 0; 16; 95.2; 
2013-05; Sud-Est; PARIS LYON; NICE VILLE; 199; 199; 0; 36; 81.9; Un incident caténaire au Creusot le 07 Mai et la panne d'un train de marchandise ont provoqué des retards très importants notament sur plusieurs trains de la liaison Paris-Nice.
2013-05; Sud-Est; PARIS LYON; PERPIGNAN; 158; 158; 0; 10; 93.7; 
2013-05; Atlantique; QUIMPER; PARIS MONTPARNASSE; 147; 147; 0; 8; 94.6; 
2013-05; Sud-Est; PARIS LYON; ANNECY; 145; 145; 0; 6; 95.9; 
2013-05; Atlantique; RENNES; PARIS MONTPARNASSE; 578; 577; 1; 39; 93.2; 
2013-05; Sud-Est; PARIS LYON; SAINT ETIENNE CHATEAUCREUX; 119; 119; 0; 11; 90.8; 
2013-05; Atlantique; ST MALO; PARIS MONTPARNASSE; 102; 102; 0; 5; 95.1; 
2013-05; Atlantique; PARIS MONTPARNASSE; ST MALO; 57; 57; 0; 3; 94.7; 
2013-05; Atlantique; ST PIERRE DES CORPS; PARIS MONTPARNASSE; 437; 436; 1; 54; 87.6; 
2013-05; Atlantique; PARIS MONTPARNASSE; ST PIERRE DES CORPS; 457; 457; 0; 36; 92.1; 
2013-05; Est; STRASBOURG; PARIS EST; 481; 481; 0; 33; 93.1; 
2013-05; Atlantique; PARIS MONTPARNASSE; TOULOUSE MATABIAU; 126; 126; 0; 17; 86.5; 
2013-05; Atlantique; PARIS MONTPARNASSE; VANNES; 187; 187; 0; 11; 94.1; 
2013-05; Sud-Est; PARIS LYON; BELLEGARDE (AIN); 275; 275; 0; 23; 91.6; 
2013-06; Sud-Est; PARIS LYON; AIX EN PROVENCE TGV; 392; 392; 0; 57; 85.5; 
2013-06; Sud-Est; PARIS LYON; CHAMBERY CHALLES LES EAUX; 209; 205; 4; 32; 84.4; 
2013-06; Nord; PARIS NORD; DUNKERQUE; 124; 121; 3; 6; 95.0; 
2013-06; Atlantique; LA ROCHELLE VILLE; PARIS MONTPARNASSE; 215; 211; 4; 8; 96.2; 
2013-06; Nord; LILLE; PARIS NORD; 598; 586; 12; 45; 92.3; 
2013-06; Sud-Est; LYON PART DIEU; MONTPELLIER; 394; 381; 13; 63; 83.5; 
2013-06; Est; PARIS EST; METZ; 303; 299; 4; 11; 96.3; 
2013-06; Sud-Est; PARIS LYON; NIMES; 338; 331; 7; 31; 90.6; 
2013-06; Est; REIMS; PARIS EST; 204; 202; 2; 12; 94.1; 
2013-06; Est; PARIS EST; REIMS; 209; 207; 2; 10; 95.2; 
2013-06; Atlantique; PARIS MONTPARNASSE; VANNES; 174; 172; 2; 15; 91.3; 
2013-07; Sud-Est; AIX EN PROVENCE TGV; PARIS LYON; 420; 420; 0; 95; 77.4; Des orages et fortes chaleurs, des vols de câbles et accidents de personne perturbent la circulation des TGV.
2013-07; Atlantique; BREST; PARIS MONTPARNASSE; 181; 181; 0; 21; 88.4; Panne d'un TER en ligne.
2013-07; Nord; PARIS NORD; DUNKERQUE; 115; 115; 0; 16; 86.1; 
2013-07; Sud-Est; GRENOBLE; PARIS LYON; 194; 194; 0; 25; 87.1; 
2013-07; Atlantique; LAVAL; PARIS MONTPARNASSE; 229; 229; 0; 29; 87.3; 
2013-07; Atlantique; LE MANS; PARIS MONTPARNASSE; 464; 464; 0; 115; 75.2; 
2013-07; Atlantique; PARIS MONTPARNASSE; ANGERS SAINT LAUD; 393; 393; 0; 62; 84.2; 
2013-07; Sud-Est; LYON PART DIEU; MONTPELLIER; 407; 407; 0; 115; 71.7; Des orages et fortes chaleurs, des vols de câbles et accidents de personne perturbent la circulation des TGV.
2013-07; Sud-Est; PARIS LYON; MONTPELLIER; 372; 372; 0; 57; 84.7; 
2013-07; Atlantique; PARIS MONTPARNASSE; QUIMPER; 186; 186; 0; 29; 84.4; 
2013-07; Sud-Est; PARIS LYON; ANNECY; 178; 178; 0; 13; 92.7; 
2013-07; Est; PARIS EST; STRASBOURG; 435; 435; 0; 34; 92.2; Accident de personne à Vendenheim (près de Strasbourg) le 18 juillet, incident caténaire en gare de Paris-Est et blocage de plusieurs voies le 30 juillet. Cet incident touche 20 TGV.
2013-07; Atlantique; PARIS MONTPARNASSE; TOULOUSE MATABIAU; 150; 150; 0; 47; 68.7; Fortes chaleurs les 7 et 8. Fortes intempéries les 26 et 27. Fuite de gazole d'un train de fret. Pannes de train. Heurt d'animal.
2013-07; Atlantique; PARIS MONTPARNASSE; TOURS; 147; 147; 0; 26; 82.3; 
2013-07; Atlantique; PARIS MONTPARNASSE; BORDEAUX ST JEAN; 653; 652; 1; 99; 84.8; Fortes chaleurs les 7 et 8 juillet. Fortes intempéries les 26 et 27 juillet. Fuite de gazole d'un train de fret. Pannes de train. Heurt d'animal.
2013-08; Sud-Est; AIX EN PROVENCE TGV; PARIS LYON; 417; 417; 0; 80; 80.8; Deux incidents caténaires, des vols de câbles, des heurts d'animaux sauvages et la chute d'une voiture sur les voies près de Marseille ont fortement perturbé la circulation des TGV en août.
2013-08; Atlantique; BREST; PARIS MONTPARNASSE; 178; 178; 0; 6; 96.6; 
2013-08; Nord; DUNKERQUE; PARIS NORD; 120; 120; 0; 3; 97.5; 
2013-08; Atlantique; PARIS MONTPARNASSE; LAVAL; 239; 238; 1; 22; 90.8; 
2013-08; Sud-Est; PARIS LYON; LE CREUSOT MONTCEAU MONTCHANIN; 206; 206; 0; 20; 90.3; 
2013-08; Atlantique; PARIS MONTPARNASSE; ANGERS SAINT LAUD; 395; 395; 0; 31; 92.2; 
2013-08; Nord; LILLE; MARSEILLE ST CHARLES; 153; 153; 0; 30; 80.4; 
2013-08; Sud-Est; LYON PART DIEU; PARIS LYON; 540; 540; 0; 20; 96.3; 
2013-08; Atlantique; RENNES; LYON PART DIEU; 79; 79; 0; 13; 83.5; Une rupture caténaire (fil d'alimentation électrique) a eu lieu en banlieue parisienne (chemin d'Anthony) et a entrainé de forts retards sur nos TGV.
2013-08; Sud-Est; PARIS LYON; MACON LOCHE; 184; 184; 0; 11; 94.0; 
2013-08; Sud-Est; MARSEILLE ST CHARLES; PARIS LYON; 460; 460; 0; 53; 88.5; 
2013-08; Atlantique; PARIS MONTPARNASSE; ANGOULEME; 348; 348; 0; 16; 95.4; 
2013-08; Est; NANCY; PARIS EST; 283; 282; 1; 7; 97.5; 
2013-08; Est; PARIS EST; NANCY; 292; 292; 0; 4; 98.6; 
2013-08; Est; NANTES; STRASBOURG; 52; 52; 0; 3; 94.2; 
2013-08; Atlantique; POITIERS; PARIS MONTPARNASSE; 504; 503; 1; 33; 93.4; 
2013-08; Atlantique; PARIS MONTPARNASSE; POITIERS; 507; 507; 0; 20; 96.1; 
2013-08; Atlantique; PARIS MONTPARNASSE; ST PIERRE DES CORPS; 445; 445; 0; 39; 91.2; 
2013-08; Sud-Est; TOULON; PARIS LYON; 321; 321; 0; 65; 79.8; Deux incidents caténaires, des vols de câbles, des heurts d'animaux sauvages et la chute d'une voiture sur les voies près de Marseille ont fortement perturbé la circulation des TGV en août.
2013-08; Sud-Est; VALENCE ALIXAN TGV; PARIS LYON; 264; 264; 0; 45; 83.0; 
2013-08; Sud-Est; PARIS LYON; VALENCE ALIXAN TGV; 276; 276; 0; 29; 89.5; 
2013-08; Atlantique; BORDEAUX ST JEAN; PARIS MONTPARNASSE; 685; 684; 1; 50; 92.7; 
2013-09; Atlantique; BREST; PARIS MONTPARNASSE; 151; 151; 0; 5; 96.7; 
2012-01; Sud-Est; AVIGNON TGV; PARIS LYON; 330; 330; 0; 24; 92.7; 
2012-01; Sud-Est; BELLEGARDE (AIN); PARIS LYON; 264; 264; 0; 32; 87.9; 
2012-01; Sud-Est; PARIS LYON; BELLEGARDE (AIN); 281; 281; 0; 41; 85.4; 
2012-02; Atlantique; BREST; PARIS MONTPARNASSE; 162; 162; 0; 12; 92.6; 
2012-02; Sud-Est; PARIS LYON; CHAMBERY CHALLES LES EAUX; 305; 305; 0; 58; 81.0; 
2012-02; Atlantique; PARIS MONTPARNASSE; LA ROCHELLE VILLE; 197; 197; 0; 50; 74.6; 
2012-02; Nord; LILLE; PARIS NORD; 588; 587; 1; 116; 80.2; 
2012-02; Nord; PARIS NORD; LILLE; 596; 595; 1; 97; 83.7; 
2012-02; Sud-Est; LYON PART DIEU; PARIS LYON; 598; 597; 1; 62; 89.6; 
2012-02; Sud-Est; MARSEILLE ST CHARLES; PARIS LYON; 455; 449; 6; 46; 89.8; 
2012-02; Est; METZ; PARIS EST; 281; 281; 0; 35; 87.5; 
2012-02; Est; NANTES; STRASBOURG; 58; 57; 1; 11; 80.7; 
2012-02; Est; STRASBOURG; NANTES; 58; 57; 1; 8; 86.0; 
2012-02; Sud-Est; PARIS LYON; NIMES; 336; 336; 0; 55; 83.6; 
2012-02; Sud-Est; PARIS LYON; PERPIGNAN; 147; 147; 0; 22; 85.0; 
2012-02; Atlantique; PARIS MONTPARNASSE; RENNES; 534; 533; 1; 74; 86.1; 
2012-02; Atlantique; ST MALO; PARIS MONTPARNASSE; 94; 94; 0; 8; 91.5; 
2012-02; Atlantique; PARIS MONTPARNASSE; TOULOUSE MATABIAU; 138; 138; 0; 35; 74.6; 
2012-02; Atlantique; VANNES; PARIS MONTPARNASSE; 165; 165; 0; 19; 88.5; 
2012-02; Sud-Est; PARIS LYON; AVIGNON TGV; 369; 366; 3; 48; 86.9; 
2012-03; Atlantique; BREST; PARIS MONTPARNASSE; 173; 173; 0; 3; 98.3; Difficultés de circulation dans l'ensemble ferroviaire de Tours-St-Pierre. Plus ponctuellement, incidents survenus sur la LGV Atlantique.
2012-03; Sud-Est; CHAMBERY CHALLES LES EAUX; PARIS LYON; 284; 284; 0; 48; 83.1; 
2012-03; Nord; PARIS NORD; DOUAI; 212; 212; 0; 21; 90.1; Circulation fortement dégradée le 5 mars avec des conséquences sur le 6 mars suite aux difficiles conditions climatiques entraînant notamment la chute d'un câble ERDF sur la ligne à grande vitesse. La ponctualité a également été impactée par les travaux importants de rénovation des voies sur la ligne à grande vitesse entre Lille et TGV - Haute Picardie.
2012-03; Sud-Est; LE CREUSOT MONTCEAU MONTCHANIN; PARIS LYON; 217; 217; 0; 27; 87.6; 
2012-03; Atlantique; PARIS MONTPARNASSE; ANGERS SAINT LAUD; 446; 446; 0; 20; 95.5; 
2012-03; Nord; LILLE; LYON PART DIEU; 308; 305; 3; 27; 91.1; Circulation fortement dégradée le 5 mars avec des conséquences sur le 6 mars suite aux difficiles conditions climatiques entraînant notamment la chute d'un câble ERDF sur la ligne à grande vitesse. La ponctualité a également été impactée par les travaux importants de rénovation des voies sur la ligne à grande vitesse entre Lille et TGV - Haute Picardie.
2012-03; Nord; LILLE; MARSEILLE ST CHARLES; 159; 158; 1; 28; 82.3; Circulation fortement dégradée le 5 mars avec des conséquences sur le 6 mars suite aux difficiles conditions climatiques entraînant notamment la chute d'un câble ERDF sur la ligne à grande vitesse. La ponctualité a également été impactée par les travaux importants de rénovation des voies sur la ligne à grande vitesse entre Lille et TGV - Haute Picardie.
2012-03; Nord; MARSEILLE ST CHARLES; LILLE; 125; 125; 0; 34; 72.8; Circulation fortement dégradée le 5 mars avec des conséquences sur le 6 mars suite aux difficiles conditions climatiques entraînant notamment la chute d'un câble ERDF sur la ligne à grande vitesse. La ponctualité a également été impactée par les travaux importants de rénovation des voies sur la ligne à grande vitesse entre Lille et TGV - Haute Picardie.
2012-03; Nord; LILLE; PARIS NORD; 628; 622; 6; 67; 89.2; Circulation fortement dégradée le 5 mars avec des conséquences sur le 6 mars suite aux difficiles conditions climatiques entraînant notamment la chute d'un câble ERDF sur la ligne à grande vitesse. La ponctualité a également été impactée par les travaux importants de rénovation des voies sur la ligne à grande vitesse entre Lille et TGV - Haute Picardie.
2012-03; Sud-Est; MARSEILLE ST CHARLES; PARIS LYON; 486; 486; 0; 22; 95.5; 
2012-03; Sud-Est; PARIS LYON; MARSEILLE ST CHARLES; 519; 519; 0; 22; 95.8; 
2012-03; Atlantique; PARIS MONTPARNASSE; ANGOULEME; 333; 333; 0; 21; 93.7; Ralentissements pour travaux de renouvellement des voies au nord de Bordeaux avec retard des TGV en provenance de Bordeaux et qui desservent St-Pierre-des-Corps.
2012-03; Sud-Est; PARIS LYON; MONTPELLIER; 351; 351; 0; 24; 93.2; 
2012-03; Est; PARIS EST; NANCY; 294; 294; 0; 15; 94.9; Incendie d'un entrepôt à hauteur de Gagny qui a généré une coupure caténaire les 10 et 11 mars 2012.
2012-03; Atlantique; PARIS MONTPARNASSE; NANTES; 592; 592; 0; 35; 94.1; 
2012-03; Atlantique; RENNES; PARIS MONTPARNASSE; 582; 582; 0; 29; 95.0; 
2012-03; Atlantique; PARIS MONTPARNASSE; ST PIERRE DES CORPS; 474; 474; 0; 36; 92.4; 
2012-03; Atlantique; PARIS MONTPARNASSE; TOULOUSE MATABIAU; 135; 135; 0; 25; 81.5; 
2012-03; Atlantique; VANNES; PARIS MONTPARNASSE; 176; 176; 0; 4; 97.7; 
2012-04; Atlantique; BREST; PARIS MONTPARNASSE; 177; 177; 0; 5; 97.2; 
2012-04; Atlantique; PARIS MONTPARNASSE; BREST; 183; 183; 0; 3; 98.4; 
2012-04; Sud-Est; PARIS LYON; CHAMBERY CHALLES LES EAUX; 229; 229; 0; 28; 87.8; 
2012-04; Nord; DUNKERQUE; PARIS NORD; 110; 110; 0; 4; 96.4; Travaux importants de rénovation des voies entre Arras et Isbergues. entre Douai et Valenciennes et sur la ligne à grande vitesse entre Lille et TGV - Haute Picardie. Les circulations ont également été impactées par une succession de dérangements d'installations ferroviaires.
2012-04; Atlantique; PARIS MONTPARNASSE; LA ROCHELLE VILLE; 210; 210; 0; 13; 93.8; 
2012-04; Atlantique; PARIS MONTPARNASSE; ANGERS SAINT LAUD; 428; 428; 0; 20; 95.3; 
2012-04; Sud-Est; LYON PART DIEU; MONTPELLIER; 363; 363; 0; 74; 79.6; Une baisse de la régularité en avril qui s'explique par une augmentation des incidents d'origine externe (trois accidents de personne, un vol de câbles sur ligne nouvelle) et une limitation de vitesse pour la réalisation de travaux au sud de Lyon.
2012-04; Atlantique; RENNES; LYON PART DIEU; 79; 79; 0; 4; 94.9; 
2012-04; Est; METZ; PARIS EST; 292; 292; 0; 16; 94.5; Une liaison au-dessus de l'objectif malgré les pertes de temps du au jet de pierre et à l'incendie à l'entrée de Paris-Est.
2012-04; Atlantique; PARIS MONTPARNASSE; POITIERS; 448; 448; 0; 28; 93.8; 
2012-04; Sud-Est; SAINT ETIENNE CHATEAUCREUX; PARIS LYON; 113; 113; 0; 10; 91.2; 
2012-04; Atlantique; ST MALO; PARIS MONTPARNASSE; 98; 98; 0; 3; 96.9; 
2012-04; Est; STRASBOURG; PARIS EST; 463; 462; 1; 52; 88.7; Un mois au dessus de l'objectif et en amélioration avec l'arrêt d'une partie des travaux.
2012-04; Nord; ARRAS; PARIS NORD; 334; 334; 0; 72; 78.4; Travaux importants de rénovation des voies entre Arras et Isbergues. entre Douai et Valenciennes et sur la ligne à grande vitesse entre Lille et TGV - Haute Picardie. Les circulations ont également été impactées par une succession de dérangements d'installations ferroviaires.
2012-04; Sud-Est; AVIGNON TGV; PARIS LYON; 366; 366; 0; 65; 82.2; 
2012-04; Sud-Est; PARIS LYON; BELLEGARDE (AIN); 268; 268; 0; 18; 93.3; 
2012-04; Atlantique; PARIS MONTPARNASSE; BORDEAUX ST JEAN; 604; 604; 0; 46; 92.4; 
2012-05; Sud-Est; AIX EN PROVENCE TGV; PARIS LYON; 451; 450; 1; 84; 81.3; 
2012-05; Nord; DOUAI; PARIS NORD; 207; 207; 0; 21; 89.9; 
2012-05; Nord; DUNKERQUE; PARIS NORD; 112; 112; 0; 5; 95.5; 
2012-05; Sud-Est; PARIS LYON; LE CREUSOT MONTCEAU MONTCHANIN; 205; 205; 0; 22; 89.3; 
2012-05; Atlantique; LE MANS; PARIS MONTPARNASSE; 469; 469; 0; 65; 86.1; Les travaux d'amélioration de la voie réalisés entre Paris et Le Mans sont à l'origine d'un part non négligeable des retards des TGV. Toutefois, quelques incidents sont tout de même à noter : problème de matériel nécessitant l'inversion de deux rames en gare du Mans; un problème attenant à la voie à la sortie de la gare de Paris-Montparnasse le 31 ainsi que des dérangements de la signalisation.
2012-05; Nord; LILLE; LYON PART DIEU; 309; 309; 0; 39; 87.4; Les circulations ont été impactées par de nombreux événements : plusieurs accidents de personne ou personnes dans les voies. des heurts d'animaux. Il y a eu également des dérangements des installations ferroviaires et quelques difficultés liées aux rames TGV
2012-05; Nord; LILLE; PARIS NORD; 603; 602; 1; 51; 91.5; 
2012-05; Nord; PARIS NORD; LILLE; 620; 619; 1; 64; 89.7; 
2012-05; Sud-Est; LYON PART DIEU; MARSEILLE ST CHARLES; 625; 625; 0; 114; 81.8; 
2012-05; Sud-Est; MONTPELLIER; LYON PART DIEU; 368; 368; 0; 61; 83.4; 
2012-05; Sud-Est; PARIS LYON; LYON PART DIEU; 625; 623; 2; 35; 94.4; 
2012-05; Sud-Est; MACON LOCHE; PARIS LYON; 174; 174; 0; 18; 89.7; 
2012-05; Est; METZ; PARIS EST; 300; 300; 0; 21; 93.0; Un légère baisse mais un résultat toujours supérieur à l'objectif malgré un baisse de la performance du matériel sur la fin de mois.
2012-05; Atlantique; PARIS MONTPARNASSE; ANGOULEME; 312; 312; 0; 19; 93.9; 
2012-05; Sud-Est; MULHOUSE VILLE; PARIS LYON; 311; 311; 0; 22; 92.9; 
2012-05; Atlantique; NANTES; PARIS MONTPARNASSE; 579; 579; 0; 18; 96.9; 
2012-05; Atlantique; PARIS MONTPARNASSE; NANTES; 554; 554; 0; 22; 96.0; 
2012-05; Sud-Est; NIMES; PARIS LYON; 394; 393; 1; 36; 90.8; 
2012-05; Atlantique; PARIS MONTPARNASSE; POITIERS; 455; 455; 0; 32; 93.0; 
2012-05; Atlantique; PARIS MONTPARNASSE; QUIMPER; 156; 156; 0; 19; 87.8; Travaux ayant lieu en Bretagne avec limitation de la vitesse de nombreux TGV occasionnant des retards.
2012-05; Sud-Est; ANNECY; PARIS LYON; 197; 197; 0; 13; 93.4; 
2012-05; Atlantique; PARIS MONTPARNASSE; RENNES; 563; 563; 0; 54; 90.4; 
2012-05; Atlantique; PARIS MONTPARNASSE; ST MALO; 55; 55; 0; 9; 83.6; Les trains circulant entre St Malo et Paris ont été retardés par des limitations de vitesses dues aux différentes zones de travaux présentes sur le parcours (travaux près de Connerré et régénération de voies de la Ligne à Grande Vitesse Atlantique à Massy) ainsi que par quelques incidents (présence d'animaux domestiques aux abords de la voie près de Dol et un accident de personne entre Dol et St Malo).
2012-05; Atlantique; ST PIERRE DES CORPS; PARIS MONTPARNASSE; 466; 466; 0; 78; 83.3; Les travaux près de Connerré (72) et de Massy (banlieue sud de Paris) ont une part de responsabilité dans l'irrégularité des trains ayant circulés entre Saint-Pierre-des-Corps et Paris. Les fortes affluences de voyageurs en raison des divers ponts du mois de mai ont aussi porté préjudice à notre production (prise en charge de 906 personnes à mobilité réduite entre autre).
2012-05; Est; STRASBOURG; PARIS EST; 471; 471; 0; 73; 84.5; Un légère baisse mais un résultat toujours supérieur à l'objectif malgré un baisse de la performance du matériel sur la fin de mois.
2012-05; Nord; ARRAS; PARIS NORD; 338; 338; 0; 32; 90.5; 
2012-05; Nord; PARIS NORD; ARRAS; 370; 370; 0; 42; 88.6; 
2012-06; Atlantique; BREST; PARIS MONTPARNASSE; 162; 162; 0; 8; 95.1; 
2012-06; Nord; DOUAI; PARIS NORD; 201; 201; 0; 21; 89.6; 
2012-06; Atlantique; LAVAL; PARIS MONTPARNASSE; 237; 237; 0; 13; 94.5; 
2012-06; Atlantique; LE MANS; PARIS MONTPARNASSE; 471; 469; 2; 63; 86.6; Quelques incidents ayant un fort impact sur la régularité des TGV.
2012-06; Nord; LYON PART DIEU; LILLE; 274; 274; 0; 76; 72.3; Liaison touchée par des vols de câbles et des accidents de personnes sur le Nord mais aussi par des dérangements d'installations ferroviaires sur le Sud-Est.
2012-06; Sud-Est; MONTPELLIER; LYON PART DIEU; 359; 359; 0; 56; 84.4; 
2012-06; Sud-Est; MACON LOCHE; PARIS LYON; 170; 170; 0; 15; 91.2; 
2012-06; Sud-Est; MARSEILLE ST CHARLES; PARIS LYON; 502; 501; 1; 42; 91.6; 
2012-06; Atlantique; PARIS MONTPARNASSE; ANGOULEME; 322; 322; 0; 24; 92.5; 
2012-06; Est; PARIS EST; NANCY; 286; 286; 0; 12; 95.8; 
2012-06; Est; NANTES; STRASBOURG; 60; 60; 0; 4; 93.3; 
2012-06; Atlantique; POITIERS; PARIS MONTPARNASSE; 493; 492; 1; 66; 86.6; Quelques incidents ayant un fort impact sur la régularité des TGV.
2012-06; Atlantique; PARIS MONTPARNASSE; QUIMPER; 160; 160; 0; 14; 91.3; 
2012-06; Est; REIMS; PARIS EST; 230; 230; 0; 25; 89.1; 
2012-06; Sud-Est; SAINT ETIENNE CHATEAUCREUX; PARIS LYON; 116; 116; 0; 13; 88.8; 
2012-06; Sud-Est; PARIS LYON; SAINT ETIENNE CHATEAUCREUX; 116; 116; 0; 7; 94.0; 
2012-06; Atlantique; PARIS MONTPARNASSE; ST PIERRE DES CORPS; 459; 459; 0; 41; 91.1; 
2011-12; Sud-Est; BESANCON FRANCHE COMTE TGV; PARIS LYON; 195; 195; 0; 17; 91.3; 
2012-01; Sud-Est; PARIS LYON; CHAMBERY CHALLES LES EAUX; 262; 262; 0; 36; 86.3; 
2012-01; Nord; DOUAI; PARIS NORD; 207; 207; 0; 26; 87.4; 
2012-01; Nord; DUNKERQUE; PARIS NORD; 109; 109; 0; 8; 92.7; 
2012-01; Sud-Est; LE CREUSOT MONTCEAU MONTCHANIN; PARIS LYON; 217; 217; 0; 30; 86.2; 
2012-01; Atlantique; LE MANS; PARIS MONTPARNASSE; 478; 478; 0; 76; 84.1; 
2012-01; Atlantique; PARIS MONTPARNASSE; ANGERS SAINT LAUD; 446; 446; 0; 25; 94.4; 
2012-01; Atlantique; LYON PART DIEU; RENNES; 31; 31; 0; 1; 96.8; 
2012-01; Sud-Est; PARIS LYON; MARSEILLE ST CHARLES; 495; 495; 0; 33; 93.3; 
2012-01; Est; METZ; PARIS EST; 301; 301; 0; 25; 91.7; 
2012-01; Est; PARIS EST; METZ; 314; 314; 0; 20; 93.6; 
2012-01; Atlantique; PARIS MONTPARNASSE; ANGOULEME; 331; 331; 0; 19; 94.3; 
2012-01; Est; MULHOUSE VILLE; PARIS LYON; 314; 314; 0; 25; 92.0; 
2012-01; Est; NANCY; PARIS EST; 301; 301; 0; 15; 95.0; 
2012-01; Atlantique; PARIS MONTPARNASSE; NANTES; 592; 592; 0; 36; 93.9; 
2012-01; Sud-Est; NICE VILLE; PARIS LYON; 162; 162; 0; 11; 93.2; 
2012-01; Sud-Est; PARIS LYON; ANNECY; 207; 207; 0; 13; 93.7; 
2012-01; Atlantique; RENNES; PARIS MONTPARNASSE; 577; 577; 0; 45; 92.2; 
2012-01; Atlantique; ST PIERRE DES CORPS; PARIS MONTPARNASSE; 470; 470; 0; 62; 86.8; 
2012-01; Est; STRASBOURG; PARIS EST; 481; 480; 1; 55; 88.5; 
2012-01; Sud-Est; VALENCE ALIXAN TGV; PARIS LYON; 244; 244; 0; 21; 91.4; 
2012-01; Sud-Est; PARIS LYON; AVIGNON TGV; 395; 395; 0; 35; 91.1; 
2012-02; Sud-Est; AIX EN PROVENCE TGV; PARIS LYON; 397; 396; 1; 75; 81.1; 
2012-02; Sud-Est; DIJON VILLE; PARIS LYON; 449; 447; 2; 68; 84.8; 
2012-02; Sud-Est; PARIS LYON; DIJON VILLE; 456; 456; 0; 39; 91.4; 
2012-02; Nord; DOUAI; PARIS NORD; 189; 189; 0; 35; 81.5; 
2012-02; Sud-Est; GRENOBLE; PARIS LYON; 237; 237; 0; 23; 90.3; 
2012-02; Atlantique; LAVAL; PARIS MONTPARNASSE; 234; 234; 0; 33; 85.9; 
2012-02; Sud-Est; LE CREUSOT MONTCEAU MONTCHANIN; PARIS LYON; 203; 203; 0; 49; 75.9; 
2012-02; Atlantique; PARIS MONTPARNASSE; LE MANS; 426; 426; 0; 98; 77.0; 
2012-02; Atlantique; PARIS MONTPARNASSE; ANGERS SAINT LAUD; 419; 419; 0; 73; 82.6; 
2012-02; Nord; LYON PART DIEU; LILLE; 255; 255; 0; 86; 66.3; 
2012-02; Sud-Est; MARSEILLE ST CHARLES; LYON PART DIEU; 554; 554; 0; 122; 78.0; 
2012-02; Sud-Est; LYON PART DIEU; MONTPELLIER; 337; 336; 1; 106; 68.5; 
2012-02; Atlantique; LYON PART DIEU; RENNES; 29; 29; 0; 10; 65.5; 
2012-02; Sud-Est; PARIS LYON; MARSEILLE ST CHARLES; 485; 479; 6; 42; 91.2; 
2012-02; Sud-Est; PARIS LYON; MONTPELLIER; 336; 336; 0; 50; 85.1; 
2012-02; Est; PARIS EST; NANCY; 276; 276; 0; 27; 90.2; 
2012-02; Atlantique; PARIS MONTPARNASSE; NANTES; 556; 556; 0; 65; 88.3; 
2012-02; Sud-Est; PERPIGNAN; PARIS LYON; 143; 143; 0; 21; 85.3; 
2012-02; Atlantique; QUIMPER; PARIS MONTPARNASSE; 140; 140; 0; 16; 88.6; 
2012-02; Sud-Est; PARIS LYON; ANNECY; 195; 195; 0; 27; 86.2; 
2012-02; Atlantique; RENNES; PARIS MONTPARNASSE; 544; 544; 0; 71; 86.9; 
2012-02; Sud-Est; SAINT ETIENNE CHATEAUCREUX; PARIS LYON; 96; 95; 1; 22; 76.8; 
2012-02; Sud-Est; VALENCE ALIXAN TGV; PARIS LYON; 227; 223; 4; 50; 77.6; 
2012-02; Nord; ARRAS; PARIS NORD; 325; 325; 0; 58; 82.2; 
2012-02; Sud-Est; AVIGNON TGV; PARIS LYON; 311; 310; 1; 50; 83.9; 
2012-03; Sud-Est; DIJON VILLE; PARIS LYON; 470; 470; 0; 50; 89.4; 
2012-03; Nord; DOUAI; PARIS NORD; 210; 209; 1; 31; 85.2; Circulation fortement dégradée le 5 mars avec des conséquences sur le 6 mars suite aux difficiles conditions climatiques entraînant notamment la chute d'un câble ERDF sur la ligne à grande vitesse. La ponctualité a également été impactée par les travaux importants de rénovation des voies sur la ligne à grande vitesse entre Lille et TGV - Haute Picardie.
2012-03; Nord; PARIS NORD; DUNKERQUE; 164; 163; 1; 7; 95.7; Circulation fortement dégradée le 5 mars avec des conséquences sur le 6 mars suite aux difficiles conditions climatiques entraînant notamment la chute d'un câble ERDF sur la ligne à grande vitesse. La ponctualité a également été impactée par les travaux importants de rénovation des voies sur la ligne à grande vitesse entre Lille et TGV - Haute Picardie.
2012-03; Sud-Est; GRENOBLE; PARIS LYON; 251; 251; 0; 11; 95.6; 
2012-03; Atlantique; PARIS MONTPARNASSE; LAVAL; 250; 250; 0; 13; 94.8; 
2012-03; Nord; LYON PART DIEU; LILLE; 270; 269; 1; 55; 79.6; Circulation fortement dégradée le 5 mars avec des conséquences sur le 6 mars suite aux difficiles conditions climatiques entraînant notamment la chute d'un câble ERDF sur la ligne à grande vitesse. La ponctualité a également été impactée par les travaux importants de rénovation des voies sur la ligne à grande vitesse entre Lille et TGV - Haute Picardie.
2012-03; Nord; PARIS NORD; LILLE; 641; 635; 6; 49; 92.3; Circulation fortement dégradée le 5 mars avec des conséquences sur le 6 mars suite aux difficiles conditions climatiques entraînant notamment la chute d'un câble ERDF sur la ligne à grande vitesse. La ponctualité a également été impactée par les travaux importants de rénovation des voies sur la ligne à grande vitesse entre Lille et TGV - Haute Picardie.
2012-03; Sud-Est; LYON PART DIEU; PARIS LYON; 640; 640; 0; 35; 94.5; 
2012-03; Sud-Est; MACON LOCHE; PARIS LYON; 184; 184; 0; 11; 94.0; 
2012-03; Sud-Est; PARIS LYON; MULHOUSE VILLE; 309; 309; 0; 23; 92.6; 
2012-03; Sud-Est; PARIS LYON; NICE VILLE; 160; 160; 0; 23; 85.6; Plusieurs zones de limitations de vitesse liées à des travaux de maintenance ou d'amélioration du réseau, pénalisent la régularité des trains sur la ligne Marseille-Nice.
2012-03; Sud-Est; NIMES; PARIS LYON; 362; 362; 0; 36; 90.1; 
2012-03; Sud-Est; PARIS LYON; PERPIGNAN; 150; 150; 0; 14; 90.7; 
2012-03; Atlantique; QUIMPER; PARIS MONTPARNASSE; 149; 149; 0; 3; 98.0; 
2012-03; Est; STRASBOURG; PARIS EST; 478; 478; 0; 37; 92.3; Incendie d'un entrepôt à hauteur de Gagny qui a généré une coupure caténaire les 10 et 11 mars 2012.
2012-03; Sud-Est; TOULON; PARIS LYON; 243; 243; 0; 33; 86.4; 
2012-03; Atlantique; TOURS; PARIS MONTPARNASSE; 227; 227; 0; 32; 85.9; 
2012-03; Sud-Est; AVIGNON TGV; PARIS LYON; 328; 328; 0; 35; 89.3; 
2012-03; Sud-Est; PARIS LYON; BELLEGARDE (AIN); 284; 284; 0; 28; 90.1; 
2012-04; Sud-Est; AIX EN PROVENCE TGV; PARIS LYON; 437; 437; 0; 71; 83.8; 
2012-04; Nord; PARIS NORD; DOUAI; 201; 200; 1; 18; 91.0; Travaux importants de rénovation des voies entre Arras et Isbergues. entre Douai et Valenciennes et sur la ligne à grande vitesse entre Lille et TGV - Haute Picardie. Les circulations ont également été impactées par une succession de dérangements d'installations ferroviaires.
2012-04; Sud-Est; PARIS LYON; GRENOBLE; 234; 234; 0; 19; 91.9; 
2012-04; Atlantique; LA ROCHELLE VILLE; PARIS MONTPARNASSE; 213; 213; 0; 23; 89.2; Avec un résultat de 89.2\%. cette relation est un peu en retrait essentiellement en raison des ralentissements pour travaux de raccordement à la ligne grande vitesse Sud-Europe-Atlantique en construction entre Poitiers et St-Pierre
2012-04; Atlantique; ANGERS SAINT LAUD; PARIS MONTPARNASSE; 467; 467; 0; 28; 94.0; 
2012-04; Sud-Est; LYON PART DIEU; MARSEILLE ST CHARLES; 600; 600; 0; 130; 78.3; Fragilité de la liaison liée à la longueur du parcours des trains assurant cette desserte et une baisse de la régularité en avril qui s'explique par une augmentation des incidents d'origine externe (trois accidents de personne, un vol de câbles sur ligne nouvelle) et une limitation de vitesse pour la réalisation de travaux au sud de Lyon
2012-04; Sud-Est; PARIS LYON; LYON PART DIEU; 615; 612; 3; 25; 95.9; 
2012-04; Sud-Est; MACON LOCHE; PARIS LYON; 174; 174; 0; 11; 93.7; 
2012-04; Sud-Est; MARSEILLE ST CHARLES; PARIS LYON; 499; 498; 1; 44; 91.2; 
2012-04; Sud-Est; MULHOUSE VILLE; PARIS LYON; 297; 297; 0; 35; 88.2; 
2012-04; Atlantique; PARIS MONTPARNASSE; NANTES; 553; 553; 0; 25; 95.5; 
2012-04; Sud-Est; PARIS LYON; PERPIGNAN; 156; 156; 0; 10; 93.6; 
2012-04; Atlantique; PARIS MONTPARNASSE; QUIMPER; 157; 157; 0; 7; 95.5; 
2012-04; Sud-Est; PARIS LYON; ANNECY; 199; 199; 0; 9; 95.5; 
2012-04; Est; REIMS; PARIS EST; 230; 229; 1; 11; 95.2; Une liaison au-dessus de l'objectif malgré les pertes de temps du au jet de pierre et à l'incendie à l'entrée de Paris-Est.
2012-04; Atlantique; ST PIERRE DES CORPS; PARIS MONTPARNASSE; 458; 458; 0; 60; 86.9; Heurt d'un chevreuil par le TGV8302 avant Vendôme, problème d'une porte sur une voiture du TGV8383 à St-Pierre-des-Corps, anomalie sur un dispositif de sécurité sur un TGV au départ de Paris-Montparnasse.
2012-04; Sud-Est; PARIS LYON; TOULON; 211; 211; 0; 37; 82.5; 
2012-04; Sud-Est; VALENCE ALIXAN TGV; PARIS LYON; 236; 236; 0; 29; 87.7; 
2012-04; Atlantique; VANNES; PARIS MONTPARNASSE; 171; 171; 0; 11; 93.6; 
2012-04; Sud-Est; PARIS LYON; BESANCON FRANCHE COMTE TGV; 229; 229; 0; 23; 90.0; 
2012-05; Sud-Est; PARIS LYON; AIX EN PROVENCE TGV; 436; 436; 0; 49; 88.8; 
2012-05; Atlantique; BREST; PARIS MONTPARNASSE; 167; 167; 0; 11; 93.4; 
2012-05; Atlantique; LAVAL; PARIS MONTPARNASSE; 245; 245; 0; 21; 91.4; 
2012-05; Atlantique; PARIS MONTPARNASSE; LE MANS; 450; 450; 0; 52; 88.4; Régularité en baisse pour cause de travaux sur la Ligne à Grande Vitesse entre Connerré et Le Mans ainsi que la régénération des voies entre Massy (banlieue sud de Paris) et Dangeau (28). L'irrégularité de ce mois est aussi due à un certains nombre d'événements ayant eu un impact important sur la régularité des TGV : transbordement au Mans suite à la panne d'un TGV, demande de secours d'un train fret de la compagnie EuroCargo Rail retardant 17 TGV de 20 minutes à 2h45, dérangements de la signalisation.
2012-05; Atlantique; PARIS MONTPARNASSE; ANGERS SAINT LAUD; 437; 437; 0; 22; 95.0; 
2012-05; Sud-Est; LYON PART DIEU; MONTPELLIER; 375; 375; 0; 61; 83.7; 
2012-05; Sud-Est; PARIS LYON; MARSEILLE ST CHARLES; 510; 509; 1; 36; 92.9; 
2012-05; Est; PARIS EST; METZ; 316; 316; 0; 12; 96.2; En amélioration par rapport aux derniers six mois.
2012-05; Atlantique; ANGOULEME; PARIS MONTPARNASSE; 346; 346; 0; 61; 82.4; Importants événements : le 2, panne d'un train fret ayant engendré des retards de 10 à 30 minutes sur 12 TGV et 2 iDTGV ; le 8, un problème de fermeture de portes sur un TGV ; le 17, panne d'un train de travaux ayant retardé 4 TGV de 10 à 20 minutes ; le 26, intervention des pompiers suite au malaise d'un voyageur (30 minutes de retard). Par ailleurs, les travaux de raccordement de la Ligne à Grande Vitesse Sud Europe Atlantique entre Paris et Bordeaux ont occasionné des retards.
2012-05; Sud-Est; MONTPELLIER; PARIS LYON; 393; 392; 1; 31; 92.1; 
2012-05; Sud-Est; PARIS LYON; MONTPELLIER; 374; 374; 0; 42; 88.8; 
2012-05; Est; NANCY; PARIS EST; 298; 298; 0; 13; 95.6; En amélioration par rapport aux derniers six mois.
2012-05; Atlantique; RENNES; PARIS MONTPARNASSE; 571; 571; 0; 42; 92.6; 
2012-05; Sud-Est; PARIS LYON; SAINT ETIENNE CHATEAUCREUX; 117; 117; 0; 11; 90.6; 
2013-11; Sud-Est; AIX EN PROVENCE TGV; PARIS LYON; 391; 391; 0; 82; 79.0; Les intempéries ayant touché la région lyonnaise les 20, 21 et 22 novembre ont provoqué des retards importants sur l'Axe TGV Sud Est.
2013-11; Atlantique; PARIS MONTPARNASSE; BREST; 182; 182; 0; 5; 97.3; 
2013-11; Nord; PARIS NORD; DOUAI; 200; 200; 0; 24; 88.0; 
2013-11; Sud-Est; LE CREUSOT MONTCEAU MONTCHANIN; PARIS LYON; 212; 212; 0; 41; 80.7; 
2013-11; Sud-Est; PARIS LYON; LE CREUSOT MONTCEAU MONTCHANIN; 199; 199; 0; 31; 84.4; 
2013-11; Atlantique; PARIS MONTPARNASSE; LE MANS; 426; 426; 0; 57; 86.6; 
2013-11; Nord; MARSEILLE ST CHARLES; LILLE; 118; 118; 0; 38; 67.8; Accidents de personne, alerte à la bombe à Lille, dérangements d'installation à Beugnatre et Croisilles, nombreuses limitations en vitesse pour cause de travaux, important épisode neigeux en Rhône Alpes les 20, 21 et 22 novembre.
2013-11; Nord; PARIS NORD; LILLE; 589; 589; 0; 70; 88.1; 
2013-11; Sud-Est; LYON PART DIEU; MARSEILLE ST CHARLES; 536; 536; 0; 125; 76.7; Les intempéries ayant touché la région lyonnaise les 20, 21 et 22 novembre ont provoqué des retards importants sur l'Axe TGV Sud Est.
2013-11; Sud-Est; MARSEILLE ST CHARLES; LYON PART DIEU; 496; 496; 0; 157; 68.3; Les intempéries ayant touché la région lyonnaise les 20, 21 et 22 novembre ont provoqué des retards importants sur l'Axe TGV Sud Est.
2013-11; Sud-Est; LYON PART DIEU; MONTPELLIER; 346; 346; 0; 75; 78.3; Les intempéries ayant touché la région lyonnaise les 20, 21 et 22 novembre ont provoqué des retards importants sur l'Axe TGV Sud Est.
2013-11; Sud-Est; PARIS LYON; MACON LOCHE; 208; 208; 0; 19; 90.9; 
2013-11; Sud-Est; PERPIGNAN; PARIS LYON; 148; 148; 0; 26; 82.4; Les intempéries ayant touché la région lyonnaise les 20, 21 et 22 novembre ont provoqué des retards importants sur l'Axe TGV Sud Est.
2013-11; Sud-Est; PARIS LYON; SAINT ETIENNE CHATEAUCREUX; 113; 113; 0; 15; 86.7; 
2013-11; Est; PARIS EST; STRASBOURG; 451; 451; 0; 27; 94.0; 
2013-11; Nord; ARRAS; PARIS NORD; 321; 321; 0; 33; 89.7; 
2013-11; Atlantique; VANNES; PARIS MONTPARNASSE; 167; 166; 1; 7; 95.8; 
2013-11; Sud-Est; AVIGNON TGV; PARIS LYON; 379; 379; 0; 78; 79.4; Les intempéries ayant touché la région lyonnaise les 20, 21 et 22 novembre ont provoqué des retards importants sur l'Axe TGV Sud Est.
2013-11; Sud-Est; PARIS LYON; AVIGNON TGV; 383; 383; 0; 55; 85.6; 
2013-11; Sud-Est; BESANCON FRANCHE COMTE TGV; PARIS LYON; 216; 216; 0; 15; 93.1; 
2013-12; Atlantique; PARIS MONTPARNASSE; BREST; 195; 194; 1; 8; 95.9; 
2013-12; Sud-Est; PARIS LYON; CHAMBERY CHALLES LES EAUX; 241; 241; 0; 33; 86.3; 
2013-12; Sud-Est; PARIS LYON; DIJON VILLE; 499; 499; 0; 23; 95.4; 
2013-12; Nord; DOUAI; PARIS NORD; 203; 203; 0; 38; 81.3; 
2013-12; Nord; DUNKERQUE; PARIS NORD; 105; 105; 0; 5; 95.2; 
2013-12; Nord; PARIS NORD; DUNKERQUE; 126; 126; 0; 9; 92.9; 
2013-12; Sud-Est; GRENOBLE; PARIS LYON; 239; 238; 1; 19; 92.0; 
2013-12; Atlantique; LA ROCHELLE VILLE; PARIS MONTPARNASSE; 224; 220; 4; 10; 95.5; 
2013-12; Atlantique; PARIS MONTPARNASSE; ANGERS SAINT LAUD; 430; 427; 3; 31; 92.7; 
2013-12; Sud-Est; LYON PART DIEU; MARSEILLE ST CHARLES; 632; 632; 0; 113; 82.1; 
2013-12; Sud-Est; MARSEILLE ST CHARLES; LYON PART DIEU; 569; 569; 0; 100; 82.4; 
2013-12; Sud-Est; MONTPELLIER; LYON PART DIEU; 372; 370; 2; 71; 80.8; 
2013-12; Atlantique; QUIMPER; PARIS MONTPARNASSE; 108; 107; 1; 13; 87.9; 
2013-12; Atlantique; RENNES; PARIS MONTPARNASSE; 576; 569; 7; 59; 89.6; 
2013-12; Sud-Est; SAINT ETIENNE CHATEAUCREUX; PARIS LYON; 50; 50; 0; 4; 92.0; 
2013-12; Atlantique; ST PIERRE DES CORPS; PARIS MONTPARNASSE; 445; 434; 11; 88; 79.7; Le 15, accident de personne à Jaunay-Clan (nord de Poitiers) : TGV 8350 et 8444 retardés respectivement de 228 et 130 minutes. Le 9, le TGV 8350 est retardé de 97 minutes au terminus après avoir heurté un animal sur ligne à grande vitesse.
2013-12; Est; PARIS EST; STRASBOURG; 455; 453; 2; 40; 91.2; 
2013-12; Atlantique; PARIS MONTPARNASSE; TOULOUSE MATABIAU; 150; 148; 2; 18; 87.8; 
2013-12; Atlantique; PARIS MONTPARNASSE; TOURS; 148; 145; 3; 19; 86.9; 
2013-12; Sud-Est; PARIS LYON; AVIGNON TGV; 451; 451; 0; 47; 89.6; 
2014-01; Sud-Est; PARIS LYON; DIJON VILLE; 521; 521; 0; 20; 96.2; 
2014-01; Atlantique; PARIS MONTPARNASSE; LA ROCHELLE VILLE; 227; 227; 0; 2; 99.1; 
2014-01; Sud-Est; LYON PART DIEU; PARIS LYON; 642; 642; 0; 28; 95.6; 
2014-01; Atlantique; LYON PART DIEU; RENNES; 30; 30; 0; 3; 90.0; 
2014-01; Sud-Est; MACON LOCHE; PARIS LYON; 201; 201; 0; 14; 93.0; 
2014-01; Est; METZ; PARIS EST; 301; 301; 0; 13; 95.7; 
2014-01; Sud-Est; PARIS LYON; NIMES; 328; 328; 0; 33; 89.9; 
2014-01; Atlantique; PARIS MONTPARNASSE; POITIERS; 522; 522; 0; 12; 97.7; 
2014-01; Atlantique; PARIS MONTPARNASSE; TOULOUSE MATABIAU; 154; 154; 0; 15; 90.3; 
2014-01; Nord; ARRAS; PARIS NORD; 343; 343; 0; 45; 86.9; 
2014-01; Sud-Est; PARIS LYON; BELLEGARDE (AIN); 238; 238; 0; 27; 88.7; 
2014-02; Sud-Est; AIX EN PROVENCE TGV; PARIS LYON; 368; 368; 0; 32; 91.3; 
2014-02; Sud-Est; PARIS LYON; GRENOBLE; 245; 245; 0; 21; 91.4; 
2014-02; Atlantique; LAVAL; PARIS MONTPARNASSE; 220; 220; 0; 17; 92.3; 
2014-02; Nord; LILLE; LYON PART DIEU; 196; 196; 0; 12; 93.9; 
2014-02; Nord; LYON PART DIEU; LILLE; 224; 224; 0; 41; 81.7; 
2014-02; Nord; MARSEILLE ST CHARLES; LILLE; 112; 112; 0; 23; 79.5; Incident matériel à Pasilly, heurt d'un chevreuil, colis suspect à Marseille, incident technique lors de travaux à Crisenoy, incident technique lié aux installations au sol à Macon, plusieurs limitations de vitesse, essentiellement à Marseille et Lapeyrouse.
2014-02; Nord; LILLE; PARIS NORD; 567; 567; 0; 52; 90.8; 
2014-02; Sud-Est; MARSEILLE ST CHARLES; LYON PART DIEU; 511; 511; 0; 70; 86.3; 
2014-02; Atlantique; RENNES; LYON PART DIEU; 56; 56; 0; 2; 96.4; 
2014-02; Est; PARIS EST; METZ; 279; 279; 0; 10; 96.4; 
2014-02; Sud-Est; MONTPELLIER; PARIS LYON; 302; 302; 0; 16; 94.7; 
2014-02; Est; PARIS EST; REIMS; 196; 196; 0; 7; 96.4; 
2014-02; Sud-Est; PARIS LYON; SAINT ETIENNE CHATEAUCREUX; 12; 12; 0; 0; 100.0; 
2014-02; Atlantique; PARIS MONTPARNASSE; ST MALO; 52; 52; 0; 3; 94.2; 
2014-02; Atlantique; PARIS MONTPARNASSE; ST PIERRE DES CORPS; 419; 419; 0; 29; 93.1; 
2014-02; Atlantique; TOURS; PARIS MONTPARNASSE; 187; 187; 0; 13; 93.0; 
2014-02; Nord; PARIS NORD; ARRAS; 310; 310; 0; 25; 91.9; 
2014-02; Sud-Est; BESANCON FRANCHE COMTE TGV; PARIS LYON; 205; 205; 0; 11; 94.6; 
2014-02; Atlantique; PARIS MONTPARNASSE; BORDEAUX ST JEAN; 570; 570; 0; 44; 92.3; 
2015-01; Sud-Est; AIX EN PROVENCE TGV; PARIS LYON; 432; 430; 2; 37; 91.4; 5 collisions avec des animaux sauvages sur la ligne à grande vitesse ont perturbé la régularité de cette relation en janvier 2015
2015-01; Sud-Est; CHAMBERY CHALLES LES EAUX; PARIS LYON; 250; 249; 1; 31; 87.6; 
2015-01; Atlantique; LAVAL; PARIS MONTPARNASSE; 219; 219; 0; 13; 94.1; 
2015-01; Atlantique; LE MANS; PARIS MONTPARNASSE; 422; 422; 0; 56; 86.7; 
2015-01; Nord; LILLE; LYON PART DIEU; 282; 280; 2; 35; 87.5; Les principaux évènements survenus au mois de janvier sont :
Chutes de neige dans le sud-est le 20 
Accident de personne : Cesseins le 01 et Vennissieux le 03
Heurts d'animaux  : Chevry le 05, Neufmoutiers le 06, Cluny le 18 et Moussy et Lapalud le 29
Plusieurs colis suspects : à Marseille le 08, Marne la Vallée le 08, Roissy les 08,10 et 11 et à Lyon les 09 et 15
Un défaut d'alimentation à Fresnoy le 06
Obstacle dans la voie à Macon le 13
Des travaux importants sur la LGV Nord entrainant des limitations de vitesse
2015-01; Nord; LYON PART DIEU; LILLE; 271; 271; 0; 60; 77.9; 
2015-01; Sud-Est; LYON PART DIEU; MARSEILLE ST CHARLES; 569; 568; 1; 116; 79.6; 5 collisions avec des animaux sauvages sur la ligne à grande vitesse ont perturbé la régularité de cette relation en janvier 2015
2015-01; Sud-Est; LYON PART DIEU; PARIS LYON; 622; 622; 0; 50; 92.0; 
2015-01; Sud-Est; MULHOUSE VILLE; PARIS LYON; 261; 259; 2; 21; 91.9; 
2015-01; Sud-Est; NIMES; PARIS LYON; 292; 292; 0; 35; 88.0; 
2015-01; Est; PARIS EST; NANCY; 284; 284; 0; 19; 93.3; 
2015-01; Sud-Est; PARIS LYON; MACON LOCHE; 210; 208; 2; 16; 92.3; 
2015-01; Sud-Est; PARIS LYON; MARSEILLE ST CHARLES; 430; 427; 3; 29; 93.2; 
2015-01; Sud-Est; PARIS LYON; MULHOUSE VILLE; 254; 254; 0; 22; 91.3; 
2015-01; Atlantique; PARIS MONTPARNASSE; ANGOULEME; 313; 313; 0; 24; 92.3; La ligne a été impactée par des incidents externes dont un accident de personne à VIVONNE le 04/01 (3 TGV retardés de 1h34 à 3h40) ainsi que la divagation d’un veau le 07/01 entre ANGOULÊME et POITIERS (5 TGV de 24’ à 1h04). Un dérangement des installations au nord de CHATELLERAULT le 15/01 a également entraîné des retards (24 TGV de 12’ à 1h50) ainsi que la panne d’un TGV sur la Ligne Grande Vitesse le 23/01 à DANGEAU (41 TGV retardés de 11’ à 3h18) et la panne d’un TER le 28/01 au sud de POITIERS (10 TGV retardés de 18’ à 3h23).
2015-01; Atlantique; PARIS MONTPARNASSE; BORDEAUX ST JEAN; 615; 614; 1; 58; 90.6; 
2015-01; Atlantique; PARIS MONTPARNASSE; BREST; 210; 210; 0; 10; 95.2; 
2015-01; Atlantique; PARIS MONTPARNASSE; LE MANS; 409; 409; 0; 46; 88.8; 
2015-01; Atlantique; PARIS MONTPARNASSE; QUIMPER; 161; 161; 0; 5; 96.9; 
2015-01; Atlantique; PARIS MONTPARNASSE; ST PIERRE DES CORPS; 431; 431; 0; 39; 91.0; 
2015-01; Atlantique; PARIS MONTPARNASSE; TOURS; 175; 175; 0; 23; 86.9; 
2015-01; Nord; PARIS NORD; LILLE; 600; 599; 1; 59; 90.2; 
2015-01; Sud-Est; PERPIGNAN; PARIS LYON; 122; 122; 0; 7; 94.3; 
2015-01; Atlantique; RENNES; PARIS MONTPARNASSE; 525; 525; 0; 32; 93.9; 
2015-01; Est; STRASBOURG; PARIS EST; 406; 406; 0; 36; 91.1; 
2015-01; Sud-Est; VALENCE ALIXAN TGV; PARIS LYON; 246; 245; 1; 29; 88.2; 
2015-01; Atlantique; VANNES; PARIS MONTPARNASSE; 212; 211; 1; 11; 94.8; 
2015-02; Sud-Est; AIX EN PROVENCE TGV; PARIS LYON; 416; 416; 0; 31; 92.5; 
2015-02; Atlantique; ANGOULEME; PARIS MONTPARNASSE; 320; 319; 1; 36; 88.7; 
2015-02; Nord; ARRAS; PARIS NORD; 311; 311; 0; 33; 89.4; 
2012-03; Atlantique; ANGERS SAINT LAUD; PARIS MONTPARNASSE; 488; 488; 0; 48; 90.2; 
2012-03; Sud-Est; LYON PART DIEU; MONTPELLIER; 360; 359; 1; 57; 84.1; 
2012-03; Sud-Est; MONTPELLIER; LYON PART DIEU; 366; 366; 0; 38; 89.6; 
2012-03; Atlantique; LYON PART DIEU; RENNES; 31; 31; 0; 3; 90.3; 
2012-03; Est; METZ; PARIS EST; 297; 297; 0; 14; 95.3; Incendie d'un entrepôt à hauteur de Gagny qui a généré une coupure caténaire les 10 et 11 mars 2012.
2012-03; Sud-Est; MULHOUSE VILLE; PARIS LYON; 310; 309; 1; 31; 90.0; 
2012-03; Sud-Est; PARIS LYON; NIMES; 348; 348; 0; 33; 90.5; 
2012-03; Atlantique; POITIERS; PARIS MONTPARNASSE; 512; 511; 1; 72; 85.9; 
2012-03; Atlantique; PARIS MONTPARNASSE; QUIMPER; 147; 147; 0; 7; 95.2; 
2012-03; Sud-Est; SAINT ETIENNE CHATEAUCREUX; PARIS LYON; 109; 109; 0; 16; 85.3; 
2012-03; Est; PARIS EST; STRASBOURG; 480; 479; 1; 32; 93.3; Incendie d'un entrepôt à hauteur de Gagny qui a généré une coupure caténaire les 10 et 11 mars 2012.
2012-03; Atlantique; TOULOUSE MATABIAU; PARIS MONTPARNASSE; 105; 105; 0; 16; 84.8; 
2012-03; Sud-Est; PARIS LYON; AVIGNON TGV; 391; 391; 0; 15; 96.2; 
2012-03; Sud-Est; BELLEGARDE (AIN); PARIS LYON; 280; 280; 0; 22; 92.1; 
2012-03; Atlantique; BORDEAUX ST JEAN; PARIS MONTPARNASSE; 650; 649; 1; 77; 88.1; Irrégularité est due essentiellement aux travaux.
2012-03; Atlantique; PARIS MONTPARNASSE; BORDEAUX ST JEAN; 636; 636; 0; 47; 92.6; Des ralentissements pour travaux de renouvellement des voies au nord de Bordeaux retardent les TGV en provenance de Bordeaux et qui desservent Angoulême.
2012-04; Sud-Est; PARIS LYON; AIX EN PROVENCE TGV; 420; 420; 0; 53; 87.4; 
2012-04; Sud-Est; GRENOBLE; PARIS LYON; 234; 234; 0; 17; 92.7; 
2012-04; Atlantique; PARIS MONTPARNASSE; LAVAL; 236; 236; 0; 9; 96.2; 
2012-04; Nord; MARSEILLE ST CHARLES; LILLE; 129; 129; 0; 33; 74.4; Conditions difficiles avec des retards liés notamment aux travaux importants sur le réseau Nord et Sud de la France. Les circulations ont également été impactées par une succession de dérangements d'installations ferroviaires.
2012-04; Est; PARIS EST; METZ; 306; 306; 0; 7; 97.7; Une liaison au-dessus de l'objectif malgré les pertes de temps du au jet de pierre et à l'incendie à l'entrée de Paris-Est.
2012-04; Atlantique; ANGOULEME; PARIS MONTPARNASSE; 348; 348; 0; 53; 84.8; Ralentissements pour travaux de raccordement à la ligne grande vitesse Sud-Europe-Atlantique en construction entre Poitiers et St-Pierre-des-Corps.
2012-04; Atlantique; PARIS MONTPARNASSE; ANGOULEME; 318; 318; 0; 28; 91.2; 
2012-04; Sud-Est; MONTPELLIER; PARIS LYON; 384; 383; 1; 24; 93.7; 
2012-04; Sud-Est; PARIS LYON; MULHOUSE VILLE; 300; 300; 0; 32; 89.3; 
2012-04; Est; PARIS EST; NANCY; 286; 286; 0; 8; 97.2; Une liaison au-dessus de l'objectif malgré les pertes de temps du au jet de pierre et à l'incendie à l'entrée de Paris-Est.
2012-04; Est; STRASBOURG; NANTES; 60; 60; 0; 4; 93.3; Un mois au dessus de l'objectif et en amélioration avec l'arrêt d'une partie des travaux.
2012-04; Sud-Est; PERPIGNAN; PARIS LYON; 152; 151; 1; 12; 92.1; 
2012-04; Atlantique; PARIS MONTPARNASSE; RENNES; 546; 546; 0; 17; 96.9; 
2012-04; Est; PARIS EST; STRASBOURG; 464; 464; 0; 33; 92.9; Un mois au dessus de l'objectif et en amélioration avec l'arrêt d'une partie des travaux.
2012-04; Atlantique; PARIS MONTPARNASSE; TOURS; 199; 199; 0; 26; 86.9; Heurt d'un chevreuil par le TGV8302 avant Vendôme, problème d'une porte sur une voiture du TGV8383 à St-Pierre-des-Corps, anomalie sur un dispositif de sécurité sur un TGV au départ de Paris-Montparnasse.
2012-04; Sud-Est; PARIS LYON; VALENCE ALIXAN TGV; 256; 256; 0; 31; 87.9; 
2012-04; Atlantique; PARIS MONTPARNASSE; VANNES; 178; 178; 0; 9; 94.9; 
2012-04; Sud-Est; BESANCON FRANCHE COMTE TGV; PARIS LYON; 207; 207; 0; 24; 88.4; 
2012-05; Nord; PARIS NORD; DOUAI; 210; 210; 0; 23; 89.0; 
2012-05; Atlantique; PARIS MONTPARNASSE; LAVAL; 244; 244; 0; 19; 92.2; 
2012-05; Sud-Est; LE CREUSOT MONTCEAU MONTCHANIN; PARIS LYON; 217; 217; 0; 28; 87.1; 
2012-05; Nord; LYON PART DIEU; LILLE; 284; 284; 0; 57; 79.9; Plusieurs accidents de personnes dans les voies, heurts d'animaux, dérangements des installations ferroviaires et quelques difficultés liées aux rames TGV.
2012-05; Sud-Est; PARIS LYON; MACON LOCHE; 185; 185; 0; 18; 90.3; 
2012-05; Sud-Est; MARSEILLE ST CHARLES; PARIS LYON; 513; 510; 3; 48; 90.6; 
2012-05; Sud-Est; PARIS LYON; MULHOUSE VILLE; 306; 306; 0; 27; 91.2; 
2012-05; Est; NANTES; STRASBOURG; 61; 61; 0; 9; 85.2; Un légère baisse mais un résultat toujours supérieur à l'objectif malgré un baisse de la performance du matériel sur la fin de mois.
2012-05; Sud-Est; NICE VILLE; PARIS LYON; 211; 211; 0; 57; 73.0; D'importantes phases de travaux d'amélioration de l'infrastructure sur le tronçon Nice Marseille nécessitent la mise en place de limitations de vitesse qui réduisent la fluidité des circulations.
2012-05; Atlantique; QUIMPER; PARIS MONTPARNASSE; 145; 145; 0; 10; 93.1; 
2012-05; Sud-Est; PARIS LYON; ANNECY; 204; 204; 0; 16; 92.2; 
2012-05; Est; REIMS; PARIS EST; 235; 234; 1; 10; 95.7; Un légère baisse mais un résultat toujours supérieur à l'objectif malgré un baisse de la performance du matériel sur la fin de mois.
2012-05; Atlantique; PARIS MONTPARNASSE; ST PIERRE DES CORPS; 455; 455; 0; 42; 90.8; 
2012-05; Est; PARIS EST; STRASBOURG; 479; 479; 0; 50; 89.6; Un légère baisse mais un résultat toujours supérieur à l'objectif malgré un baisse de la performance du matériel sur la fin de mois.
2012-05; Sud-Est; PARIS LYON; TOULON; 218; 217; 1; 32; 85.3; 
2012-05; Atlantique; TOULOUSE MATABIAU; PARIS MONTPARNASSE; 112; 112; 0; 13; 88.4; Travaux nécessitant le ralentissement de la vitesse des TGV en divers points sur le parcours (travaux concernant l'alimentation électrique près de Toulouse, renouvellement de la voie et raccordement de la Ligne à Grande Vitesse Sud Europe Atlantique à la ligne classique entre Bordeaux et Tours, travaux près de Connerré (72) ainsi que régénération des voies de la Ligne à Grande Vitesse Atlantique près de Massy (banlieue sud de Paris).
2012-05; Sud-Est; PARIS LYON; VALENCE ALIXAN TGV; 261; 261; 0; 26; 90.0; 
2012-05; Sud-Est; AVIGNON TGV; PARIS LYON; 388; 386; 2; 64; 83.4; 
2012-05; Atlantique; BORDEAUX ST JEAN; PARIS MONTPARNASSE; 624; 624; 0; 74; 88.1; Retards dus à des limitations de vitesse pour l'exécution des travaux préparatoires pour la construction de la Ligne à Grande Vitesse Sud Europe Atlantique entre Paris et Bordeaux.
2012-06; Sud-Est; PARIS LYON; DIJON VILLE; 427; 427; 0; 20; 95.3; 
2012-06; Sud-Est; GRENOBLE; PARIS LYON; 238; 238; 0; 24; 89.9; 
2012-06; Atlantique; LA ROCHELLE VILLE; PARIS MONTPARNASSE; 219; 218; 1; 23; 89.4; Quelques incidents ayant un fort impact sur la régularité des TGV.
2012-06; Atlantique; PARIS MONTPARNASSE; ANGERS SAINT LAUD; 429; 429; 0; 19; 95.6; 
2012-06; Sud-Est; LYON PART DIEU; MONTPELLIER; 364; 364; 0; 58; 84.1; 
2012-06; Sud-Est; PARIS LYON; LYON PART DIEU; 617; 617; 0; 24; 96.1; 
2012-06; Atlantique; RENNES; LYON PART DIEU; 81; 81; 0; 6; 92.6; 
2012-06; Est; NANCY; PARIS EST; 290; 290; 0; 14; 95.2; 
2012-06; Sud-Est; PARIS LYON; NIMES; 372; 372; 0; 36; 90.3; 
2012-06; Atlantique; PARIS MONTPARNASSE; POITIERS; 445; 445; 0; 26; 94.2; 
2012-06; Atlantique; RENNES; PARIS MONTPARNASSE; 562; 561; 1; 31; 94.5; 
2012-06; Atlantique; ST PIERRE DES CORPS; PARIS MONTPARNASSE; 459; 459; 0; 60; 86.9; Quelques incidents ayant un fort impact sur la régularité des TGV.
2012-06; Atlantique; VANNES; PARIS MONTPARNASSE; 170; 170; 0; 9; 94.7; 
2012-06; Sud-Est; AVIGNON TGV; PARIS LYON; 362; 362; 0; 60; 83.4; 
2014-03; Atlantique; BREST; PARIS MONTPARNASSE; 160; 160; 0; 6; 96.3; 
2014-03; Nord; PARIS NORD; DUNKERQUE; 124; 124; 0; 6; 95.2; 
2014-03; Atlantique; LA ROCHELLE VILLE; PARIS MONTPARNASSE; 225; 225; 0; 7; 96.9; 
2014-03; Atlantique; ANGERS SAINT LAUD; PARIS MONTPARNASSE; 449; 449; 0; 25; 94.4; 
2014-03; Nord; MARSEILLE ST CHARLES; LILLE; 124; 124; 0; 28; 77.4; Défaut d'alimentation électrique à Montanay le 01. Agression d'un agent à Lyon impactant les 01 et 02. Vol de câbles aux Mazes le 01 et à Lille le 29. Problème matériel à Arsy le 02. Heurt d'un animal à St Georges d'Espéranche le 13 et à Cluny le 21. Dérangement d'installations au Creusot du 16 au 18 et à Macon le 24. Accident de personne à Aix le 17 et à Vallauris le 21. Incendie dans le tunnel de Marseille le 26. Nombreuses limitations de vitesse pour travaux, notamment à Lapalud les 12, 13 et 14.
2014-03; Sud-Est; PARIS LYON; MARSEILLE ST CHARLES; 479; 479; 0; 28; 94.2; 
2014-03; Sud-Est; MONTPELLIER; PARIS LYON; 336; 336; 0; 33; 90.2; 
2014-03; Atlantique; NANTES; PARIS MONTPARNASSE; 537; 537; 0; 21; 96.1; 
2014-03; Sud-Est; PARIS LYON; NICE VILLE; 169; 169; 0; 19; 88.8; 
2014-03; Sud-Est; PERPIGNAN; PARIS LYON; 159; 159; 0; 13; 91.8; 
2014-03; Sud-Est; PARIS LYON; PERPIGNAN; 158; 158; 0; 16; 89.9; La relation a été impactée par un accident de personne et un vol de câbles près de Montpellier le 1er, le déraillement d'un Fret près de Narbonne le 3, et 2 heurts d'animaux sur lignes à grande vitesse le 21.
2014-03; Atlantique; PARIS MONTPARNASSE; QUIMPER; 147; 147; 0; 8; 94.6; 
2014-03; Sud-Est; PARIS LYON; ANNECY; 150; 150; 0; 7; 95.3; 
2014-03; Atlantique; TOURS; PARIS MONTPARNASSE; 214; 214; 0; 26; 87.9; 
2014-03; Sud-Est; AVIGNON TGV; PARIS LYON; 410; 410; 0; 62; 84.9; 
2014-03; Sud-Est; PARIS LYON; BELLEGARDE (AIN); 240; 240; 0; 24; 90.0; 
2014-03; Atlantique; BORDEAUX ST JEAN; PARIS MONTPARNASSE; 659; 657; 2; 69; 89.5; 
2014-04; Sud-Est; AIX EN PROVENCE TGV; PARIS LYON; 433; 433; 0; 53; 87.8; 
2014-04; Atlantique; PARIS MONTPARNASSE; BREST; 181; 181; 0; 7; 96.1; 
2014-04; Nord; DOUAI; PARIS NORD; 194; 194; 0; 17; 91.2; 
2014-04; Nord; PARIS NORD; DUNKERQUE; 120; 119; 1; 12; 89.9; 
2014-04; Sud-Est; PARIS LYON; GRENOBLE; 245; 245; 0; 22; 91.0; 
2014-04; Sud-Est; LE CREUSOT MONTCEAU MONTCHANIN; PARIS LYON; 213; 213; 0; 38; 82.2; 
2014-04; Atlantique; LE MANS; PARIS MONTPARNASSE; 459; 459; 0; 39; 91.5; 
2014-04; Nord; LILLE; LYON PART DIEU; 209; 209; 0; 22; 89.5; Les principaux évènements survenus au mois d'avril sont : Vol de câble à Oignies le 11 ; Accident de personne à Lille le 17 et au Creusot le 25 ; Heurt de sanglier à Solers le 15 et Lacour d'Arcenay le 16 ; Personnes dans les voies à Sathonay le 11 ; Dérangement des installations à Oignies le 02 et le 17 et au Fresnoy le 28.
2014-04; Nord; MARSEILLE ST CHARLES; LILLE; 123; 123; 0; 26; 78.9; Les principaux évènements survenus au mois d'avril sont : Vol de câble à Oignies le 11 ; Accident de personne à Lille le 17 et au Creusot le 25 ; Heurt de sanglier à Roquemaure le 14, à Solers le 15 et Lacour d'Arcenay le 16, et d’un cheuvreuil à St Georges le 27 ; Personnes dans les voies à Sathonay le 11 ; Dérangement des installations à Oignies le 02 et le 17, à Valence le 16, à St Marcel le 26 et au Fresnoy le 28 ; Nombreuses limitations de vitesse pour travaux, notamment à Marseille pour la construction de la 3ème voie.
2014-04; Nord; PARIS NORD; LILLE; 614; 611; 3; 68; 88.9; 
2014-04; Sud-Est; MONTPELLIER; LYON PART DIEU; 386; 386; 0; 44; 88.6; 
2014-04; Atlantique; RENNES; LYON PART DIEU; 58; 58; 0; 6; 89.7; 
2014-04; Sud-Est; NICE VILLE; PARIS LYON; 214; 214; 0; 35; 83.6; 
2014-04; Atlantique; POITIERS; PARIS MONTPARNASSE; 489; 489; 0; 20; 95.9; 
2014-04; Atlantique; ST MALO; PARIS MONTPARNASSE; 95; 95; 0; 3; 96.8; 
2014-04; Atlantique; TOURS; PARIS MONTPARNASSE; 203; 203; 0; 8; 96.1; 
2013-09; Atlantique; LE MANS; PARIS MONTPARNASSE; 401; 401; 0; 55; 86.3; 
2013-09; Nord; LYON PART DIEU; LILLE; 216; 216; 0; 48; 77.8; Accidents de personnes et dérangements d'installations.
2013-09; Sud-Est; MONTPELLIER; LYON PART DIEU; 308; 308; 0; 61; 80.2; 
2013-09; Atlantique; RENNES; LYON PART DIEU; 54; 54; 0; 9; 83.3; 
2013-09; Sud-Est; PARIS LYON; MACON LOCHE; 167; 167; 0; 5; 97.0; 
2013-09; Est; PARIS EST; METZ; 273; 273; 0; 2; 99.3; 
2013-09; Sud-Est; PARIS LYON; MONTPELLIER; 281; 281; 0; 32; 88.6; 
2013-09; Est; STRASBOURG; NANTES; 54; 54; 0; 5; 90.7; 
2013-09; Sud-Est; NICE VILLE; PARIS LYON; 180; 180; 0; 25; 86.1; Heurt d'une personne à Valence, panne d'un train Intercité près de Toulon, plusieurs incidents techniques suite à des orages en fin de mois.
2013-09; Sud-Est; NIMES; PARIS LYON; 296; 296; 0; 40; 86.5; 
2013-09; Sud-Est; PERPIGNAN; PARIS LYON; 135; 135; 0; 16; 88.1; 
2013-09; Atlantique; PARIS MONTPARNASSE; QUIMPER; 119; 119; 0; 6; 95.0; 
2013-09; Est; REIMS; PARIS EST; 185; 185; 0; 6; 96.8; 
2013-09; Atlantique; PARIS MONTPARNASSE; RENNES; 476; 475; 1; 39; 91.8; 
2013-09; Sud-Est; PARIS LYON; SAINT ETIENNE CHATEAUCREUX; 103; 103; 0; 17; 83.5; 
2013-09; Sud-Est; TOULON; PARIS LYON; 221; 221; 0; 25; 88.7; 
2013-09; Atlantique; PARIS MONTPARNASSE; TOULOUSE MATABIAU; 130; 130; 0; 29; 77.7; Bagages abandonnés, dérangements de signalisation, panne d'un train travaux, heurt d'un sanglier, accident de personne.
2013-09; Nord; ARRAS; PARIS NORD; 292; 292; 0; 35; 88.0; 
2013-09; Atlantique; PARIS MONTPARNASSE; BORDEAUX ST JEAN; 559; 558; 1; 59; 89.4; 
2013-10; Atlantique; BREST; PARIS MONTPARNASSE; 158; 158; 0; 24; 84.8; Événements occasionnant des retards sur des TGV, souvent dans les deux sens de circulations : heurt d’un chevreuil, incident caténaire, détournement suite à la collision avec un camion, accident de personne, défaut d’alimentation caténaire.
2013-10; Sud-Est; PARIS LYON; DIJON VILLE; 480; 480; 0; 19; 96.0; 
2013-10; Nord; PARIS NORD; DOUAI; 185; 185; 0; 14; 92.4; 
2013-10; Nord; PARIS NORD; DUNKERQUE; 130; 130; 0; 2; 98.5; 
2013-10; Atlantique; PARIS MONTPARNASSE; LAVAL; 240; 238; 2; 54; 77.3; Événements occasionnant des retards sur des TGV, souvent dans les deux sens de circulations : heurt d’un chevreuil, incident caténaire, détournement suite à la collision avec un camion, accident de personne, défaut d’alimentation caténaire.
2013-10; Nord; LYON PART DIEU; LILLE; 138; 138; 0; 18; 87.0; 
2013-10; Nord; LILLE; MARSEILLE ST CHARLES; 155; 155; 0; 32; 79.4; Mois d'octobre principalement perturbé dans la deuxième quinzaine : orage et tempête sur Lille avec des arbres dans la caténaire occasionnant dans les deux cas des manques tension, accidents de personne, alerte à la bombe à Lille le 19 octobre, dérangements d'installations, nombreuses limitations de vitesse suite à des travaux essentiellement au sud de Lyon.
2013-10; Sud-Est; MONTPELLIER; LYON PART DIEU; 308; 308; 0; 61; 80.2; 
2013-10; Sud-Est; LYON PART DIEU; PARIS LYON; 648; 648; 0; 24; 96.3; 
2013-10; Sud-Est; PARIS LYON; LYON PART DIEU; 648; 647; 1; 26; 96.0; 
2013-10; Sud-Est; MACON LOCHE; PARIS LYON; 199; 199; 0; 31; 84.4; 
2013-10; Sud-Est; NICE VILLE; PARIS LYON; 198; 197; 1; 41; 79.2; Travaux de rénovation de l'infrastructure sur la ligne à grande Vitesse et sur le tronçon Marseille-Nice.
2013-10; Sud-Est; PARIS LYON; PERPIGNAN; 157; 157; 0; 27; 82.8; Six accidents de personne ont perturbé cette liaison en octobre.
2013-10; Atlantique; QUIMPER; PARIS MONTPARNASSE; 79; 79; 0; 9; 88.6; Événements occasionnant des retards sur des TGV, souvent dans les deux sens de circulations : heurt d’un chevreuil, incident caténaire, détournement suite à la collision avec un camion, accident de personne, défaut d’alimentation caténaire.
2013-10; Sud-Est; PARIS LYON; ANNECY; 177; 177; 0; 9; 94.9; 
2013-10; Atlantique; PARIS MONTPARNASSE; RENNES; 573; 571; 2; 105; 81.6; Événements occasionnant des retards sur des TGV, souvent dans les deux sens de circulations : heurt d’un chevreuil, incident caténaire, détournement suite à la collision avec un camion, accident de personne, défaut d’alimentation caténaire.
2013-10; Atlantique; PARIS MONTPARNASSE; ST MALO; 62; 62; 0; 5; 91.9; 
2013-10; Atlantique; PARIS MONTPARNASSE; ST PIERRE DES CORPS; 475; 475; 0; 61; 87.2; Événements occasionnant des retards sur des TGV, souvent dans les deux sens de circulations : présence de ballast sur les rails, arbre tombé aux abords de la voie avec endommagement du pantographe d'un train (bras mécanique permettant l'alimentation électrique du train), incident caténaire à l'arrivée sur Paris, accident de personne, heurts d'animaux, problème d'alimentation de la caténaire.
2013-10; Atlantique; TOULOUSE MATABIAU; PARIS MONTPARNASSE; 84; 84; 0; 18; 78.6; Événements occasionnant des retards sur des TGV, souvent dans les deux sens de circulations : présence de ballast sur les rails, arbre tombé aux abords de la voie avec endommagement du pantographe d'un train (bras mécanique permettant l'alimentation électrique du train), incident caténaire à l'arrivée sur Paris, accident de personne, heurts d'animaux, problème d'alimentation de la caténaire.
2013-10; Atlantique; TOURS; PARIS MONTPARNASSE; 183; 183; 0; 28; 84.7; 
2013-10; Atlantique; PARIS MONTPARNASSE; TOURS; 152; 152; 0; 23; 84.9; 
2013-10; Atlantique; VANNES; PARIS MONTPARNASSE; 175; 174; 1; 20; 88.5; Événements occasionnanrt des retards sur des TGV, souvent dans les deux sens de circulations : le 13 octobre, heurt d’un chevreuil à Dollon (72) et incident caténaire à Marcoussis (91).
2013-10; Sud-Est; PARIS LYON; AVIGNON TGV; 390; 389; 1; 47; 87.9; 
2013-11; Sud-Est; CHAMBERY CHALLES LES EAUX; PARIS LYON; 197; 196; 1; 41; 79.1; Les intempéries ayant touché la région lyonnaise les 20, 21 et 22 novembre ont provoqué des retards importants sur l'Axe TGV Sud Est.
2013-11; Sud-Est; DIJON VILLE; PARIS LYON; 456; 456; 0; 46; 89.9; 
2013-11; Sud-Est; PARIS LYON; DIJON VILLE; 458; 458; 0; 31; 93.2; 
2013-11; Nord; DUNKERQUE; PARIS NORD; 110; 110; 0; 4; 96.4; 
2013-11; Sud-Est; PARIS LYON; GRENOBLE; 224; 210; 14; 23; 89.0; 
2013-11; Atlantique; LAVAL; PARIS MONTPARNASSE; 237; 236; 1; 12; 94.9; 
2013-11; Atlantique; PARIS MONTPARNASSE; LAVAL; 226; 226; 0; 13; 94.2; 
2013-11; Atlantique; ANGERS SAINT LAUD; PARIS MONTPARNASSE; 446; 444; 2; 47; 89.4; 
2013-11; Nord; LILLE; PARIS NORD; 586; 586; 0; 63; 89.2; 
2013-11; Sud-Est; LYON PART DIEU; PARIS LYON; 612; 611; 1; 44; 92.8; 
2013-11; Sud-Est; PARIS LYON; LYON PART DIEU; 604; 604; 0; 41; 93.2; 
2013-11; Est; METZ; PARIS EST; 285; 285; 0; 35; 87.7; 
2013-11; Atlantique; PARIS MONTPARNASSE; ANGOULEME; 313; 312; 1; 27; 91.3; 
2013-11; Sud-Est; PARIS LYON; MONTPELLIER; 306; 306; 0; 42; 86.3; Les intempéries ayant touché la région lyonnaise les 20, 21 et 22 novembre ont provoqué des retards importants sur l'Axe TGV Sud Est.
2013-11; Atlantique; PARIS MONTPARNASSE; NANTES; 520; 520; 0; 46; 91.2; 
2013-11; Atlantique; ST MALO; PARIS MONTPARNASSE; 100; 100; 0; 3; 97.0; 
2013-11; Sud-Est; TOULON; PARIS LYON; 249; 249; 0; 51; 79.5; Les intempéries ayant touché la région lyonnaise les 20, 21 et 22 novembre ont provoqué des retards importants sur l'Axe TGV Sud Est.
2013-11; Atlantique; TOULOUSE MATABIAU; PARIS MONTPARNASSE; 90; 90; 0; 21; 76.7; Retards en raison de nombreux événements particulièrement perturbants, notamment : incident caténaire à Cérons (33) le 12, heurt de plusieurs sangliers sur ligne à grande vitesse le 19, accident de personne à St Cyr (37) le 23, accident de personne à St Benoit (86) le 26, accident de personne à St Médard (33) le 27.
2013-11; Sud-Est; BELLEGARDE (AIN); PARIS LYON; 254; 254; 0; 46; 81.9; 
2013-11; Atlantique; PARIS MONTPARNASSE; BORDEAUX ST JEAN; 621; 620; 1; 52; 91.6; 
2013-12; Sud-Est; PARIS LYON; AIX EN PROVENCE TGV; 430; 430; 0; 63; 85.3; 
2013-12; Sud-Est; LE CREUSOT MONTCEAU MONTCHANIN; PARIS LYON; 221; 221; 0; 23; 89.6; 
2013-12; Nord; LILLE; LYON PART DIEU; 255; 255; 0; 23; 91.0; 
2013-12; Sud-Est; LYON PART DIEU; MONTPELLIER; 393; 391; 2; 87; 77.7; Quatre accidents de personne entre Montpellier et Nîmes et quatre heurts d'animaux sauvages sur la ligne à grande vitesse ont dégradé la régularité en décembre.
2013-12; Atlantique; RENNES; LYON PART DIEU; 75; 74; 1; 8; 89.2; 
2013-12; Sud-Est; PARIS LYON; MARSEILLE ST CHARLES; 476; 476; 0; 33; 93.1; 
2013-12; Sud-Est; PARIS LYON; MONTPELLIER; 318; 318; 0; 33; 89.6; Quatre accidents de personne entre Montpellier et Nîmes et quatre heurts d'animaux sauvages sur la ligne à grande vitesse ont dégradé la régularité en décembre.
2013-12; Sud-Est; PARIS LYON; MULHOUSE VILLE; 316; 316; 0; 24; 92.4; 
2013-12; Atlantique; NANTES; PARIS MONTPARNASSE; 573; 564; 9; 36; 93.6; 
2013-12; Atlantique; PARIS MONTPARNASSE; NANTES; 555; 548; 7; 36; 93.4; 
2013-12; Est; NANTES; STRASBOURG; 51; 50; 1; 3; 94.0; 
2013-12; Atlantique; POITIERS; PARIS MONTPARNASSE; 502; 493; 9; 35; 92.9; 
2013-12; Sud-Est; PARIS LYON; ANNECY; 159; 159; 0; 15; 90.6; 
2013-12; Atlantique; PARIS MONTPARNASSE; RENNES; 567; 562; 5; 27; 95.2; 
2013-12; Atlantique; PARIS MONTPARNASSE; ST MALO; 56; 56; 0; 2; 96.4; 
2013-12; Atlantique; PARIS MONTPARNASSE; ST PIERRE DES CORPS; 469; 461; 8; 51; 88.9; 
2013-12; Sud-Est; PARIS LYON; BELLEGARDE (AIN); 240; 240; 0; 43; 82.1; 
2013-12; Sud-Est; BESANCON FRANCHE COMTE TGV; PARIS LYON; 225; 225; 0; 11; 95.1; 
2014-01; Sud-Est; AIX EN PROVENCE TGV; PARIS LYON; 406; 406; 0; 51; 87.4; 
2014-01; Nord; PARIS NORD; DOUAI; 207; 207; 0; 12; 94.2; 
2014-01; Atlantique; LA ROCHELLE VILLE; PARIS MONTPARNASSE; 227; 227; 0; 2; 99.1; 
2014-01; Sud-Est; PARIS LYON; LE CREUSOT MONTCEAU MONTCHANIN; 210; 210; 0; 26; 87.6; 
2014-01; Atlantique; PARIS MONTPARNASSE; LE MANS; 457; 457; 0; 30; 93.4; 
2014-01; Sud-Est; LYON PART DIEU; MARSEILLE ST CHARLES; 648; 647; 1; 131; 79.8; Travaux d'amélioration de l'infrastructure entre Lyon et Valence.
2014-01; Sud-Est; MARSEILLE ST CHARLES; LYON PART DIEU; 562; 561; 1; 113; 79.9; Travaux d'amélioration de l'infrastructure entre Lyon et Valence.
2014-01; Sud-Est; PARIS LYON; MONTPELLIER; 328; 328; 0; 30; 90.9; 
2014-01; Est; NANTES; STRASBOURG; 37; 37; 0; 5; 86.5; 
2014-01; Est; STRASBOURG; NANTES; 39; 39; 0; 6; 84.6; 
2014-01; Sud-Est; NIMES; PARIS LYON; 337; 337; 0; 41; 87.8; 
2014-01; Sud-Est; PARIS LYON; PERPIGNAN; 150; 150; 0; 16; 89.3; 
2014-01; Est; REIMS; PARIS EST; 212; 211; 1; 6; 97.2; 
2014-01; Atlantique; ST MALO; PARIS MONTPARNASSE; 101; 101; 0; 4; 96.0; 
2014-01; Nord; PARIS NORD; ARRAS; 340; 340; 0; 18; 94.7; 
2014-01; Atlantique; VANNES; PARIS MONTPARNASSE; 162; 162; 0; 11; 93.2; 
2014-01; Atlantique; BORDEAUX ST JEAN; PARIS MONTPARNASSE; 659; 658; 1; 66; 90.0; 
2014-02; Sud-Est; DIJON VILLE; PARIS LYON; 439; 439; 0; 32; 92.7; 
2014-02; Sud-Est; PARIS LYON; DIJON VILLE; 485; 485; 0; 15; 96.9; 
2014-02; Nord; PARIS NORD; DUNKERQUE; 116; 116; 0; 4; 96.6; 
2012-06; Atlantique; TOULOUSE MATABIAU; PARIS MONTPARNASSE; 106; 105; 1; 14; 86.7; Quelques incidents ayant un fort impact sur la régularité des TGV.
2012-06; Nord; ARRAS; PARIS NORD; 338; 338; 0; 35; 89.6; 
2012-06; Sud-Est; PARIS LYON; BELLEGARDE (AIN); 261; 261; 0; 33; 87.4; 
2012-06; Sud-Est; BESANCON FRANCHE COMTE TGV; PARIS LYON; 199; 199; 0; 13; 93.5; 
2012-06; Sud-Est; PARIS LYON; BESANCON FRANCHE COMTE TGV; 225; 225; 0; 17; 92.4; 
2012-06; Atlantique; BORDEAUX ST JEAN; PARIS MONTPARNASSE; 628; 626; 2; 58; 90.7; 
2012-06; Atlantique; PARIS MONTPARNASSE; BORDEAUX ST JEAN; 609; 609; 0; 51; 91.6; 
2014-12; Sud-Est; LYON PART DIEU; PARIS LYON; 644; 644; 0; 25; 96.1; 
2014-12; Atlantique; LYON PART DIEU; RENNES; 31; 31; 0; 4; 87.1; Cette ligne a été touché par des incidents externes dont un accident de personne à Laval le 01/12 (20 TGV retardés de 16min à 3h19) \& le heurt d'un sanglier le 22/12 sur la Ligne à Grande Vitesse près de St Arnoult (30 TGV retardés de 11min à 1h18).
Des dérangements des installations ont également entraîné de retards : à Valenton (en région parisienne) notamment les 18, 19 \& 21/12 avec près d'une quinzaine de trains retardés par jour \& sur la ligne à Grande Vitesse Sud Est également le 3/12 (21 TGV). Un incident électrique dans la région lyonnaise le 11/12 retarde 32 TGV \& un rail fissuré à Neau (près de Laval) le 29/12 (18 TGV retardés de 12' à 1h15).
2014-12; Sud-Est; MONTPELLIER; LYON PART DIEU; 369; 362; 7; 74; 79.6; Les intempéries en région Languedoc Roussillon ont perturbé la régularité de cette relation en décembre.
2014-12; Est; NANTES; STRASBOURG; 43; 43; 0; 3; 93.0; 
2014-12; Sud-Est; NICE VILLE; PARIS LYON; 187; 187; 0; 20; 89.3; 
2014-12; Sud-Est; NIMES; PARIS LYON; 328; 326; 2; 62; 81.0; Les intempéries en région Languedoc Roussillon ont perturbé la régularité de cette relation en décembre.
2014-12; Sud-Est; PARIS LYON; AIX EN PROVENCE TGV; 437; 437; 0; 29; 93.4; 
2014-12; Sud-Est; PARIS LYON; AVIGNON TGV; 498; 498; 0; 33; 93.4; 
2014-12; Sud-Est; PARIS LYON; LYON PART DIEU; 631; 631; 0; 18; 97.1; 
2014-12; Sud-Est; PARIS LYON; MULHOUSE VILLE; 312; 310; 2; 15; 95.2; 
2014-12; Atlantique; QUIMPER; PARIS MONTPARNASSE; 149; 149; 0; 17; 88.6; La ligne a été affectée par des incidents externes dont un accident de personne à Laval le 01/12 (20 TGV retardés de 16min à 3h19) \& le heurt d'un sanglier sur la Ligne à Grande Vitesse près de St Arnoult le 22/12 (30 TGV retardés de 11min à 1h18);
Des dérangements des installations ont également entraîné des retards: le 25/12 notamment entre Paris Montparnasse \& Massy (32 TGV de 25' à 2h38) \& un rail fissuré à Neau (près de Laval) le 29/12 (18 TGV de 12' à 1h15). A noter également, la défaillance Matériel d'un train d'une autre compagnie ferroviaire en sortie du Mans le 26/12 (20 TGV retardés de 14' à 2h35). 
2014-12; Est; STRASBOURG; PARIS EST; 476; 476; 0; 55; 88.4; 
2015-01; Atlantique; ANGERS SAINT LAUD; PARIS MONTPARNASSE; 435; 434; 1; 24; 94.5; 
2015-01; Sud-Est; AVIGNON TGV; PARIS LYON; 530; 527; 3; 64; 87.9; 5 collisions avec des animaux sauvages sur la ligne à grande vitesse ont perturbé la régularité de cette relation en janvier 2015
2015-01; Sud-Est; BELLEGARDE (AIN); PARIS LYON; 158; 156; 2; 17; 89.1; Sur la ligne à grande vitesse, 3 collisions avec des animaux sauvages et 1 accident de personne ont perturbé la régualrité de cette relation en janvier 2015
2015-01; Atlantique; BORDEAUX ST JEAN; PARIS MONTPARNASSE; 653; 650; 3; 63; 90.3; La ligne a été impactée par des incidents externes dont deux accidents de personne au sud de POITIERS le 04/01 (3 TGV retardés de 1h34 à 3h40) ainsi que la divagation d’un veau le 07/01 entre ANGOULÊME et POITIERS (5 TGV de 24’ à 1h04) et le signalement de colis suspects le 14/01 et le 16/01 en gare de BORDEAUX (retardant respectivement 6 TGV de 18’ à 1h33 et 6 TGV de 17’ à 1h31). Un dérangement des installations au nord de CHATELLERAULT le 15/01 a également entraîné des retards (24 TGV de 12’ à 1h50) ainsi que la panne d’un TGV sur la Ligne Grande Vitesse le 23/01 (41 TGV retardés de 11’ à 3h18) et la panne d’un TER le 28/01 au sud de POITIERS (10 TGV retardés de 18’ à 3h23). Le heurt d’un train fret avec un train en stationnement à BORDEAUX le 13/01 a perturbé les circulations (6 TGV retardés de 8’ à 1h33).
2015-01; Atlantique; BREST; PARIS MONTPARNASSE; 212; 212; 0; 12; 94.3; 
2015-01; Sud-Est; DIJON VILLE; PARIS LYON; 416; 414; 2; 28; 93.2; 
2015-01; Atlantique; LA ROCHELLE VILLE; PARIS MONTPARNASSE; 210; 209; 1; 9; 95.7; 
2015-01; Sud-Est; LE CREUSOT MONTCEAU MONTCHANIN; PARIS LYON; 209; 209; 0; 25; 88.0; 
2015-01; Sud-Est; MARSEILLE ST CHARLES; PARIS LYON; 411; 410; 1; 30; 92.7; 
2015-01; Est; METZ; PARIS EST; 223; 223; 0; 23; 89.7; 
2015-01; Atlantique; NANTES; PARIS MONTPARNASSE; 521; 520; 1; 40; 92.3; 
2015-01; Est; NANTES; STRASBOURG; 39; 39; 0; 3; 92.3; 
2015-01; Est; PARIS EST; METZ; 231; 231; 0; 24; 89.6; 
2015-01; Est; PARIS EST; REIMS; 213; 213; 0; 19; 91.1; 
2015-01; Est; PARIS EST; STRASBOURG; 418; 418; 0; 30; 92.8; 
2015-01; Sud-Est; PARIS LYON; ANNECY; 159; 157; 2; 10; 93.6; 
2015-01; Sud-Est; PARIS LYON; BELLEGARDE (AIN); 171; 169; 2; 20; 88.2; 
2015-01; Sud-Est; PARIS LYON; GRENOBLE; 233; 232; 1; 23; 90.1; Plusieurs incidents techniques et pannes de TER entre Lyon et Grenoble ont perturbé la régularité de cette relation en Janvier 2015
2015-01; Sud-Est; PARIS LYON; TOULON; 231; 228; 3; 31; 86.4; 5 collisions avec des animaux sauvages sur la ligne à grande vitesse ont perturbé la régularité de cette relation en janvier 2015
2015-01; Atlantique; PARIS MONTPARNASSE; ST MALO; 91; 90; 1; 5; 94.4; 
2015-01; Nord; PARIS NORD; DUNKERQUE; 250; 250; 0; 19; 92.4; 
2015-01; Nord; PARIS NORD; ARRAS; 301; 301; 0; 30; 90.0; 
2015-01; Est; REIMS; PARIS EST; 205; 205; 0; 9; 95.6; 
2015-01; Atlantique; RENNES; LYON PART DIEU; 84; 83; 1; 19; 77.1; 
2015-01; Sud-Est; SAINT ETIENNE CHATEAUCREUX; PARIS LYON; 112; 112; 0; 14; 87.5; 
2015-01; Atlantique; ST PIERRE DES CORPS; PARIS MONTPARNASSE; 409; 409; 0; 48; 88.3; 
2015-02; Sud-Est; AVIGNON TGV; PARIS LYON; 508; 507; 1; 56; 89.0; 
2015-02; Atlantique; BREST; PARIS MONTPARNASSE; 214; 210; 4; 25; 88.1; Cette ligne a été touchée par un incident majeur le 13/02 : un problème caténaire à Laval retarde 83 TGV de 7min à 5h59. On compte également 3 incidents externes dont deux survenus sur la Ligne à Grande Vitesse : un accident de personne à Rouvray le 7/02 (19 TGV retardés de 30min à 4h04) \& un autre le 18/02 près de Paris (26 TGV retardés de 12 à 48min) ; le heurt d’un chevreuil à Dollon le 01/02 retarde 33 TGV (11min à 2h34). Par ailleurs, un dérangement des installations à Massy  le 11/02 touche 16 TGV (11 à 38min) \& on compte également deux incidents caténaires les 16 \& 18/02 : le premier sur la Ligne à Grande Vitesse (37 TGV de 16min à 1h41) \& le second à Massy-Verrières en région parisienne (22 TGV de 13min à 3h37). Enfin une défaillance de Matériel à Massy le 19/02 pénalise 16 TGV (15min \& 1h47).
2015-02; Nord; DUNKERQUE; PARIS NORD; 240; 240; 0; 11; 95.4; 
2015-02; Atlantique; LAVAL; PARIS MONTPARNASSE; 217; 217; 0; 24; 88.9; 
2015-02; Atlantique; LE MANS; PARIS MONTPARNASSE; 421; 420; 1; 77; 81.7; 
2015-02; Sud-Est; LYON PART DIEU; MONTPELLIER; 334; 333; 1; 69; 79.3; Les travaux de raccordement du réseau actuel à la nouvelle ligne à grande vitesse (qui contournera Nîmes et Montpellier à partir de 2017) ont généré des ralentissements en février
2015-02; Sud-Est; MONTPELLIER; LYON PART DIEU; 297; 297; 0; 46; 84.5; 
2015-02; Est; PARIS EST; REIMS; 194; 190; 4; 9; 95.3; 
2015-02; Sud-Est; PARIS LYON; CHAMBERY CHALLES LES EAUX; 218; 218; 0; 39; 82.1; 
2015-02; Sud-Est; PARIS LYON; PERPIGNAN; 144; 144; 0; 9; 93.8; 
2015-02; Atlantique; PARIS MONTPARNASSE; BREST; 215; 214; 1; 15; 93.0; Cette ligne a été touchée par un incident majeur le 13/02 : un problème caténaire à Laval retarde 83 TGV de 7min à 5h59. On compte également 3 incidents externes dont deux survenus sur la Ligne à Grande Vitesse : un accident de personne à Rouvray le 7/02 (19 TGV retardés de 30min à 4h04) \& un autre le 18/02 près de Paris (26 TGV retardés de 12 à 48min) ; le heurt d’un chevreuil à Dollon le 01/02 retarde 33 TGV (11min à 2h34). Par ailleurs, un dérangement des installations à Massy  le 11/02 touche 16 TGV (11 à 38min) \& on compte également deux incidents caténaires les 16 \& 18/02 : le premier sur la Ligne à Grande Vitesse (37 TGV de 16min à 1h41) \& le second à Massy-Verrières en région parisienne (22 TGV de 13min à 3h37). Enfin une défaillance de Matériel à Massy le 19/02 pénalise 16 TGV (15min \& 1h47).
2015-02; Atlantique; PARIS MONTPARNASSE; POITIERS; 464; 463; 1; 32; 93.1; 
2015-02; Atlantique; PARIS MONTPARNASSE; ST MALO; 92; 91; 1; 10; 89.0; Cette ligne a été touchée par un incident majeur le 13/02 : un problème caténaire à Laval retarde 83 TGV de 7min à 5h59. On compte également 3 incidents externes dont deux survenus sur la Ligne à Grande Vitesse : un accident de personne à Rouvray le 7/02 (19 TGV retardés de 30min à 4h04) \& un autre le 18/02 près de Paris (26 TGV retardés de 12 à 48min) ; le heurt d’un chevreuil à Dollon le 01/02 retarde 33 TGV (11min à 2h34). Par ailleurs, un dérangement des installations à Massy  le 11/02 touche 16 TGV (11 à 38min) \& on compte également deux incidents caténaires les 16 \& 18/02 : le premier sur la Ligne à Grande Vitesse (37 TGV de 16min à 1h41) \& le second à Massy-Verrières en région parisienne (22 TGV de 13min à 3h37). Enfin une défaillance de Matériel à Massy le 19/02 pénalise 16 TGV (15min \& 1h47).
2015-02; Atlantique; RENNES; LYON PART DIEU; 91; 91; 0; 17; 81.3; Cette ligne a été touchée par deux incidents majeurs : un problème caténaire à Laval le 13/02 (83 TGV de 7min à 5h59) \& une absence d’alimentation électrique à Lyon Part Dieu le 26-02 (72 TGV jusqu’à 1h58 de retard). On compte également 5 incidents externes dont quatre survenus sur les Lignes à Grande Vitesse : un accident de personne à Rouvray le 7/02 (19 TGV retardés de 30min à 4h04) \& un autre le 18/02 près de Paris (26 TGV retardés de 12 à 48min) ; le heurt d’un chevreuil à Dollon le 01/02 retarde 33 TGV (11min à 2h34) \& celui d’un sanglier vers Mâcon touche 27 TGV (6 à 59min) ; les limitations de vitesse sur la ligne à Grande Vitesse Sud-Est le 21-02 suite à la présence de neige retardent 81 TGV (5 à 27min). Par ailleurs, un dérangement des installations à Massy  le 11/02 touche 16 TGV (11 à 38min) \& on compte également trois incidents caténaires les 3, 16 \& 18/02 : l’un près de Lyon (Grenay) le 3/02 (32 TGV de 5min à 1h28), le suivant sur la Ligne à Grande Vitesse Atlantique (37 TGV de 16min à 1h41) \& le troisième à Massy-Verrières en région parisienne (22 TGV de 13min à 3h37). Enfin une défaillance de Matériel à Massy le 19/02 pénalise 16 TGV (15min \& 1h47). 
2015-02; Est; STRASBOURG; NANTES; 35; 35; 0; 4; 88.6; 
2015-02; Sud-Est; TOULON; PARIS LYON; 196; 196; 0; 29; 85.2; 
2015-02; Sud-Est; VALENCE ALIXAN TGV; PARIS LYON; 228; 228; 0; 21; 90.8; 
2015-03; Sud-Est; AIX EN PROVENCE TGV; PARIS LYON; 419; 419; 0; 15; 96.4; 
2015-03; Sud-Est; DIJON VILLE; PARIS LYON; 411; 411; 0; 12; 97.1; 
2015-03; Nord; LYON PART DIEU; LILLE; 275; 274; 1; 50; 81.8; 
2015-03; Sud-Est; MACON LOCHE; PARIS LYON; 183; 183; 0; 10; 94.5; 
2015-03; Sud-Est; MARSEILLE ST CHARLES; PARIS LYON; 398; 397; 1; 12; 97.0; 
2015-03; Est; NANTES; STRASBOURG; 40; 40; 0; 3; 92.5; 
2015-03; Est; PARIS EST; METZ; 288; 288; 0; 19; 93.4; 
2015-03; Est; PARIS EST; NANCY; 295; 295; 0; 14; 95.3; 
2015-03; Est; PARIS EST; REIMS; 217; 217; 0; 8; 96.3; 
2015-03; Est; PARIS EST; STRASBOURG; 443; 442; 1; 29; 93.4; 
2015-03; Sud-Est; PARIS LYON; BESANCON FRANCHE COMTE TGV; 199; 199; 0; 7; 96.5; 
2015-03; Sud-Est; PARIS LYON; MACON LOCHE; 197; 197; 0; 2; 99.0; 
2015-03; Sud-Est; PARIS LYON; MONTPELLIER; 288; 288; 0; 23; 92.0; 
2015-03; Sud-Est; PARIS LYON; TOULON; 213; 212; 1; 28; 86.8; 
2015-03; Atlantique; PARIS MONTPARNASSE; ANGERS SAINT LAUD; 445; 445; 0; 19; 95.7; 
2015-03; Atlantique; PARIS MONTPARNASSE; LA ROCHELLE VILLE; 222; 222; 0; 6; 97.3; Cette OD a été touchée par les événements suivants:  un heurt de chevreuil sur la ligne grande vitesse à Vendôme le 2 (28 TGV; de 13' à 58'), un accident de personne à Poitiers le 8 (6 TGV; de 2h06 à 3h28), un feu de traverses à l'entrée de Paris Montparnasse le 14 (6 TGV; 12' à 33') ainsi que la divagation d'un chien sur la ligne grande vitesse à St Arnoult le 31 (13 TGV; de 12' à 36'). On compte par ailleurs 4 dérangements d'installations sur la ligne grande vitesse les 1, 2, 16 et 23 touchant jusqu'à 7 trains par jour avec des retards allant de 12' à 1h05 ainsi qu'un autre dérangement au niveau de  Poitiers le 23 (6 TGV; 18' à 2h10).
2015-03; Atlantique; PARIS MONTPARNASSE; NANTES; 554; 554; 0; 30; 94.6; 
2015-03; Atlantique; PARIS MONTPARNASSE; QUIMPER; 173; 172; 1; 12; 93.0; Cette OD a été touchée par les événements suivants:  un heurt de chevreuil sur la ligne grande vitesse le 2 (28 TGV; de 13' à 58'), 2 accidents de personne à Vannes le 6 (6 TGV; de 13' à 1h51) et à Cesson le 26 (11 TGV; 12' à 2h24'), un feu de traverses à l'entrée de Paris Montparnasse le 14 (6 TGV; 12' à 33'), un incendie aux abords des voies vers Laval le 23  (7 TGV; 22' à 2h05) ainsi que la divagation d'un chien sur la ligne grande vitesse à St Arnoult le 31 (13 TGV; de 12' à 36'). On compte par ailleurs 4 dérangements d'installations sur la ligne grande vitesse les 1, 2, 16 et 23 touchant jusqu'à 7 trains par jour avec des retards allant de 12' à 1h05 et un incident caténaire le 22 vers Redon (9 TGV; de 12' à 80').
2015-03; Atlantique; PARIS MONTPARNASSE; TOULOUSE MATABIAU; 152; 152; 0; 7; 95.4; 
2015-03; Nord; PARIS NORD; LILLE; 628; 628; 0; 40; 93.6; 
2015-04; Sud-Est; ANNECY; PARIS LYON; 191; 190; 1; 18; 90.5; 
2015-04; Sud-Est; BESANCON FRANCHE COMTE TGV; PARIS LYON; 220; 220; 0; 7; 96.8; 
2015-04; Nord; MARSEILLE ST CHARLES; LILLE; 213; 213; 0; 66; 69.0; Les principaux évènements survenus au mois d'avril sont :
3 accidents de personne : Marne la Vallée et Choisy le Roi le 05 ainsi que Vergeze le 30
Plusieurs heurts d'animaux  : Allan le 08, Fresnoy le 14 et Sully le 21
Des colis suspects à Roissy les 01 et 30
Défauts d'alimentations : à Cesseins le 07, Vennissieux le 10 et à Aix le 17
Dérangement d'installations : Oignies le 03, Montanay le 08 et Sathonay le 24
Des travaux importants sur la LGV  entrainant des limitations de vitesse, essentiellement à Ressons et Lapalud
2015-04; Est; METZ; PARIS EST; 272; 272; 0; 93; 65.8; 
2015-04; Sud-Est; MULHOUSE VILLE; PARIS LYON; 309; 308; 1; 21; 93.2; 
2015-04; Atlantique; NANTES; PARIS MONTPARNASSE; 549; 549; 0; 41; 92.5; 
2015-04; Est; PARIS EST; METZ; 286; 286; 0; 24; 91.6; 
2015-04; Est; PARIS EST; STRASBOURG; 428; 428; 0; 28; 93.5; 
2015-04; Sud-Est; PARIS LYON; GRENOBLE; 222; 220; 2; 14; 93.6; 
2015-04; Sud-Est; PARIS LYON; LYON PART DIEU; 614; 613; 1; 39; 93.6; 
2015-04; Sud-Est; PARIS LYON; MULHOUSE VILLE; 295; 295; 0; 15; 94.9; 
2015-04; Sud-Est; PARIS LYON; NIMES; 344; 344; 0; 52; 84.9; 
2015-04; Sud-Est; PARIS LYON; TOULON; 277; 277; 0; 70; 74.7; Des travaux de modernisation de l'infrastructure ont perturbé la régularité de cette relation en Avril.
2015-04; Sud-Est; PARIS LYON; VALENCE ALIXAN TGV; 280; 279; 1; 33; 88.2; 
2015-04; Atlantique; PARIS MONTPARNASSE; LAVAL; 237; 237; 0; 6; 97.5; 
2015-04; Atlantique; RENNES; PARIS MONTPARNASSE; 570; 570; 0; 25; 95.6; 
2015-04; Sud-Est; SAINT ETIENNE CHATEAUCREUX; PARIS LYON; 110; 110; 0; 16; 85.5; 
2015-04; Est; STRASBOURG; PARIS EST; 465; 465; 0; 70; 84.9; 
2012-05; Atlantique; PARIS MONTPARNASSE; VANNES; 183; 183; 0; 21; 88.5; Travaux près de Connerré (72) et régénération des voies de la Ligne à Grande Vitesse Atlantique à Massy (banlieue sud de Paris), problème d'ouverture d'un passage à niveau pour le passage du train.
2012-05; Sud-Est; BELLEGARDE (AIN); PARIS LYON; 264; 263; 1; 22; 91.6; 
2012-06; Sud-Est; DIJON VILLE; PARIS LYON; 427; 427; 0; 35; 91.8; 
2012-06; Nord; PARIS NORD; DOUAI; 205; 205; 0; 32; 84.4; 
2012-06; Atlantique; PARIS MONTPARNASSE; LA ROCHELLE VILLE; 218; 218; 0; 24; 89.0; Quelques incidents ayant un fort impact sur la régularité des TGV.
2012-06; Atlantique; PARIS MONTPARNASSE; LE MANS; 442; 442; 0; 38; 91.4; 
2012-06; Atlantique; ANGERS SAINT LAUD; PARIS MONTPARNASSE; 467; 466; 1; 26; 94.4; 
2012-06; Nord; LILLE; LYON PART DIEU; 300; 300; 0; 37; 87.7; 
2012-06; Nord; LILLE; MARSEILLE ST CHARLES; 151; 151; 0; 32; 78.8; Liaison touchée par des vols de câbles et des accidents de personnes sur le Nord mais aussi par des dérangements d'installations ferroviaires sur le Sud-Est.
2012-06; Sud-Est; LYON PART DIEU; PARIS LYON; 621; 621; 0; 40; 93.6; 
2012-06; Est; METZ; PARIS EST; 290; 290; 0; 38; 86.9; 
2012-06; Sud-Est; MULHOUSE VILLE; PARIS LYON; 277; 277; 0; 25; 91.0; 
2012-06; Atlantique; NANTES; PARIS MONTPARNASSE; 571; 569; 2; 29; 94.9; 
2012-06; Atlantique; PARIS MONTPARNASSE; NANTES; 556; 556; 0; 27; 95.1; 
2012-06; Sud-Est; NIMES; PARIS LYON; 385; 385; 0; 48; 87.5; 
2012-06; Sud-Est; PARIS LYON; PERPIGNAN; 150; 150; 0; 13; 91.3; 
2012-06; Atlantique; PARIS MONTPARNASSE; RENNES; 550; 550; 0; 27; 95.1; 
2012-06; Atlantique; ST MALO; PARIS MONTPARNASSE; 98; 98; 0; 4; 95.9; 
2012-06; Atlantique; PARIS MONTPARNASSE; ST MALO; 56; 56; 0; 3; 94.6; 
2012-06; Est; STRASBOURG; PARIS EST; 464; 463; 1; 44; 90.5; 
2012-06; Atlantique; PARIS MONTPARNASSE; TOULOUSE MATABIAU; 134; 134; 0; 19; 85.8; Quelques incidents ayant un fort impact sur la régularité des TGV.
2012-06; Sud-Est; BELLEGARDE (AIN); PARIS LYON; 253; 253; 0; 20; 92.1; 
2015-01; Nord; DUNKERQUE; PARIS NORD; 252; 252; 0; 23; 90.9; 
2015-01; Nord; MARSEILLE ST CHARLES; LILLE; 210; 210; 0; 52; 75.2; Les principaux évènements survenus au mois de janvier sont :
Chutes de neige dans le sud-est le 20 
Accident de personne : Cesseins le 01 et Vennissieux le 03
Heurts d'animaux  : Chevry le 05, Neufmoutiers le 06, Cluny le 18 et Moussy et Lapalud le 29
Plusieurs colis suspects : à Marseille le 08, Marne la Vallée le 08, Roissy les 08,10 et 11 et à Lyon les 09 et 15
Un défaut d'alimentation à Fresnoy le 06
Obstacle dans la voie à Macon le 13
Des travaux importants sur la LGV Nord entrainant des limitations de vitesse
2015-01; Sud-Est; MONTPELLIER; PARIS LYON; 292; 292; 0; 25; 91.4; 
2015-01; Sud-Est; PARIS LYON; AIX EN PROVENCE TGV; 419; 414; 5; 32; 92.3; 5 collisions avec des animaux sauvages sur la ligne à grande vitesse ont perturbé la régularité de cette relation en janvier 2015
2015-01; Sud-Est; PARIS LYON; AVIGNON TGV; 483; 480; 3; 47; 90.2; 
2015-01; Sud-Est; PARIS LYON; LE CREUSOT MONTCEAU MONTCHANIN; 195; 195; 0; 24; 87.7; 
2015-01; Sud-Est; PARIS LYON; LYON PART DIEU; 603; 601; 2; 38; 93.7; 
2015-01; Sud-Est; PARIS LYON; NIMES; 320; 319; 1; 34; 89.3; 
2015-01; Atlantique; PARIS MONTPARNASSE; LA ROCHELLE VILLE; 210; 210; 0; 12; 94.3; 
2015-01; Atlantique; PARIS MONTPARNASSE; POITIERS; 482; 482; 0; 24; 95.0; 
2015-01; Atlantique; PARIS MONTPARNASSE; TOULOUSE MATABIAU; 143; 142; 1; 20; 85.9; 
2015-01; Atlantique; PARIS MONTPARNASSE; VANNES; 177; 176; 1; 9; 94.9; 
2015-01; Atlantique; ST MALO; PARIS MONTPARNASSE; 89; 89; 0; 4; 95.5; 
2015-01; Est; STRASBOURG; NANTES; 33; 33; 0; 3; 90.9; 
2015-01; Sud-Est; TOULON; PARIS LYON; 205; 201; 4; 23; 88.6; 
2015-01; Atlantique; TOULOUSE MATABIAU; PARIS MONTPARNASSE; 168; 166; 2; 17; 89.8; La ligne a été impactée par des incidents externes dont deux accidents de personne à VIVONNE le 04/01 (3 TGV retardés de 1h34 à 3h40) et à CADAUJAC le 13/01 (3 TGV retardés de 13’ à 5h08) – ainsi que la divagation d’un veau le 07/01 entre ANGOULÊME et POITIERS (5 TGV de 24’ à 1h04) et le signalement de colis suspects le 14/01 et le 16/01 en gare de BORDEAUX (retardant respectivement 6 TGV de 18’ à 1h33 et 6 TGV de 17’ à 1h31). Un dérangement des installations au nord de CHATELLERAULT le 15/01 a également entraîné des retards (24 TGV de 12’ à 1h50) ainsi que la panne d’un TGV sur la Ligne Garnde Vitesse le 23/01 à DANGEAU (41 TGV retardés de 11’ à 3h18) et la panne d’un TER le 28/01 au sud de POITIERS (10 TGV retardés de 18’ à 3h23).
2015-02; Atlantique; ANGERS SAINT LAUD; PARIS MONTPARNASSE; 426; 425; 1; 60; 85.9; 
2015-02; Sud-Est; LE CREUSOT MONTCEAU MONTCHANIN; PARIS LYON; 197; 197; 0; 17; 91.4; 
2015-02; Est; NANTES; STRASBOURG; 36; 36; 0; 2; 94.4; 
2015-02; Est; PARIS EST; STRASBOURG; 400; 393; 7; 33; 91.6; 
2015-02; Sud-Est; PARIS LYON; BESANCON FRANCHE COMTE TGV; 199; 199; 0; 9; 95.5; 
2015-02; Sud-Est; PARIS LYON; LYON PART DIEU; 574; 573; 1; 24; 95.8; 
2015-02; Sud-Est; PARIS LYON; MONTPELLIER; 294; 293; 1; 19; 93.5; 
2015-02; Atlantique; PARIS MONTPARNASSE; ANGERS SAINT LAUD; 403; 403; 0; 36; 91.1; 
2015-02; Atlantique; PARIS MONTPARNASSE; BORDEAUX ST JEAN; 577; 576; 1; 44; 92.4; 
2015-02; Nord; PARIS NORD; DUNKERQUE; 239; 239; 0; 16; 93.3; 
2015-02; Nord; PARIS NORD; ARRAS; 280; 280; 0; 25; 91.1; 
2015-02; Sud-Est; PERPIGNAN; PARIS LYON; 114; 114; 0; 12; 89.5; 
2015-02; Est; REIMS; PARIS EST; 191; 188; 3; 7; 96.3; 
2015-03; Sud-Est; ANNECY; PARIS LYON; 181; 181; 0; 6; 96.7; 
2015-03; Sud-Est; AVIGNON TGV; PARIS LYON; 508; 506; 2; 47; 90.7; 
2015-03; Nord; DOUAI; PARIS NORD; 209; 208; 1; 24; 88.5; 
2015-03; Nord; DUNKERQUE; PARIS NORD; 265; 265; 0; 15; 94.3; 
2015-03; Sud-Est; LYON PART DIEU; PARIS LYON; 594; 594; 0; 21; 96.5; 
2015-03; Sud-Est; MONTPELLIER; LYON PART DIEU; 332; 332; 0; 70; 78.9; Les travaux de raccordement du réseau actuel à la nouvelle ligne à grande vitesse (qui contournera Nîmes et Montpellier à partir de 2017), ainsi que des travaux sur ligne classique près de Montpellier ont généré des ralentissements en mars
2015-03; Sud-Est; PARIS LYON; AIX EN PROVENCE TGV; 395; 395; 0; 17; 95.7; 
2015-03; Sud-Est; PARIS LYON; DIJON VILLE; 427; 427; 0; 9; 97.9; 
2015-03; Sud-Est; PARIS LYON; LE CREUSOT MONTCEAU MONTCHANIN; 186; 186; 0; 4; 97.8; 
2015-03; Sud-Est; PARIS LYON; LYON PART DIEU; 573; 573; 0; 10; 98.3; 
2015-03; Sud-Est; PARIS LYON; PERPIGNAN; 140; 140; 0; 14; 90.0; 
2015-03; Sud-Est; PARIS LYON; SAINT ETIENNE CHATEAUCREUX; 104; 104; 0; 6; 94.2; 
2015-03; Atlantique; PARIS MONTPARNASSE; ANGOULEME; 329; 329; 0; 16; 95.1; 
2015-03; Atlantique; PARIS MONTPARNASSE; BREST; 227; 227; 0; 5; 97.8; 
2015-03; Atlantique; PARIS MONTPARNASSE; TOURS; 191; 191; 0; 15; 92.1; 
2015-03; Atlantique; PARIS MONTPARNASSE; VANNES; 190; 190; 0; 16; 91.6; Cette OD a été touchée par les événements suivants:  un heurt de chevreuil sur la ligne grande vitesse le 2 (28 TGV; de 13' à 58'), 2 accidents de personne à Vannes le 6 (6 TGV; de 13' à 1h51) et à Cesson le 26 (11 TGV; 12' à 2h24'), un feu de traverses à l'entrée de Paris Montparnasse le 14 (6 TGV; 12' à 33'), un incendie aux abords des voies vers Laval le 23  (7 TGV; 22' à 2h05) ainsi que la divagation d'un chien sur la ligne grande vitesse à St Arnoult le 31 (13 TGV; de 12' à 36'). On compte par ailleurs 4 dérangements d'installations sur la ligne grande vitesse les 1, 2, 16 et 23 touchant jusqu'à 7 trains par jour avec des retards allant de 12' à 1h05 et un incident caténaire le 22 vers Redon (9 TGV; de 12' à 80').
2015-03; Sud-Est; PERPIGNAN; PARIS LYON; 114; 114; 0; 14; 87.7; 
2015-03; Atlantique; ST MALO; PARIS MONTPARNASSE; 102; 102; 0; 5; 95.1; 
2015-03; Sud-Est; TOULON; PARIS LYON; 192; 190; 2; 29; 84.7; 
2015-04; Nord; DOUAI; PARIS NORD; 198; 198; 0; 27; 86.4; 
2015-04; Nord; DUNKERQUE; PARIS NORD; 255; 255; 0; 17; 93.3; 
2015-04; Atlantique; LAVAL; PARIS MONTPARNASSE; 234; 234; 0; 7; 97.0; 
2015-04; Atlantique; LE MANS; PARIS MONTPARNASSE; 455; 455; 0; 39; 91.4; 
2015-04; Nord; LILLE; PARIS NORD; 580; 579; 1; 50; 91.4; 
2015-04; Nord; LYON PART DIEU; LILLE; 273; 273; 0; 89; 67.4; Les principaux évènements survenus au mois d'avril sont :
3 accidents de personne : Marne la Vallée et Choisy le Roi le 05 ainsi que Vergeze le 30
Plusieurs heurts d'animaux  : Allan le 08, Fresnoy le 14 et Sully le 21
Des colis suspects à Roissy les 01 et 30
Défauts d'alimentations : à Cesseins le 07, Vennissieux le 10 et à Aix le 17
Dérangement d'installations : Oignies le 03, Montanay le 08 et Sathonay le 24
Des travaux importants sur la LGV  entrainant des limitations de vitesse, essentiellement à Ressons et Lapalud
2015-04; Sud-Est; MACON LOCHE; PARIS LYON; 198; 198; 0; 20; 89.9; 
2015-04; Est; NANTES; STRASBOURG; 55; 55; 0; 4; 92.7; 
2015-04; Sud-Est; PARIS LYON; BESANCON FRANCHE COMTE TGV; 212; 212; 0; 12; 94.3; 
2015-04; Sud-Est; PARIS LYON; MARSEILLE ST CHARLES; 446; 446; 0; 44; 90.1; 
2015-04; Sud-Est; PARIS LYON; PERPIGNAN; 163; 162; 1; 24; 85.2; 3 heurts d'animaux sur ligne à grande vitesse et 2 accidents de personne ont généré des retards importants en Avril 
2015-04; Atlantique; PARIS MONTPARNASSE; ANGERS SAINT LAUD; 430; 430; 0; 13; 97.0; 
2015-04; Atlantique; PARIS MONTPARNASSE; BORDEAUX ST JEAN; 632; 632; 0; 42; 93.4; 
2015-04; Atlantique; PARIS MONTPARNASSE; LA ROCHELLE VILLE; 216; 215; 1; 10; 95.3; 
2015-04; Atlantique; PARIS MONTPARNASSE; POITIERS; 503; 503; 0; 20; 96.0; 
2015-04; Atlantique; PARIS MONTPARNASSE; QUIMPER; 200; 200; 0; 8; 96.0; 
2015-04; Atlantique; PARIS MONTPARNASSE; RENNES; 552; 552; 0; 21; 96.2; 
2015-04; Atlantique; PARIS MONTPARNASSE; ST PIERRE DES CORPS; 447; 447; 0; 33; 92.6; 
2015-04; Nord; PARIS NORD; ARRAS; 302; 302; 0; 20; 93.4; 
2015-04; Atlantique; POITIERS; PARIS MONTPARNASSE; 488; 487; 1; 31; 93.6; 
2015-04; Atlantique; QUIMPER; PARIS MONTPARNASSE; 221; 221; 0; 5; 97.7; 
2015-04; Atlantique; ST PIERRE DES CORPS; PARIS MONTPARNASSE; 428; 428; 0; 45; 89.5; 
2015-04; Atlantique; TOURS; PARIS MONTPARNASSE; 199; 199; 0; 21; 89.4; 
2015-04; Atlantique; VANNES; PARIS MONTPARNASSE; 245; 245; 0; 5; 98.0; 
2015-05; Atlantique; ANGERS SAINT LAUD; PARIS MONTPARNASSE; 455; 454; 1; 27; 94.1; 
2015-05; Sud-Est; GRENOBLE; PARIS LYON; 211; 211; 0; 7; 96.7; 
2015-05; Est; METZ; PARIS EST; 273; 273; 0; 59; 78.4; 
2015-05; Sud-Est; MONTPELLIER; PARIS LYON; 349; 348; 1; 52; 85.1; 
2015-05; Sud-Est; PARIS LYON; ANNECY; 138; 100; 38; 15; 85.0; 
2015-05; Sud-Est; PARIS LYON; BESANCON FRANCHE COMTE TGV; 201; 201; 0; 6; 97.0; 
2015-02; Atlantique; BORDEAUX ST JEAN; PARIS MONTPARNASSE; 619; 618; 1; 53; 91.4; Cette destination a été touchée par 4 incidents externes dont deux survenus sur la Ligne à Grande Vitesse : un accident de personne à Rouvray le 7/02 (19 TGV retardés de 30min à 4h04) \& un autre le 18/02 près de Paris (26 TGV retardés de 12 à 48min) ; le heurt d’un chevreuil à Dollon le 01/02 retarde 33 TGV (11min à 2h34), celui d’un sanglier le 20/02 vers Angoulême touche 5 TGV (14min à 4h) . Par ailleurs, un dérangement lié à une autre entreprise ferroviaire de transport de marchandises le 4/02 au Nord de Poitiers ralentit 6 TGV (13 à 54min), un autre à Massy  le 11/02 touche 16 TGV (11 à 38min) \& on compte également deux incidents caténaires les 16 \& 18/02 : le premier sur la Ligne à Grande Vitesse (37 TGV de 16min à 1h41) \& le second à Massy-Verrières en région parisienne (22 TGV de 13min à 3h37). Enfin une défaillance de Matériel à Massy le 19/02 pénalise 16 TGV (15min \& 1h47).
2015-02; Nord; DOUAI; PARIS NORD; 190; 190; 0; 19; 90.0; 
2015-02; Atlantique; LA ROCHELLE VILLE; PARIS MONTPARNASSE; 205; 204; 1; 12; 94.1; Cette destination a été touchée par 5 incidents externes dont deux survenus sur la Ligne à Grande Vitesse : un accident de personne à Rouvray le 7/02 (19 TGV retardés de 30min à 4h04) \& un autre le 18/02 près de Paris (26 TGV retardés de 12 à 48min), la présence d’une personne suicidaire à Mauzé le 4/02 touche 2 TGV (55min \& 1h21) \& des jets de pierres avec une personne blessée à proximité des voies vers Poitiers le 9/02 (2 TGV 34min \& 1h36) ; le heurt d’un chevreuil à Dollon le 1/02 retarde également 33 TGV (11min à 2h34). Par ailleurs, un dérangement lié à une autre entreprise ferroviaire de transport de marchandises le 4/02 au Nord de Poitiers ralentit 6 TGV (13 à 54min), un à Massy le 11/02 touche 16 TGV (11 à 38min), un autre à Surgères le 21/02 (2 TGV de 23 à 42min) \& un défaut de la voie près de St Maixent le 19/02 (6 TGV 12 à 27min). On compte également deux incidents caténaires les 16 \& 18/02 : le premier sur la Ligne à Grande Vitesse (37 TGV de 16min à 1h41) \& le second à Massy-Verrières en région parisienne (22 TGV de 13min à 3h37). Enfin une défaillance de Matériel à Massy le 19/02 pénalise 16 TGV (15min \& 1h47).
2015-02; Nord; LILLE; MARSEILLE ST CHARLES; 192; 192; 0; 36; 81.3; 
2015-02; Sud-Est; MARSEILLE ST CHARLES; LYON PART DIEU; 498; 498; 0; 90; 81.9; 
2015-02; Est; METZ; PARIS EST; 266; 261; 5; 29; 88.9; 
2015-02; Sud-Est; MONTPELLIER; PARIS LYON; 269; 269; 0; 21; 92.2; 
2015-02; Sud-Est; MULHOUSE VILLE; PARIS LYON; 290; 290; 0; 15; 94.8; 
2015-02; Atlantique; NANTES; PARIS MONTPARNASSE; 506; 505; 1; 94; 81.4; 
2015-02; Sud-Est; PARIS LYON; LE CREUSOT MONTCEAU MONTCHANIN; 187; 187; 0; 7; 96.3; 
2015-02; Sud-Est; PARIS LYON; MULHOUSE VILLE; 276; 276; 0; 15; 94.6; 
2015-02; Sud-Est; PARIS LYON; SAINT ETIENNE CHATEAUCREUX; 101; 101; 0; 17; 83.2; 
2015-02; Atlantique; PARIS MONTPARNASSE; LAVAL; 208; 208; 0; 18; 91.3; 
2015-02; Atlantique; PARIS MONTPARNASSE; QUIMPER; 141; 141; 0; 12; 91.5; Cette ligne a été touchée par un incident majeur le 13/02 : un problème caténaire à Laval retarde 83 TGV de 7min à 5h59. On compte également 6 incidents externes dont deux survenus sur la Ligne à Grande Vitesse : un accident de personne à Rouvray le 7/02 (19 de 30min à 4h04), deux le 14/02 à Questembert (3 TGV 16min \& 1h50) et à Auray (1 TGV à 1h05), un autre le 18/02 près de Paris (26 TGV retardés de 12 à 48min) \& à Quimperlé le 27/02 (1 TGV à 1h46) ; le heurt d’un chevreuil à Dollon le 01/02 retarde 33 TGV (11min à 2h34). Par ailleurs, un dérangement des installations à Massy  le 11/02 touche 16 TGV (11 à 38min) \& on compte également deux incidents caténaires les 16 \& 18/02 : le premier sur la Ligne à Grande Vitesse (37 TGV de 16min à 1h41) \& le second à Massy-Verrières en région parisienne (22 TGV de 13min à 3h37). Enfin une défaillance de Matériel à Massy le 19/02 pénalise 16 TGV (15min \& 1h47).
2015-02; Atlantique; PARIS MONTPARNASSE; RENNES; 504; 503; 1; 54; 89.3; Cette ligne a été touchée par un incident majeur le 13/02 : un problème caténaire à Laval retarde 83 TGV de 7min à 5h59. On compte également 3 incidents externes dont deux survenus sur la Ligne à Grande Vitesse : un accident de personne à Rouvray le 7/02 (19 TGV retardés de 30min à 4h04) \& un autre le 18/02 près de Paris (26 TGV retardés de 12 à 48min) ; le heurt d’un chevreuil à Dollon le 01/02 retarde 33 TGV (11min à 2h34). Par ailleurs, un dérangement des installations à Massy  le 11/02 touche 16 TGV (11 à 38min) \& on compte également deux incidents caténaires les 16 \& 18/02 : le premier sur la Ligne à Grande Vitesse (37 TGV de 16min à 1h41) \& le second à Massy-Verrières en région parisienne (22 TGV de 13min à 3h37). Enfin une défaillance de Matériel à Massy le 19/02 pénalise 16 TGV (15min \& 1h47).
2015-02; Atlantique; RENNES; PARIS MONTPARNASSE; 524; 524; 0; 64; 87.8; Cette ligne a été touchée par un incident majeur le 13/02 : un problème caténaire à Laval retarde 83 TGV de 7min à 5h59. On compte également 3 incidents externes dont deux survenus sur la Ligne à Grande Vitesse : un accident de personne à Rouvray le 7/02 (19 TGV retardés de 30min à 4h04) \& un autre le 18/02 près de Paris (26 TGV retardés de 12 à 48min) ; le heurt d’un chevreuil à Dollon le 01/02 retarde 33 TGV (11min à 2h34). Par ailleurs, un dérangement des installations à Massy  le 11/02 touche 16 TGV (11 à 38min) \& on compte également deux incidents caténaires les 16 \& 18/02 : le premier sur la Ligne à Grande Vitesse (37 TGV de 16min à 1h41) \& le second à Massy-Verrières en région parisienne (22 TGV de 13min à 3h37). Enfin une défaillance de Matériel à Massy le 19/02 pénalise 16 TGV (15min \& 1h47).
2015-02; Atlantique; TOURS; PARIS MONTPARNASSE; 188; 188; 0; 27; 85.6; 
2015-03; Atlantique; ANGERS SAINT LAUD; PARIS MONTPARNASSE; 472; 472; 0; 28; 94.1; 
2015-03; Nord; ARRAS; PARIS NORD; 343; 343; 0; 36; 89.5; 
2015-03; Sud-Est; BELLEGARDE (AIN); PARIS LYON; 229; 229; 0; 6; 97.4; 
2015-03; Sud-Est; GRENOBLE; PARIS LYON; 210; 210; 0; 3; 98.6; 
2015-03; Nord; LILLE; LYON PART DIEU; 277; 276; 1; 15; 94.6; 
2015-03; Nord; LILLE; MARSEILLE ST CHARLES; 209; 209; 0; 31; 85.2; 
2015-03; Est; METZ; PARIS EST; 294; 292; 2; 50; 82.9; 
2015-03; Est; NANCY; PARIS EST; 294; 294; 0; 15; 94.9; 
2015-03; Sud-Est; PARIS LYON; ANNECY; 140; 140; 0; 9; 93.6; 
2015-03; Sud-Est; PARIS LYON; AVIGNON TGV; 454; 454; 0; 23; 94.9; 
2015-03; Sud-Est; PARIS LYON; CHAMBERY CHALLES LES EAUX; 219; 219; 0; 18; 91.8; 
2015-03; Sud-Est; PARIS LYON; GRENOBLE; 215; 214; 1; 4; 98.1; 
2015-03; Sud-Est; PARIS LYON; MULHOUSE VILLE; 278; 278; 0; 7; 97.5; 
2015-03; Sud-Est; PARIS LYON; VALENCE ALIXAN TGV; 263; 262; 1; 12; 95.4; 
2015-03; Atlantique; PARIS MONTPARNASSE; BORDEAUX ST JEAN; 645; 645; 0; 27; 95.8; 
2015-03; Atlantique; PARIS MONTPARNASSE; POITIERS; 509; 509; 0; 16; 96.9; 
2015-03; Nord; PARIS NORD; ARRAS; 314; 313; 1; 17; 94.6; 
2015-03; Est; REIMS; PARIS EST; 213; 213; 0; 10; 95.3; 
2015-03; Atlantique; RENNES; LYON PART DIEU; 102; 101; 1; 8; 92.1; Cette OD a été touchée par les événements suivants:  un heurt de chevreuil sur la ligne grande vitesse à Vendôme le 2 (28 TGV; de 13' à 58'), un accident de personne à Piolenc dans le Sud-Est le 28  (3 TGV; 31' à 92') ainsi que la divagation d'un chien sur la ligne grande vitesse à St Arnoult le 31 (13 TGV; de 12' à 36'). On compte par ailleurs 4 dérangements d'installations sur la ligne grande vitesse les 1, 2, 16 et 23 touchant jusqu'à 7 trains par jour avec des retards allant de 12' à 1h05 ainsi qu'un incident d'origine électrique sur la région parisienne (6 TGV; 8' à 15').
2015-03; Atlantique; RENNES; PARIS MONTPARNASSE; 576; 576; 0; 35; 93.9; 
2015-03; Est; STRASBOURG; PARIS EST; 476; 474; 2; 37; 92.2; 
2015-03; Atlantique; TOULOUSE MATABIAU; PARIS MONTPARNASSE; 181; 181; 0; 17; 90.6; Cette OD a été touchée par les événements suivants:  un heurt de chevreuil sur la ligne grande vitesse à Vendôme le 2 (28 TGV; de 13' à 58'), un accident de personne à Poitiers le 8 (6 TGV; de 2h06 à 3h28), un feu de traverses à l'entrée de Paris Montparnasse le 14 (6 TGV; 12' à 33') ainsi que la divagation d'un chien sur la ligne grande vitesse à St Arnoult le 31 (13 TGV; de 12' à 36'). On compte par ailleurs 4 dérangements d'installations sur la ligne grande vitesse les 1, 2, 16 et 23 touchant de jusqu'à 7 trains par jour avec des retards allant de 12' à 1h05 ainsi qu'un autre dérangement au niveau de  Poitiers le 23 (6 TGV; 18' à 2h10). On signalera également 4 actes de malvaillances (jets, intrusions) entre Toulouse et Bordeaux.
2015-03; Atlantique; TOURS; PARIS MONTPARNASSE; 206; 206; 0; 15; 92.7; 
2015-03; Sud-Est; VALENCE ALIXAN TGV; PARIS LYON; 227; 227; 0; 35; 84.6; 
2015-04; Atlantique; ANGERS SAINT LAUD; PARIS MONTPARNASSE; 453; 453; 0; 28; 93.8; 
2015-04; Nord; ARRAS; PARIS NORD; 331; 331; 0; 36; 89.1; 
2015-04; Sud-Est; AVIGNON TGV; PARIS LYON; 578; 576; 2; 131; 77.3; Des travaux de modernisation de l'infrastructure ont perturbé la régularité de cette relation en Avril.
2015-04; Sud-Est; BELLEGARDE (AIN); PARIS LYON; 246; 246; 0; 23; 90.7; 
2015-04; Atlantique; BORDEAUX ST JEAN; PARIS MONTPARNASSE; 663; 661; 2; 47; 92.9; Cette ligne a été touchée notamment par des causes externes: un accident de personne le 05/04 en région parisienne (7 TGV retardés pour 333min) \& à Vayres près de Bordeaux le 13/04 (14 TGV impactés pour 1043min), mais également un colis suspect en gare de Paris Montparnasse le 08/04 (8 TGV touchés pour 119min) \& des jets de projectiles le 21/04 près de Libourne (6 TGV retardés pour 163min). Des dérangements des installations ont également entraîné des retards notamment sur la ligne à grande vitesse les 07/04 (20 TGV retardés pour 324min), 17/04 (14 TGV pour 393min) \& 19/04 (8  circulations impactées pour 174min). Par ailleurs, le heurt d'une charge déplacée sur un train de marchandises vers Angoulême le 08/04 a touché 11 TGV pour 819min.
2015-04; Atlantique; LA ROCHELLE VILLE; PARIS MONTPARNASSE; 215; 214; 1; 8; 96.3; 
2015-04; Sud-Est; LE CREUSOT MONTCEAU MONTCHANIN; PARIS LYON; 214; 214; 0; 31; 85.5; 
2015-04; Nord; LILLE; MARSEILLE ST CHARLES; 207; 207; 0; 42; 79.7; Les principaux évènements survenus au mois d'avril sont :
2 accidents de personne : Marne la Vallée et Choisy le Roi le 05
Plusieurs heurts d'animaux  : Allan le 08, Fresnoy le 14 et Sully le 21
Des colis suspects à Roissy les 01 et 30
Défauts d'alimentations : à Cesseins le 07, Vennissieux le 10 et à Aix le 17
Dérangement d'installations : Oignies le 03, Montanay le 08 et Sathonay le 24
Des travaux importants sur la LGV  entrainant des limitations de vitesse, essentiellement à Ressons et Lapalud
2015-04; Sud-Est; LYON PART DIEU; MONTPELLIER; 360; 360; 0; 80; 77.8; Des travaux de modernisation de l'infrastructure ont perturbé la régularité de cette relation en Avril.
2015-04; Sud-Est; MARSEILLE ST CHARLES; PARIS LYON; 437; 437; 0; 48; 89.0; 4 heurts d'animaux sur ligne à grande vitesse et un accident de personne au Vert de Maisons ont généré des retards importants en Avril 
2015-04; Est; NANCY; PARIS EST; 282; 281; 1; 13; 95.4; 
2015-04; Sud-Est; NICE VILLE; PARIS LYON; 213; 211; 2; 66; 68.7; Des travaux de modernisation de l'infrastructure ont perturbé la régularité de cette relation en Avril.
2015-04; Sud-Est; PARIS LYON; CHAMBERY CHALLES LES EAUX; 211; 211; 0; 15; 92.9; 
2015-04; Sud-Est; PARIS LYON; DIJON VILLE; 466; 466; 0; 15; 96.8; 
2015-04; Sud-Est; PARIS LYON; LE CREUSOT MONTCEAU MONTCHANIN; 199; 199; 0; 22; 88.9; 
2015-04; Sud-Est; PARIS LYON; MACON LOCHE; 209; 209; 0; 7; 96.7; 
2015-04; Sud-Est; PARIS LYON; MONTPELLIER; 342; 341; 1; 47; 86.2; 3 heurts d'animaux sur ligne à grande vitesse et 2 accidents de personne ont généré des retards importants en Avril 
2015-04; Sud-Est; PARIS LYON; SAINT ETIENNE CHATEAUCREUX; 111; 111; 0; 10; 91.0; 
2015-04; Atlantique; PARIS MONTPARNASSE; BREST; 228; 228; 0; 6; 97.4; Cette ligne a été touchée notamment par des causes externes: un accident  de personne à Cesson-Sévigné près de Rennes le 04-04 (7 TGV retardés pour 583min), en région parisienne le 05/04  (7 TGV retardés pour 333min) mais également un colis suspect en gare de Paris Montparnasse le 08/04 (8 TGV touchés pour 119min), un incendie aux abords des voies à Port Brillet (Laval) le 13/04 (3 TGV retardés pour 77min) \& le heurt d'une biche à Voutré - entre Le Mans \& Laval - le 23/04 avec 5 TGV retardés pour 151min.  Des dérangements des installations ont également entraîné des retards notamment sur l'ensemble de la Ligne à Grande Vitesse le 07-04 (12 circulations impactées pour 150min),  à Laval le 28/04 (9 TGV retardés pour 302min) \& à Rennes le 29/04 (16 TGV impactés pour 314min). Par ailleurs, la panne d'un train assurant les travaux à St Pierre-la-Cour (entre Laval \& Rennes) le 09-04 a retardé 9 TGV pour 361min.
2015-04; Atlantique; PARIS MONTPARNASSE; TOULOUSE MATABIAU; 148; 148; 0; 14; 90.5; 
2015-04; Est; REIMS; PARIS EST; 204; 204; 0; 8; 96.1; 
2015-04; Atlantique; RENNES; LYON PART DIEU; 100; 100; 0; 10; 90.0; 
2015-04; Atlantique; ST MALO; PARIS MONTPARNASSE; 98; 98; 0; 2; 98.0; Cette ligne a été touchée notamment par des causes externes: un accident de personne Cesson-Sévigné - près de Rennes - le 04-04 (7 TGV retardés pour 583min), en région parisienne le 05/04  (7 TGV retardés pour 333min)  mais également un colis suspect en gare de Paris Montparnasse le 08/04 (8 TGV touchés pour 119min), un incendie aux abords des voies à Port Brillet - Laval - le 13/04 (3 TGV retardés pour 77min) \& le heurt d'une biche à Voutré - entre Le Mans \& Laval - le 23/04 avec 5 TGV retardés pour 151min.  Des dérangements des installations ont également entraîné des retards notamment  sur l'ensemble de la Ligne à Grande Vitesse le 07-04 (12 circulations impactées pour 150min), à Laval le 28/04 (9 TGV retardés pour 302min) \& à Rennes le 29/04 (16 TGV impactés pour 314min). Par ailleurs, la panne d'un train assurant les travaux à St Pierre-la-Cour - entre Laval \& Rennes- le 09-04 a retardé9 TGV pour 361min.
2015-04; Est; STRASBOURG; NANTES; 55; 55; 0; 1; 98.2; 
2015-04; Sud-Est; VALENCE ALIXAN TGV; PARIS LYON; 255; 253; 2; 53; 79.1; Des travaux de modernisation de l'infrastructure ont perturbé la régularité de cette relation en Avril.
2015-05; Sud-Est; AIX EN PROVENCE TGV; PARIS LYON; 457; 457; 0; 57; 87.5; 
2015-05; Sud-Est; ANNECY; PARIS LYON; 157; 120; 37; 12; 90.0; 
2015-05; Nord; ARRAS; PARIS NORD; 328; 328; 0; 37; 88.7; 
2015-05; Sud-Est; BELLEGARDE (AIN); PARIS LYON; 233; 233; 0; 23; 90.1; 
2015-05; Nord; DOUAI; PARIS NORD; 202; 202; 0; 26; 87.1; 
2015-05; Sud-Est; LE CREUSOT MONTCEAU MONTCHANIN; PARIS LYON; 213; 213; 0; 41; 80.8; 
2015-05; Sud-Est; MACON LOCHE; PARIS LYON; 186; 186; 0; 21; 88.7; 
2015-05; Nord; MARSEILLE ST CHARLES; LILLE; 219; 218; 1; 58; 73.4; Toujours des travaux importants sur la LGV  entrainant des limitations de vitesse, essentiellement à Lapalud. Autres événements :
03/05 : dérangement d'aiguille lors de travaux au Creusot
05 et 28/05 : panne d'une rame sur LGV Nord
08/05 : accident de personne à Lapalud
20/05 : acte de malveillance (jets de pierres) à Vémars
25/05 : occupation des voies à l'escale de Lille Europe
2015-05; Sud-Est; MONTPELLIER; LYON PART DIEU; 331; 330; 1; 74; 77.6; 
2015-05; Sud-Est; MULHOUSE VILLE; PARIS LYON; 307; 307; 0; 13; 95.8; 
2015-05; Est; PARIS EST; REIMS; 208; 208; 0; 5; 97.6; 
2015-05; Sud-Est; PARIS LYON; BELLEGARDE (AIN); 230; 230; 0; 28; 87.8; 
2015-05; Sud-Est; PARIS LYON; NICE VILLE; 230; 230; 0; 44; 80.9; 
2015-05; Sud-Est; PARIS LYON; NIMES; 343; 343; 0; 51; 85.1; 
2015-05; Sud-Est; PARIS LYON; PERPIGNAN; 167; 167; 0; 20; 88.0; 
2015-05; Sud-Est; PARIS LYON; SAINT ETIENNE CHATEAUCREUX; 114; 114; 0; 6; 94.7; 
2015-05; Atlantique; PARIS MONTPARNASSE; NANTES; 537; 536; 1; 37; 93.1; 
2015-05; Atlantique; PARIS MONTPARNASSE; RENNES; 544; 544; 0; 29; 94.7; 
2015-05; Atlantique; PARIS MONTPARNASSE; TOURS; 178; 178; 0; 14; 92.1; 
2015-05; Atlantique; PARIS MONTPARNASSE; VANNES; 207; 207; 0; 13; 93.7; 
2015-05; Atlantique; POITIERS; PARIS MONTPARNASSE; 490; 489; 1; 18; 96.3; 
2015-05; Est; REIMS; PARIS EST; 202; 202; 0; 7; 96.5; 
2015-05; Sud-Est; TOULON; PARIS LYON; 266; 266; 0; 51; 80.8; 
2015-05; Sud-Est; VALENCE ALIXAN TGV; PARIS LYON; 249; 249; 0; 53; 78.7; 
2015-06; Atlantique; BREST; PARIS MONTPARNASSE; 227; 226; 1; 20; 91.2; 
2015-06; Sud-Est; CHAMBERY CHALLES LES EAUX; PARIS LYON; 188; 188; 0; 21; 88.8; 
2015-06; Nord; DOUAI; PARIS NORD; 204; 204; 0; 33; 83.8; 
2015-06; Sud-Est; GRENOBLE; PARIS LYON; 228; 228; 0; 9; 96.1; 
2015-06; Atlantique; LA ROCHELLE VILLE; PARIS MONTPARNASSE; 208; 208; 0; 24; 88.5; Cette OD a été touchée le 4 par un rail cassé couplé à un affaissement caténaire à la sortie de Paris (172 TGV; 7664mn), le 5 par diverses limitations de vitesses liées aux fortes chaleurs dont une à la sortie de Paris reduisant fortement le débit des trains (82TGV; 985mn), 2 incendies aux abords des voies entre Droué et Vendôme le 25 (48TGV; 1826mn) et le 30 (30TGV; 1773mn), le 3 par l'arrêt de sécurité d'un train dans les tunnels avant Paris (22TGV; 1035mn) lié à une presence dans les voies en sortie de gare Montparnasse (16TGV; 419mn),  le 22 par le heurt de 2 chevreuils sur la branche sud de la ligne grande vitesse (15TGV; 328mn), le 4 par un dérangement d'installation à Lusignan (15 TGV; 654mn)
2015-06; Nord; LILLE; LYON PART DIEU; 268; 267; 1; 33; 87.6; 
2015-06; Sud-Est; MONTPELLIER; LYON PART DIEU; 325; 321; 4; 83; 74.1; Des travaux de modernisation de l'infrastructure ont perturbé la régularité de cette relation en Juin
2015-06; Est; PARIS EST; METZ; 299; 299; 0; 28; 90.6; 
2015-06; Sud-Est; PARIS LYON; BESANCON FRANCHE COMTE TGV; 215; 215; 0; 13; 94.0; 
2015-06; Sud-Est; PARIS LYON; NICE VILLE; 206; 205; 1; 63; 69.3; Des travaux de modernisation de l'infrastructure ont perturbé la régularité de cette relation en Juin
2014-04; Sud-Est; AVIGNON TGV; PARIS LYON; 467; 467; 0; 79; 83.1; 
2014-04; Sud-Est; PARIS LYON; BELLEGARDE (AIN); 232; 232; 0; 28; 87.9; 
2014-04; Sud-Est; PARIS LYON; BESANCON FRANCHE COMTE TGV; 213; 213; 0; 14; 93.4; 
2014-04; Atlantique; PARIS MONTPARNASSE; BORDEAUX ST JEAN; 630; 630; 0; 15; 97.6; 
2014-05; Atlantique; BREST; PARIS MONTPARNASSE; 173; 173; 0; 5; 97.1; 
2014-05; Nord; DOUAI; PARIS NORD; 198; 198; 0; 20; 89.9; 
2014-05; Nord; PARIS NORD; DUNKERQUE; 115; 115; 0; 5; 95.7; 
2014-05; Atlantique; PARIS MONTPARNASSE; LA ROCHELLE VILLE; 223; 223; 0; 11; 95.1; 
2014-05; Atlantique; LE MANS; PARIS MONTPARNASSE; 460; 460; 0; 46; 90.0; 
2014-05; Atlantique; PARIS MONTPARNASSE; LE MANS; 448; 448; 0; 31; 93.1; 
2014-05; Atlantique; ANGERS SAINT LAUD; PARIS MONTPARNASSE; 462; 462; 0; 19; 95.9; 
2014-05; Sud-Est; LYON PART DIEU; MONTPELLIER; 399; 399; 0; 66; 83.5; 
2014-05; Sud-Est; MONTPELLIER; LYON PART DIEU; 394; 393; 1; 66; 83.2; 
2014-05; Atlantique; NANTES; PARIS MONTPARNASSE; 560; 560; 0; 18; 96.8; 
2014-05; Est; NANTES; STRASBOURG; 58; 58; 0; 7; 87.9; Accident de personne le 2 mai,  extinction le 7 mai du tableau optique du poste d'aiguillage qui télécommande Vendenheim. Le 24 mai, une rame duplex inapte occasionne un retard important.
2014-05; Sud-Est; NICE VILLE; PARIS LYON; 209; 209; 0; 44; 78.9; Cinq heurts d'animaux sauvages et deux accidents de personnes sur lignes à grande vitesse ont pénalisé la régularité de cette liaison.
2014-05; Sud-Est; NIMES; PARIS LYON; 360; 360; 0; 36; 90.0; 
2014-05; Atlantique; QUIMPER; PARIS MONTPARNASSE; 142; 142; 0; 7; 95.1; 
2014-05; Sud-Est; ANNECY; PARIS LYON; 135; 135; 0; 6; 95.6; 
2014-05; Sud-Est; PARIS LYON; TOULON; 214; 214; 0; 23; 89.3; 
2014-05; Atlantique; TOULOUSE MATABIAU; PARIS MONTPARNASSE; 94; 94; 0; 2; 97.9; 
2014-05; Atlantique; PARIS MONTPARNASSE; TOULOUSE MATABIAU; 137; 137; 0; 16; 88.3; La liaison a subi plusieurs incidents qui ont paralisé le trafic : avarie caténaire le 10 à Angoulême, présomption de fuite de matières dangereuses sur un train de marchandises à Poitiers le 15, un accident de personne à Futuroscope le 27 et présomption d'avarie caténaire à l'entrée de Paris le 28. Les travaux à Bordeaux avec la perte de 2 voies au nord de Bordeaux et en gare génèrent par ailleurs des difficultés de circulation en cas d'aléas. 
2014-05; Sud-Est; BELLEGARDE (AIN); PARIS LYON; 218; 217; 1; 24; 88.9; 
2014-06; Sud-Est; PARIS LYON; AIX EN PROVENCE TGV; 367; 367; 0; 46; 87.5; 
2014-06; Atlantique; BREST; PARIS MONTPARNASSE; 170; 158; 12; 16; 89.9; Le 9, des travaux rendus tardivement près de Rennes retardent 14 TGV et de violents orages en région parisienne ont entraîné de multiples dérangements d'installations sur la Ligne à Grande Vitesse. Le 25, un incendie dans une maison à proximité des voies près du Mans retarde trois TGV. A ces incidents s'ajoutent les retards liés à la période de grève nationale de juin.
2014-06; Nord; PARIS NORD; DOUAI; 194; 175; 19; 26; 85.1; 
2014-06; Nord; PARIS NORD; DUNKERQUE; 109; 87; 22; 5; 94.3; 
2014-06; Sud-Est; LE CREUSOT MONTCEAU MONTCHANIN; PARIS LYON; 205; 190; 15; 63; 66.8; La circulation des TGV a été fortement perturbée par le mouvement social du 11 au 22 juin.
2014-06; Sud-Est; MONTPELLIER; LYON PART DIEU; 361; 278; 83; 43; 84.5; La circulation des TGV a été fortement perturbée par le mouvement social du 11 au 22 juin.
2014-06; Sud-Est; MARSEILLE ST CHARLES; PARIS LYON; 117; 116; 1; 44; 62.1; 
2014-06; Sud-Est; PARIS LYON; MULHOUSE VILLE; 285; 254; 31; 31; 87.8; 
2014-06; Sud-Est; PARIS LYON; NIMES; 318; 282; 36; 33; 88.3; 
2014-06; Atlantique; QUIMPER; PARIS MONTPARNASSE; 127; 107; 20; 17; 84.1; Le 9, des travaux rendus tardivement près de Rennes retardent 14 TGV de 1h24 à 2h13 et de violents orages en région parisienne ont entraîné de multiples dérangements d'installations sur la Ligne à Grande Vitesse: 4 TGV de 39' à 1h53. Le 25, un incendie dans une maison à proximité des voies près du Mans retarde 3 TGV. A ces incidents s'ajoutent les retards liés à la période de grève nationale de juin..
2014-06; Est; PARIS EST; REIMS; 203; 188; 15; 19; 89.9; 
2014-06; Atlantique; RENNES; PARIS MONTPARNASSE; 541; 466; 75; 54; 88.4; Le 9, des travaux rendus tardivement près de Rennes retardent 14 TGV et de violents orages en région parisienne ont entraîné de multiples dérangements d'installations sur la Ligne à Grande Vitesse. Le 25, un incendie dans une maison à proximité des voies près du Mans retarde 3 TGV. A ces incidents s'ajoutent les retards liés à la période de grève nationale de juin.
2014-06; Atlantique; PARIS MONTPARNASSE; RENNES; 525; 448; 77; 53; 88.2; Le 9, des travaux rendus tardivement près de Rennes retardent 9 TGV et de violents orages en région parisienne ont entraîné de multiples dérangements d'installations sur la Ligne à Grande Vitesse. Le 25, un incendie dans une maison à proximité des voies près du Mans retarde 4 TGV. A ces incidents s'ajoutent les retards liés à la période de grève nationale de juin.
2014-06; Atlantique; PARIS MONTPARNASSE; ST PIERRE DES CORPS; 436; 349; 87; 54; 84.5; 
2014-06; Sud-Est; AVIGNON TGV; PARIS LYON; 437; 382; 55; 83; 78.3; La circulation des TGV a été fortement perturbée par le mouvement social du 11 au 22 juin.
2014-06; Atlantique; BORDEAUX ST JEAN; PARIS MONTPARNASSE; 588; 489; 99; 81; 83.4; 
2014-07; Sud-Est; DIJON VILLE; PARIS LYON; 471; 470; 1; 36; 92.3; 
2014-07; Nord; DUNKERQUE; PARIS NORD; 117; 117; 0; 5; 95.7; 
2014-07; Sud-Est; PARIS LYON; GRENOBLE; 201; 201; 0; 16; 92.0; 
2014-07; Atlantique; PARIS MONTPARNASSE; LE MANS; 433; 433; 0; 63; 85.5; 
2014-07; Nord; MARSEILLE ST CHARLES; LILLE; 127; 127; 0; 46; 63.8; Accident de personne à Nimes le 18, découverte d'un corps à Lyon le 2 et présence d'une personne suicidaire à Montanay le 16, heurt d'un animal au Creusot le 23, déraillement d'un wagon de fret à Lyon le 18 impactant également le 19, intempéries de type orage à Piolenc le 20 et à Avignon le 25, quelques dérangements d'installations notament à Valence le 3, Macon le 6, Vianges le 8, Lille le 11 et à Vemars le 12, incident matériel le 19 à Hattencourt, plusieurs limitations de vitesse suite à travaux essentiellement à Cesseins, Montanay et Marseille.
2014-07; Sud-Est; MONTPELLIER; PARIS LYON; 382; 382; 0; 47; 87.7; 
2014-07; Sud-Est; PARIS LYON; MONTPELLIER; 373; 373; 0; 40; 89.3; 
2014-07; Est; NANTES; STRASBOURG; 51; 51; 0; 9; 82.4; 
2014-07; Sud-Est; SAINT ETIENNE CHATEAUCREUX; PARIS LYON; 33; 33; 0; 2; 93.9; 
2014-07; Sud-Est; PARIS LYON; SAINT ETIENNE CHATEAUCREUX; 24; 24; 0; 1; 95.8; 
2014-07; Atlantique; PARIS MONTPARNASSE; ST MALO; 59; 59; 0; 3; 94.9; 
2014-07; Est; STRASBOURG; PARIS EST; 455; 454; 1; 62; 86.3; 
2014-07; Est; PARIS EST; STRASBOURG; 433; 433; 0; 37; 91.5; 
2014-07; Atlantique; TOULOUSE MATABIAU; PARIS MONTPARNASSE; 92; 92; 0; 17; 81.5; Le 9 la panne d'un train à Marcoussis, sur la Ligne à Grande Vitesse, retarde 28 TGV de 8min à 1h35. Le 10 un acte de malveillance affectant un train à St Léger, sur la Ligne à Grande Vitesse, retarde 43 TGV de 14min à 4h30, le 20 un arbre sur la voie suite aux intempéries entre Angoulême \& Mouthiers retarde 16 TGV de 14min à 3h25.
2014-07; Atlantique; PARIS MONTPARNASSE; TOULOUSE MATABIAU; 150; 150; 0; 31; 79.3; Le 9 la panne d'un train à Marcoussis, sur la Ligne à Grande Vitesse, retarde 28 TGV de 8min à 1h35. Le 10 un acte de malveillance affectant un train à St Léger, sur la Ligne à Grande Vitesse, retarde 43 TGV de 14min à 4h30. Le 20 un arbre sur la voie suite aux intempéries entre Angoulême et Mouthiers retarde 16 TGV de 14min à 3h25.
2014-07; Atlantique; PARIS MONTPARNASSE; TOURS; 146; 146; 0; 16; 89.0; 
2014-07; Sud-Est; PARIS LYON; VALENCE ALIXAN TGV; 275; 275; 0; 21; 92.4; 
2014-07; Atlantique; PARIS MONTPARNASSE; VANNES; 195; 195; 0; 9; 95.4; 
2014-07; Sud-Est; BESANCON FRANCHE COMTE TGV; PARIS LYON; 216; 215; 1; 12; 94.4; 
2014-08; Nord; ARRAS; PARIS NORD; 320; 319; 1; 29; 90.9; 
2014-08; Nord; DOUAI; PARIS NORD; 175; 175; 0; 17; 90.3; 
2014-08; Sud-Est; MARSEILLE ST CHARLES; LYON PART DIEU; 571; 571; 0; 141; 75.3; En août, la régularité des TGV entre Paris et la Méditerranée a été fortement impactée par 8 accidents de personnes et par un acte de malveillance près du Creusot.
2014-08; Sud-Est; MARSEILLE ST CHARLES; PARIS LYON; 443; 443; 0; 33; 92.6; 
2014-08; Est; NANCY; PARIS EST; 282; 282; 0; 21; 92.6; 
2014-08; Sud-Est; NICE VILLE; PARIS LYON; 267; 267; 0; 59; 77.9; En août, la régularité des TGV entre Paris et la Méditerranée a été fortement impactée par 8 accidents de personnes et par un acte de malveillance près du Creusot.
2014-08; Sud-Est; PARIS LYON; AVIGNON TGV; 576; 576; 0; 64; 88.9; 
2014-08; Sud-Est; PARIS LYON; GRENOBLE; 203; 203; 0; 17; 91.6; 
2014-08; Sud-Est; PARIS LYON; MARSEILLE ST CHARLES; 476; 476; 0; 34; 92.9; 
2014-08; Sud-Est; PARIS LYON; NICE VILLE; 255; 255; 0; 54; 78.8; En août, la régularité des TGV entre Paris et la Méditerranée a été fortement impactée par 8 accidents de personnes et par un acte de malveillance près du Creusot.
2014-08; Atlantique; PARIS MONTPARNASSE; POITIERS; 507; 507; 0; 14; 97.2; 
2014-08; Atlantique; PARIS MONTPARNASSE; RENNES; 556; 556; 0; 28; 95.0; 
2014-08; Atlantique; PARIS MONTPARNASSE; ST PIERRE DES CORPS; 441; 441; 0; 30; 93.2; 
2014-08; Atlantique; PARIS MONTPARNASSE; VANNES; 196; 196; 0; 9; 95.4; 
2014-08; Nord; PARIS NORD; LILLE; 515; 514; 1; 23; 95.5; 
2014-08; Est; STRASBOURG; NANTES; 50; 50; 0; 5; 90.0; 
2014-08; Atlantique; TOULOUSE MATABIAU; PARIS MONTPARNASSE; 92; 92; 0; 12; 87.0; 
2014-09; Sud-Est; AIX EN PROVENCE TGV; PARIS LYON; 426; 426; 0; 59; 86.2; 
2014-09; Sud-Est; BESANCON FRANCHE COMTE TGV; PARIS LYON; 216; 216; 0; 8; 96.3; 
2014-09; Sud-Est; GRENOBLE; PARIS LYON; 239; 239; 0; 25; 89.5; 
2014-09; Nord; LYON PART DIEU; LILLE; 214; 213; 1; 51; 76.1; Accidents de personne à Toulon le 1er et Roeux le 16, tentative de suicide à Avignon le 8, heurts d'animal à Haute Picardie le 3 et à Piolenc le 29, dérangement des installations à Nimes le 21, Croisilles le 22, Marne le 25 et Roeux le 26, violents orages à Piolenc le 15 et Toulon le 19, défaillance matériel à Macon le 23, incendie en gare de Manduel le 21, défaut d'alimentation à Nimes le 25 et  Lyon le 26, très importantes innondations sur Montpellier les 29 et 30.
2014-09; Est; PARIS EST; NANCY; 285; 285; 0; 8; 97.2; 
2014-09; Sud-Est; PARIS LYON; AVIGNON TGV; 487; 487; 0; 48; 90.1; 
2014-09; Sud-Est; PARIS LYON; BELLEGARDE (AIN); 220; 220; 0; 12; 94.5; 
2014-09; Sud-Est; PARIS LYON; MACON LOCHE; 206; 206; 0; 5; 97.6; 
2014-09; Sud-Est; PARIS LYON; MONTPELLIER; 309; 304; 5; 33; 89.1; 
2014-09; Sud-Est; PARIS LYON; MULHOUSE VILLE; 306; 306; 0; 19; 93.8; 
2014-09; Sud-Est; PARIS LYON; TOULON; 197; 197; 0; 25; 87.3; 
2014-09; Atlantique; PARIS MONTPARNASSE; LE MANS; 429; 429; 0; 40; 90.7; 
2014-09; Atlantique; PARIS MONTPARNASSE; ST MALO; 51; 51; 0; 2; 96.1; 
2014-09; Sud-Est; SAINT ETIENNE CHATEAUCREUX; PARIS LYON; ; ; ; ; ; 
2014-09; Sud-Est; VALENCE ALIXAN TGV; PARIS LYON; 266; 261; 5; 49; 81.2; 
2014-10; Atlantique; BORDEAUX ST JEAN; PARIS MONTPARNASSE; 664; 663; 1; 72; 89.1; 
2014-10; Nord; DOUAI; PARIS NORD; 202; 202; 0; 30; 85.1; 
2014-10; Sud-Est; LE CREUSOT MONTCEAU MONTCHANIN; PARIS LYON; 221; 221; 0; 44; 80.1; 
2014-10; Atlantique; LE MANS; PARIS MONTPARNASSE; 497; 496; 1; 116; 76.6; 
2014-10; Nord; LILLE; MARSEILLE ST CHARLES; 124; 124; 0; 22; 82.3; 
2014-10; Sud-Est; MARSEILLE ST CHARLES; PARIS LYON; 462; 462; 0; 39; 91.6; Plusieurs accidents de personne et heurts d'animaux sur lignes à grande vitesse et en région Languedoc-Roussillon ont perturbé la régularité de cette relation en octobre
2014-10; Sud-Est; MONTPELLIER; LYON PART DIEU; 339; 336; 3; 46; 86.3; Plusieurs accidents de personne et heurts d'animaux sur lignes à grande vitesse et en région Languedoc-Roussillon ont perturbé la régularité de cette relation en octobre
2014-10; Sud-Est; MULHOUSE VILLE; PARIS LYON; 310; 310; 0; 23; 92.6; 
2014-10; Sud-Est; PARIS LYON; AIX EN PROVENCE TGV; 457; 457; 0; 39; 91.5; 
2014-10; Sud-Est; PARIS LYON; ANNECY; 151; 151; 0; 6; 96.0; 
2014-02; Atlantique; PARIS MONTPARNASSE; LA ROCHELLE VILLE; 202; 202; 0; 20; 90.1; 
2014-02; Sud-Est; LYON PART DIEU; MARSEILLE ST CHARLES; 586; 585; 1; 62; 89.4; 
2014-02; Atlantique; LYON PART DIEU; RENNES; 28; 28; 0; 1; 96.4; 
2014-02; Sud-Est; MACON LOCHE; PARIS LYON; 183; 183; 0; 11; 94.0; 
2014-02; Sud-Est; PARIS LYON; MACON LOCHE; 208; 208; 0; 8; 96.2; 
2014-02; Sud-Est; MARSEILLE ST CHARLES; PARIS LYON; 400; 400; 0; 16; 96.0; 
2014-02; Sud-Est; PARIS LYON; MARSEILLE ST CHARLES; 433; 433; 0; 11; 97.5; 
2014-02; Atlantique; ANGOULEME; PARIS MONTPARNASSE; 298; 298; 0; 40; 86.6; 
2014-02; Atlantique; PARIS MONTPARNASSE; ANGOULEME; 300; 300; 0; 23; 92.3; 
2014-02; Sud-Est; MULHOUSE VILLE; PARIS LYON; 290; 290; 0; 14; 95.2; 
2014-02; Sud-Est; NICE VILLE; PARIS LYON; 174; 174; 0; 18; 89.7; 
2014-02; Sud-Est; PARIS LYON; NIMES; 295; 295; 0; 16; 94.6; 
2014-02; Atlantique; QUIMPER; PARIS MONTPARNASSE; 132; 132; 0; 4; 97.0; 
2014-02; Sud-Est; PARIS LYON; ANNECY; 136; 136; 0; 5; 96.3; 
2014-02; Atlantique; RENNES; PARIS MONTPARNASSE; 520; 520; 0; 40; 92.3; 
2014-02; Est; PARIS EST; STRASBOURG; 402; 402; 0; 31; 92.3; 
2014-02; Sud-Est; TOULON; PARIS LYON; 228; 228; 0; 20; 91.2; 
2014-02; Atlantique; PARIS MONTPARNASSE; TOURS; 135; 135; 0; 14; 89.6; 
2014-02; Sud-Est; VALENCE ALIXAN TGV; PARIS LYON; 248; 248; 0; 22; 91.1; 
2014-02; Atlantique; VANNES; PARIS MONTPARNASSE; 140; 140; 0; 5; 96.4; 
2014-02; Atlantique; PARIS MONTPARNASSE; VANNES; 146; 146; 0; 8; 94.5; 
2014-02; Sud-Est; AVIGNON TGV; PARIS LYON; 383; 383; 0; 49; 87.2; 
2014-02; Sud-Est; BELLEGARDE (AIN); PARIS LYON; 234; 234; 0; 17; 92.7; 
2015-05; Sud-Est; DIJON VILLE; PARIS LYON; 447; 443; 4; 21; 95.3; 
2015-05; Nord; DUNKERQUE; PARIS NORD; 254; 252; 2; 16; 93.7; 
2015-05; Nord; LILLE; LYON PART DIEU; 275; 275; 0; 24; 91.3; Toujours des travaux importants sur la LGV  entrainant des limitations de vitesse, essentiellement à Lapalud. Autres événements :
03/05 : dérangement d'aiguille lors de travaux au Creusot
05 et 28/05 : panne d'une rame sur LGV Nord
08/05 : accident de personne à Lapalud
20/05 : acte de malveillance (jets de pierres) à Vémars
25/05 : occupation des voies à l'escale de Lille Europe
2015-05; Nord; LYON PART DIEU; LILLE; 281; 280; 1; 83; 70.4; Toujours des travaux importants sur la LGV  entrainant des limitations de vitesse, essentiellement à Lapalud. Autres événements :
03/05 : dérangement d'aiguille lors de travaux au Creusot
05 et 28/05 : panne d'une rame sur LGV Nord
08/05 : accident de personne à Lapalud
20/05 : acte de malveillance (jets de pierres) à Vémars
25/05 : occupation des voies à l'escale de Lille Europe
2015-05; Sud-Est; MARSEILLE ST CHARLES; PARIS LYON; 435; 435; 0; 50; 88.5; 
2015-05; Est; PARIS EST; NANCY; 271; 271; 0; 9; 96.7; 
2015-05; Sud-Est; PARIS LYON; AIX EN PROVENCE TGV; 453; 453; 0; 46; 89.8; 
2015-05; Sud-Est; PARIS LYON; AVIGNON TGV; 535; 535; 0; 62; 88.4; 
2015-05; Sud-Est; PARIS LYON; CHAMBERY CHALLES LES EAUX; 166; 166; 0; 12; 92.8; 
2015-05; Sud-Est; PARIS LYON; VALENCE ALIXAN TGV; 279; 279; 0; 22; 92.1; 
2015-05; Atlantique; PARIS MONTPARNASSE; LA ROCHELLE VILLE; 189; 189; 0; 12; 93.7; 
2015-05; Atlantique; PARIS MONTPARNASSE; LAVAL; 235; 235; 0; 11; 95.3; 
2015-05; Atlantique; PARIS MONTPARNASSE; LE MANS; 430; 430; 0; 38; 91.2; 
2015-05; Atlantique; PARIS MONTPARNASSE; QUIMPER; 191; 191; 0; 11; 94.2; 
2015-05; Nord; PARIS NORD; DUNKERQUE; 251; 250; 1; 24; 90.4; 
2015-05; Nord; PARIS NORD; ARRAS; 303; 303; 0; 22; 92.7; 
2015-05; Atlantique; QUIMPER; PARIS MONTPARNASSE; 210; 210; 0; 14; 93.3; Cette ligne a été retardée par : un accident de personne à Yvre-l'Evêque le 14/05 (15 TGV soit 1456 minutes), une disjonction sur la LGV le 25/05 (6 TGV soit 87 minutes), heurt d'un animal sur la LGV le 12/05 (30 TGV soit 1799 minutes) et heurt d'une voiture le 06/05 à Vitré (18 TGV soit 1704 minutes)
2015-05; Atlantique; RENNES; PARIS MONTPARNASSE; 563; 563; 0; 37; 93.4; Cette ligne a été retardée par : un accident de personne à Yvre-l'Evêque le 14/05 (15 TGV soit 1456 minutes), une disjonction sur la LGV le 25/05 (6 TGV soit 87 minutes), heurt d'un animal sur la LGV le 12/05 (30 TGV soit 1799 minutes) et heurt d'une voiture le 06/05 à Vitré (18 TGV soit 1704 minutes)
2015-06; Nord; ARRAS; PARIS NORD; 334; 333; 1; 58; 82.6; 
2015-06; Nord; LYON PART DIEU; LILLE; 267; 267; 0; 83; 68.9; Toujours des travaux importants sur la LGV  entrainant des limitations de vitesse, essentiellement à Lapalud. Autres événements :
05/06 : plusieurs dérangements d'installations sur LGV Nord ainsi qu'à Lyon
08/06 : dérangements d'aiguilles à Arsy et à Sathonay 
13 et 15/06 : heurt de chevreuils sur LGV Nord
19:06 : panne d'une rame TGV avec rupture caténaire à Lunel
28/06 : colis suspects à Lille et Marseille + dérangement d'installation sur LGV SE
30/06 : vol de câble à Sainghin + effets de la chaleur
Pannes d'une  rame les 11, 13 ,  27, 30 juin
2015-06; Sud-Est; MACON LOCHE; PARIS LYON; 198; 196; 2; 20; 89.8; 
2015-06; Sud-Est; MARSEILLE ST CHARLES; LYON PART DIEU; 556; 555; 1; 182; 67.2; Des travaux de modernisation de l'infrastructure ont perturbé la régularité de cette relation en Juin
2015-06; Sud-Est; PARIS LYON; BELLEGARDE (AIN); 233; 233; 0; 27; 88.4; 
2015-06; Sud-Est; PARIS LYON; CHAMBERY CHALLES LES EAUX; 172; 172; 0; 22; 87.2; 
2015-06; Sud-Est; PARIS LYON; LE CREUSOT MONTCEAU MONTCHANIN; 200; 200; 0; 22; 89.0; 
2015-06; Sud-Est; PARIS LYON; LYON PART DIEU; 614; 612; 2; 50; 91.8; 
2015-06; Sud-Est; PARIS LYON; MACON LOCHE; 205; 205; 0; 15; 92.7; 
2015-06; Sud-Est; PARIS LYON; MULHOUSE VILLE; 302; 302; 0; 17; 94.4; 
2015-06; Atlantique; PARIS MONTPARNASSE; ST MALO; 96; 96; 0; 14; 85.4; Cette OD a été touchée le 4 par un rail cassé couplé à un affaissement caténaire à la sortie de Paris (172 TGV; 7664mn), le 5 par diverses limitations de vitesses liées aux fortes chaleurs dont une à la sortie de Paris reduisant fortement le débit des trains (82TGV; 985mn), le 26 par la rupture d'un appareil de voie en sortie de ligne grande vitesse avant Le Mans (52TGV; 923mn), le 3 par l'arrêt de sécurité d'un train dans les tunnels avant Paris (22TGV; 1035mn) lié à une presence dans les voies en sortie de gare Montparnasse (16TGV; 419mn); le 9 par un dérangement d'installation sur la branche ouest de la ligne grande vitesse (18TGV; 169mn)
2015-06; Est; STRASBOURG; NANTES; 55; 55; 0; 3; 94.5; 
2015-06; Est; STRASBOURG; PARIS EST; 467; 467; 0; 54; 88.4; 
2015-05; Sud-Est; PARIS LYON; GRENOBLE; 218; 217; 1; 19; 91.2; 
2015-05; Sud-Est; PARIS LYON; MONTPELLIER; 340; 340; 0; 37; 89.1; 
2015-05; Atlantique; PARIS MONTPARNASSE; ANGERS SAINT LAUD; 425; 425; 0; 25; 94.1; 
2015-05; Atlantique; PARIS MONTPARNASSE; ANGOULEME; 322; 321; 1; 14; 95.6; 
2015-05; Atlantique; PARIS MONTPARNASSE; BREST; 212; 212; 0; 11; 94.8; Cette ligne a été retardée par : un accident de personne à Yvre-l'Evêque le 14/05 (15 TGV soit 1456 minutes), une disjonction sur la LGV le 25/05 (6 TGV soit 87 minutes), heurt d'un animal sur la LGV le 12/05 (30 TGV soit 1799 minutes) et heurt d'une voiture le 06/05 à Vitré (18 TGV soit 1704 minutes)
2015-05; Atlantique; PARIS MONTPARNASSE; POITIERS; 502; 501; 1; 26; 94.8; 
2015-05; Atlantique; PARIS MONTPARNASSE; ST PIERRE DES CORPS; 441; 441; 0; 32; 92.7; 
2015-05; Sud-Est; PERPIGNAN; PARIS LYON; 155; 155; 0; 30; 80.6; 
2015-05; Sud-Est; SAINT ETIENNE CHATEAUCREUX; PARIS LYON; 110; 110; 0; 13; 88.2; 
2015-05; Atlantique; ST MALO; PARIS MONTPARNASSE; 105; 105; 0; 5; 95.2; Cette ligne a été retardée par : un accident de personne à Yvre-l'Evêque le 14/05 (15 TGV soit 1456 minutes), une disjonction sur la LGV le 25/05 (6 TGV soit 87 minutes), heurt d'un animal sur la LGV le 12/05 (30 TGV soit 1799 minutes) et heurt d'une voiture le 06/05 à Vitré (18 TGV soit 1704 minutes)
2015-05; Atlantique; ST PIERRE DES CORPS; PARIS MONTPARNASSE; 424; 423; 1; 32; 92.4; 
2015-05; Est; STRASBOURG; PARIS EST; 452; 452; 0; 29; 93.6; 
2015-05; Atlantique; TOURS; PARIS MONTPARNASSE; 195; 195; 0; 10; 94.9; 
2015-05; Atlantique; VANNES; PARIS MONTPARNASSE; 232; 232; 0; 14; 94.0; 
2015-06; Sud-Est; AIX EN PROVENCE TGV; PARIS LYON; 452; 452; 0; 54; 88.1; 
2015-06; Atlantique; ANGOULEME; PARIS MONTPARNASSE; 328; 326; 2; 46; 85.9; 
2015-06; Atlantique; LAVAL; PARIS MONTPARNASSE; 228; 228; 0; 17; 92.5; Cette OD a été touchée le 4 par un rail cassé couplé à un affaissement caténaire à la sortie de Paris (172 TGV; 7664mn), le 5 par diverses limitations de vitesses liées aux fortes chaleurs dont une à la sortie de Paris reduisant fortement le débit des trains (82TGV; 985mn), le 26 par la rupture d'un appareil de voie en sortie de ligne grande vitesse avant Le Mans (52TGV; 923mn), le 3 par l'arrêt de sécurité d'un train dans les tunnels avant Paris (22TGV; 1035mn); le 9 un dérangement d'installation sur la branche ouest de la ligne grande vitesse (18TGV; 169mn)
2015-06; Sud-Est; LE CREUSOT MONTCEAU MONTCHANIN; PARIS LYON; 215; 214; 1; 24; 88.8; 
2015-06; Nord; LILLE; MARSEILLE ST CHARLES; 207; 206; 1; 65; 68.4; Toujours des travaux importants sur la LGV  entrainant des limitations de vitesse, essentiellement à Lapalud. Autres événements :
05/06 : plusieurs dérangements d'installations sur LGV Nord ainsi qu'à Lyon
08/06 : dérangements d'aiguilles à Arsy et à Sathonay 
13 et 15/06 : heurt de chevreuils sur LGV Nord
19:06 : panne d'une rame TGV avec rupture caténaire à Lunel
28/06 : colis suspects à Lille et Marseille + dérangement d'installation sur LGV SE
30/06 : vol de câble à Sainghin + effets de la chaleur
Pannes d'une  rame les 11, 13 ,  27, 30 juin
2015-06; Nord; MARSEILLE ST CHARLES; LILLE; 209; 209; 0; 68; 67.5; Toujours des travaux importants sur la LGV  entrainant des limitations de vitesse, essentiellement à Lapalud. Autres événements :
05/06 : plusieurs dérangements d'installations sur LGV Nord ainsi qu'à Lyon
08/06 : dérangements d'aiguilles à Arsy et à Sathonay 
13 et 15/06 : heurt de chevreuils sur LGV Nord
19:06 : panne d'une rame TGV avec rupture caténaire à Lunel
28/06 : colis suspects à Lille et Marseille + dérangement d'installation sur LGV SE
30/06 : vol de câble à Sainghin + effets de la chaleur
Pannes d'une  rame les 11, 13 ,  27, 30 juin
2015-06; Sud-Est; MARSEILLE ST CHARLES; PARIS LYON; 435; 435; 0; 46; 89.4; 
2015-06; Est; METZ; PARIS EST; 289; 289; 0; 45; 84.4; 
2015-06; Sud-Est; MULHOUSE VILLE; PARIS LYON; 313; 313; 0; 16; 94.9; 
2015-06; Est; NANTES; STRASBOURG; 55; 55; 0; 8; 85.5; 
2015-06; Sud-Est; NIMES; PARIS LYON; 338; 338; 0; 78; 76.9; Des travaux de modernisation de l'infrastructure ont perturbé la régularité de cette relation en Juin
2015-06; Est; PARIS EST; NANCY; 288; 288; 0; 17; 94.1; 
2015-06; Est; PARIS EST; REIMS; 208; 208; 0; 19; 90.9; 
2015-06; Sud-Est; PARIS LYON; SAINT ETIENNE CHATEAUCREUX; 114; 113; 1; 11; 90.3; 
2015-06; Sud-Est; PARIS LYON; VALENCE ALIXAN TGV; 282; 282; 0; 33; 88.3; 
2015-06; Atlantique; PARIS MONTPARNASSE; BORDEAUX ST JEAN; 625; 623; 2; 78; 87.5; Cette OD a été touchée le 4 par un rail cassé couplé à un affaissement caténaire à la sortie de Paris (172 TGV; 7664mn), le 5 par diverses limitations de vitesses liées aux fortes chaleurs dont une à la sortie de Paris reduisant fortement le débit des trains (82TGV; 985mn), 2 incendies aux abords des voies entre Droué et Vendôme le 25 (48TGV; 1826mn) et le 30 (30TGV; 1773mn), le 3 par l'arrêt de sécurité d'un train dans les tunnels avant Paris (22TGV; 1035mn) lié à une presence dans les voies en sortie de gare Montparnasse (16TGV; 419mn),  le 22 par le heurt de 2 chevreuils sur la branche sud de la ligne grande vitesse (15TGV; 328mn)
2015-06; Atlantique; PARIS MONTPARNASSE; BREST; 207; 206; 1; 18; 91.3; Cette OD a été touchée le 4 par un rail cassé couplé à un affaissement caténaire à la sortie de Paris (172 TGV; 7664mn), le 5 par diverses limitations de vitesses liées aux fortes chaleurs dont une à la sortie de Paris reduisant fortement le débit des trains (82TGV; 985mn), le 26 par la rupture d'un appareil de voie en sortie de ligne grande vitesse avant Le Mans (52TGV; 923mn), le 3 par l'arrêt de sécurité d'un train dans les tunnels avant Paris (22TGV; 1035mn) lié à une presence dans les voies en sortie de gare Montparnasse (16TGV; 419mn); le 9 par un dérangement d'installation sur la branche ouest de la ligne grande vitesse (18TGV; 169mn)
2015-06; Atlantique; PARIS MONTPARNASSE; TOULOUSE MATABIAU; 149; 149; 0; 39; 73.8; Cette OD a été touchée le 4 par un rail cassé couplé à un affaissement caténaire à la sortie de Paris (172 TGV; 7664mn), le 5 par diverses limitations de vitesses liées aux fortes chaleurs dont une à la sortie de Paris reduisant fortement le débit des trains (82TGV; 985mn), 2 incendies aux abords des voies entre Droué et Vendôme le 25 (48TGV; 1826mn) et le 30 (30TGV; 1773mn), le 3 par l'arrêt de sécurité d'un train dans les tunnels avant Paris (22TGV; 1035mn) lié à une presence dans les voies en sortie de gare Montparnasse (16TGV; 419mn),  le 22 par le heurt de 2 chevreuils sur la branche sud de la ligne grande vitesse (15TGV; 328mn)
2015-06; Atlantique; PARIS MONTPARNASSE; VANNES; 217; 217; 0; 32; 85.3; Cette OD a été touchée le 4 par un rail cassé couplé à un affaissement caténaire à la sortie de Paris (172 TGV; 7664mn), le 5 par diverses limitations de vitesses liées aux fortes chaleurs dont une à la sortie de Paris reduisant fortement le débit des trains (82TGV; 985mn), le 26 par la rupture d'un appareil de voie en sortie de ligne grande vitesse avant Le Mans (52TGV; 923mn), le 3 par l'arrêt de sécurité d'un train dans les tunnels avant Paris (22TGV; 1035mn) lié à une presence dans les voies en sortie de gare Montparnasse (16TGV; 419mn); le 9 par un dérangement d'installation sur la branche ouest de la ligne grande vitesse (18TGV; 169mn)
2015-06; Nord; PARIS NORD; DUNKERQUE; 257; 257; 0; 27; 89.5; 
2015-06; Nord; PARIS NORD; ARRAS; 308; 308; 0; 26; 91.6; 
2015-06; Atlantique; POITIERS; PARIS MONTPARNASSE; 455; 454; 1; 60; 86.8; 
2015-06; Atlantique; QUIMPER; PARIS MONTPARNASSE; 213; 213; 0; 17; 92.0; Cette OD a été touchée le 4 par un rail cassé couplé à un affaissement caténaire à la sortie de Paris (172 TGV; 7664mn), le 5 par diverses limitations de vitesses liées aux fortes chaleurs dont une à la sortie de Paris reduisant fortement le débit des trains (82TGV; 985mn), le 26 par la rupture d'un appareil de voie en sortie de ligne grande vitesse avant Le Mans (52TGV; 923mn), le 3 par l'arrêt de sécurité d'un train dans les tunnels avant Paris (22TGV; 1035mn) lié à une presence dans les voies en sortie de gare Montparnasse (16TGV; 419mn); le 9 par un dérangement d'installation sur la branche ouest de la ligne grande vitesse (18TGV; 169mn)
2015-06; Sud-Est; SAINT ETIENNE CHATEAUCREUX; PARIS LYON; 109; 109; 0; 8; 92.7; 
2015-06; Sud-Est; TOULON; PARIS LYON; 243; 243; 0; 64; 73.7; Des travaux de modernisation de l'infrastructure ont perturbé la régularité de cette relation en Juin
2015-06; Atlantique; PARIS MONTPARNASSE; ANGOULEME; 318; 316; 2; 43; 86.4; Cette OD a été touchée le 4 par un rail cassé couplé à un affaissement caténaire à la sortie de Paris (172 TGV; 7664mn), le 5 par diverses limitations de vitesses liées aux fortes chaleurs dont une à la sortie de Paris reduisant fortement le débit des trains (82TGV; 985mn), 2 incendies aux abords des voies entre Droué et Vendôme le 25 (48TGV; 1826mn) et le 30 (30TGV; 1773mn), le 3 par l'arrêt de sécurité d'un train dans les tunnels avant Paris (22TGV; 1035mn) lié à une presence dans les voies en sortie de gare Montparnasse (16TGV; 419mn),  le 22 par le heurt de 2 chevreuils sur la branche sud de la ligne grande vitesse (15TGV; 328mn)
2015-06; Atlantique; PARIS MONTPARNASSE; LA ROCHELLE VILLE; 213; 213; 0; 24; 88.7; Cette OD a été touchée le 4 par un rail cassé couplé à un affaissement caténaire à la sortie de Paris (172 TGV; 7664mn), le 5 par diverses limitations de vitesses liées aux fortes chaleurs dont une à la sortie de Paris reduisant fortement le débit des trains (82TGV; 985mn), 2 incendies aux abords des voies entre Droué et Vendôme le 25 (48TGV; 1826mn) et le 30 (30TGV; 1773mn), le 3 par l'arrêt de sécurité d'un train dans les tunnels avant Paris (22TGV; 1035mn) lié à une presence dans les voies en sortie de gare Montparnasse (16TGV; 419mn),  le 22 par le heurt de 2 chevreuils sur la branche sud de la ligne grande vitesse (15TGV; 328mn), le 4 par un dérangement d'installation à Lusignan (15 TGV; 654mn)
2015-06; Atlantique; PARIS MONTPARNASSE; RENNES; 533; 532; 1; 66; 87.6; 
2015-06; Sud-Est; PERPIGNAN; PARIS LYON; 138; 138; 0; 29; 79.0; Des travaux de modernisation de l'infrastructure ont perturbé la régularité de cette relation en Juin
2015-06; Est; REIMS; PARIS EST; 205; 205; 0; 10; 95.1; 
2015-06; Atlantique; TOURS; PARIS MONTPARNASSE; 200; 200; 0; 28; 86.0; 
2014-10; Sud-Est; PARIS LYON; AVIGNON TGV; 506; 506; 0; 35; 93.1;
    \end{Verbatim}

    \#

Informatique tc3 (Projet Web) - TD1

    \#\#

Base de données SQLite, Firefox et SQLite manager, utilisation via
Python

    \subsubsection{1. Préambule}\label{pruxe9ambule}

    Dans le cadre de ce TD vous serez amenés à créer une base de données, à
l'alimenter à partir de données ouvertes (Open Data) librement
disponibles sur l'internet, et à développer quelques programmes Python
pour utiliser ces données, en extraire divers résultats, et en obtenir
facilement des représentations graphiques.

    Les données dont vous aurez besoin se trouvent sur le serveur
pédagogique. Vous devrez :

créer un répertoire sur votre machine pour les fichiers relatifs à ce
TD,

récupérer sur le serveur pédagogique le sujet (fichier .ipynb) et les
fichiers csv, et les placer dans le répertoire que vous venez de créer

démarrer un serveur de notebook dans ce même répertoire avec la commande
ipython notebook ou toute méthode que vous jugerez utile pour consulter
le présent sujet,

effectuer au fur et à mesure le travail demandé, qui consiste en général
à remplir les cellules du présent notebook avec du code Python répondant
à la question.

    \subsubsection{2. Lecture d'un fichier de
données}\label{lecture-dun-fichier-de-donnuxe9es}

    L'objectif de cette question est récupérer les données qui se trouvent
dans le fichier regularite-mensuelle-tgv.csv.

Ce fichier est au format CSV. Il s'agit d'un fichier contenant du texte
(qui peut à ce titre être ouvert et consulté à l'aide du bloc-notes) qui
représente un tableau de données, dont chaque ligne contient une liste
de valeurs séparées par des virgules (ou tout autre caractère
séparateur).

Le contenu du fichier est constitué de données séparées par le caractère
; point-virgule. Lorsqu'une valeur contient elle-même le caractère
séparateur, cette valeur est délimitée par des " guillemets doubles.

Il est encodé en UTF-8. C'est la convention choisie pour représenter les
caractères alphabétiques et autres, par un code numérique sur le disque
et/ou dans la mémoire de l'ordinateur.

    2.1 - En vous aidant de la doc python pour lire un fichier csv,
reproduisez le premier exemple de programme du § 14.1.1 en l'adaptant
pour lire le fichier regularite-mensuelle-tgv.csv. N'oubliez pas de
préciser le caractère séparateur (delimiter) et celui servant à
délimiter les valeurs (quotechar). Si vous observez des caractères
bizarres dans le résultat, consultez la doc python et spécifiez le
format du fichier dans l'instruction open.

Vous devriez observer un résultat similaire à celui-ci :

N.B. Une fois la liste obtenue, si elle vous paraît trop longue vous
pouvez cliquer dans la colonne de gauche pour la replier et éviter
d'encombrer la fenêtre du sujet\ldots{}

    \subsubsection{3. Création d'une base
SQL}\label{cruxe9ation-dune-base-sql}

    3.1 - En vous aidant du premier exemple de code de la doc python sur
l'utilisation de bases de données SQLite, créez une base de données
nommée sncf.sqlite et observez dans le répertoire du TD l'appartion du
fichier correspondant après exécution du code. 

    \begin{Verbatim}[commandchars=\\\{\}]
{\color{incolor}In [{\color{incolor}2}]:} \PY{c+c1}{\PYZsh{} votre code ici}
        
        \PY{k+kn}{import} \PY{n+nn}{sqlite3}
        \PY{n}{conn} \PY{o}{=} \PY{n}{sqlite3}\PY{o}{.}\PY{n}{connect}\PY{p}{(}\PY{l+s+s1}{\PYZsq{}}\PY{l+s+s1}{sncf.sqlite}\PY{l+s+s1}{\PYZsq{}}\PY{p}{)}
\end{Verbatim}

    3.2 - Toujours en vous basant sur la doc python, ainsi que sur la
syntaxe SQL supportée par la base SQLite, créez une table nommée
regularite\_tgv avec les champs suivants :

N.B. Pour permettre l'exécution répétée de votre cellule de code,
ajoutez une instruction SQL pour supprimer la table si elle existe,
avant de la recréer\ldots{}

    \begin{Verbatim}[commandchars=\\\{\}]
{\color{incolor}In [{\color{incolor} }]:} \PY{c+c1}{\PYZsh{} votre code ici}
        
        \PY{k+kn}{import} \PY{n+nn}{sqlite3}
        \PY{n}{conn} \PY{o}{=} \PY{n}{sqlite3}\PY{o}{.}\PY{n}{connect}\PY{p}{(}\PY{l+s+s1}{\PYZsq{}}\PY{l+s+s1}{sncf.sqlite}\PY{l+s+s1}{\PYZsq{}}\PY{p}{)}
        \PY{n}{c} \PY{o}{=} \PY{n}{conn}\PY{o}{.}\PY{n}{cursor}\PY{p}{(}\PY{p}{)}
        \PY{c+c1}{\PYZsh{}c.execute(\PYZdq{}DROP TABLE IF EXISTS regularite\PYZus{}tgv\PYZdq{})}
        \PY{c+c1}{\PYZsh{} Create table}
        \PY{c+c1}{\PYZsh{}c.execute(\PYZdq{}CREATE TABLE regularite\PYZus{}tgv(id INTEGER PRIMARY KEY, annee INTEGER, mois INTEGER, axe text, depart text, arrive text, nb\PYZus{}prevus INTEGER, nb\PYZus{}train real,nb\PYZus{}annules INTEGER, nb\PYZus{}retards INTEGER, regularite real, commentaire text)\PYZdq{})}
        
        \PY{c+c1}{\PYZsh{} Insert a row of data}
        \PY{c+c1}{\PYZsh{}c.execute(\PYZdq{}INSERT INTO stocks VALUES (\PYZsq{}2006\PYZhy{}01\PYZhy{}05\PYZsq{},\PYZsq{}BUY\PYZsq{},\PYZsq{}RHAT\PYZsq{},100,35.14)\PYZdq{})}
        
        \PY{c+c1}{\PYZsh{} Save (commit) the changes}
        \PY{c+c1}{\PYZsh{}conn.commit()}
        
        \PY{c+c1}{\PYZsh{} We can also close the connection if we are done with it.}
        \PY{c+c1}{\PYZsh{} Just be sure any changes have been committed or they will be lost.}
        \PY{c+c1}{\PYZsh{}conn.close()}
        \PY{k}{for} \PY{n}{row} \PY{o+ow}{in} \PY{n}{c}\PY{o}{.}\PY{n}{execute}\PY{p}{(}\PY{l+s+s1}{\PYZsq{}}\PY{l+s+s1}{SELECT * FROM regularite\PYZus{}tgv}\PY{l+s+s1}{\PYZsq{}}\PY{p}{)}\PY{p}{:}
                \PY{n+nb}{print}\PY{p}{(}\PY{n}{row}\PY{p}{)}
\end{Verbatim}

    Autant le succès du programme de création de la base de données
(question 3.1) peut s'observer par l'apparition du fichier sncf.sqlite,
autant nous sommes aveugles pour vérifier la bonne exécution du
programme de création de la table.

Qu'à cela ne tienne : il nous faut un logiciel nous permettant
d'observer le contenu d'une base de données SQLite. Ou mieux : à
observer et à interagir avec une base de données SQLIte (créer,
supprimer ou modifier des tables et des enregistrements, effectuer des
requêtes SQL\ldots{}).

Un tel outil existe : il s'appelle SQLite Manager et se présente sous la
forme d'un add-on pour le navigateur Web Firefox. Firefox possède
lui-même l'insigne avantage de pouvoir s'installer aussi bien sous
Windows, que Mac ou Linux.

    3.3 - Téléchargez et installez Firefox si vous ne l'avez pas déjà, puis
installez l'add-on SQLite Manager.

Démarrez SQLite Manager et ouvrez la base de données sncf.sqlite pour
constater l'existence de la table regularite\_tgv. Si tel n'est pas le
cas, corrigez votre programme de la question 3.2 jusqu'à y arriver. 

    \subsubsection{4. Remplissage de la table de régularité des
TGV}\label{remplissage-de-la-table-de-ruxe9gularituxe9-des-tgv}

    Il est maintenant temps de passer aux choses sérieuses et de transférer
les données depuis le fichier CSV dans la base de données, et plus
particulièrement dans la table que nous venons de créer.

    4.1 - A partir du code de la question 2.1 permettant de lire le fichier
CSV, créez un programme permettant d'insérer les données dans la table
regularite\_tgv. Attention au format de la date : il s'agit d'une chaîne
de caractères dans le fichier CSV, et de deux entiers respectivement
pour le mois et l'année dans la base de données.

N.B. Lors de l'écriture d'un enregistrement dans la table, si on met le
champ id à NULL, il sera automatiquement géré par SQLite pour éviter
tout doublon. C'est la magie du type entier, clé primaire.

Vérifiez l'apparition des données et la bonne exécution de votre
programme à l'aide de SQLite Manager : 

    \begin{Verbatim}[commandchars=\\\{\}]
{\color{incolor}In [{\color{incolor} }]:} \PY{c+c1}{\PYZsh{} votre code ici}
        \PY{k+kn}{import} \PY{n+nn}{csv}
        \PY{k}{with} \PY{n+nb}{open}\PY{p}{(}\PY{l+s+s1}{\PYZsq{}}\PY{l+s+s1}{regularite\PYZhy{}mensuelle\PYZhy{}tgv.csv}\PY{l+s+s1}{\PYZsq{}}\PY{p}{,} \PY{l+s+s1}{\PYZsq{}}\PY{l+s+s1}{r}\PY{l+s+s1}{\PYZsq{}}\PY{p}{,} \PY{n}{encoding}\PY{o}{=}\PY{l+s+s1}{\PYZsq{}}\PY{l+s+s1}{utf\PYZhy{}8}\PY{l+s+s1}{\PYZsq{}}\PY{p}{)} \PY{k}{as} \PY{n}{csvfile}\PY{p}{:}
            \PY{n}{reader} \PY{o}{=} \PY{n}{csv}\PY{o}{.}\PY{n}{reader}\PY{p}{(}\PY{n}{csvfile}\PY{p}{,} \PY{n}{delimiter}\PY{o}{=}\PY{l+s+s1}{\PYZsq{}}\PY{l+s+s1}{;}\PY{l+s+s1}{\PYZsq{}}\PY{p}{,} \PY{n}{quotechar}\PY{o}{=}\PY{l+s+s1}{\PYZsq{}}\PY{l+s+s1}{\PYZdq{}}\PY{l+s+s1}{\PYZsq{}}\PY{p}{)}
            \PY{n}{got\PYZus{}header}\PY{o}{=} \PY{k+kc}{False}
            \PY{k}{for} \PY{n}{row} \PY{o+ow}{in} \PY{n}{reader}\PY{p}{:}
                \PY{k}{if} \PY{p}{(}\PY{o+ow}{not} \PY{n}{got\PYZus{}header}\PY{p}{)}\PY{p}{:}
                    \PY{n}{got\PYZus{}header}\PY{o}{=}\PY{k+kc}{True}
                    \PY{k}{continue}
                \PY{n}{annee}\PY{p}{,}\PY{n}{mois}\PY{o}{=} \PY{n}{row}\PY{p}{[}\PY{l+m+mi}{0}\PY{p}{]}\PY{o}{.}\PY{n}{split}\PY{p}{(}\PY{n}{sep}\PY{o}{=}\PY{l+s+s1}{\PYZsq{}}\PY{l+s+s1}{\PYZhy{}}\PY{l+s+s1}{\PYZsq{}}\PY{p}{)}
                \PY{n}{c}\PY{o}{.}\PY{n}{execute}\PY{p}{(}\PY{l+s+s1}{\PYZsq{}}\PY{l+s+s1}{INSERT INTO regularite\PYZus{}tgv VALUES (NULL,?,?,?,?,?,?,?,?,?,?,?)}\PY{l+s+s1}{\PYZsq{}}\PY{p}{,}\PY{p}{(}\PY{n}{annee}\PY{p}{,}\PY{n}{mois}\PY{p}{)}\PY{o}{+}\PY{n+nb}{tuple}\PY{p}{(}\PY{n}{v} \PY{k}{for} \PY{n}{v} \PY{o+ow}{in} \PY{n}{row}\PY{p}{[}\PY{l+m+mi}{1}\PY{p}{:}\PY{p}{]}\PY{p}{)}\PY{p}{)}
                \PY{n}{conn}\PY{o}{.}\PY{n}{commit}\PY{p}{(}\PY{p}{)}
\end{Verbatim}

    4.2 - Comme il est peu probable que vous y arriviez du premier coup, il
sera nécessaire de rejouer le programme de remplissage plusieurs fois
sans pour autant créer de doublons dans la table.

Ajoutez une instruction pour vider la table avant de la remplir.

    \begin{Verbatim}[commandchars=\\\{\}]
{\color{incolor}In [{\color{incolor}5}]:} \PY{c+c1}{\PYZsh{} votre code ici}
\end{Verbatim}

    4.3 - Vous l'aurez noté, le remplissage de la table n'est pas
instantané\ldots{} En vous référant à la doc python, encadrez la boucle
de remplissage de la table, des instructions nécessaires pour afficher
le temps d'exécution de cette portion de code. 

    \begin{Verbatim}[commandchars=\\\{\}]
{\color{incolor}In [{\color{incolor}6}]:} \PY{c+c1}{\PYZsh{} votre code ici}
\end{Verbatim}

    \subsubsection{5. Exploitation de la table de régularité des
TGV}\label{exploitation-de-la-table-de-ruxe9gularituxe9-des-tgv}

    5.1 - Ecrivez un programme Python qui affiche le nombre
d'enregistrements dans la table.

    \begin{Verbatim}[commandchars=\\\{\}]
{\color{incolor}In [{\color{incolor}7}]:} \PY{c+c1}{\PYZsh{} votre code ici}
\end{Verbatim}

    5.2 - Ecrivez un programme Python qui affiche le nombre d'axes TGV cités
dans la base ?

    \begin{Verbatim}[commandchars=\\\{\}]
{\color{incolor}In [{\color{incolor}8}]:} \PY{c+c1}{\PYZsh{} votre code ici}
\end{Verbatim}

    5.3 - Ecrivez un programme Python qui affiche la liste des axes TGV
cités dans la base.

    \begin{Verbatim}[commandchars=\\\{\}]
{\color{incolor}In [{\color{incolor}9}]:} \PY{c+c1}{\PYZsh{} votre code ici}
\end{Verbatim}

    5.4 - Ecrivez un programme Python qui affiche le nombre et la liste des
gares de départ sur l'axe Sud-Est.

    \begin{Verbatim}[commandchars=\\\{\}]
{\color{incolor}In [{\color{incolor}10}]:} \PY{c+c1}{\PYZsh{} votre code ici}
\end{Verbatim}

    5.5 - Ecrivez un programme Python qui affiche le nombre de trains au
départ de Paris Gare de Lyon en 2014.

    \begin{Verbatim}[commandchars=\\\{\}]
{\color{incolor}In [{\color{incolor}11}]:} \PY{c+c1}{\PYZsh{} votre code ici}
\end{Verbatim}

    5.6 - Ecrivez un programme Python qui affiche le nombre de trains partis
de Paris Gare de Lyon en 2014 et arrivés avec du retard.

    \begin{Verbatim}[commandchars=\\\{\}]
{\color{incolor}In [{\color{incolor}12}]:} \PY{c+c1}{\PYZsh{} votre code ici}
\end{Verbatim}

    5.7 - Ecrivez un programme Python qui affiche le nombre de trains
annulés au départ de Paris Gare de Lyon en 2014.

    \begin{Verbatim}[commandchars=\\\{\}]
{\color{incolor}In [{\color{incolor}13}]:} \PY{c+c1}{\PYZsh{} votre code ici}
\end{Verbatim}

    5.8 - Ecrivez un programme Python qui affiche le contenu de la table
regularite\_tgv mois par mois, dans l'ordre, au départ de Paris Gare de
Lyon, vers Lyon Part Dieu.

    \begin{Verbatim}[commandchars=\\\{\}]
{\color{incolor}In [{\color{incolor}14}]:} \PY{c+c1}{\PYZsh{} votre code ici}
\end{Verbatim}

    5.9 - En vous référant au tutoriel pyplot, tracez la courbe de
régularité correspondant à ces données. 

    \begin{Verbatim}[commandchars=\\\{\}]
{\color{incolor}In [{\color{incolor}15}]:} \PY{c+c1}{\PYZsh{} votre code ici}
\end{Verbatim}

    5.10 - Consultez le tutoriel et la doc, puis retracez ce graphique avec
un titre et un axe des abscisses correctement annoté (mois et années). 

    \begin{Verbatim}[commandchars=\\\{\}]
{\color{incolor}In [{\color{incolor}16}]:} \PY{c+c1}{\PYZsh{} votre code ici}
\end{Verbatim}

    5.11 - Le résultat n'étant pas très esthétique, pourquoi ne pas essayer
d'afficher les points de mesure superposés à une courbe dérivable ?
Voyez la doc puis retracez cette courbe en interpolant une dizaine de
points entre deux points de mesure à l'aide d'une fonction spline. 

    \begin{Verbatim}[commandchars=\\\{\}]
{\color{incolor}In [{\color{incolor}17}]:} \PY{c+c1}{\PYZsh{} votre code ici}
\end{Verbatim}

    5.12 - Affichez l'ensemble des commentaires (dernier champ de la table)
pour les TVG concernés par cette courbe.

    \begin{Verbatim}[commandchars=\\\{\}]
{\color{incolor}In [{\color{incolor}18}]:} \PY{c+c1}{\PYZsh{} votre code ici}
\end{Verbatim}

    5.13 - Affichez les commentaires pour l'ensemble des destinations au
départ de Paris Gare de Lyon, en évitant les doublons.

    \begin{Verbatim}[commandchars=\\\{\}]
{\color{incolor}In [{\color{incolor}19}]:} \PY{c+c1}{\PYZsh{} votre code ici}
\end{Verbatim}

    \subsubsection{6. Nouvelles données}\label{nouvelles-donnuxe9es}

    6.1 - Créez une table ponctualite\_transilien :

et chargez-la avec les données du fichier
ponctualite-mensuelle-transilien.csv.

    \begin{Verbatim}[commandchars=\\\{\}]
{\color{incolor}In [{\color{incolor}20}]:} \PY{c+c1}{\PYZsh{} votre code ici}
\end{Verbatim}

    6.2 - Affichez le nombre de lignes (de train) répertoriées dans cette
table, et leur liste avec id, type, code et nom correctement formattés.

    \begin{Verbatim}[commandchars=\\\{\}]
{\color{incolor}In [{\color{incolor}21}]:} \PY{c+c1}{\PYZsh{} votre code ici}
\end{Verbatim}

    6.3 - Créez une table couleur\_transilien :

et remplissez-la avec les informations du fichier
code-couleur-transilien.csv en calculant les valeurs du champ
code\_ligne à partir des informations du fichier, de manière à ce que
les valeurs obtenues soient cohérentes avec celles du champ éponyme de
la table ponctualite\_transilien (clé étrangère).

    \begin{Verbatim}[commandchars=\\\{\}]
{\color{incolor}In [{\color{incolor}22}]:} \PY{c+c1}{\PYZsh{} votre code ici}
\end{Verbatim}

    6.4 - Tracez un graphique supperposant les courbes de ponctualité des
lignes de RER en respectant les couleurs.

Attention à corriger au mieux l'absence d'information (il y a une valeur
de ponctualité manquante pour le RER B).

Exemple de résultat : 

    \begin{Verbatim}[commandchars=\\\{\}]
{\color{incolor}In [{\color{incolor}23}]:} \PY{c+c1}{\PYZsh{} votre code ici}
\end{Verbatim}

    \begin{Verbatim}[commandchars=\\\{\}]
{\color{incolor}In [{\color{incolor}24}]:} \PY{c+c1}{\PYZsh{} votre code ici}
\end{Verbatim}


    % Add a bibliography block to the postdoc
    
    
    
    \end{document}
